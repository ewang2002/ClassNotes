\documentclass[letterpaper]{article}
\usepackage[margin=1in]{geometry}
\usepackage[utf8]{inputenc}
\usepackage{textcomp}
\usepackage{amssymb}
\usepackage{natbib}
\usepackage{graphicx}
\usepackage{gensymb}
\usepackage{amsthm, amsmath, mathtools}
\usepackage[dvipsnames]{xcolor}
\usepackage{enumerate}
\usepackage{mdframed}
\usepackage[most]{tcolorbox}
\usepackage{csquotes}
% https://tex.stackexchange.com/questions/13506/how-to-continue-the-framed-text-box-on-multiple-pages

\tcbuselibrary{theorems}

\newcommand{\R}{\mathbb{R}}
\newcommand{\Z}{\mathbb{Z}}
\newcommand{\N}{\mathbb{N}}
\newcommand{\Q}{\mathbb{Q}}
\newcommand{\C}{\mathbb{C}}
\newcommand{\code}[1]{\texttt{#1}}
\newcommand{\mdiamond}{$\diamondsuit$}
\newcommand{\PowerSet}{\mathcal{P}}
\newcommand{\Mod}[1]{\ (\mathrm{mod}\ #1)}
\DeclareMathOperator{\lcm}{lcm}

%\newtheorem*{theorem}{Theorem}
%\newtheorem*{definition}{Definition}
%\newtheorem*{corollary}{Corollary}
%\newtheorem*{lemma}{Lemma}
\newtheorem*{proposition}{Proposition}


\newtcbtheorem[number within=section]{theorem}{Theorem}
{colback=green!5,colframe=green!35!black,fonttitle=\bfseries}{th}

\newtcbtheorem[number within=section]{definition}{Definition}
{colback=blue!5,colframe=blue!35!black,fonttitle=\bfseries}{def}

\newtcbtheorem[number within=section]{corollary}{Corollary}
{colback=yellow!5,colframe=yellow!35!black,fonttitle=\bfseries}{cor}

\newtcbtheorem[number within=section]{lemma}{Lemma}
{colback=red!5,colframe=red!35!black,fonttitle=\bfseries}{lem}

\newtcbtheorem[number within=section]{example}{Example}
{colback=white!5,colframe=white!35!black,fonttitle=\bfseries}{def}

\newtcbtheorem[number within=section]{note}{Important Note}{
        enhanced,
        sharp corners,
        attach boxed title to top left={
            xshift=-1mm,
            yshift=-5mm,
            yshifttext=-1mm
        },
        top=1.5em,
        colback=white,
        colframe=black,
        fonttitle=\bfseries,
        boxed title style={
            sharp corners,
            size=small,
            colback=red!75!black,
            colframe=red!75!black,
        } 
    }{impnote}
\usepackage[utf8]{inputenc}
\usepackage[english]{babel}
\usepackage{fancyhdr}
\usepackage[hidelinks]{hyperref}

\pagestyle{fancy}
\fancyhf{}
\rhead{Math 170A}
\chead{Friday, January 20, 2023}
\lhead{Lecture 5}
\rfoot{\thepage}

\setlength{\parindent}{0pt}

\newcommand{\0}{\mathbf{0}}
\newcommand{\y}{\mathbf{y}}
\renewcommand{\b}{\mathbf{b}}
\newcommand{\x}{\mathbf{x}}
\newcommand{\e}{\mathbf{e}}

\begin{document}

\section{Gaussian Elimination with Pivoting (1.8)}
In the previous section, we discussed Gaussian Elimination \emph{without} row interchanges (i.e., pivoting). In this section, we will now permit row interchanges. Consider the following system, 
\[\begin{bmatrix}
    0 & 4 & 1 \\ 
    1 & 3 & 4 \\ 
    2 & 2 & 5
\end{bmatrix} \begin{bmatrix}
    1 \\ 3 \\ 4
\end{bmatrix}.\]
Notice that this matrix has several properties: 
\begin{itemize}
    \item It's invertible ($\det(A) \neq 0$), therefore a unique solution exists. 
    \item However, using Gaussian Elimination \emph{without} row interchanges would fail. 
\end{itemize}
However, we can resolve this by switching two rows, e.g., R1 and R3, to get 
\[\begin{bmatrix}
    2 & 2 & 5 \\ 
    1 & 3 & 4 \\ 
    0 & 4 & 1
\end{bmatrix} \begin{bmatrix}
    4 \\ 3 \\ 1
\end{bmatrix}.\]
When deciding which two rows to switch, the rule is to use the row with the \emph{first} entry containing the \textbf{largest absolute value}\footnote{The reason why we choose the largest absolute value is because dividing by small numbers can cause errors, and these errors can accumulate.}. The idea is that we'll divide by first element in next step for Gaussian elimination.

\subsection{Relation to the Permutation Matrix}
Note taht a permutation matrix $P$ is a matrix which only contains 0's and 1's such that each row and each column has exactly one entry equal to 1. 

\begin{center}
    \begin{tabular}{p{3in}|p{3in}}
        \textbf{Example of Permutation Matrix} & \textbf{\underline{Not} a Permutation Matrix} \\ 
        \hline 
        \[\begin{bmatrix}
            0 & 1 & 0 \\ 
            1 & 0 & 0 \\ 
            0 & 0 & 1 
        \end{bmatrix}\] & \[\begin{bmatrix}
            0 & 1 & 1 \\ 
            1 & 0 & 0 \\ 
            0 & 0 & 1
        \end{bmatrix}\]
        This matrix has two 1's in the first row and last column.
    \end{tabular}
\end{center}
How does the permutation matrix work in relation to row interchanges?

\begin{mdframed}
    (Example.) Suppose we multiply a permutation matrix, 
    \[P = \begin{bmatrix}
        0 & 0 & 1 \\ 
        0 & 1 & 0 \\ 
        1 & 0 & 0
    \end{bmatrix},\]
    on the left. 
    \[\begin{bmatrix}
        0 & 0 & 1 \\ 
        0 & 1 & 0 \\ 
        1 & 0 & 0
    \end{bmatrix} \begin{bmatrix}
        0 & 4 & 1 \\ 
        1 & 3 & 4 \\ 
        2 & 2 & 5
    \end{bmatrix} = \begin{bmatrix}
        2 & 2 & 5 \\ 
        1 & 3 & 4 \\ 
        0 & 4 & 1
    \end{bmatrix}.\]
    This corresponds to switching R1 and R3. 
\end{mdframed}

\begin{mdframed}
    (Example.) Suppose we multiply the same permutation matrix as in the previous example on the right. Then, 
    \[\begin{bmatrix}
        0 & 4 & 1 \\ 
        1 & 3 & 4 \\ 
        2 & 2 & 5
    \end{bmatrix} \begin{bmatrix}
        0 & 0 & 1 \\ 
        0 & 1 & 0 \\ 
        1 & 0 & 0
    \end{bmatrix} = \begin{bmatrix}
        1 & 4 & 0 \\ 
        4 & 3 & 1 \\ 
        5 & 2 & 2
    \end{bmatrix}.\]
    Here, this corresponds to switching C1 and C3. 
\end{mdframed}

\subsection{PLU Decomposition}
Gaussian Elimination with pivoting leads to a decomposition of the form $PA = LU$, also known as \emph{PLU decomposition}, where 
\begin{itemize}
    \item $P$ is a permutation matrix,
    \item $L$ is a lower-triangular matrix, and 
    \item $U$ is an upper-triangular matrix. The PLU exists if $A$ is invertible. 
\end{itemize}
\textbf{Remarks:}
\begin{itemize}
    \item The PLU exists if $A$ is invertible. 
    \item In MATLAB, we can run \code{[P, L, U] = lu(A)}, where 
    \[P \cdot A = L \cdot U \implies A = P^T \cdot L \cdot U.\]
\end{itemize}

\subsubsection{Finding PLU Decomposition}
We'll find the PLU decomposition through an example. 

\begin{mdframed}
    (Example.) Suppose we want to find the PLU decomposition of 
    \[A = \begin{bmatrix}
        0 & 4 & 1 \\ 
        1 & 3 & 4 \\ 
        2 & 2 & 5
    \end{bmatrix}.\]

    \begin{itemize}
        \item Step 1: First, let's switch R1 and R3. 
        \[P_1 = \begin{bmatrix}
            0 & 0 & 1 \\ 
            0 & 1 & 0 \\ 
            1 & 0 & 0 
        \end{bmatrix}, \qquad P_1 A = \begin{bmatrix}
            2 & 2 & 5 \\ 
            1 & 3 & 4 \\ 
            0 & 4 & 1
        \end{bmatrix}.\]

        \item Step 2: Next, let's perform the operation $R_2 \mapsto R_2 - \frac{1}{2} R_1$.
        \[L_1 = \begin{bmatrix}
            1 & 0 & 0 \\ 
            -\frac{1}{2} & 1 & 0 \\ 
            0 & 0 & 1
        \end{bmatrix}, \qquad L_1 P_1 A = L_1 \begin{bmatrix}
            2 & 2 & 5 \\ 
            1 & 3 & 4 \\ 
            0 & 4 & 1
        \end{bmatrix} = \begin{bmatrix}
            2 & 2 & 5 \\ 
            0 & 2 & \frac{3}{2} \\ 
            0 & 4 & 1
        \end{bmatrix}.\]

        \item Step 3: Next, notice that \textbf{4} is a pivot (the 2 in the second row is smaller than 4). let's switch $R_2$ and $R_3$.
        \[P_2 = \begin{bmatrix}
            1 & 0 & 0 \\ 
            0 & 0 & 1 \\ 
            0 & 1 & 0
        \end{bmatrix}, \qquad P_2 L_1 P_1 A = \begin{bmatrix}
            2 & 2 & 5 \\ 
            0 & 4 & 1 \\ 
            0 & 2 & \frac{3}{2}
        \end{bmatrix}.\]

        \item Step 4: We can now perform the operation $R_3 \mapsto R_3 - \frac{1}{2}R_2$. 
        \[L_2 = \begin{bmatrix}
            1 & 0 & 0 \\ 
            0 & 1 & 0 \\ 
            0 & -\frac{1}{2} & 1
        \end{bmatrix}, \qquad L_2 P_2 L_1 P_1 A = \begin{bmatrix}
            2 & 2 & 5 \\ 
            0 & 4 & 1 \\ 
            0 & 0 & 1
        \end{bmatrix}.\]
    \end{itemize}
    Here, we find our upper-triangular matrix \[U = L_2 P_2 L_1 P_1 A = \begin{bmatrix}
        2 & 2 & 5 \\ 
        0 & 4 & 1 \\ 
        0 & 0 & 1
    \end{bmatrix}.\] Now that we found the upper-triangular matrix $U$, we need to find the decomposition. We don't know what the $P$ or $L$ matrices are. Notice that, with $L_2 P_2 L_1 P_1$, we can do 
    \[P_2 L_1 = \tilde{L_1} P_2.\] Then, 
    \[L_2 P_2 L_1 P_1 = \underbrace{L_2 \tilde{L_1}}_{\tilde{L}} \overbrace{P_2 P_1}^{P}.\]

    Therefore, if \[P_2 L_1 = \begin{bmatrix}
        1 & 0 & 0 \\ 
        0 & 0 & 1 \\ 
        -\frac{1}{2} & 1 & 0
    \end{bmatrix},\] then
    \begin{equation*}
        \begin{aligned}
            P_2 L_1 &= \tilde{L_1} P_2 \\ 
                &\implies \tilde{L_1} = P_2 L_1 P_2^{-1} \\ 
                &\implies \tilde{L_1} = \begin{bmatrix}
                    1 & 0 & 0 \\ 
                    0 & 0 & 1 \\ 
                    -\frac{1}{2} & 1 & 0
                \end{bmatrix} P_2^{-1} \\ 
                &\implies \tilde{L_1} = \begin{bmatrix}
                    1 & 0 & 0 \\ 
                    0 & 0 & 1 \\ 
                    -\frac{1}{2} & 1 & 0
                \end{bmatrix} \begin{bmatrix}
                    1 & 0 & 0 \\ 
                    0 & 0 & 1 \\ 
                    0 & 1 & 0
                \end{bmatrix} \\ 
                &\implies \tilde{L_1} = \begin{bmatrix}
                    1 & 0 & 0 \\ 
                    0 & 1 & 0 \\ 
                    -\frac{1}{2} & 0 & 1
                \end{bmatrix}.
        \end{aligned}
    \end{equation*}
    Therefore, going back to 
    \[L_2 P_2 L_1 P_1 = \underbrace{L_2 \tilde{L_1}}_{\tilde{L}} \overbrace{P_2 P_1}^{P},\]
    we have 
    \[\tilde{L}PA = U \implies PA = \tilde{L}^{-1} U.\]
    From this, we find 
    \[\tilde{L}^{-1} = \begin{bmatrix}
        1 & 0 & 0 \\ 
        0 & 1 & 0 \\ 
        \frac{1}{2} & 0 & 1
    \end{bmatrix}.\]
    Finally, defining $L = \tilde{L}^{-1}$ gives us 
    \[PA = LU.\]
\end{mdframed}

\end{document}