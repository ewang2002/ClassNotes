\documentclass[letterpaper]{article}
\usepackage[margin=1in]{geometry}
\usepackage[utf8]{inputenc}
\usepackage{textcomp}
\usepackage{amssymb}
\usepackage{natbib}
\usepackage{graphicx}
\usepackage{gensymb}
\usepackage{amsthm, amsmath, mathtools}
\usepackage[dvipsnames]{xcolor}
\usepackage{enumerate}
\usepackage{mdframed}
\usepackage[most]{tcolorbox}
\usepackage{csquotes}
% https://tex.stackexchange.com/questions/13506/how-to-continue-the-framed-text-box-on-multiple-pages

\tcbuselibrary{theorems}

\newcommand{\R}{\mathbb{R}}
\newcommand{\Z}{\mathbb{Z}}
\newcommand{\N}{\mathbb{N}}
\newcommand{\Q}{\mathbb{Q}}
\newcommand{\C}{\mathbb{C}}
\newcommand{\code}[1]{\texttt{#1}}
\newcommand{\mdiamond}{$\diamondsuit$}
\newcommand{\PowerSet}{\mathcal{P}}
\newcommand{\Mod}[1]{\ (\mathrm{mod}\ #1)}
\DeclareMathOperator{\lcm}{lcm}

%\newtheorem*{theorem}{Theorem}
%\newtheorem*{definition}{Definition}
%\newtheorem*{corollary}{Corollary}
%\newtheorem*{lemma}{Lemma}
\newtheorem*{proposition}{Proposition}


\newtcbtheorem[number within=section]{theorem}{Theorem}
{colback=green!5,colframe=green!35!black,fonttitle=\bfseries}{th}

\newtcbtheorem[number within=section]{definition}{Definition}
{colback=blue!5,colframe=blue!35!black,fonttitle=\bfseries}{def}

\newtcbtheorem[number within=section]{corollary}{Corollary}
{colback=yellow!5,colframe=yellow!35!black,fonttitle=\bfseries}{cor}

\newtcbtheorem[number within=section]{lemma}{Lemma}
{colback=red!5,colframe=red!35!black,fonttitle=\bfseries}{lem}

\newtcbtheorem[number within=section]{example}{Example}
{colback=white!5,colframe=white!35!black,fonttitle=\bfseries}{def}

\newtcbtheorem[number within=section]{note}{Important Note}{
        enhanced,
        sharp corners,
        attach boxed title to top left={
            xshift=-1mm,
            yshift=-5mm,
            yshifttext=-1mm
        },
        top=1.5em,
        colback=white,
        colframe=black,
        fonttitle=\bfseries,
        boxed title style={
            sharp corners,
            size=small,
            colback=red!75!black,
            colframe=red!75!black,
        } 
    }{impnote}
\usepackage[utf8]{inputenc}
\usepackage[english]{babel}
\usepackage{fancyhdr}
\usepackage[hidelinks]{hyperref}

\pagestyle{fancy}
\fancyhf{}
\rhead{Math 170A}
\chead{Monday, January 09, 2023}
\lhead{Lecture 1}
\rfoot{\thepage}

\setlength{\parindent}{0pt}

\begin{document}

\section{Matrix Multiplication (Section 1.1)}
Consider an $n \times m$ matrix, or a matrix with $n$ rows and $m$ columns:
\[
    A = \begin{bmatrix}
        a_{11} & a_{12} & \hdots & a_{1m} \\ 
        a_{21} & a_{22} & \hdots & a_{2m} \\ 
        \vdots & \vdots &        & \vdots \\ 
        a_{n1} & a_{n2} & \hdots & a_{nm}
    \end{bmatrix}.
\]
The entries of $A$ might be real or complex numbers although, for now, we'll assume they're real. Suppose we have an $m$-tuple (or vector) of real numbers: 
\[\mathbf{x} = \begin{bmatrix}
    x_1 \\ x_2 \\ \vdots \\ x_m
\end{bmatrix}.\]

\subsection{Multiplying Matrix by Vector}
Suppose $A$ is an $n \times \boxed{m}$ matrix and $\mathbf{x}$ is a vector with $m$ elements (i.e., a $\boxed{m} \times 1$ matrix). Let's suppose we wanted to find $A\mathbf{x}$. There are two ways we can do this. 

\subsubsection{Solving as a Single Component}
We can solve for
\[A\mathbf{x} = \mathbf{b},\]
where 
\[\mathbf{b} = \begin{bmatrix}
    b_1 \\ b_2 \\ \vdots \\ b_n
\end{bmatrix}.\]
Here, 
\[b_i = a_{i1} x_1 + a_{i2} x_2 + \hdots + a_{im} x_{m} = \sum_{j = 1}^{m} a_{ij} x_j.\]

\begin{mdframed}
    (Example.) Consider
    \[A = \begin{bmatrix}
        1 & 2 & 3 \\ 
        4 & 5 & 6
    \end{bmatrix} \qquad \mathbf{x} = \begin{bmatrix}
        7 \\ 8 \\ 9
    \end{bmatrix}.\]
    We note that 
    \[A\mathbf{x} = \begin{bmatrix}
        b_1 \\ 
        b_2
    \end{bmatrix},\]
    where 
    \[b_1 = a_{11}x_1 + a_{12}x_2 + a_{13}x_3 = 1(7) + 2(8) + 3(9) = 50\]
    and 
    \[b_2 = a_{21}x_1 + a_{22}x_2 + a_{23}x_3 = 4(7) + 5(8) + 6(9) = 122.\]
\end{mdframed}
We can write some code to perform this operation for us:
\begin{algorithmic}
    \State $\mathbf{b} \gets \mathbf{0}$
    \For{$i = 1, \hdots, n$}
        \For{$i = 1, \hdots, m$}
            \State $b_i \gets b_i + a_{ij} x_j$
        \EndFor
    \EndFor
\end{algorithmic}
Note that the $j$-loop accumulates the inner product $b_i$. 
    
\subsubsection{Solving as a Formula}
We can also solve for $A\mathbf{x} = \mathbf{b}$ by considering the following:
\[\begin{bmatrix}
    b_1 \\ b_2 \\ \vdots \\ b_n 
\end{bmatrix} = \begin{bmatrix}
    a_{11} \\ a_{21} \\ \vdots \\ a_{n1}
\end{bmatrix} x_1 + \begin{bmatrix}
    a_{12} \\ a_{22} \\ \vdots \\ a_{n2}
\end{bmatrix} x_2 + \hdots + \begin{bmatrix}
    a_{1m} \\ a_{2m} \\ \vdots \\ a_{nm}
\end{bmatrix} x_m.\]
This shows that $\mathbf{b}$ is a linear combination of the columns of $A$. 

\begin{mdframed}
    (Example.) Consider
    \[A = \begin{bmatrix}
        1 & 2 & 3 \\ 
        4 & 5 & 6
    \end{bmatrix} \qquad \mathbf{x} = \begin{bmatrix}
        7 \\ 8 \\ 9
    \end{bmatrix}.\]
    Using this approach, we now have 
    \[\begin{bmatrix}
        1 \\ 4
    \end{bmatrix} 7 + \begin{bmatrix}
        2 \\ 5
    \end{bmatrix} 8 + \begin{bmatrix}
        3 \\ 6
    \end{bmatrix} 9 = \begin{bmatrix}
        7 \\ 28
    \end{bmatrix} + \begin{bmatrix}
        16 \\ 40
    \end{bmatrix} + \begin{bmatrix}
        27 \\ 54
    \end{bmatrix} = \begin{bmatrix}
        50 \\ 122
    \end{bmatrix}.\]
\end{mdframed}

\begin{proposition}
    If $\mathbf{b} = A\mathbf{x}$, then $\mathbf{b}$ is a linear combination of the columns of $A$. If we let $A_j$ denote the $j$th column of $A$, we have 
    \[\mathbf{b} = \sum_{j = 1}^{m} A_j x_j.\]
\end{proposition}
We can also express this as pseudocode: 
\begin{algorithmic}
    \State $\mathbf{b} \gets \mathbf{0}$
    \For{$j = 1, \hdots, m$}
        \State $\mathbf{b} \gets \mathbf{b} + A_j x_j$
    \EndFor
\end{algorithmic}
If we use a loop to perform each vector operation, the code becomes:
\begin{algorithmic}
    \State $\mathbf{b} \gets \mathbf{0}$
    \For{$j = 1, \hdots, m$}
        \For{$i = 1, \hdots, n$}
            \State $b_i \gets b_i + a_{ij} x_j$
        \EndFor 
    \EndFor
\end{algorithmic}
Notice how the loops for rows and columns are interchangeable.

\subsection{Flop Counts}
We note that real numbers are normally stored in computers in a floating-point format. The arithmetic operations that a computer performs on these numbers are called \emph{floating-point operations}, or \emph{flops}. So, 
\[b_i \gets b_i + a_{ij}x_j\]
involves two flops: one floating-point multiply and one floating-point add. \textbf{Essentially}, we would like to count the number of operations on real numbers, or the number of addition, subtraction, multiplication, and division operations, within our program.

\begin{mdframed}
    (Example.) Consider 
    \[\mathbf{v} = \begin{bmatrix}
        v_1 \\ v_2 \\ \vdots \\ v_n
    \end{bmatrix} \qquad \mathbf{w} = \begin{bmatrix}
        w_1 \\ w_2 \\ \vdots \\ w_n
    \end{bmatrix}.\]
    Let's compute the inner product\footnote{A generalization of the dot product}:
    \begin{equation*}
        \begin{aligned}
            \cyclic{\mathbf{v}, \mathbf{w}} &= \mathbf{v}^T \cdot \mathbf{w} = \begin{bmatrix}
                v_1 & v_2 & \hdots & v_n
            \end{bmatrix} \begin{bmatrix}
                w_1 \\ w_2 \\ \vdots \\ w_n
            \end{bmatrix} \\ 
                &= v_1 \cdot w_1 + v_2 \cdot w_2 + \hdots + v_n \cdot w_n \\ 
                &= \sum_{i = 1}^{n} v_i w_i.
        \end{aligned}
    \end{equation*}
    There are $n - 1$ addition operations and $n$ multiplication operations, for a total of $2n - 1$ flop count.
\end{mdframed}

\subsubsection{Big-O Notation}
We note that $\mathbf{v}^T \mathbf{w}$, where $\mathbf{v}, \mathbf{w} \in \R^n$ needs $2n$ flops. This is the same thing as saying that the operation takes $\BigO(n)$ time as $n \mapsto \infty$. This means that we can disregard the implied constant, which is 2 in this case. The reason why we can do this is because, as $n$ grows larger, the constant doesn't really matter all that much. 

\bigskip 

For instance, if $n$ is large, then $\BigO(n)$ is faster than $\BigO(n^2)$, and $\BigO(n^2)$ is faster than $\BigO(n^3)$. 

\end{document}