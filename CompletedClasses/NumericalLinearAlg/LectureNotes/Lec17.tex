\documentclass[letterpaper]{article}
\usepackage[margin=1in]{geometry}
\usepackage[utf8]{inputenc}
\usepackage{textcomp}
\usepackage{amssymb}
\usepackage{natbib}
\usepackage{graphicx}
\usepackage{gensymb}
\usepackage{amsthm, amsmath, mathtools}
\usepackage[dvipsnames]{xcolor}
\usepackage{enumerate}
\usepackage{mdframed}
\usepackage[most]{tcolorbox}
\usepackage{csquotes}
% https://tex.stackexchange.com/questions/13506/how-to-continue-the-framed-text-box-on-multiple-pages

\tcbuselibrary{theorems}

\newcommand{\R}{\mathbb{R}}
\newcommand{\Z}{\mathbb{Z}}
\newcommand{\N}{\mathbb{N}}
\newcommand{\Q}{\mathbb{Q}}
\newcommand{\C}{\mathbb{C}}
\newcommand{\code}[1]{\texttt{#1}}
\newcommand{\mdiamond}{$\diamondsuit$}
\newcommand{\PowerSet}{\mathcal{P}}
\newcommand{\Mod}[1]{\ (\mathrm{mod}\ #1)}
\DeclareMathOperator{\lcm}{lcm}

%\newtheorem*{theorem}{Theorem}
%\newtheorem*{definition}{Definition}
%\newtheorem*{corollary}{Corollary}
%\newtheorem*{lemma}{Lemma}
\newtheorem*{proposition}{Proposition}


\newtcbtheorem[number within=section]{theorem}{Theorem}
{colback=green!5,colframe=green!35!black,fonttitle=\bfseries}{th}

\newtcbtheorem[number within=section]{definition}{Definition}
{colback=blue!5,colframe=blue!35!black,fonttitle=\bfseries}{def}

\newtcbtheorem[number within=section]{corollary}{Corollary}
{colback=yellow!5,colframe=yellow!35!black,fonttitle=\bfseries}{cor}

\newtcbtheorem[number within=section]{lemma}{Lemma}
{colback=red!5,colframe=red!35!black,fonttitle=\bfseries}{lem}

\newtcbtheorem[number within=section]{example}{Example}
{colback=white!5,colframe=white!35!black,fonttitle=\bfseries}{def}

\newtcbtheorem[number within=section]{note}{Important Note}{
        enhanced,
        sharp corners,
        attach boxed title to top left={
            xshift=-1mm,
            yshift=-5mm,
            yshifttext=-1mm
        },
        top=1.5em,
        colback=white,
        colframe=black,
        fonttitle=\bfseries,
        boxed title style={
            sharp corners,
            size=small,
            colback=red!75!black,
            colframe=red!75!black,
        } 
    }{impnote}
\usepackage[utf8]{inputenc}
\usepackage[english]{babel}
\usepackage{fancyhdr}
\usepackage[hidelinks]{hyperref}
\usepackage{csquotes}

\pagestyle{fancy}
\fancyhf{}
\rhead{Math 170A}
\chead{Friday, February 24, 2023}
\lhead{Lecture 18}
\rfoot{\thepage}

\setlength{\parindent}{0pt}
\newcommand{\0}{\mathbf{0}}
\newcommand{\y}{\mathbf{y}}
\renewcommand{\b}{\mathbf{b}}
\newcommand{\x}{\mathbf{x}}
\newcommand{\e}{\mathbf{e}}
\newcommand{\rr}{\mathbf{r}}
\newcommand{\vv}{\mathbf{v}}
\renewcommand{\u}{\mathbf{u}}

\begin{document}

\section{Pseudoinverse (4.3)}
We will solve the least squares with the SVD with full rank matrix $A$. Using SVD also works when the rank of $A$ is not full. Recall that 
\[A \in \R^{n \times m} \qquad n \geq m \qquad \min_{\x \in \R^m} ||\b - A\x||_2^2.\]

\subsection{Brief Review}
Recall that, for a full rank $A$, i.e., $\text{rank}(A) = m$, we can use full QR decomposition ($A = QR$) or reduced QR decomposition ($A = \hat{Q}\hat{R}$). In particular, for reduced QR, $A = \hat{Q}\hat{R}$ where $\hat{Q} \in \R^{n \times m}$ and $\hat{R} \in \R^{m \times m}$. Recall that $A\x = \b$ as well, so 
\begin{equation*}
    \begin{aligned}
        A\x &= \b \\ 
            &\implies \hat{Q}\hat{R}\x = \b \\ 
            &\implies \hat{Q}^{T}\hat{Q}\hat{R}\x = \hat{Q}^{T}\b \\ 
            &\implies \hat{R}\x = \hat{Q}^{T}\b && \text{Since } Q^T Q = I.
    \end{aligned}
\end{equation*}
Remember that, because $\hat{Q}$ is orthogonal, $\hat{Q} \hat{Q}^T = I_{n \times n}$ and $\hat{Q}^T \hat{Q} = I_{m \times m}$. Now, remember that if $\hat{R}$ has full rank, then it'll look like 
\[\hat{R} = \begin{bmatrix}
    * & * & * & * & \hdots & * \\ 
    0 & * & * & * & \hdots & * \\ 
    0 & 0 & * & * & \hdots & * \\ 
    0 & 0 & 0 & * & \hdots & * \\ 
    \vdots & \vdots & \vdots & \vdots & \ddots & \vdots \\ 
    0 & 0 & 0 & 0 & \hdots & * \\ 
\end{bmatrix}.\]
Then, the associated equation will have a unique solution since it has $m$ equations and $m$ unknowns. \textbf{Now}, what if $\text{rank}(R) = r < m$? Then, it is not full rank and it'll look like 
\[\hat{R} = \begin{bmatrix}
    * & * & * & * & \hdots & * \\ 
    0 & * & * & * & \hdots & * \\ 
    0 & 0 & * & * & \hdots & * \\ 
    0 & 0 & 0 & 0 & \hdots & 0 \\ 
    \vdots & \vdots & \vdots & \vdots & \ddots & \vdots \\ 
    0 & 0 & 0 & 0 & \hdots & 0 \\ 
\end{bmatrix}.\]
In particular, some of the diagonal entries are 0's. So, $\hat{R}\x = \hat{Q}^T \b$. We have $m$ unknowns, but we only have $r$ independent equations. Those equations are not helpful since we can't use them to solve for the unknowns -- we'll end up with infinitely many solutions. So, we'll have to choose one of these infinitely many solutions based on $||\x||_2$. 

\subsection{A New Problem}
We now have a new problem to solve. 
\begin{itemize}
    \item Find $\min||\b - A\x||_2^2$ (infinitely many solutions.)
    \item Pick $\x$ with minimal $||\x||_2$. 
\end{itemize}
With the above two minimizers, there \emph{will} be a unique solution $\x$. We will see that the unique $\x$ can be written as $\x = A^+ \b$, where $A^+$ is known as the \textbf{pseudoinverse}\footnote{Note that if $A$ is invertible, then $\x = A^{-1}\b$, so in some sense $A^+$ is mimicking $A^{-1}$.}.

\bigskip 

With this in mind, how do we find the least squares solution of the minimal 2-norm? With $A = U\Sigma V^T$, we have 
\begin{equation*}
    \begin{aligned}
        ||\b - A\x||_2^2 &= ||\b - U\Sigma V^T \x||_2^2 \\ 
            &= ||U(U^T \b) - U(\Sigma V^T \x)||_2^2 && \text{Recall that } UU^T = U^T U = I \\ 
            &= ||U(U^T \b - \Sigma V^T \x)||_2^2 \\ 
            &= ||U^T \b - \Sigma V^T \x||_2^2 \\ 
            &= \left|\left| \underbrace{\begin{bmatrix}
                \hat{c} \\ 
                d
            \end{bmatrix}}_{U^T \b \in \R^{n \times 1}} - \left[
                \begin{array}{c|c}
                    \hat{\Sigma} & \makebox{0} \\ 
                    \hline 
                    \makebox{0} & \makebox{0}
                \end{array}
            \right] \underbrace{\begin{bmatrix}
                \hat{y} \\ z
            \end{bmatrix}}_{V^T \x \in \R^{m \times 1}} \right|\right|_2^2 && \text{See remark.} \\ 
            &= \left|\left| \begin{bmatrix}
                \hat{c} \\ 
                d
            \end{bmatrix} - \begin{bmatrix}
                \hat{\Sigma} \hat{y} \\ 
                \0
            \end{bmatrix} \right|\right|_2^2  \\ 
            &= \left|\left| \begin{bmatrix}
                \hat{c} - \hat{\Sigma} \hat{y} \\ 
                d
            \end{bmatrix} \right|\right|_2^2 \\ 
            &= ||\hat{c} - \hat{\Sigma} \hat{y}||_2^2 + ||d||_2^2
    \end{aligned}
\end{equation*}
\textbf{Remarks:}
\begin{itemize}
    \item Note that $\hat{c} \in \R^r$ and $d \in \R^{n - r}$ and $\hat{y} \in \R^r$ and $z \in \R^{m - r}$.
    \item Additionally, \[\Sigma = \left[
        \begin{array}{c|c}
            \hat{\Sigma} & \makebox{0} \\ 
            \hline 
            \makebox{0} & \makebox{0}
        \end{array}
    \right]\] where $\hat{\Sigma}$ is an $r \times r$ matrix. 

    \item $d$ is independent of $\x$ when minimizing over $\x$. 
\end{itemize}
In any case, we want to minimize $||\hat{c} - \hat{\Sigma} \hat{y}||_2^2$. This can be done by solving \[\hat{c} = \hat{\Sigma}\hat{y}.\] This yields \[\begin{bmatrix}
    \sigma_1 & & & & \\ 
    & \sigma_2 & & & \\ 
    & & \sigma_3 & & \\ 
    & & & \ddots & \\ 
    & & & & \sigma_r
\end{bmatrix} \begin{bmatrix}
    \hat{y}_1 \\ \hat{y}_2 \\ \vdots \\ \hat{y}_r
\end{bmatrix} = \begin{bmatrix}
    \hat{c}_1 \\ \hat{c}_2 \\ \vdots \\ \hat{c}_r
\end{bmatrix}.\]
And this gives us 
\[\hat{y}_i = \frac{\hat{c}_i}{\sigma_i}, i = 1, \hdots, r.\] To find $\x$, we do \[V^T \x = \begin{bmatrix}
    \hat{y} \\ z
\end{bmatrix} \implies \x = V\begin{bmatrix}
    \hat{y} \\ z 
\end{bmatrix}.\]
No matter the value of $z$, it won't affect the value of $\hat{y}$. We can define many $\x$ values such that the least square problems has the minimum. However, we want to define the unique solution $\x$, so how do we choose $z$ so we can find the minimized $||\x||_2$? 
\[||\x||_2^2 = \left|\left|V \begin{bmatrix}
    \hat{y} \\ z
\end{bmatrix}\right|\right|_2^2 = \left|\left|\begin{bmatrix}
    \hat{y} \\ 
    z
\end{bmatrix}\right|\right|_2^2 = ||\hat{y}||_2^2 + ||z||_2^2,\] where recall that $V$ is orthogonal. So, $||\x||_2^2$ is minimized when $z = \0$. So, in summary, $\boxed{\x = V\begin{bmatrix}
    \hat{y} \\ \0
\end{bmatrix}}$ is the unique solution.

% last few minutes of lecture 19 for summary

\end{document}