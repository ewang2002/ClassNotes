\documentclass[letterpaper]{article}
\usepackage[margin=1in]{geometry}
\usepackage[utf8]{inputenc}
\usepackage{textcomp}
\usepackage{amssymb}
\usepackage{natbib}
\usepackage{graphicx}
\usepackage{gensymb}
\usepackage{amsthm, amsmath, mathtools}
\usepackage{xcolor}
\usepackage{enumerate}
\usepackage{framed}
\usepackage{tcolorbox}
\tcbuselibrary{theorems}

\newcommand{\R}{\mathbb{R}}
\newcommand{\Z}{\mathbb{Z}}
\newcommand{\N}{\mathbb{N}}
\newcommand{\Q}{\mathbb{Q}}
\newcommand{\code}[1]{\texttt{#1}}
\newcommand{\mdiamond}{$\diamondsuit$}

%\newtheorem*{theorem}{Theorem}
%\newtheorem*{definition}{Definition}
\newtheorem*{proposition}{Proposition}
%\newtheorem*{corollary}{Corollary}
%\newtheorem*{lemma}{Lemma}

\newtcbtheorem[number within=section]{theorem}{Theorem}
{colback=green!5,colframe=green!35!black,fonttitle=\bfseries}{def}

\newtcbtheorem[number within=section]{definition}{Definition}
{colback=blue!5,colframe=blue!35!black,fonttitle=\bfseries}{def}

\newtcbtheorem[number within=section]{corollary}{Corollary}
{colback=yellow!5,colframe=yellow!35!black,fonttitle=\bfseries}{def}

\newtcbtheorem[number within=section]{lemma}{Lemma}
{colback=red!5,colframe=red!35!black,fonttitle=\bfseries}{def}
\usepackage[utf8]{inputenc}
\usepackage[english]{babel}
\usepackage{fancyhdr}
\usepackage[hidelinks]{hyperref}
\usepackage{csquotes}

\pagestyle{fancy}
\fancyhf{}
\rhead{Math 170A}
\chead{Wednesday, February 22, 2023}
\lhead{Lecture 17}
\rfoot{\thepage}

\setlength{\parindent}{0pt}
\newcommand{\0}{\mathbf{0}}
\newcommand{\y}{\mathbf{y}}
\renewcommand{\b}{\mathbf{b}}
\newcommand{\x}{\mathbf{x}}
\newcommand{\e}{\mathbf{e}}
\newcommand{\rr}{\mathbf{r}}
\newcommand{\vv}{\mathbf{v}}
\renewcommand{\u}{\mathbf{u}}

\begin{document}

\section{Singular Value Decomposition, Continued (4.1, 4.2)}
(Continued from previous notes.)

\subsection{Relationship to Norm and Condition Number}
Recall that we defined the matrix 2-norm as  
\[||A|| = \max_{\x \neq \0} \frac{||A\x||_2}{||\x||} = \sigma_1,\] where $\sigma_1$ is the largest singular value. Note that this definition also makes sense for $A \in \R^{n \times m}$. 

\begin{theorem}{}{}
    \[||A||_2 = \sigma_1.\]
\end{theorem}

Since $A$ and $A^T$ have the same singular values, we have the following corollary. 
\begin{corollary}{}{}
    \[||A||_2 = ||A^T||_2.\]
\end{corollary}
Since $A$ is nonsingular, $A$ has full rank, i.e., rank $n$. $A$ has $n$ strictly positive singualr values, $\sigma_1 \geq \sigma_2 \geq \hdots \geq \sigma_n > 0$. Now, 
\[A^{-1} Av_i = A^{-1}(\sigma_i u-i) \implies v_i = \sigma_i A^{-1} u_i \implies A^{-1} u_i = \frac{1}{\sigma_i} v_i,\] so in particular we can map each $\sigma$ like so: 
\begin{center}
    \begin{tabular}{p{2in} p{2in}}
        \[A\] & \[A^{-1}\] \\ 
        \[v_1 \xrightarrow[\sigma_1]{} u_1\]
        \[v_2 \xrightarrow[\sigma_2]{} u_2\]
        \[v_3 \xrightarrow[\sigma_3]{} u_3\]
        \[\vdots\]
        \[v_n \xrightarrow[\sigma_n]{} u_n\] &
        \[u_1 \xrightarrow[\sigma_1^{-1}]{} v_1\]
        \[u_2 \xrightarrow[\sigma_2^{-1}]{} v_2\]
        \[u_3 \xrightarrow[\sigma_3^{-1}]{} v_3\]
        \[\vdots\]
        \[u_{n} \xrightarrow[\sigma_n^{-1}]{} v_n\]
    \end{tabular}
\end{center}
This tells us that the singular values of $A^{-1}$ must be \[\frac{1}{\sigma_n} \geq \frac{1}{\sigma_{n - 1}} \geq \hdots \geq \frac{1}{\sigma_2} \geq \frac{1}{\sigma_1} > 0\]
such that 
\[\Sigma^{-1} = \begin{bmatrix}
    \frac{1}{\sigma_1} & 0 & 0 & 0 \\ 
    0 & \frac{1}{\sigma_2} & 0 & 0 \\ 
    0 & 0 & \ddots & 0 \\ 
    0 & 0 & 0 & \frac{1}{\sigma_n} 
\end{bmatrix}.\]
And, in particular, \[||A^{-1}||_2 = \frac{1}{\sigma_n} \qquad ||A||_2 = \sigma_1.\]

\begin{theorem}{}{}
    Let $A \in \R^{n \times n}$ be a nonsingular matrix with singular values $\sigma_1 \geq \hdots \geq \sigma_n > 0$. Then, 
    \[\kappa_{2}(A) = \frac{\sigma_1}{\sigma_n}.\]
\end{theorem}

\subsection{More on SVD}
Remember that there are two types of SVD: 
\begin{itemize}
    \item Full SVD.
    \[A = U\Sigma V^T,\]
    where $A$ is $n \times m$, $U$ is $n \times n$, $\Sigma$ is $n \times m$, and $V^T$ is $m \times m$. Here, $\text{rank}(A) = r \leq m$ and $n \geq m$. 

    \item Reduced SVD 
    \[A = \hat{U}\hat{\Sigma}\hat{V^T},\]
    where $A$ is $n \times m$, $\hat{U}$ is $n \times r$, $\hat{\Sigma}$ is $r \times r$, and $\hat{V}^T$ is $r \times m$.
\end{itemize}
In any case, we now know that \[||A||_2 = \sigma_1 \qquad \kappa_{2}(A) = \frac{\sigma_1}{\sigma_n}.\]

\subsection{Rank-1 Decomposition}
\begin{theorem}{}{}
    Let $A \in \R^{n \times m}$ be a nonzero matrix with rank $r$. Let $\sigma_1, \hdots, \sigma_r$ be the singular values of $A$, with associated right and left singular vectors $v_1, \hdots, v_r$ and $u_1, \hdots, u_r$, respectively. Then, 
    \[A = \sum_{j = 1}^{r} \sigma_j u_j v_j^T,\]
    where $u_j \in \R^{n}$, $v_j \in \R^{m}$, and $u_j v_j^T \in \R^{n \times m}$. 
\end{theorem}
To see why this theorem works, 
\begin{equation*}
    \begin{aligned}
        A &= \hat{U}\hat{\Sigma}\hat{V^T} \\ 
            &= \underbrace{\begin{bmatrix}
                u_1 & v_2 & \hdots & u_r
            \end{bmatrix}}_{\hat{U}} \underbrace{\begin{bmatrix}
                \sigma_1 & 0 & 0 & 0 \\ 
                0 & \sigma_2 & 0 & 0 \\ 
                0 & 0 & \ddots & 0 \\ 
                0 & 0 & 0 & \sigma_r
            \end{bmatrix}}_{\hat{\Sigma}} \underbrace{\begin{bmatrix}
                v_1^T \\ v_2^T \\ \vdots \\ v_r^T
            \end{bmatrix}}_{\hat{V}} \\ 
            &= \begin{bmatrix}
                \sigma_1 u_1 & \sigma_2 u_2 & \hdots & \sigma_r u_r
            \end{bmatrix} \begin{bmatrix}
                v_1^T \\ v_2^T \\ \vdots \\ v_r^T
            \end{bmatrix} \\ 
            &= \sigma_1 u_1 v_1^T + \sigma_2 u_2 v_2^T + \hdots + \sigma_r u_r v_r^T \\ 
            &= \sum_{i = 1}^{r} \sigma_i u_i v_{i}^T.
    \end{aligned}
\end{equation*}
This is called the \textbf{rank-1 decomposition} because $A$ is written as a sum of rank-1 matrices ($u_i v_i^T$ is a rank-1 matrix.)

\subsubsection{Low Rank Approximation}
We know that \[A = \sum_{i = 1}^{r} \sigma_i u_i v_i^T = \sigma_1 u_1 v_1^T + \sigma_2 u_2 v_2^T + \hdots + \sigma_r u_r v_r^T.\] We also know that $\sigma_1 \geq \sigma_2 \geq \hdots \geq \sigma_r$. So, we can choose some $k \leq r$ and define \[A_k = \sum_{i = 1}^{k} \sigma_i u_i v_i^T.\] Then, $A_k$ is called the rank-$k$ approximation of $A$ (it's of type ``low-rank'' when $k < r$). 

\bigskip 

Essentially, we cut-off parts of the sum belonging to small singular values, producing an approximation ($A_k$) to the original matrix ($A$). It should, then, be noted that $\text{rank}(A_k) = k$, with each $u_i v_i^T$ having rank 1. 

\begin{theorem}{}{}
    \[||A - A_k||_2 = \sigma_{k + 1}.\]
\end{theorem}

\begin{proof}
    We know that $A = \hat{U}\hat{\Sigma}\hat{V}^T$ and $A_k = \hat{U}\hat{\Sigma}_k \hat{V}^T$. So, \[\hat{\Sigma} = \begin{bmatrix}
        \sigma_1 & & &  \\ 
        & \sigma_2 & & \\ 
        & & \ddots & \\ 
        & & & \sigma_r
    \end{bmatrix} \qquad \hat{\Sigma}_k = \begin{bmatrix}
        \sigma_1 & & &  \\ 
        & \ddots & & \\ 
        & & \sigma_k & \\ 
        & & & 0 
    \end{bmatrix}.\]
    From this, \[\begin{aligned}
        A - A_k &= \hat{U}(\hat{\Sigma} - \hat{\Sigma}_k) \hat{V}^T \\ 
            &= \hat{U}\begin{bmatrix}
                0 & & & & & & \\ 
                & 0 & & & & & \\ 
                & & \ddots & & & & \\ 
                & & & 0 & & & \\ 
                & & & & \sigma_{k + 1} & & \\ 
                & & & & & \ddots & \\ 
                & & & & & & \sigma_r
            \end{bmatrix} \hat{V}^T
    \end{aligned},\]
    as desired.
\end{proof}
In addition, $A_k$ is the matrix of rank $k$ that is closest to $A$ (in the 2-norm). In other words, $\min||A - B||_2$ with minimum over all matrix $B$ of rank $k$ is $A_k$. 


\end{document}