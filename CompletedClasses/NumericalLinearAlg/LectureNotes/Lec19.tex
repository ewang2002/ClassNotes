\documentclass[letterpaper]{article}
\usepackage[margin=1in]{geometry}
\usepackage[utf8]{inputenc}
\usepackage{textcomp}
\usepackage{amssymb}
\usepackage{natbib}
\usepackage{graphicx}
\usepackage{gensymb}
\usepackage{amsthm, amsmath, mathtools}
\usepackage[dvipsnames]{xcolor}
\usepackage{enumerate}
\usepackage{mdframed}
\usepackage[most]{tcolorbox}
\usepackage{csquotes}
% https://tex.stackexchange.com/questions/13506/how-to-continue-the-framed-text-box-on-multiple-pages

\tcbuselibrary{theorems}

\newcommand{\R}{\mathbb{R}}
\newcommand{\Z}{\mathbb{Z}}
\newcommand{\N}{\mathbb{N}}
\newcommand{\Q}{\mathbb{Q}}
\newcommand{\C}{\mathbb{C}}
\newcommand{\code}[1]{\texttt{#1}}
\newcommand{\mdiamond}{$\diamondsuit$}
\newcommand{\PowerSet}{\mathcal{P}}
\newcommand{\Mod}[1]{\ (\mathrm{mod}\ #1)}
\DeclareMathOperator{\lcm}{lcm}

%\newtheorem*{theorem}{Theorem}
%\newtheorem*{definition}{Definition}
%\newtheorem*{corollary}{Corollary}
%\newtheorem*{lemma}{Lemma}
\newtheorem*{proposition}{Proposition}


\newtcbtheorem[number within=section]{theorem}{Theorem}
{colback=green!5,colframe=green!35!black,fonttitle=\bfseries}{th}

\newtcbtheorem[number within=section]{definition}{Definition}
{colback=blue!5,colframe=blue!35!black,fonttitle=\bfseries}{def}

\newtcbtheorem[number within=section]{corollary}{Corollary}
{colback=yellow!5,colframe=yellow!35!black,fonttitle=\bfseries}{cor}

\newtcbtheorem[number within=section]{lemma}{Lemma}
{colback=red!5,colframe=red!35!black,fonttitle=\bfseries}{lem}

\newtcbtheorem[number within=section]{example}{Example}
{colback=white!5,colframe=white!35!black,fonttitle=\bfseries}{def}

\newtcbtheorem[number within=section]{note}{Important Note}{
        enhanced,
        sharp corners,
        attach boxed title to top left={
            xshift=-1mm,
            yshift=-5mm,
            yshifttext=-1mm
        },
        top=1.5em,
        colback=white,
        colframe=black,
        fonttitle=\bfseries,
        boxed title style={
            sharp corners,
            size=small,
            colback=red!75!black,
            colframe=red!75!black,
        } 
    }{impnote}
\usepackage[utf8]{inputenc}
\usepackage[english]{babel}
\usepackage{fancyhdr}
\usepackage[hidelinks]{hyperref}
\usepackage{csquotes}

\pagestyle{fancy}
\fancyhf{}
\rhead{Math 170A}
\chead{Wednesday, March 01, 2023}
\lhead{Lecture 20}
\rfoot{\thepage}

\setlength{\parindent}{0pt}
\newcommand{\0}{\mathbf{0}}
\newcommand{\y}{\mathbf{y}}
\renewcommand{\b}{\mathbf{b}}
\newcommand{\x}{\mathbf{x}}
\newcommand{\e}{\mathbf{e}}
\newcommand{\rr}{\mathbf{r}}
\newcommand{\vv}{\mathbf{v}}
\renewcommand{\u}{\mathbf{u}}

\begin{document}

\section{The Power Method (5.3)}
Let $A \in \C^{n \times n}$, and assume that $A$ is semisimple. Let $\lambda_1, \lambda_2, \hdots, \lambda_n$ denote the eigenvalues associated with the linearly independent eigenvectors, $v_1, \hdots, v_n$, respectively. Assume that the vectors are ordered so that $|\lambda_1| \geq |\lambda_2| \geq \hdots \geq |\lambda_n|$. If $|\lambda_1| > |\lambda_2|$, then $\lambda_1$ is called the \textbf{dominant eigenvalue}\footnote{Basically, the largest absolute eigenvalue.} and $v_1$ is called the \textbf{dominant eigenvector} of $A$.  

\subsection{The Iterative Power Method}
Assuming we have $|\lambda_1| > |\lambda_2|$ as described above (otherwise, this method may not work), the general idea behind the iterative power method is that we can pick $q \in \R^n$ randomly. Then, we can form the sequence of vectors
\[q, Aq, A^2 q, A^3 q, \hdots.\]
To calculate this sequence, we don't necessarily need to form the powers of $A$ explicitly. Each vector in the sequence can be obtained by multiplying the previous vector by $A$, e.g., $A^{j + 1}q = A(A^j q)$. It's easy to show that the sequence converge, in a sense, to a dominant eigenvector, for almost all choices of $q$. Since $v_1, \hdots, v_n$ form a basis for $\C^n$, there exists constants $c_1, \hdots, c_n$ such that  
\[q = c_1 v_1 + c_2 v_2 + \hdots + c_n v_n.\]
We don't know what $v_1, \hdots, v_n$ are, so we don't know what $c_1, \hdots, c_n$ are, either. However, it's clear that, for any choice of $q$, $c_1$ will be nonzero. The argument that follows is valid for every $q$ for which $c_1 \neq 0$; multiplying by $A$, we have 
\begin{equation*}
    \begin{aligned}
        Aq &= c_1 Av_1 + c_2 Ac_2 + \hdots + c_n Av_n \\ 
            &= c_1 \lambda_1 v_1 + c_2 \lambda_2 v_2 + \hdots + c_n  \lambda_n v_n.
    \end{aligned}
\end{equation*}
Similarly, 
\begin{equation*}
    \begin{aligned}
        A^2 q &= A(c_1 \lambda_1 v_1 + c_2 \lambda_2 v_2 + \hdots + c_n  \lambda_n v_n) \\
            &= c_1 \lambda_1 (Av_1) + c_2 \lambda_2 (Av_2) + \hdots + c_n \lambda_n (Av_n) \\ 
            &= c_1 \lambda_1 (\lambda_1 v_1) + c_2 \lambda_2 (\lambda_2 v_2) + \hdots + c_n \lambda_n (\lambda_n v_n) \\ 
            &= c_1 \lambda_1^2 v_1 + c_2 \lambda_2^2 v_2 + \hdots + c_n \lambda_n^2 v_n.
    \end{aligned}
\end{equation*}
In general, we have 
\begin{equation*}
    \begin{aligned}
        A^j q &= c_1 \lambda_1^j v_1 + c_2 \lambda_2^j v_2 + \hdots + c_n \lambda_n^j v_n \\ 
            &= \lambda_1^j \left(c_1 v_1 + c_2 \left(\frac{\lambda_2}{\lambda_1}\right)^j v_2 + \hdots + c_n \left(\frac{\lambda_n}{\lambda_1}\right)^j v_n\right).
    \end{aligned}
\end{equation*}
So, 
\[\frac{1}{\lambda_1^j} A^i q = c_1 v_1 + c_2 \left(\frac{\lambda_2}{\lambda_1}\right)^j v_2 + \hdots + c_n \left(\frac{\lambda_n}{\lambda_1}\right)^j v_n.\]
Notice that $\lim_{j \mapsto \infty} \left(\frac{\lambda_i}{\lambda_1}\right)^j = 0$ (because $|\lambda_1| > |\lambda_2| \geq |\lambda_3| \geq \hdots \geq |\lambda_n|$), so \[\lim_{j \mapsto \infty} \frac{1}{\lambda_1^j} A^j q = c_1 v_1,\]
the dominant eigenvector.

\bigskip 

\textbf{Remark:} In fact, when we use the power method to converge to a dominant eigenvector, we need to know all the eigenvalues and then check whether they're strictly greater or not. So, this only works if $\lambda_1$ is known and it is strictly greater than $\lambda_2$ and so on. 

\subsection{Scaling}
Notice how we went from $A_j q$ to $\frac{1}{\lambda_1^j} A^j q$? In some sense, we can say we're scaling $A_j q$. Now, what if we don't know what the value of $\lambda_1$ is, but we need to do \emph{some} scaling to get a reasonable convergence? In this case, we can still do some scaling, but not necessarily with $\lambda_1$. 

\bigskip 

Let's start with a random $q \in \R^n$; let $q_0 = q$. We want to use the iterative power formula method, 
\[\boxed{q_{j + 1} = \frac{1}{s_{j + 1}} Aq_j \qquad j = 0, 1, 2, \hdots},\]
where $\frac{1}{s_{j + 1}}$ is a scalar. Here, 
\[s_{j + 1} = ||Aq_j||_\infty.\]
In particular, all entries of $q_{j + 1}$ have absolute value $\leq 1$. With this scaling, as $j \mapsto \infty$, 
\[q_j \mapsto \text{Dominant eigenvector.}\]
\[s_j \mapsto \text{Dominant eigenvalue (in absolute value sense).}\]
Notice how $s_j$ will eventually converge to the absolute value of the dominant eigenvalue. What if we want the actual value of $\lambda_1$? There is another version of the scalar. which is just 
\[s_{j + 1} = \sgn((Aq_{j})_i) \cdot ||Aq_j||_\infty,\]
where $i$ is the index of the first entry of the vector $Aq_j$ such that $|(Aq_j)_{i}| = ||Aq_j||_\infty$, i.e., the absolute value of the entry at index $i$ is equal to the infinity norm of $Aq_j$. Then, $\sgn((Aq_{j})_i)$ is the sign function, which returns either $-1$ or 1 based on the sign of $(Aq_{j})_i$.  

\begin{mdframed}
    (Example.) Suppose \[Aq_j = \begin{bmatrix}
        -1 \\ 0 \\ 1/2 \\ 1 \\ 0
    \end{bmatrix}.\]
    We know that \[||Aq_j||_\infty = 1.\] So, \[s_{j + 1} = ||Aq_j||_\infty = 1.\]
    To find the sign, we note that there are two values in $Aq_j$ such that its absolute value equals $||Aq_j||_\infty = 1$; 
    \begin{itemize}
        \item Value $-1$ at index $i = 1$ (top value),
        \item Value $1$ at index $i = 4$ (second-to-bottom value).
    \end{itemize}
    We want the \emph{first} entry, so $i = 1$. Therefore, $\sgn((Aq_j)_1) = \sgn(-1) = -1$ and so 
    \[s_{j + 1} = \sgn((Aq_j)_i) \cdot ||Aq_j||_\infty = -1 \cdot 1 = -1.\]
\end{mdframed}

\subsubsection{Stopping Criterion}
Because this method is an iterative method, we need to stop at some point. We can set a threshold at $\epsilon > 0$. Stop the iteration when $||q_{j + 1} - q_j||_\infty < \epsilon$, basically $q_{j + 1} \approx q_j$. So, 
\[\frac{1}{s_{j + 1}}Aq_j \approx q_j \implies Aq_j \approx s_{j + 1} q_j.\]
Notice how this formula looks very similar to $Av = \lambda v$; in that sense, we can say that $q_j$ is the approximated eigenvector and $s_{j + 1}$ is the approximated eigenvalue.

\subsubsection{Flop Count and Rate of Convergence}
Recall again $|\lambda_1| > |\lambda_2| \geq \hdots \geq |\lambda_n|$. Let's look at the flop count; notice that we have $\BigO(n^2)$ in every step of the iteration (matrix-vector multiplication). If we do $N$ iterations, then the overall flop count is $\BigO(Nn^2)$. 

\bigskip 

Additionally, the convergence of the power method can be \emph{slow}. In particular, 
\begin{itemize}
    \item If $|\lambda_2 / \lambda_1|$ is small (e.g., $|1 / 1000|$), this means that $|\lambda_1| \gg |\lambda_2$ and convergence is fast. 
    \item If $|\lambda_2 / \lambda_1| \approx 1$ (e.g., $|0.99 / 1|$), then $|\lambda_1| \approx |\lambda_2|$ and convergence is slow.
\end{itemize}

\begin{mdframed}
    (Example.) Suppose \[A = \begin{bmatrix}
        3 & 1 \\ 1 & 1
    \end{bmatrix}.\] Suppose you need to apply 3 steps of the power method to approximate $\lambda_1$ and $v_1$. 

    \begin{itemize}
        %  We know that $||q_0||_\infty = 1$.
        \item $j = 0$: Let's start with\footnote{Remember that the initial vector is randomly chosen.} $q_0 = \begin{bmatrix}
            1 \\ 1
        \end{bmatrix}$. Then, 
        \[Aq_0 = \begin{bmatrix}
            3 & 1 \\ 1 & 1
        \end{bmatrix} \begin{bmatrix}
            1 \\ 1
        \end{bmatrix} = \begin{bmatrix}
            4 \\ 2
        \end{bmatrix}.\]
        We're now interested in the sign of $s_{1}$. To find the sign, we need to find the index $i$ of the first entry of the vector $Aq_0$ such that $|(Aq_0)_i| = ||Aq_0||_\infty$. We know that $||Aq_0||_\infty = \max\{4, 2\} = 4$, so we find that $i = 1$ and so $\sgn((Aq_0)_1) = \sgn(4) = 1$. 

        \bigskip 

        Thus, $s_{1} = 1 \cdot 4 = 4$ and so \[q_1 = \frac{1}{s_1}Aq_0 = \frac{1}{4}Aq_0 = \begin{bmatrix}
            1 \\ 1/2
        \end{bmatrix}.\]

        \item $j = 1$: With $q_1 = \begin{bmatrix}
            1 \\ 1/2
        \end{bmatrix}$, we have \[Aq_1 = \begin{bmatrix}
            3 & 1 \\ 1 & 1
        \end{bmatrix} \begin{bmatrix}
            1 \\ 1 / 2
        \end{bmatrix} = \begin{bmatrix}
            7 / 2 \\ 3 / 2
        \end{bmatrix}.\]
        We know that $||Aq_1||_\infty = \frac{7}{2}$ and so we find the index at $i = 1$ and $\sgn((Aq_1)_1) = \sgn(7 / 2) = 1$. Thus, $s_2 = 1 \cdot \frac{7}{2} = \frac{7}{2}$ and so 
        \[q_2 = \frac{1}{s_2}Aq_1 = \frac{2}{7}\begin{bmatrix}
            7/2 \\ 3/2
        \end{bmatrix} = \begin{bmatrix}
            1 \\ 3 / 7
        \end{bmatrix}.\]

        \item $j = 3$: With $q_2 = \begin{bmatrix}
            1 \\ 3/7
        \end{bmatrix}$, we have 
        \[Aq_2 = \begin{bmatrix}
            3 & 1 \\ 1 & 1
        \end{bmatrix} \begin{bmatrix}
            1 \\ 3/7
        \end{bmatrix} = \begin{bmatrix}
            24 / 7 \\ 10 / 7
        \end{bmatrix}.\] We find that $||Aq_2||_\infty = 24 / 7$ and so, again, $i = 1$ and $\sgn((Aq_2)_1) = \sgn(24 / 7) = 1$. Thus, $s_3 = 1 \cdot \frac{24}{7} = \frac{24}{7}$ and 
        \[q_3 = \frac{1}{s_3}Aq_2 = \frac{7}{24}\begin{bmatrix}
            24 / 7 \\ 10 / 7
        \end{bmatrix} = \begin{bmatrix}
            1 \\ 5 / 12
        \end{bmatrix}.\]
    \end{itemize}
    With this, we find that $s_3 = 24 / 7 \approx 3.429$ and $q_3 = \begin{bmatrix}
        1 \\ 5/12
    \end{bmatrix} \approx \begin{bmatrix}
        1 \\ 0.4167
    \end{bmatrix}$. In actuality, the eigenvalue is $\lambda_1 = 3.4142$ and the eigenvector is $v_1 = \begin{bmatrix}
        1 \\ 0.4142
    \end{bmatrix}$, so after \emph{three} steps, the approximation is very close.
\end{mdframed}
\textbf{Remarks:}
\begin{itemize}
    \item The power method is easy to implement. 
    \item The power method does \emph{not} converge if $|\lambda_1| = |\lambda_2|$. 
\end{itemize}


\end{document}