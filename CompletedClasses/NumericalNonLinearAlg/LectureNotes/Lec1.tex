\documentclass[letterpaper]{article}
\usepackage[margin=1in]{geometry}
\usepackage[utf8]{inputenc}
\usepackage{textcomp}
\usepackage{amssymb}
\usepackage{natbib}
\usepackage{graphicx}
\usepackage{gensymb}
\usepackage{amsthm, amsmath, mathtools}
\usepackage[dvipsnames]{xcolor}
\usepackage{enumerate}
\usepackage{mdframed}
\usepackage[most]{tcolorbox}
\usepackage{csquotes}
% https://tex.stackexchange.com/questions/13506/how-to-continue-the-framed-text-box-on-multiple-pages

\tcbuselibrary{theorems}

\newcommand{\R}{\mathbb{R}}
\newcommand{\Z}{\mathbb{Z}}
\newcommand{\N}{\mathbb{N}}
\newcommand{\Q}{\mathbb{Q}}
\newcommand{\C}{\mathbb{C}}
\newcommand{\code}[1]{\texttt{#1}}
\newcommand{\mdiamond}{$\diamondsuit$}
\newcommand{\PowerSet}{\mathcal{P}}
\newcommand{\Mod}[1]{\ (\mathrm{mod}\ #1)}
\DeclareMathOperator{\lcm}{lcm}

%\newtheorem*{theorem}{Theorem}
%\newtheorem*{definition}{Definition}
%\newtheorem*{corollary}{Corollary}
%\newtheorem*{lemma}{Lemma}
\newtheorem*{proposition}{Proposition}


\newtcbtheorem[number within=section]{theorem}{Theorem}
{colback=green!5,colframe=green!35!black,fonttitle=\bfseries}{th}

\newtcbtheorem[number within=section]{definition}{Definition}
{colback=blue!5,colframe=blue!35!black,fonttitle=\bfseries}{def}

\newtcbtheorem[number within=section]{corollary}{Corollary}
{colback=yellow!5,colframe=yellow!35!black,fonttitle=\bfseries}{cor}

\newtcbtheorem[number within=section]{lemma}{Lemma}
{colback=red!5,colframe=red!35!black,fonttitle=\bfseries}{lem}

\newtcbtheorem[number within=section]{example}{Example}
{colback=white!5,colframe=white!35!black,fonttitle=\bfseries}{def}

\newtcbtheorem[number within=section]{note}{Important Note}{
        enhanced,
        sharp corners,
        attach boxed title to top left={
            xshift=-1mm,
            yshift=-5mm,
            yshifttext=-1mm
        },
        top=1.5em,
        colback=white,
        colframe=black,
        fonttitle=\bfseries,
        boxed title style={
            sharp corners,
            size=small,
            colback=red!75!black,
            colframe=red!75!black,
        } 
    }{impnote}
\usepackage[utf8]{inputenc}
\usepackage[english]{babel}
\usepackage{fancyhdr}
\usepackage[hidelinks]{hyperref}

\pagestyle{fancy}
\fancyhf{}
\rhead{Math 170B}
\chead{Monday, April 03, 2023}
\lhead{Lecture 1}
\rfoot{\thepage}

\setlength{\parindent}{0pt}

\begin{document}

\section{Basic Concepts \& Taylor's Theorem (Section 1.1)}
Let $f(x): \R \mapsto \R$ be a general function (typically nonlinear). We may also write $f([a, b]): [a, b] \mapsto \R$ to denote a general function over an interval $[a, b]$. We also write $C^n (\R)$ or $C^n ([a, b])$ to denote the \emph{classes} of $n$-times continuously differentiable functions. We write $C^0 (\R) = C(\R)$ to mean the class of only continuous functions. 

\begin{mdframed}
    (Example.) $f(x) = |x|$ is continuous but is not differentiable at $x = 0$. Thus, $f(x) = |x|$ is in $C^0 (\R)$. 

    \bigskip 

    $f(x) = e^x$ is in $C^\infty (\R)$.
\end{mdframed}

\begin{mdframed}
    (Exercise.) Show that \[f(x) = \begin{cases}
        x^2 \sin\left(\frac{1}{x}\right) & x \neq 0 \\ 
        0 & x = 0
    \end{cases}\] is once differentiable (but not twice).

    \begin{mdframed}
        First, we need to check that $f$ is continuous. First, we know that \[\lim_{x \mapsto \infty} x^2 \sin\left(\frac{1}{x}\right) = 0,\] so $f(x)$ is continuous. Next, 
        \begin{enumerate}[(a)]
            \item When $x \neq 0$, then $f(x) = x^2 \sin\left(\frac{1}{x}\right)$ and \[f'(x) = 2x\sin\left(\frac{1}{x}\right) - \cos\left(\frac{1}{x}\right).\] Thus, $f(x)$ is differentiable when $x \neq 0$. 
            \item When $x = 0$, we note that \[f'(0) = \lim_{x \mapsto 0} \frac{f(x) - f(0)}{x - 0} = \lim_{x \mapsto 0} \frac{x^2 \sin\left(\frac{1}{x}\right)}{x} = \lim_{x \mapsto 0} x\sin\left(\frac{1}{x}\right) = 0.\] Thus, $f(x)$ is differentiable when $x = 0$.
        \end{enumerate}
        Therefore, $f(x)$ is differentiable. With this in mind, we know that 
        \[f'(x) = \begin{cases}
            2x\sin\left(\frac{1}{x}\right) - \cos\left(\frac{1}{x}\right) & x \neq 0 \\ 
            0 & x = 0
        \end{cases}.\]
        But, $\lim_{x \mapsto 0} f'(x)$ doesn't exist since $\lim_{x \mapsto 0^+} f'(x) \neq \lim_{x \mapsto 0^-} f'(x)$. Because differentiability implies continuity, $f'(x)$ is not differentiable. 
    \end{mdframed}
\end{mdframed}

\begin{mdframed}
    (Exercise.) With $[a, b] = [1, 3]$ and $f(x) = 3 - 2x + x^2$, find $\xi$ in the Mean-Value Theorem: $f(b) - f(a) = f'(\xi) (b - a)$. 

    \begin{mdframed}
        Recall that $\xi \in [a, b]$. We know from the problem that $f(b) - f(a) = f'(\xi) (b - a)$, and so rearranging the terms gives us 
        \[f'(\xi) = \frac{f(b) - f(a)}{b - a} = \frac{f(3) - f(1)}{3 - 1} = \frac{6 - 2}{3 - 1} = \frac{4}{2} = 2.\] We also know that \[f'(x) = 2x - 2,\] so we need to find $f'(\xi) = 2$. This gives us \[2\xi - 2 = 2 \implies \xi = 2.\]
    \end{mdframed}
\end{mdframed}

\subsection{Taylor Series}
\begin{theorem}{Taylor Series with Lagrange Remainder}{}
    If $f \in C^m ([a, b])$, and if the derivative $f^{(m + 1)}$ exists on the open interval $(a, b)$, then for any points $x, c \in [a, b]$, 
    \[f(x) = \sum_{k = 0}^{m} \frac{f^{(k)} (c)}{k!} (x - c)^k + E_{m}(x),\] 
    where $E_{m}(x)$, the remainder (or error) term, is  
    \[E_{m}(x) = \frac{f^{(m + 1)}(\xi)}{(m + 1)!} (x - c)^{m + 1},\] 
    where $c < \xi < x$ or $x < \xi < c$ depending on the values of $x$ and $c$. 
\end{theorem}
\textbf{Remark:} Note that $f^{(k)}(x)$ is the $k$th derivative of $f(x)$. So, given $f(x)$, you will need to find $f'(x)$, $f''(x)$, and so on, and then generalize these derivatives. See the below examples for more information.

\begin{mdframed}
    (Example.) Suppose $f(x) = \ln(x)$ with interval $[a, b] = [1, 10]$ and $c = e^1.$ Let $|x - c| < 1$ (i.e., $x$ is relatively close to $c$). Then, 
    \[f^{(1)}(x) = f'(x) = \frac{1}{x}.\]
    \[f^{(2)}(x) = f''(x) = -\frac{1}{x^2}.\]
    \[f^{(3)}(x) = f'''(x) = \frac{2}{x^3}.\]
    \[f^{(4)}(x) = -2 \cdot 3 \frac{1}{x^4}.\]
    \[f^{(5)}(x) = 2 \cdot 3 \cdot 4 \frac{1}{x^5}.\]
    \[\vdots\]
    \[f^{(k)}(x) = (-1)^{k - 1} (k - 1)! \frac{1}{x^k}\]
    for $k = 1, 2, \hdots$. Then, 
    \[E_m (x) = \frac{1}{(m + 1)!} f^{(m + 1)}(\xi) (x - c)^{m + 1}.\] Using the value of $c = e^1$, \[f^{(k)}(c) = (-1)^{k - 1}(k - 1)! \frac{1}{e^k}.\] Combining everything, we end up with 
    \begin{equation*}
        \begin{aligned}
            f(x) &= \sum_{k = 0}^{m} \frac{f^{(k)} (c)}{k!} (x - c)^k + E_{m}(x) && \text{General Taylor Series} \\ 
                &= \frac{f^{(0)}(c)}{0!} (x - c)^0 + \sum_{k = 1}^{m} \frac{f^{(k)}(c)}{k!} (x - c)^k + E_{m}(x) && \text{Separate the first term in summation} \\ 
                &= f(c) + \sum_{k = 1}^{m} f^{(k)}(c) \frac{1}{k!} (x - c)^k + E_{m}(x) \\ 
                &= f(c) + \sum_{k = 1}^{m} \left((-1)^{k - 1}(k - 1)! \frac{1}{e^k}\right) \frac{1}{k!} (x - c)^k + E_{m}(x) \\ 
                &= f(c) + \sum_{k = 1}^{m} (-1)^{k - 1} \frac{1}{e^k} \frac{1}{k} (x - c)^k + E_{m}(x) \\ 
                &= f(e) + \sum_{k = 1}^{m} (-1)^{k - 1} \frac{1}{e^k} \frac{1}{k} (x - e)^k + E_{m}(x) && c = e \\ 
                &= 1 + \sum_{k = 1}^{m} (-1)^{k - 1} \frac{(x - e)^k }{k e^k} + E_{m}(x) 
        \end{aligned}
    \end{equation*}
    How many terms in this approximation do we need in order for the error to be below a certain amount? In other words, what is the minimum $m$ so that a Taylor expansion is accurate up to $\frac{1}{\alpha} \cdot 10^{-7}$? We have 
    \[|E_m (m)| \leq \frac{1}{\alpha} \cdot 10^{-7}.\]
    We already computed the remainder, so 
    \[\left| \frac{1}{(m + 1)} f^{(m + 1)} (\xi) (x - e)^{m + 1}\right| \leq \frac{1}{\alpha} \cdot 10^{-7}.\] Using $|x - e| < 1$, we want to find $m$. In any case, $|\xi| < 1$. 
\end{mdframed}

\begin{mdframed}
    (Exercise.) Consider the function $f(x) = \ln(x)$. 
    \begin{enumerate}[(a)]
        \item Determine the Taylor series of $f(x)$ using Taylor's Theorem, with the interval $[a, b] = [1, 2]$ and $c = 1$.
        \begin{mdframed}
            From the previous example, we know that 
            \[f^{(k)}(x) = (-1)^{k - 1} (k - 1)! \frac{1}{x^k}.\]
            Therefore, 
            \[f^{(k)}(c)  = f^{(k)}(1) = (-1)^{k - 1} (k - 1)!.\]
            So, using Taylor's Theorem, we have 
            \begin{equation*}
                \begin{aligned}
                    f(x) &= \sum_{k = 0}^{m} \frac{f^{(k)} (c)}{k!} (x - c)^k + E_{m}(x) \\ 
                        &= \sum_{k = 1}^{m} \frac{f^{(k)} (c)}{k!} (x - c)^k + E_{m}(x) \\ 
                        &= \sum_{k = 1}^{m} \frac{(-1)^{k - 1} (k - 1)!}{k!} (x - c)^k + E_{m}(x) \\ 
                        &= \sum_{k = 1}^{m} \frac{(-1)^{k - 1}}{k} (x - 1)^k + E_{m}(x).
                \end{aligned}
            \end{equation*}
            Note that the summation started at $k = 1$ because $f^{(0)}(1) = f(1) = \ln(1) = 0$. Thus, 
            \begin{equation*}
                \begin{aligned}
                    E_{m}(x) &= \frac{1}{(m + 1)!} f^{(m + 1)}(\xi) (x - 1)^{m + 1} \\ 
                        &= \frac{1}{(m + 1)!} (-1)^{m + 1 - 1} (m + 1 - 1)! \frac{1}{\xi^{m + 1}} (x - 1)^{m + 1} \\
                        &= (-1)^{m} \frac{1}{m + 1} \frac{1}{\xi^{m + 1}} (x - 1)^{m + 1} \\  
                \end{aligned}
            \end{equation*}
            Remember that $E_{m}(x)$ tells us how the polynomial approximation differs from $\ln(x)$. Note that this term is not a polynomial because $\xi$ depents on $x$ in a nonpolynomial way. In any case, writing out the polynomial formula (found in part (a)) for $\ln(x)$ gives us 
            \[\ln(x) = (x - 1) - \frac{1}{2}(x - 1)^2 + \frac{1}{3}(x - 1)^3 - \hdots + (-1)^{n - 1} \frac{1}{m}(x - 1)^m + E_{m}(x).\]
            The error, then, is given by 
            \[|E_{m}(x)| = \frac{1}{m + 1} \frac{1}{\xi^{m + 1}} (x - 1)^{m + 1} < \frac{1}{m + 1}(x - 1)^{m + 1}.\]
        \end{mdframed}
        
        \item How many terms in the series need to be used to compute $\ln(2)$ with accuracy of one part in $10^8$? 
        \begin{mdframed}
            With $x = 2$, we know that 
            \[f(2) = \ln(2) = \sum_{k = 1}^{m} \frac{(-1)^{k - 1}}{k} + E_{m}(2) = \sum_{k = 1}^{m} \frac{(-1)^{k - 1}}{k} + E_{m}(2),\]
            and we know that $|E_{m}(2)| < \frac{1}{m + 1}(2 - 1)^{m + 1} = \frac{1}{m + 1}$. Since $E_{m}(2)$ is the numerical error, to compute $\ln(2)$ with the desired accuracy, we need to find $m$ such that $E_{m}(2) \leq 10^{-8}$. This means that we can solve the inequality for $m$: 
            \[|E_{m}(2)| < \frac{1}{m + 1} \leq 10^{-8}.\] 
        \end{mdframed}
        
    \end{enumerate}
\end{mdframed}



\end{document}