\documentclass[letterpaper]{article}
\usepackage[margin=1in]{geometry}
\usepackage[utf8]{inputenc}
\usepackage{textcomp}
\usepackage{amssymb}
\usepackage{natbib}
\usepackage{graphicx}
\usepackage{gensymb}
\usepackage{amsthm, amsmath, mathtools}
\usepackage{xcolor}
\usepackage{enumerate}
\usepackage{framed}
\usepackage{tcolorbox}
\tcbuselibrary{theorems}

\newcommand{\R}{\mathbb{R}}
\newcommand{\Z}{\mathbb{Z}}
\newcommand{\N}{\mathbb{N}}
\newcommand{\Q}{\mathbb{Q}}
\newcommand{\code}[1]{\texttt{#1}}
\newcommand{\mdiamond}{$\diamondsuit$}

%\newtheorem*{theorem}{Theorem}
%\newtheorem*{definition}{Definition}
\newtheorem*{proposition}{Proposition}
%\newtheorem*{corollary}{Corollary}
%\newtheorem*{lemma}{Lemma}

\newtcbtheorem[number within=section]{theorem}{Theorem}
{colback=green!5,colframe=green!35!black,fonttitle=\bfseries}{def}

\newtcbtheorem[number within=section]{definition}{Definition}
{colback=blue!5,colframe=blue!35!black,fonttitle=\bfseries}{def}

\newtcbtheorem[number within=section]{corollary}{Corollary}
{colback=yellow!5,colframe=yellow!35!black,fonttitle=\bfseries}{def}

\newtcbtheorem[number within=section]{lemma}{Lemma}
{colback=red!5,colframe=red!35!black,fonttitle=\bfseries}{def}
\usepackage[utf8]{inputenc}
\usepackage[english]{babel}
\usepackage{fancyhdr}
\usepackage[hidelinks]{hyperref}

\pagestyle{fancy}
\fancyhf{}
\rhead{Math 170B}
\chead{Monday, May 15, 2023}
\lhead{Lecture 19}
\rfoot{\thepage}

\setlength{\parindent}{0pt}

\begin{document}

\section{Trigonometric Interpolation (Section 6.12)}
Recall, for one-dimensional interpolation, there is a unique $P(x_j) = f(x_j)$ for $0 \leq j \leq m - 1$. However, when $f$ is periodic, trigonometric functions for interpolation are appropriate.

\subsection{Connection to Complex Variables}
If $f$ has a period of $2\pi$ and a continuous first derivative, then the \textbf{Fourier series/analysis} is defined by\footnote{The tilde means that this converges for functions that satisfies several properties.}
\[ f \sim \frac{a_0}{2} + \sum_{k = 1}^{\infty} (a_{k} \cos(kx) + b_k \sin(kx)), \]
where \[a_k = \frac{1}{\pi} \int_{-\pi}^{\pi} f(t)\cos(kt) dt,\]
\[b_k = \frac{1}{\pi} \int_{-\pi}^{\pi} f(t) \sin(kt) dt.\]
A connection to complex variables is \textbf{Euler's formula}, which is 
\[e^{i\theta} = \cos(\theta) + i\sin(\theta),\] where $i = \sqrt{-1}$. The complex conjugate of Euler's formula is 
\[\overline{e^{i\theta}} = \cos(\theta) - i\sin(\theta).\]

\subsection{Special Exponential Functions}
We can also define the function, 
\[E_{k}(x) = e^{ikx}, \qquad k = 0, \pm 1, \pm 2, \pm 3, \hdots.\]
$E_{k}$ is special because it forms a ``basis'' of linearly independent functions.

\subsubsection{Some Notation}
Some notation to keep in mind: 
\begin{itemize}
    \item The inner product of two functions is defined by
    \[\cyclic{f, g} = \frac{1}{2\pi} \int_{-\pi}^{\pi} f(x) \overline{g(x)} dx.\]

    \item The ``discrete inner product'' is defined by
    \[\cyclic{f, g}_{N} = \frac{1}{N} \sum_{j = 0}^{N - 1} f\left(\frac{2\pi j}{N}\right) \overline{g\left(\frac{2\pi j}{N}\right)},\]
    with $N$ is the number of points being used.
\end{itemize}

Using the special exponential function, along with Euler's formula, we note that, for $k \neq m$, 
\[\begin{aligned}
    \cyclic{E_k, E_m} &= \frac{1}{2\pi} \int_{-\pi}^{\pi} e^{ikx} \overline{e^{imx}} dx \\ 
        &= \frac{1}{2\pi} \int_{-\pi}^{\pi} e^{ikx} e^{-imx} dx \\ 
        &= \frac{1}{2\pi} \int_{-\pi}^{\pi} e^{ix(k - m)} dx \\ 
        &= \frac{1}{2\pi(k - m)} e^{ix(k - m)} \rvert_{-\pi}^{\pi} \\ 
        &= \frac{e^{i\pi (k - m)} - e^{-i\pi(k - m)}}{2\pi i (k - m)} \\ 
        &= \hdots \\ 
        &= 0. 
\end{aligned}\]
However, if $k = m$, we have 
\[\begin{aligned}
    \cyclic{E_{k}, E_{m}} &= \cyclic{E_{k}, E_{k}} \\ 
        &= \frac{1}{2\pi} \int_{-\pi}^{\pi} 1 dx \\ 
        &= 1.
\end{aligned}\]
So, $E_{k}(x)$ is a basis for trigonometric functions. 

\subsection{Exponential Polynomial \& Interpolation}
We can define $P(x)$ to be 
\begin{equation}
    P(x) = \sum_{k = 0}^{m} c_{k} E_{k}(x) = \sum_{k = 0}^{m} c_{k} e^{ikx} = \sum_{k = 0}^{m} c_{k} \left(e^{ix}\right)^k.
\end{equation}
Suppose now we want to interpolate a polynomial at equally spaced points, $x_{j} = \frac{2\pi j}{N}.$ Then, we can use the discrete inner product, 
\[P(x) = \sum_{k = 0}^{N - 1} c_{k} E_{k}(x), \qquad c_{k} = \cyclic{f, E_{k}}_{N} = \frac{1}{N} \sum_{j = 0}^{N - 1} f\left(\frac{2\pi j}{N}\right) \overline{E_{k}\left(\frac{2\pi j}{N}\right)}.\]

\begin{mdframed}
    (Example.) Suppose $N = 2$. We want to find an explicit formula of $P$ given $f$. So, $x_{0} = 0$ and $x_{1} = \frac{2\pi}{2} = \pi$ and
    \[c_{0} = \frac{1}{2} \sum_{j = 0}^{2 - 1} f\left(\frac{2\pi j}{2}\right) e^{-i \cdot 0 \cdot j\pi} = \frac{1}{2} (f(0) + f(\pi)).\]
    \[c_{1} = \frac{1}{2} \sum_{j = 0}^{2 - 1} f\left(\frac{2\pi j}{2}\right) e^{-i \cdot 2\pi} = \frac{1}{2}\left(f(0) + f(\pi)e^{-i\pi}\right) = \frac{1}{2} (f(0) - f(\pi)),\]
    where $e^{-i\pi} = \cos(\pi) - i\sin(\pi) = -1 - i(0) = -1$. Then, combining this yields 
    \[P(x) = c_{0}E_{0}(x) + c_{1}E_{1}(x) = \frac{1}{2}(f(0) + f(\pi)) + \frac{1}{2}(f(0) - f(\pi)) e^{ix}.\]
\end{mdframed}

\subsection{Complex Fourier Series}
Let \[f(x) \sim \sum_{k = -\infty}^{\infty} \hat{f}(x) e^{ikx},\]
where \[\hat{f}(x) = \frac{1}{2\pi} \int_{-\pi}^{\pi} f(t) e^{-ikt} dt = \frac{1}{2\pi} \int_{-\infty}^{\infty} f(t) (\cos(kt) - i\sin(kt)) dt = \frac{1}{2}(a_k - i b_k).\]
If $f(t)$ is real, then it corresponds to the ``real'' Fourier series. Let $a_{k} = a_{-k}$ (i.e., even proeprty), $b_{k} = -b_{-k}$, $c_{k} = \frac{1}{2}(a_{k} - ib_{k})$, $b_{0} = 0$. Then, 
\[f \sim \sum_{k = -\infty}^{\infty} \hat{f}(k) e^{ikx} = \sum_{k = -\infty}^{\infty} \frac{1}{2} (a_{k} - ib_{k})(\cos(kt) + i\sin(kt)).\]
Here, the imaginary part is 0 (represented by the summation). In this sense, 
\[\sum_{k = -\infty}^{\infty} \frac{1}{2}(a_{k} \sin(kx) - b_{k}\cos(kx))\] is the imaginary part.

\end{document}