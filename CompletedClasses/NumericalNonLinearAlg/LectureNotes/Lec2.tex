\documentclass[letterpaper]{article}
\usepackage[margin=1in]{geometry}
\usepackage[utf8]{inputenc}
\usepackage{textcomp}
\usepackage{amssymb}
\usepackage{natbib}
\usepackage{graphicx}
\usepackage{gensymb}
\usepackage{amsthm, amsmath, mathtools}
\usepackage{xcolor}
\usepackage{enumerate}
\usepackage{framed}
\usepackage{tcolorbox}
\tcbuselibrary{theorems}

\newcommand{\R}{\mathbb{R}}
\newcommand{\Z}{\mathbb{Z}}
\newcommand{\N}{\mathbb{N}}
\newcommand{\Q}{\mathbb{Q}}
\newcommand{\code}[1]{\texttt{#1}}
\newcommand{\mdiamond}{$\diamondsuit$}

%\newtheorem*{theorem}{Theorem}
%\newtheorem*{definition}{Definition}
\newtheorem*{proposition}{Proposition}
%\newtheorem*{corollary}{Corollary}
%\newtheorem*{lemma}{Lemma}

\newtcbtheorem[number within=section]{theorem}{Theorem}
{colback=green!5,colframe=green!35!black,fonttitle=\bfseries}{def}

\newtcbtheorem[number within=section]{definition}{Definition}
{colback=blue!5,colframe=blue!35!black,fonttitle=\bfseries}{def}

\newtcbtheorem[number within=section]{corollary}{Corollary}
{colback=yellow!5,colframe=yellow!35!black,fonttitle=\bfseries}{def}

\newtcbtheorem[number within=section]{lemma}{Lemma}
{colback=red!5,colframe=red!35!black,fonttitle=\bfseries}{def}
\usepackage[utf8]{inputenc}
\usepackage[english]{babel}
\usepackage{fancyhdr}
\usepackage[hidelinks]{hyperref}

\pagestyle{fancy}
\fancyhf{}
\rhead{Math 170B}
\chead{Wednesday, April 05, 2023}
\lhead{Lecture 2}
\rfoot{\thepage}

\setlength{\parindent}{0pt}

\begin{document}
\section{Order (Section 1.2)}
Let's begin by looking at some different sequences. We'll be using \textbf{orders} to compare those sequences.
\begin{definition}{Big-$\BigO$}{} 
    Let $x_m$ and $\alpha_m$ be two sequences. We say that $x_m = \BigO(\alpha_m)$ if, for $m \mapsto \infty$, we have $\frac{x_m}{\alpha_m} \leq C$ for some constant $C$.
\end{definition}

%Let 
% \[x_m = \frac{1}{m}, \qquad \alpha_m = \frac{1}{\ln(m)},\]
% with $m \geq 2$. 

\begin{mdframed}
    (Example.) Suppose $x_m = \frac{1}{m} + \frac{1}{m^2}$ and $\alpha_m = \frac{1}{m}$. Then, \[\frac{x_m}{\alpha_m} = 1 + \frac{1}{m} \leq C\] for $m \mapsto \infty$ and $C \in \R$. Therefore, we say that \[x_m = \BigO\left(\frac{1}{m}\right) = \BigO(\alpha_m).\]
\end{mdframed}

\begin{definition}{Little-o}{}
    Let $x_m$ and $\alpha_m$ be two sequences, both tending to 0 as $m \mapsto \infty$. We say that $x_m = o(\alpha_m)$ if, for $m \mapsto \infty$, we have $\frac{x_m}{\alpha_m} \mapsto 0$. 
\end{definition}
\textbf{Remarks:} 
\begin{itemize}
    \item If something is little-$o$, then it will also be Big-$\BigO$. 
    \item If $x_n \mapsto 0$ and $\alpha_n \mapsto 0$ and $x_n = \BigO(\alpha_n)$, then $x_n$ converges to 0 at least as rapidly as $\alpha_n$ does. If $x_n = o(\alpha_n)$, then $x_n$ converges to 0 more rapidly than $\alpha_n$. 
\end{itemize}

\begin{mdframed}
    (Example.) Let $x_m = \frac{1}{m}$ and $\alpha_m = \frac{1}{\ln(m)}$. Then, as $m \mapsto \infty$, we have \[\frac{x_m}{\alpha_m} = \frac{\ln(m)}{m} \mapsto 0.\]
    Then, we can say \[x_m = o\left(\frac{1}{\ln(m)}\right) = o(\alpha_m).\]
\end{mdframed}

\begin{mdframed}
    (Example.) 
    \begin{itemize}
        \item $\frac{m + 1}{m^2} = \frac{1}{m} + \frac{1}{m^2} = \BigO\left(\frac{1}{m}\right).$
        \item $\frac{m + 1}{\sqrt{m}} = \BigO(\sqrt{m}).$
        \item $\frac{1}{m\ln(m)} = o\left(\frac{1}{m}\right).$
    \end{itemize}
\end{mdframed}

\subsection{Functions}
There are analogous definitions of Big-$\BigO$ and little-$o$ for functions. 

\begin{definition}{Big-$\BigO$ \& Little-o}{}
    For functions $f(x)$, $g(x)$, 
    \begin{itemize}
        \item we say that $f(x) = \BigO(g(x))$ if $\frac{f(x)}{g(x)} \leq C$ as $x \mapsto x^*$ and for some constant $C$. 
        \item we say that $f(x) = o(g(x))$ if $\frac{f(x)}{g(x)} \mapsto 0$ as $x \mapsto x^*$. 
    \end{itemize}
\end{definition}

\begin{mdframed}
    (Example.) We know that the Taylor Series \[\begin{aligned}
        \frac{1}{1 - x} &= 1 + x + x^2 + x^3 + \hdots \\
            &= 1 + x + \hdots + \BigO(x^2)
    \end{aligned}\] for $|x| < 1$. Likewise, \[\frac{1}{1 - x} = 1 + x + o(x).\] Here, 
    \begin{itemize}
        \item In the first equation, the $\BigO(x^2)$ means that the remaining terms have order $x^2$. 
        \item Likewise, $o(x^2)$ means that the remaining terms tend to 0 faster than $x^2$. 
    \end{itemize}
\end{mdframed}
% TODO check

\subsection{Polynomials}
We know that polynomials can approximate functions (like we've seen through Taylor Series). That is, the polynomial \[p(x) = a_0 + a_1 x + a_2 x^2 + a_3 x^3 + \hdots + a_m x^m\] can be used to approximate functions. Note that
\begin{equation*}
    \begin{aligned}
        p(x) &= a_0 + a_1 x + a_2 x^2 + a_3 x^3 + \hdots + a_m x^m \\ 
            &= a_0 + x(a_1 + a_2 x + a_3 x^2 + \hdots + a_m x^{m - 1}) \\ 
            &= a_0 + x(a_1 + x(a_2 + a_3 x + \hdots + a_m x^{m - 2}))
    \end{aligned}
\end{equation*}
Suppose we want to evaluate this polynomial at some $x$. Using this process, known as \textbf{Horner's Method}, we can compute a polynomial at some $x$. 
\begin{verbatim}
    p = a_m 
    for i = m - 1 : 0 
        p = px + a_i
    end\end{verbatim}
This algorithm uses one loop, and does not store anything else (i.e., does not use any intermediate values). In terms of flop count, this is $\BigO(m)$ flops. 

\subsection{Mean-Value Theorem}
\begin{theorem}{Mean-Value Theorem}{}
    For $\xi \in [a, b]$ and $f \in C^1$, \[f(b) - f(a) = f'(\xi)(b - a). \]
\end{theorem}

\subsection{Function Representation}
Functions can be explicit or implicit. Generally, we'll consider functions in the explicit form, \[y = f(x).\]
An implicit form may look like\footnote{If we have $G(x, y) = C$ for some constant $C$, then we can subtract $C$ on both sides to get $G(x, y) - C = 0$.} \[G(x, y) = 0.\]
\begin{mdframed}
    (Example.) 
    \begin{itemize}
        \item $y = x^2$ is a simple explicit function. 
        \item $y^2 + x^2 = 0$ is a simple implicit function. 
    \end{itemize}
\end{mdframed}
Essentially, an explicit function has one variable whereas an implicit function has two variables. 

\begin{theorem}{Implicit Function Theorem}{}
    Let $G$ be a function of two real variables defined and continuously differentiable in a neighborhood of $(x_0, y_0)$. If $G(x_0, y_0) = 0$ and $\frac{\delta G}{\delta y} \neq 0$ at $(x_0, y_0)$, then there is a continuously differentiable function $f$ defined such that $f(x_0) = y_0$ and $G(x, f(x)) = 0$. 
\end{theorem}

\begin{mdframed}
    (Example.) Suppose $x^2 + y^2 = 2$. Then, $G = x^2 + y^2 - 2$ and we also know that, for example, $G(1, 1) = 0$. Then, $x_0 = y_0 = 1$ and \[\frac{\partial G}{\partial y}(x_0, y_0) = 2y_0 = 2 \neq 0.\] Then, there is an implicit function $y$ around $(1, 1)$. We know that $G = 0$ and $y^2 = 2 - x^2 \implies y_1, y_2 = \pm\sqrt{2 - x^2}$, which means $x_0 = 1$.
\end{mdframed}

\end{document}