\documentclass[letterpaper]{article}
\usepackage[margin=1in]{geometry}
\usepackage[utf8]{inputenc}
\usepackage{textcomp}
\usepackage{amssymb}
\usepackage{natbib}
\usepackage{graphicx}
\usepackage{gensymb}
\usepackage{amsthm, amsmath, mathtools}
\usepackage[dvipsnames]{xcolor}
\usepackage{enumerate}
\usepackage{mdframed}
\usepackage[most]{tcolorbox}
\usepackage{csquotes}
% https://tex.stackexchange.com/questions/13506/how-to-continue-the-framed-text-box-on-multiple-pages

\tcbuselibrary{theorems}

\newcommand{\R}{\mathbb{R}}
\newcommand{\Z}{\mathbb{Z}}
\newcommand{\N}{\mathbb{N}}
\newcommand{\Q}{\mathbb{Q}}
\newcommand{\C}{\mathbb{C}}
\newcommand{\code}[1]{\texttt{#1}}
\newcommand{\mdiamond}{$\diamondsuit$}
\newcommand{\PowerSet}{\mathcal{P}}
\newcommand{\Mod}[1]{\ (\mathrm{mod}\ #1)}
\DeclareMathOperator{\lcm}{lcm}

%\newtheorem*{theorem}{Theorem}
%\newtheorem*{definition}{Definition}
%\newtheorem*{corollary}{Corollary}
%\newtheorem*{lemma}{Lemma}
\newtheorem*{proposition}{Proposition}


\newtcbtheorem[number within=section]{theorem}{Theorem}
{colback=green!5,colframe=green!35!black,fonttitle=\bfseries}{th}

\newtcbtheorem[number within=section]{definition}{Definition}
{colback=blue!5,colframe=blue!35!black,fonttitle=\bfseries}{def}

\newtcbtheorem[number within=section]{corollary}{Corollary}
{colback=yellow!5,colframe=yellow!35!black,fonttitle=\bfseries}{cor}

\newtcbtheorem[number within=section]{lemma}{Lemma}
{colback=red!5,colframe=red!35!black,fonttitle=\bfseries}{lem}

\newtcbtheorem[number within=section]{example}{Example}
{colback=white!5,colframe=white!35!black,fonttitle=\bfseries}{def}

\newtcbtheorem[number within=section]{note}{Important Note}{
        enhanced,
        sharp corners,
        attach boxed title to top left={
            xshift=-1mm,
            yshift=-5mm,
            yshifttext=-1mm
        },
        top=1.5em,
        colback=white,
        colframe=black,
        fonttitle=\bfseries,
        boxed title style={
            sharp corners,
            size=small,
            colback=red!75!black,
            colframe=red!75!black,
        } 
    }{impnote}
\usepackage[utf8]{inputenc}
\usepackage[english]{babel}
\usepackage{fancyhdr}
\usepackage[hidelinks]{hyperref}

\pagestyle{fancy}
\fancyhf{}
\rhead{CSE 130}
\chead{Friday, April 22, 2022}
\lhead{Lecture 12}
\rfoot{\thepage}

\setlength{\parindent}{0pt}

\begin{document}

\section{Representing Complex Data}

\subsection{Building Data Types}
There are three ways to build complex types/values: 
\begin{itemize}
    \item \textbf{Product Types} (each-of): a value of \code{T} contains a value of \code{T1} and a value of \code{T2}.
    \item \textbf{Sum Type} (one-of): a value of \code{T} contains a value of \code{T1} \emph{or} a value of \code{T2}.
\end{itemize}

\subsubsection{Product Type}
Suppose we wanted to represent a date as a tuple of \code{Int}s. 
\begin{verbatim}
    deadlineDate :: (Int, Int, Int)
    deadlineDate = (4, 29, 2022)

    deadlineTime :: (Int, Int, Int)
    deadlineTime = (11, 59, 59)

    -- Deadline date extended by one day
    extendedDate :: (Int, Int, Int) -> (Int, Int, Int)\end{verbatim}
There are a few issues here.
\begin{itemize}
    \item This is verbose and unreadable.
    \begin{mdframed}[]
        A \textbf{type synonym} for \code{T} is a name that can be used interchangeably with \code{T}. For example, 
        \begin{verbatim}
    type Date = (Int, Int, Int)
    type Time = (Int, Int, Int)
    
    deadlineDate :: Date
    deadlineDate = (4, 29, 2022)
    
    deadlineTime :: Time
    deadlineTime = (11, 59, 59)
    
    -- | Deadline date extended by one day
    extendedDate :: Date -> Date
        \end{verbatim}
    \end{mdframed}
    \item You can put the time into the \code{extendedDate} function.
    \begin{mdframed}[]
        We want \code{extendedDate deadlineTime} to fail at compile-time. A solution is to construct two different \textbf{datatypes}.
        \begin{verbatim}
    data Date = Date Int Int Int
    data Time = Time Int Int Int 
    -- constructor ^ ^^^^^^^^^^^ parameter types
    deadlineDate :: Date 
    deadlineDate = Date 4 29 2022 

    deadlineTime :: Time 
    deadlineTime = Time 11 59 59
        \end{verbatim}
        
    \end{mdframed}
\end{itemize}

\begin{mdframed}[]
    (Quiz.) Consider the following datatype. 
    \begin{verbatim}
        data Date = Date Int Int Int \end{verbatim}
    What would GHCi say to the following?  
    \begin{verbatim}
        >:t Date 4 29 2022\end{verbatim}
    
    \begin{enumerate}[(a)]
        \item Syntax error. 
        \item Type error. 
        \item \code{(Int, Int, Int)}
        \item \code{Date}
        \item \code{Date Int Int Int}
    \end{enumerate}

    \begin{mdframed}[]
        The answer is \textbf{D}. 
    \end{mdframed}
\end{mdframed}

\subsubsection{Record Syntax}
Consider the following datatype: 
\begin{verbatim}
    data Date = Date Int Int Int \end{verbatim}
It might be hard for us to tell what each \code{Int} means. So, we can use Haskell's \textbf{record syntax} to name the constructor parameters:
\begin{verbatim}
    data Date = Date {
        month   :: Int, 
        day     :: Int, 
        year    :: Int 
    }\end{verbatim}
Then, we can do:
\begin{verbatim}
    deadlineDate = Date { month = 4, date = 29, year = 2022 }\end{verbatim}
To extract a field value, we can treat the name as a function: 
\begin{verbatim}
    deadlineMonth = month deadlineDate\end{verbatim}


\subsubsection{Sum Type}
Suppose I want to represent a \emph{text document} with simple markup. Each paragraph is either 
\begin{itemize}
    \item Plain text (\code{String}).
    \item Heading: level and text (\code{Int} and \code{String}).
    \item List: whether it is ordered and items (\code{Bool} and \code{[String]}).
\end{itemize}
Now, let's suppose we store all paragraphs in a list: 
\begin{verbatim}
    doc = [
        (1, "Notes from 130"),                                        -- Level 1 heading 
        "There are two types of languages:",                          -- Plain text
        (True, ["those people complain about", "those no one uses"])  -- Ordered list
    ]
\end{verbatim}
This won't work because lists need to have one type only. The solution is to use \textbf{sum types} -- construct a new type that is a sum of (one-of) the three operations: 
\begin{verbatim}
    data Paragraph = PText String 
        | PHeading Int String 
        | PList Bool [String]
\end{verbatim}

\begin{mdframed}[]
    (Quiz.) Consider the following datatype:
    \begin{verbatim}
        data Paragraph = PText String | PHeading Int String | PList Bool [String]\end{verbatim}
    What would GHCi say to 
    \begin{verbatim}
        >:t PText "Hey there!"\end{verbatim}
    
    \begin{enumerate}[(a)]
        \item Syntax error 
        \item Type error 
        \item \code{PText}
        \item \code{String}
        \item \code{Paragraph}
    \end{enumerate}

    \begin{mdframed}[]
        Tha answer is \textbf{E}. Here, the type of \code{PText} is 
        \begin{verbatim}
            PText :: String -> Paragraph\end{verbatim}
    \end{mdframed}
\end{mdframed}
So, to construct a sum type, we use the notation 
\begin{verbatim}
    data T = C1 T11 .. T1k 
        | C2 T21 .. T2l 
        | .. 
        | Cn Tn1 .. Tnm \end{verbatim}
Here, \code{T} is the new datatype and \code{C1 .. Cn} are the constructors of \code{T}. A value of type \code{T} is either 
\begin{itemize}
    \item \code{C1 v1 .. vk} with \code{v1 :: T1i}
    \item or \code{C2 v1 .. vl} with \code{vi :: T2i}
    \item ... 
    \item or \code{Cn v1 .. vm} with \code{vi :: Tni}
\end{itemize}
How would we actually interpret a sum type? Using pattern matching!
\begin{verbatim}
    html :: Paragraph -> String 
    html (PText str)        = ... 
    html (PHeading lvl str) = ...
    html (PList ord items)  = ...
\end{verbatim}

\subsubsection{Dangers of Pattern Matching}
\begin{itemize}
    \item Consider the following: 
    \begin{verbatim}
        html :: Paragraph -> String 
        html (PText str)        = ... 
        html (PList ord items)  = ... \end{verbatim}
\end{itemize}

\subsubsection{Pattern Matching Syntax}
We can also use pattern-matching in a program using the \code{case} expression. For example, 
\begin{verbatim}
    html :: Paragraph -> String 
    html p = 
        case p of 
            PText str           -> ...
            PHeading lvl str    -> ... 
            PList ord items     -> ... \end{verbatim}

\begin{mdframed}[]
    (Quiz.) Consider the following datatype:
    \begin{verbatim}
        data Paragraph = PText String | PHeading Int String | PList Bool [String]\end{verbatim}
    What is the type of the following?
    \begin{verbatim}
        case PText "Hey there!" of 
            PText _      -> 1
            PHeading _ _ -> 2
            PList _ _    -> 3\end{verbatim}
    
    \begin{enumerate}[(a)]
        \item Syntax error 
        \item Type error 
        \item \code{Paragraph}
        \item \code{Int}
        \item \code{Paragraph -> Int}
    \end{enumerate} 

    \begin{mdframed}[]
        The answer is \textbf{D}. Here, we passed in an value \code{PText "Hey there!"}, which matches with the first branch and returns \code{1}.
    \end{mdframed}
\end{mdframed}

\begin{mdframed}[]
    (Quiz.) Consider the following datatype:
    \begin{verbatim}
        data Paragraph = PText String | PHeading Int String | PList Bool [String]\end{verbatim}
    What is the type of the following? 
    \begin{verbatim}
        case PText "Hey there!" of 
            PText str       -> str 
            PHeading lvl _  -> lvl 
            PList ord _     -> ord \end{verbatim}

    \begin{enumerate}[(a)]
        \item Syntax error
        \item Type error 
        \item \code{String}
        \item \code{Paragraph}
        \item \code{Paragraph -> String}
    \end{enumerate}

    \begin{mdframed}[]
        The answer is \textbf{B}. The \code{case} expression takes in a \code{Paragraph} and appear to return either a \code{String} (the first branch) or an \code{Int} (the second branch) or a \code{Bool} (the third branch). However, it is required that the return type is the same across all branches. 
    \end{mdframed}    
\end{mdframed}

\end{document}