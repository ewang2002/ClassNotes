\documentclass[letterpaper]{article}
\usepackage[margin=1in]{geometry}
\usepackage[utf8]{inputenc}
\usepackage{textcomp}
\usepackage{amssymb}
\usepackage{natbib}
\usepackage{graphicx}
\usepackage{gensymb}
\usepackage{amsthm, amsmath, mathtools}
\usepackage[dvipsnames]{xcolor}
\usepackage{enumerate}
\usepackage{mdframed}
\usepackage[most]{tcolorbox}
\usepackage{csquotes}
% https://tex.stackexchange.com/questions/13506/how-to-continue-the-framed-text-box-on-multiple-pages

\tcbuselibrary{theorems}

\newcommand{\R}{\mathbb{R}}
\newcommand{\Z}{\mathbb{Z}}
\newcommand{\N}{\mathbb{N}}
\newcommand{\Q}{\mathbb{Q}}
\newcommand{\C}{\mathbb{C}}
\newcommand{\code}[1]{\texttt{#1}}
\newcommand{\mdiamond}{$\diamondsuit$}
\newcommand{\PowerSet}{\mathcal{P}}
\newcommand{\Mod}[1]{\ (\mathrm{mod}\ #1)}
\DeclareMathOperator{\lcm}{lcm}

%\newtheorem*{theorem}{Theorem}
%\newtheorem*{definition}{Definition}
%\newtheorem*{corollary}{Corollary}
%\newtheorem*{lemma}{Lemma}
\newtheorem*{proposition}{Proposition}


\newtcbtheorem[number within=section]{theorem}{Theorem}
{colback=green!5,colframe=green!35!black,fonttitle=\bfseries}{th}

\newtcbtheorem[number within=section]{definition}{Definition}
{colback=blue!5,colframe=blue!35!black,fonttitle=\bfseries}{def}

\newtcbtheorem[number within=section]{corollary}{Corollary}
{colback=yellow!5,colframe=yellow!35!black,fonttitle=\bfseries}{cor}

\newtcbtheorem[number within=section]{lemma}{Lemma}
{colback=red!5,colframe=red!35!black,fonttitle=\bfseries}{lem}

\newtcbtheorem[number within=section]{example}{Example}
{colback=white!5,colframe=white!35!black,fonttitle=\bfseries}{def}

\newtcbtheorem[number within=section]{note}{Important Note}{
        enhanced,
        sharp corners,
        attach boxed title to top left={
            xshift=-1mm,
            yshift=-5mm,
            yshifttext=-1mm
        },
        top=1.5em,
        colback=white,
        colframe=black,
        fonttitle=\bfseries,
        boxed title style={
            sharp corners,
            size=small,
            colback=red!75!black,
            colframe=red!75!black,
        } 
    }{impnote}
\usepackage[utf8]{inputenc}
\usepackage[english]{babel}
\usepackage{fancyhdr}
\usepackage[hidelinks]{hyperref}

\pagestyle{fancy}
\fancyhf{}
\rhead{CSE 130}
\chead{Friday, May 27, 2022}
\lhead{Lecture 26}
\rfoot{\thepage}

\setlength{\parindent}{0pt}

\begin{document}

\section{Monads}
\subsection{Writing Apps with Monads}
Why is it so hard to write a program that prints ``\code{Hello world!}'' in Haskell?

\bigskip 

Note that Haskell is \emph{pure}. A program is an expression that evaluates to a value and does \emph{nothing else}. So, a function of type \code{Int -> Int} computes a \emph{single integer output} from a single integer input and does \emph{nothing else}. Moreover, it always returns the same output given the same integer. Specifically, evaluation must not have any side effects. 

\bigskip 

Haskell has a special type called \code{IO}, which we can think of as a \code{Recipe}.
\begin{verbatim}
    type Recipe a = IO a \end{verbatim}

So, when executing a program, Haskell looks for a special value 
\begin{verbatim}
    main :: Recipe ()\end{verbatim}
This is a Recipe for everything a program should do, and does not return a special value.

\bigskip 

Note that 
\begin{itemize}
    \item A function of type \code{Int -> Int} still computes a single integer output from a single integer input and does nothing else.
    \item A function of type \code{Int -> Recipe Int} computes an Int-recipe from a single integer input and does nothing else.
    \item Handing this recipe to \code{main} will possibly result in these ``side effects.''
\end{itemize}
So, writing ``\code{Hello World}'' is as simple as 
\begin{verbatim}
    main :: Recipe ()
    main = putStrLn "Hello, world!"\end{verbatim}
Note that \code{putStrLn} has the following definition 
\begin{verbatim}
    putStrLn :: String -> Recipe ()\end{verbatim}
Compiling and running like so gives us 
\begin{verbatim}
    $ ghc hello.hs 
    $ ./hello 
    Hello, world!\end{verbatim}
How would we have multiple print statements, though? 
\begin{mdframed}[]
    (Quiz.) Suppose we have a function \code{combine} that lets us combine recipes like so: 
    \begin{verbatim}
        main :: Recipe ()
        main = combine (putStrLn "Hello,") (putStrLn "World!")\end{verbatim}

    \begin{enumerate}[(a)]
        \item \code{() -> () -> ()}
        \item \code{Recipe () -> Recipe () -> Recipe ()}
        \item \code{Recipe a -> Recipe a -> Recipe a}
        \item \code{Recipe a -> Recipe b -> Recipe b}
        \item \code{Recipe a -> Recipe b -> Recipe a}
    \end{enumerate}

    \begin{mdframed}[]
        The answer depends. For this particular example, \textbf{B} is the answer. However, this could be generalized, which we will discuss later. 
    \end{mdframed}
\end{mdframed}

\subsubsection{Using Intermediate Results}
Suppose we wanted to write a program that 
\begin{itemize}
    \item asks for the user's \code{name} using 
    \begin{verbatim}
        getLine :: Recipe String\end{verbatim}
    \item prints out a greeting with that name using 
    \begin{verbatim}
        putStrLn :: String -> Recipe ()\end{verbatim}
\end{itemize}

How do we pass the output of the first recipe into the second recipe? 
\begin{mdframed}[]
    (Quiz.) Suppose you have two recipes.
    \begin{verbatim}
        crack       :: Recipe Yolk 
        eggBatter   :: Yolk -> Recipe Batter  \end{verbatim}
    and we want to get 
    \begin{verbatim}
        mkBatter :: Recipe Batter 
        mkBatter :: crack `combineWithResult` eggBatter\end{verbatim}
    What should the type of \code{combineWithResult} be? 

    \begin{enumerate}[(a)]
        \item \code{Yolk -> Batter -> Batter}
        \item \code{Recipe Yolk -> (Yolk -> Recipe Batter) -> Recipe Batter}
        \item \code{Recipe a -> (a -> Recipe a) -> Recipe a}
        \item \code{Recipe a -> (a -> Recipe b) -> Recipe b}
        \item \code{Recipe Yolk -> (Yolk -> Recipe Batter) -> Recipe ()}
    \end{enumerate}

    \begin{mdframed}[]
        The answer is \textbf{D}. Note that B is a more specific case of D. As noted, this is similar in nature to the type of \code{>>=} (bind). 
    \end{mdframed}
\end{mdframed}
In fact, since \code{Recipe}s are \code{Monad}s, we can do 
\begin{verbatim}
    main :: Recipe ()
    main = getLine >>= \name -> putStrLn ("Hello, " ++ name ++ "!")\end{verbatim}
or (in a more procedural style),
\begin{verbatim}
    main :: Recipe ()
    main = do 
                name <- getLine
                putStrLn ("Hello, " ++ name ++ "!")\end{verbatim}
Now, expanding on this program to ask the user for the name multiple times gives us 
\begin{verbatim}
    doQuit :: Recipe a 
    doQuit =  exitWith ExitSuccess

    putStrFlush :: String -> Recipe ()
    putStrFlush str = do 
        putStr str 
        hFlush stdout 

    main :: Recipe ()
    main = do 
                putStrFlush "Your name: "
                name <- getLine
                if name == ":quit"
                    then doQuit 
                    else do 
                        putStrLn ("Hello, " ++ name ++ "!")
                        main\end{verbatim}

\end{document}