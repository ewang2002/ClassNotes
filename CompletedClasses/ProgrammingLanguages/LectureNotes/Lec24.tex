\documentclass[letterpaper]{article}
\usepackage[margin=1in]{geometry}
\usepackage[utf8]{inputenc}
\usepackage{textcomp}
\usepackage{amssymb}
\usepackage{natbib}
\usepackage{graphicx}
\usepackage{gensymb}
\usepackage{amsthm, amsmath, mathtools}
\usepackage[dvipsnames]{xcolor}
\usepackage{enumerate}
\usepackage{mdframed}
\usepackage[most]{tcolorbox}
\usepackage{csquotes}
% https://tex.stackexchange.com/questions/13506/how-to-continue-the-framed-text-box-on-multiple-pages

\tcbuselibrary{theorems}

\newcommand{\R}{\mathbb{R}}
\newcommand{\Z}{\mathbb{Z}}
\newcommand{\N}{\mathbb{N}}
\newcommand{\Q}{\mathbb{Q}}
\newcommand{\C}{\mathbb{C}}
\newcommand{\code}[1]{\texttt{#1}}
\newcommand{\mdiamond}{$\diamondsuit$}
\newcommand{\PowerSet}{\mathcal{P}}
\newcommand{\Mod}[1]{\ (\mathrm{mod}\ #1)}
\DeclareMathOperator{\lcm}{lcm}

%\newtheorem*{theorem}{Theorem}
%\newtheorem*{definition}{Definition}
%\newtheorem*{corollary}{Corollary}
%\newtheorem*{lemma}{Lemma}
\newtheorem*{proposition}{Proposition}


\newtcbtheorem[number within=section]{theorem}{Theorem}
{colback=green!5,colframe=green!35!black,fonttitle=\bfseries}{th}

\newtcbtheorem[number within=section]{definition}{Definition}
{colback=blue!5,colframe=blue!35!black,fonttitle=\bfseries}{def}

\newtcbtheorem[number within=section]{corollary}{Corollary}
{colback=yellow!5,colframe=yellow!35!black,fonttitle=\bfseries}{cor}

\newtcbtheorem[number within=section]{lemma}{Lemma}
{colback=red!5,colframe=red!35!black,fonttitle=\bfseries}{lem}

\newtcbtheorem[number within=section]{example}{Example}
{colback=white!5,colframe=white!35!black,fonttitle=\bfseries}{def}

\newtcbtheorem[number within=section]{note}{Important Note}{
        enhanced,
        sharp corners,
        attach boxed title to top left={
            xshift=-1mm,
            yshift=-5mm,
            yshifttext=-1mm
        },
        top=1.5em,
        colback=white,
        colframe=black,
        fonttitle=\bfseries,
        boxed title style={
            sharp corners,
            size=small,
            colback=red!75!black,
            colframe=red!75!black,
        } 
    }{impnote}
\usepackage[utf8]{inputenc}
\usepackage[english]{babel}
\usepackage{fancyhdr}
\usepackage[hidelinks]{hyperref}

\pagestyle{fancy}
\fancyhf{}
\rhead{CSE 130}
\chead{Monday, May 23, 2022}
\lhead{Lecture 24}
\rfoot{\thepage}

\setlength{\parindent}{0pt}

\begin{document}

\section{Monads}

\subsection{Functors}
Recall our use of \emph{higher-order functions} to abstract code patterns for \emph{lists}. In particular, we made use of the \code{map} higher-order function to 
\begin{itemize}
    \item render the values of a list, 
    \item square the values of a list,
    \item and more.
\end{itemize}
What about trees? 
\begin{itemize}
    \item Let's suppose we wanted to render the values of a \emph{tree}, where a tree is defined by 
    \begin{verbatim}
    data Tree a 
        = Leaf 
        | Node a (Tree a) (Tree a)\end{verbatim}
    So, to render the values of a tree, we can do 
    \begin{verbatim}
    showTree :: Tree Int -> Tree String
    showTree Leaf         = Leaf 
    showTree (Node v l r) = Node (show x) (showTree l) (showTree r)\end{verbatim}

    \item Now, let's suppose we wanted to square the values of a tree. We can use the same pattern as used in the previous part, changing the return type and the return values appropriately.
\end{itemize}

We can write a generalization of this by writing a \code{map} function for trees. 
\begin{verbatim}
    mapTree :: (a -> b) -> Tree a -> Tree b
    mapTree f Leaf          = Leaf 
    mapTree f (Node v l r)  = ... \end{verbatim}
But, observe the following: 
\begin{verbatim}
    type List a = [a]
    mapList :: (a -> b) -> List a -> List b    -- List
    mapTree :: (a -> b) -> Tree a -> Tree b    -- Tree\end{verbatim}
Notice how we have essentially the same signature for both \code{List}s and \code{Tree}s. 

\subsubsection{Class for Mapping}
We can make a typeclass to model mapping over some datatypes (not all datatypes support mapping over them).

\begin{verbatim}
    class Functor t where 
        fmap :: (a -> b) -> t a -> t b\end{verbatim}

Then, we can do 
\begin{verbatim}
    instance Functor [] where 
        fmap = mapList 

    instance Functor Tree where 
        fmap = mapTree\end{verbatim}




\subsection{Monads}
Consider the following \code{Expr} data type. 
\begin{verbatim}
    data Expr 
        = Num   Int 
        | Plus  Expr Expr
        | Div   Expr Expr 
        deriving (Show) 
    
    eval :: Expr -> Int 
    eval (Num n)        = n
    eval (Plus e1 e2)   = eval e1   +   eval e2 
    eval (Div  e1 e2)   = eval e1 `div` eval e2\end{verbatim}

So, for example, we can run 
\begin{verbatim}
    $ eval (Div (Num 6) (Num 2))
    3 \end{verbatim}
But, if we were in an interpreter like Nano, the following can crash Nano: 
\begin{verbatim}
    $ eval (Div (Num 6) (Num 0))
    *** Exception: divide by zero \end{verbatim}

Let us introduce a new data type which will handle errors for us.
\begin{verbatim}
    data Result a
        = Error String 
        | Value a \end{verbatim}
So, instead of returning \code{Int}, this will now return \code{Result Int}, where 
\begin{itemize}
    \item If a sub-expression has a divide-by-zero, then return \code{Error ...}.
    \item If all sub-expressions are safe, then we can return the actual \code{Value v}.
\end{itemize}
Therefore, 
\begin{verbatim}
    eval :: Expr -> Result Int 
    eval (Num n)            = Value n 
    eval (Plus e1 e2)       = 
        case eval e1 of 
            Error err1  -> Error err1 
            Value v1    -> case eval e2 of 
                            Error err2  -> Error err2 
                            Value v1    -> Value (v1 + v2)
    
    eval (Div e1 e2)        = 
        case eval e1 of 
            Error err1  -> Error err1 
            Value v1    -> case eval e2 of 
                            Error err2  -> Error err2 
                            Value v1    -> if v2 == 0 
                                            then Error ("DBZ: " ++ show e2)
                                            else Value (v1 `div` v2)\end{verbatim}

Note that this works -- this doesn't crash the interpreter. However, there is a lot of repetition. 





\end{document}