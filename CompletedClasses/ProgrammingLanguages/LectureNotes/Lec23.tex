\documentclass[letterpaper]{article}
\usepackage[margin=1in]{geometry}
\usepackage[utf8]{inputenc}
\usepackage{textcomp}
\usepackage{amssymb}
\usepackage{natbib}
\usepackage{graphicx}
\usepackage{gensymb}
\usepackage{amsthm, amsmath, mathtools}
\usepackage[dvipsnames]{xcolor}
\usepackage{enumerate}
\usepackage{mdframed}
\usepackage[most]{tcolorbox}
\usepackage{csquotes}
% https://tex.stackexchange.com/questions/13506/how-to-continue-the-framed-text-box-on-multiple-pages

\tcbuselibrary{theorems}

\newcommand{\R}{\mathbb{R}}
\newcommand{\Z}{\mathbb{Z}}
\newcommand{\N}{\mathbb{N}}
\newcommand{\Q}{\mathbb{Q}}
\newcommand{\C}{\mathbb{C}}
\newcommand{\code}[1]{\texttt{#1}}
\newcommand{\mdiamond}{$\diamondsuit$}
\newcommand{\PowerSet}{\mathcal{P}}
\newcommand{\Mod}[1]{\ (\mathrm{mod}\ #1)}
\DeclareMathOperator{\lcm}{lcm}

%\newtheorem*{theorem}{Theorem}
%\newtheorem*{definition}{Definition}
%\newtheorem*{corollary}{Corollary}
%\newtheorem*{lemma}{Lemma}
\newtheorem*{proposition}{Proposition}


\newtcbtheorem[number within=section]{theorem}{Theorem}
{colback=green!5,colframe=green!35!black,fonttitle=\bfseries}{th}

\newtcbtheorem[number within=section]{definition}{Definition}
{colback=blue!5,colframe=blue!35!black,fonttitle=\bfseries}{def}

\newtcbtheorem[number within=section]{corollary}{Corollary}
{colback=yellow!5,colframe=yellow!35!black,fonttitle=\bfseries}{cor}

\newtcbtheorem[number within=section]{lemma}{Lemma}
{colback=red!5,colframe=red!35!black,fonttitle=\bfseries}{lem}

\newtcbtheorem[number within=section]{example}{Example}
{colback=white!5,colframe=white!35!black,fonttitle=\bfseries}{def}

\newtcbtheorem[number within=section]{note}{Important Note}{
        enhanced,
        sharp corners,
        attach boxed title to top left={
            xshift=-1mm,
            yshift=-5mm,
            yshifttext=-1mm
        },
        top=1.5em,
        colback=white,
        colframe=black,
        fonttitle=\bfseries,
        boxed title style={
            sharp corners,
            size=small,
            colback=red!75!black,
            colframe=red!75!black,
        } 
    }{impnote}
\usepackage[utf8]{inputenc}
\usepackage[english]{babel}
\usepackage{fancyhdr}
\usepackage[hidelinks]{hyperref}

\pagestyle{fancy}
\fancyhf{}
\rhead{CSE 130}
\chead{Friday, May 20, 2022}
\lhead{Lecture 23}
\rfoot{\thepage}

\setlength{\parindent}{0pt}

\begin{document}
\section{Type Classes}
\subsection{Using Type Classes}
To motivate this, we will build a small library for \emph{Environments} mapping keys \code{k} to values \code{v}. Recall that, in Nano, we represented environments as \code{[(Id, Value)]}; however, what if we want to represent keys that are not \code{Id} or values that are not \code{Value}?

\bigskip 

Let us define a new \textbf{polymorphic datatype} \code{Env}. 
\begin{verbatim}
    data Env k v 
        = Def   v                   -- Default value to be used for missing keys 
        | Bind  k v (Env k v)       -- Bind key 'k' to value 'v'
        deriving (Show)\end{verbatim}

So, for example, 
\begin{verbatim}
    $ let env0 = add "cat" 10.0 (add "dog" 20.0 (Def 0))

    $ get "cat" env0 
    10 

    $ get "dog" env0 
    20 

    $ get "horse" env0 
    0\end{verbatim}

Let us implement some of the key functions. 
\begin{itemize}
    \item \code{add}: Adds a \code{key} and \code{val} pair, returning a new environment. 
    \begin{verbatim}
        add :: k -> v -> Env k v -> Env k v 
        add key val env = Bind key val env 
        -- or
        -- add = Bind\end{verbatim}

    \item \code{get}: Gets the value associated with the \code{key}.
        \begin{verbatim}
            get :: k -> Env k v -> v 
            get key (Def v)         = v
            get key (Bind k v env)
                | k == key      = v
                | otherwise     = get key env \end{verbatim}
\end{itemize}
Note that this gives a type error, especially for \code{get}. The issue is that we require \code{k} and \code{key} to have \code{Eq} snce we're comparing two keys. So,
\begin{verbatim}
    get :: Eq k => k -> Env k v -> v    -- Changed this line 
    get key (Def v)         = v
    get key (Bind k v env)
        | k == key      = v
        | otherwise     = get key env \end{verbatim}




\subsection{Explicit Type Annotations}
Consider the standard typeclass \code{Read}, where its simplified implementation is shown below: 
\begin{verbatim}
    class Read a where 
        read :: String -> a\end{verbatim}

Note that \code{Read} is the \emph{opposite} of \code{Show}. 
\begin{itemize}
    \item It requires that every instance \code{T} can parse a string and turn it into \code{T}.
    \item Just like with \code{Show}, most standard types are instances of \code{Read}.
\end{itemize}

\begin{mdframed}[]
    (Quiz.) What does the expression \code{read "2"} evaluate to? 
    \begin{enumerate}[(a)]
        \item Type error
        \item \code{"2"}
        \item \code{2}
        \item \code{2.0}
        \item Run-time error 
    \end{enumerate}

    \begin{mdframed}[]
        The answer is \textbf{A}. There are multiple ways to ``read'' \code{"2"}. In general, note that the definition of \code{read} has that the return type is \code{a}, a generic type. 
    \end{mdframed}
\end{mdframed}

So, \textbf{explicit type annotation} is needed to tell Haskell what to convert the string to. 
\begin{verbatim}
    $ (read "2") :: Int 
    2 

    $ (read "2") :: Float 
    2.0 

    $ (read "2") :: String 
    **** Exception: Prelude.read: no parse 
    
    $ (read "\"2\"") :: String 
    "2"

    $ read "()"
    ()\end{verbatim}


\subsection{Creating Type Classes}
Type classes are useful for many different things. Let's see an example with \textbf{JSON}. Here's an example JSON: 
\begin{verbatim}
    { "name"    : "Nadia"
    , "age"     : 37.0
    , "likes"   : [ "poke", "coffee", "pasta" ]
    , "hates"   : [ "beets" , "milk" ]
    , "lunches" : [ {"day" : "mon", "loc" : "rubios"}
                , {"day" : "tue", "loc" : "home"}
                , {"day" : "wed", "loc" : "curry up now"}
                , {"day" : "thu", "loc" : "home"}
                , {"day" : "fri", "loc" : "santorini"} ]
    }\end{verbatim}
Each JSON value is either 
\begin{itemize}
    \item a base value like a string, number, or boolean,
    \item an (ordered) array of values, or 
    \item an object, i.e. a set of string-value pairs. 
\end{itemize}

\subsubsection{JSON Datatype}
We can represent a subset of JSON values with the Haskell data type 
\begin{verbatim}
    data JVal 
        = JStr  String 
        | JNum  Double 
        | JBool Bool 
        | JObj  [(String, JVal)]
        | JArr  [JVal]
        deriving (Eq, Ord, Show)\end{verbatim}
So, the example JSON would look like 
\begin{verbatim}
    js1 =
        JObj [("name", JStr "Nadia")
            ,("age",  JNum 36.0)
            ,("likes",   JArr [ JStr "poke", JStr "coffee", JStr "pasta"])
            ,("hates",   JArr [ JStr "beets", JStr "milk"])
            ,("lunches", JArr [ JObj [("day",  JStr "mon")
                                        ,("loc",  JStr "rubios")]
                                , JObj [("day",  JStr "tue")
                                        ,("loc",  JStr "home")]
                                , JObj [("day",  JStr "wed")
                                        ,("loc",  JStr "curry up now")]
                                , JObj [("day",  JStr "thu")
                                        ,("loc",  JStr "home")]
                                , JObj [("day",  JStr "fri")
                                        ,("loc",  JStr "santorini")]
                                ])
            ]  \end{verbatim}
This is a pain to write out. Instead, let us serialize Haskell Values to JSON. 
\begin{itemize}
    \item Base types \code{String}, \code{Double}, \code{Bool} are serialized as base JSON values. 
    \item Lists are serialized into JSON arrays. 
    \item Lists of key-value pairs are serialized into JSON objects.
\end{itemize}

\subsubsection{Type Classes}
We can define a type class 
\begin{verbatim}
    class JSON a where 
        toJson :: a -> JVal \end{verbatim}
so that a type \code{a} can be converted to JSON. Then, we can work on the basic types: 
\begin{verbatim}
    instance JSON Double where 
        toJson = JNum

    instance JSON Bool where 
        toJson = JBool
    
    instance JSON String where 
        toJson = JStr\end{verbatim}
We can also work on more complicated types. 
\begin{verbatim}
    instance JSON a => JSON [a] where 
        toJson xs = JArr [toJson x | x <- xs]\end{verbatim}
Here, if \code{a} is an instance of \code{JSON}, then there is a generic recipe to convert lists of \code{a} values. Similarly, for key-value lists, we have: 
\begin{verbatim}
    instance (JSON a) => JSON [(String a)] where 
        toJson kvs = JObj [ (k, toJson v) | (k, v) <- kvs]\end{verbatim}


\end{document}