\documentclass[letterpaper]{article}
\usepackage[margin=1in]{geometry}
\usepackage[utf8]{inputenc}
\usepackage{textcomp}
\usepackage{amssymb}
\usepackage{natbib}
\usepackage{graphicx}
\usepackage{gensymb}
\usepackage{amsthm, amsmath, mathtools}
\usepackage[dvipsnames]{xcolor}
\usepackage{enumerate}
\usepackage{mdframed}
\usepackage[most]{tcolorbox}
\usepackage{csquotes}
% https://tex.stackexchange.com/questions/13506/how-to-continue-the-framed-text-box-on-multiple-pages

\tcbuselibrary{theorems}

\newcommand{\R}{\mathbb{R}}
\newcommand{\Z}{\mathbb{Z}}
\newcommand{\N}{\mathbb{N}}
\newcommand{\Q}{\mathbb{Q}}
\newcommand{\C}{\mathbb{C}}
\newcommand{\code}[1]{\texttt{#1}}
\newcommand{\mdiamond}{$\diamondsuit$}
\newcommand{\PowerSet}{\mathcal{P}}
\newcommand{\Mod}[1]{\ (\mathrm{mod}\ #1)}
\DeclareMathOperator{\lcm}{lcm}

%\newtheorem*{theorem}{Theorem}
%\newtheorem*{definition}{Definition}
%\newtheorem*{corollary}{Corollary}
%\newtheorem*{lemma}{Lemma}
\newtheorem*{proposition}{Proposition}


\newtcbtheorem[number within=section]{theorem}{Theorem}
{colback=green!5,colframe=green!35!black,fonttitle=\bfseries}{th}

\newtcbtheorem[number within=section]{definition}{Definition}
{colback=blue!5,colframe=blue!35!black,fonttitle=\bfseries}{def}

\newtcbtheorem[number within=section]{corollary}{Corollary}
{colback=yellow!5,colframe=yellow!35!black,fonttitle=\bfseries}{cor}

\newtcbtheorem[number within=section]{lemma}{Lemma}
{colback=red!5,colframe=red!35!black,fonttitle=\bfseries}{lem}

\newtcbtheorem[number within=section]{example}{Example}
{colback=white!5,colframe=white!35!black,fonttitle=\bfseries}{def}

\newtcbtheorem[number within=section]{note}{Important Note}{
        enhanced,
        sharp corners,
        attach boxed title to top left={
            xshift=-1mm,
            yshift=-5mm,
            yshifttext=-1mm
        },
        top=1.5em,
        colback=white,
        colframe=black,
        fonttitle=\bfseries,
        boxed title style={
            sharp corners,
            size=small,
            colback=red!75!black,
            colframe=red!75!black,
        } 
    }{impnote}
\usepackage[utf8]{inputenc}
\usepackage[english]{babel}
\usepackage{fancyhdr}
\usepackage[hidelinks]{hyperref}

\pagestyle{fancy}
\fancyhf{}
\rhead{CSE 130}
\chead{Friday, May 13, 2022}
\lhead{Lecture 20}
\rfoot{\thepage}

\setlength{\parindent}{0pt}

\begin{document}
\section{Lexing and Parsing}
\subsection{Using the Parser Generators}
There are two special types of files. 
\begin{itemize}
    \item \code{Lexer.x} (or any \code{.x} file): here, you will be writing the specifications for the lexer here (the token format). We will use the Haskell tool \code{alex} to translate this file to \code{Lexer.hs} (i.e., generate the lexer).
    \item \code{Parser.y} (or any \code{.y} file): here, you will describe the form of the parser (the grammar). This will be fed to a tool called \code{happy}, which will be translated to the file \code{Parser.hs}.
\end{itemize}

\subsubsection{Regular Expressions}
Before we talk about lexing, we first need to talk about regular expressions. Simply put, a regular expression is a sequence of characters that specifies a search pattern in text. For our \textbf{token format} file (used by \code{alex}), regular expressions have one of the following form\footnote{These may differ in different languages.}: 
\begin{itemize}
    \item a \textbf{character class}: written as a bunch of characters in a square bracket, this particular regular expression will match any of the characters in the brackets. In other words, for the character class \code{[c1 c2 ... cn]}, this will match \emph{any one} of the characters \code{c1 c2 ... cn}. 
    
    \bigskip 

    There are several special character classes. 
    \begin{itemize}
        \item \code{[0-9]} will match any \emph{digit}.
        \item \code{[a-z]} will match any \emph{lowercase letter}.
        \item \code{[A-Z]} will match any \emph{uppercase letter}.
        \item \code{[a-z A-Z]} will match \emph{any letter}. 
    \end{itemize}
    If a character class has one character, then you do not need to put any brackets; in other words, \code{[c1]} is equivalent to \code{c1}. 

    \item \code{R1 R2} matches a string \code{s1 ++ s2}, where \code{s1} matches \code{R1} and \code{s2} matches \code{R2}. For example, \code{[0-9] [0-9]} matches any two-digit string. 
    % do spaces not matter here?

    \item \code{R+} matches \emph{one or more repetitions} of what \code{R} matches. For example, \code{[0-9]+} matches a natural number. 
    
    \item \code{R*} matches \emph{zero or more repetitions} of what \code{R} matches. 
\end{itemize}

\begin{mdframed}[]
    (Quiz.) Which of the following strings are matched by \code{[a-z A-Z] [a-z A-Z 0-9]*}?

    \begin{enumerate}[(a)]
        \item (Empty string)
        \item \code{5}
        \item \code{x5}
        \item \code{x}
        \item C and D 
    \end{enumerate}

    \begin{mdframed}[]
        The answer is \textbf{E}. The regular expression matches any \emph{one} letter (\code{[a-z A-Z]}) followed by \emph{zero oe more} letters and digits (\code{[a-z A-Z 0-9]*}). 
    \end{mdframed}
\end{mdframed}

We can also name regular expressions. For example, instead of writing out \code{[0-9]} or \code{[a-z A-Z]} over and over again, we can instead write 
\begin{verbatim}
    $digit = [0-9]
    $alpha = [a-z A-Z]\end{verbatim}
Then, instead of writing \code{[a-z A-Z] [a-z A-Z 0-9]*}, we can write 
\begin{verbatim}
    $alpha [$alpha $digit]*\end{verbatim}

\subsubsection{Lexing}
We use the tool \code{alex} to generate the lexer. We want to give \code{alex} a \code{.x} file, describing the lexer. There are several components in a \code{.x} file. 

\begin{itemize}
    \item First, you need to list the kinds of \textbf{tokens} we have in the language.
    \begin{verbatim}
        data Token
            = NUM    AlexPosn Int
            | ID     AlexPosn String
            | PLUS   AlexPosn
            | MINUS  AlexPosn
            | MUL    AlexPosn
            | DIV    AlexPosn
            | LPAREN AlexPosn
            | RPAREN AlexPosn
            | EOF    AlexPosn\end{verbatim}
    We use a Haskell datatype to represent these possible tokens. This is completely normal Haskell syntax.
    
    \bigskip 
    
    Note that each constructor has the parameter \code{AlexPosn}. The whole point of this is to store information regarding where the token was found\footnote{Otherwise, you do not need to worry about this.}. Then, we might have some other parameters (e.g., \code{NUM} with \code{Int}) to \emph{store} the data associated with this type.

    \item Next, you need to list the \textbf{token rules}.
    \begin{verbatim}
        -- Regex:                       Function: AlexPosn -> String -> Token
        \+                            { \p _ -> PLUS   p }
        \-                            { \p _ -> MINUS  p }
        \*                            { \p _ -> MUL    p }
        \/                            { \p _ -> DIV    p }
        \(                            { \p _ -> LPAREN p }
        \)                            { \p _ -> RPAREN p }
        $alpha [$alpha $digit \_ \']* { \p s -> ID     p s }
        $digit+                       { \p s -> NUM p (read s) }\end{verbatim}
    Each line is a token rule. The content on the left-hand side is a \textbf{regular expression} that tells you what the token looks like. The content on the right-hand side tells \code{alex} how to construct this \code{Token} type from this information. 

    \bigskip 

    For example, let's look at 
    \begin{verbatim}
        \+                            { \p _ -> PLUS   p }\end{verbatim}
    Obviously, the string to the left is a regular expression. It should be noted that a lot of the regular expressions have a backslash prepended to it (for example, this \code{+} regular expression has a backslash before it). The reason for this is because these characters, or at least some of them, have a special meaning in regular expressions, so we need to \emph{escape} them so we interpret the \code{+} literally as opposed to in a regular expression class. 

    \bigskip 

    Now, what is going on at the right side? On the right, we have a \textbf{Haskell function} surrounded by curly braces. Each function has \textbf{two arguments}: 
    \begin{itemize}
        \item The first argument is the position (\code{AlexPosn}). 
        \item The second one is the string that was parsed. 
    \end{itemize}
    For this, we consider a more interesting example:
    \begin{verbatim}
        $alpha [$alpha $digit \_ \']*       { \p s -> ID     p s }\end{verbatim}
    What this regular expression is saying is  to match anything that looks like an identifier (starts with a letter followed by any letter, digit, underscore, or tick). Then, the function takes in two arguments, \code{p} (for position) and \code{s} (the string that was parsed) and returns an \code{ID} with the position and string parsed. 

    \bigskip 

    Note that the last regular expression above is no different; the only difference is that we have \code{read s}, which converts the string read into an \code{Int}.



    \item Finally, we need to run the lexer. From the token rules, \code{alex} will generate a function \code{alexScan} which
    \begin{itemize}
        \item Given the input string, find the longest prefix \code{p} that matches one of the rules. 
        \item If \code{p} is empty, it fails (at the beginning of the string, there's something that doesn't match).
        \item Otherwise, it will convert \code{p} into a token and returns the \emph{rest of the string}. 
    \end{itemize}
    As implied, this function will be executed in a loop so you can read the whole input string. Note that, in the programming assignments, we wrap this function into a function 
    \begin{verbatim}
        parseTokens :: String -> Either ErrMsg [Token]\end{verbatim}
    which repeatedly calls \code{alexScan} until it consumes the whole input string or fails. Note that the return type here means that we will return \emph{either} \code{ErrMsg} or \code{[Token]}.

    \bigskip 

    Thus, testing the function gives us 
    \begin{verbatim}
        $ parseTokens "23 + 4 / off -"
        Right [ NUM (AlexPosn 0 1 1) 23
            , PLUS (AlexPosn 3 1 4)
            , NUM (AlexPosn 5 1 6) 4
            , DIV (AlexPosn 7 1 8)
            , ID (AlexPosn 9 1 10) "off"
            , MINUS (AlexPosn 13 1 14) 
            ]
        
        $ parseTokens "%"
        Left "lexical error at 1 line, 1 column"\end{verbatim}
\end{itemize}

\begin{mdframed}[]
    (Quiz.) What is the result of \code{parseTokens} \code{"92zoo"} (positions omitted for readability)?

    \begin{enumerate}[(a)]
        \item Lexical error 
        \item \code{[ID "92zoo"]}
        \item \code{[NUM "92"]}
        \item \code{[NUM "92", ID "zoo"]}
    \end{enumerate}

    \begin{mdframed}[]
        The answer can either be \textbf{A} or \textbf{D}, depending  on whether you have whitespace sensitivity. If we care about whitespace, then \textbf{A} is the answer. If we do not care about whitespace, then \textbf{D} is the answer. 
    \end{mdframed}
\end{mdframed}
\textbf{Remark:} In a real programming langauge, you probably care about whitespace sensitivity.

\subsubsection{Parser}
Just like how \code{alex} is for lexing, we use \code{happy} for the parser. We input to \code{happy} a \code{.y} file that describes the \emph{grammar}. As always, there are several components. 

\begin{itemize}
    \item Note that something like 
    \begin{verbatim}
        data Aexpr 
            = AConst  Int
            | AVar    Id
            | APlus   Aexpr Aexpr
            | AMinus  Aexpr Aexpr
            | AMul    Aexpr Aexpr\end{verbatim}
    is the \emph{abstract} syntax. This tells us that there is a binary operation called (for example) \code{APlus} with two \code{Aexpr}, but this doesn't tell us \emph{what} this might look like in source code. Thus, what we want is the \emph{concrete syntax}; in particular,
    \begin{itemize}
        \item What programs look like when written as text. 
        \item How to map that text into the abstract syntax. 
    \end{itemize}
    What is a grammar? A grammar is a recursive definition of a set of parse trees; in particular, a grammar is made of 
    \begin{itemize}
        \item \textbf{Terminals}: the leaves of the tree (corresponds to tokens).
        \item \textbf{Nonterminals}: the internal nodes of the tree (corresponds to more complex constructs, all kinds of subexpressions). 
        \item \textbf{Production Rules}: describes how to ``produce'' a non-terminal from terminals and other non-terminals (i.e., what children each nonterminal can have).
    \end{itemize}
    So, consider the following example:
    \begin{verbatim}
        Aexpr :      -- Non-term Aexpr can be either:
              | TNUM             -- Term of format "number", or
              | ID               -- Term of format "identifier", or
              | '(' Aexpr ')'    -- Term '(', non-term Aexpr, term ')'
              | Aexpr '*' Aexpr  -- Non-term Aexpr, term '*', non-term Aexpr
              | Aexpr '+' Aexpr  -- Non-term Aexpr, term '+', non-term Aexpr
              | Aexpr '-' Aexpr  -- Non-term Aexpr, term '-', non-term Aexpr\end{verbatim}
    Here, we have a single nonterminal (\code{Aexpr}). Then, our terminals are things like \code{TNUM}, \code{ID}, \code{'('}, \code{')'}, \code{'*'}, \code{'+'}, and \code{'-'}. Notice that we have 6 production rules, which describes how our \code{Aexpr} can be represented as a sequence of other things. For example, 
    \begin{itemize}
        \item \code{Aexpr} can be written as a \code{TNUM}.
        \item \code{Aexpr} can be written as an \code{ID}.
        \item \code{Aexpr} can be written as an \code{Aexpr} surrounded by parentheses.
        \item And so on. 
    \end{itemize}
    So, a parse tree for the string \code{(x + 2)} will look like 
    \begin{verbatim}
                                    Aexpr 
                                    / | \ 
                                   /  |  \ 
                                  /   |   \ 
                                 /    |    \ 
                               '('    |    ')'
                                    Aexpr
                                    / | \ 
                                   /  |  \ 
                                  /  '+'  \ 
                                 /         \
                              Aexpr      Aexpr 
                                |          |
                                |          |
                                |          |
                            'x' (ID)   '2' (TNUM)\end{verbatim}
    \begin{mdframed}[]
        (Quiz.) Using the above grammar of \code{Aexpr}, which string \emph{cannot} be parsed as \code{Aexpr}?
    
        \begin{enumerate}[(a)]
            \item \code{x}
            \item \code{x 5}
            \item \code{(x +) 5}
            \item \code{x + 5 + 1}
            \item B and C 
        \end{enumerate}
    
        \begin{mdframed}[]
            The answer is \textbf{E}. For B, there is no rule that lets us generate both an identifier and number. For C, there is no rule that lets us generate \code{(x +)}.
        \end{mdframed}
    \end{mdframed}

    \item So far, our grammar tells us whether a string is syntactically correct or not. However, we want to convert our string to an AST. 
    \begin{verbatim}
        parse :: String -> AExpr \end{verbatim}
    An \textbf{attribute grammar} associates a value with each node in the parse tree such that 
    \begin{itemize}
        \item Each production is annotated with a rule.
        \item Each rule computes the value\footnote{The AST (i.e. the Haskell value of type \code{AExpr}.)} of a non-terminal from the values of its children.  
    \end{itemize}
    So, the attribute grammar for arithmetic expression is as follows: 
    \begin{verbatim}
        --      Format                    Value
        Aexpr : TNUM                    { AConst $1    }
              | ID                      { AVar   $1    }
              | '(' Aexpr ')'           { $2           }
              | Aexpr '*' Aexpr         { AMul   $1 $3 }
              | Aexpr '+' Aexpr         { APlus  $1 $3 } 
              | Aexpr '-' Aexpr         { AMinus $1 $3 }
    \end{verbatim}
    Like with the token rules, there is a format for each expression. 
    \begin{itemize}
        \item The left-hand side is a context-free grammar (just like what we saw in the example above). 
        \item The right-hand side is the attributes part. In particular, you have a \textbf{Haskell expression} surrounded by curly braces. This will then construct the value associated with each node. So, essentially, you can think of the right-hand side as a regular Haskell expression, with the dollar signs.
    \end{itemize}
    What do the dollar signs mean? Well,
    \begin{itemize}
        \item \code{\$1} refers to the \emph{value} of the first child. 
        \item \code{\$2} refers to the value of the second child. 
        \item And so on. 
    \end{itemize}
    Let's take this line (from the grammar) as an example: 
    \begin{verbatim}
        Aexpr '+' Aexpr         { APlus  $1 $3 } \end{verbatim}
    To construct the value, i.e. the AST, for the plus node, you should apply the constructor \code{APlus} to the value of the first child (\code{\$1}) and the value of the third child (\code{\$3}). Here, each child (\code{\$n}) is annotated below:
    \begin{verbatim}
        Aexpr '+' Aexpr         { APlus  $1 $3 } 
    --  $1    $2  $3\end{verbatim}
    So, there's no reason for us to use the second child, \code{\$2}, since it's just literally \code{'+'}. So, going back to the parse tree for \code{(x + 2)}, the AST would look like 
    So, a parse tree for the string \code{(x + 2)} will look like 
    \begin{verbatim}
                                    Aexpr           => APlus (AVar "x") (AConst 2)
                                    / | \ 
                                   /  |  \ 
                                  /   |   \ 
                                 /    |    \ 
                               '('    |    ')'
                                    Aexpr           => APlus (AVar "x") (AConst 2)
                                    / | \ 
                                   /  |  \ 
                                  /  '+'  \ 
                                 /         \
            AVar "x" <=       Aexpr      Aexpr      => AConst 2
                                |          |
                                |          |
                                |          |
                            'x' (ID)   '2' (TNUM)\end{verbatim}
    How do we compute the value of a terminal? In particular, 
    \begin{itemize}
        \item how do we map terminal \code{'x'} to string \code{"x"}?
        \item how do we map terminal \code{'2'} to integer \code{2}?
    \end{itemize}

    \item We have to describe what the values of the terminals are to \code{happy}. Recall that the terminals correspond to the tokens returned by the lexer. So, in the \code{.y} file, we have to declare which termianls in the rules correspond to which tokens from the \code{Token} datatype. 
    
    \begin{verbatim}
        -- Terminals:   Tokens from lexer:
            TNUM        { NUM _ $$ }
            ID          { ID _ $$  }
            '+'         { PLUS _   }
            '-'         { MINUS _  }
            '*'         { MUL _    }
            '/'         { DIV _    }
            '('         { LPAREN _ }
            ')'         { RPAREN _ }\end{verbatim}
    Each thing on the left is a terminal (which appears in the production rules). Each thing on the right is a Haskell pattern for the datatype \code{Token}. So, the \code{\$\$} designates one parameter of a token constructor as the \textbf{value} of that token. 
\end{itemize}

\begin{mdframed}[]
    (Quiz.) What is the value of the root \code{AExpr} node when parsing \code{1 + 2 + 3}?

    \begin{verbatim}
        Aexpr : TNUM                    { AConst $1    }
              | ID                      { AVar   $1    }
              | '(' Aexpr ')'           { $2           }
              | Aexpr '*' Aexpr         { AMul   $1 $3 }
              | Aexpr '+' Aexpr         { APlus  $1 $3 } 
              | Aexpr '-' Aexpr         { AMinus $1 $3 }\end{verbatim}

    \begin{enumerate}[(a)]
        \item Cannot be parsed as \code{AExpr}.
        \item \code{6}
        \item \code{APlus (APlus (AConst 1) (AConst 2)) (AConst 3)}
        \item \code{APlus (AConst 1) (APlus (AConst 2) (AConst 3))}
    \end{enumerate}

    \begin{mdframed}[]
        The answer is both \textbf{C} and \textbf{D}. If we draw parse trees, we would be able to get both answers.
    \end{mdframed}
\end{mdframed}



\end{document}