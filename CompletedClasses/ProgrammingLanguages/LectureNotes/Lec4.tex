\documentclass[letterpaper]{article}
\usepackage[margin=1in]{geometry}
\usepackage[utf8]{inputenc}
\usepackage{textcomp}
\usepackage{amssymb}
\usepackage{natbib}
\usepackage{graphicx}
\usepackage{gensymb}
\usepackage{amsthm, amsmath, mathtools}
\usepackage[dvipsnames]{xcolor}
\usepackage{enumerate}
\usepackage{mdframed}
\usepackage[most]{tcolorbox}
\usepackage{csquotes}
% https://tex.stackexchange.com/questions/13506/how-to-continue-the-framed-text-box-on-multiple-pages

\tcbuselibrary{theorems}

\newcommand{\R}{\mathbb{R}}
\newcommand{\Z}{\mathbb{Z}}
\newcommand{\N}{\mathbb{N}}
\newcommand{\Q}{\mathbb{Q}}
\newcommand{\C}{\mathbb{C}}
\newcommand{\code}[1]{\texttt{#1}}
\newcommand{\mdiamond}{$\diamondsuit$}
\newcommand{\PowerSet}{\mathcal{P}}
\newcommand{\Mod}[1]{\ (\mathrm{mod}\ #1)}
\DeclareMathOperator{\lcm}{lcm}

%\newtheorem*{theorem}{Theorem}
%\newtheorem*{definition}{Definition}
%\newtheorem*{corollary}{Corollary}
%\newtheorem*{lemma}{Lemma}
\newtheorem*{proposition}{Proposition}


\newtcbtheorem[number within=section]{theorem}{Theorem}
{colback=green!5,colframe=green!35!black,fonttitle=\bfseries}{th}

\newtcbtheorem[number within=section]{definition}{Definition}
{colback=blue!5,colframe=blue!35!black,fonttitle=\bfseries}{def}

\newtcbtheorem[number within=section]{corollary}{Corollary}
{colback=yellow!5,colframe=yellow!35!black,fonttitle=\bfseries}{cor}

\newtcbtheorem[number within=section]{lemma}{Lemma}
{colback=red!5,colframe=red!35!black,fonttitle=\bfseries}{lem}

\newtcbtheorem[number within=section]{example}{Example}
{colback=white!5,colframe=white!35!black,fonttitle=\bfseries}{def}

\newtcbtheorem[number within=section]{note}{Important Note}{
        enhanced,
        sharp corners,
        attach boxed title to top left={
            xshift=-1mm,
            yshift=-5mm,
            yshifttext=-1mm
        },
        top=1.5em,
        colback=white,
        colframe=black,
        fonttitle=\bfseries,
        boxed title style={
            sharp corners,
            size=small,
            colback=red!75!black,
            colframe=red!75!black,
        } 
    }{impnote}
\usepackage[utf8]{inputenc}
\usepackage[english]{babel}
\usepackage{fancyhdr}
\usepackage[hidelinks]{hyperref}

\pagestyle{fancy}
\fancyhf{}
\rhead{CSE 130}
\chead{Monday, April 04, 2022}
\lhead{Lecture 4}
\rfoot{\thepage}

\setlength{\parindent}{0pt}

\begin{document}

\section{The Lambda Calculus (Continued)}

\subsection{Normal Forms}
A \textbf{redux} is a $\alpha$-term of the form 
\begin{verbatim}
    (\x -> E1) E2
\end{verbatim}
An $\alpha$-term is in \textbf{normal form} if it contains no reduxes. In normal form, you cannot apply more $\beta$-steps.

\subsection{Semantics: Evaluation}
A $\lambda$-term \code{E} evalutes to \code{E'} if 
\begin{enumerate}
    \item There is a sequence of steps 
    \begin{verbatim}
        E =?> E1 =?> ... =?> EN =?> E'
    \end{verbatim}
    where each \code{=?>} is either \code{=a>} or \code{=b>} and $N \geq 0$.

    \item \code{E'} is in normal form. 
\end{enumerate}

As an example, consider the following evaluation: 
\begin{verbatim}
    eval test :
        (\x -> x x) (\x -> x)
        =a> (\x -> x x) (\z -> z)
        =b> (\z -> z) (\z -> z)
        =b> (\z -> z)
\end{verbatim}

\subsection{Non-Terminating Evaluation}
Consider the following program: 
\begin{verbatim}
    (\x -> x x) (\x -> x x)
    =a> (\x -> x x) (\y -> y y)
    =b> (\y -> y y) (\y -> y y)
    =a> (\x -> x x) (\x -> x x)
\end{verbatim}
This program can actually loop back to itself and thus reduce to a normal form. This combinator is called $\Omega$. 

\end{document}