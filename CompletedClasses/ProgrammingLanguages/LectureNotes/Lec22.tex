\documentclass[letterpaper]{article}
\usepackage[margin=1in]{geometry}
\usepackage[utf8]{inputenc}
\usepackage{textcomp}
\usepackage{amssymb}
\usepackage{natbib}
\usepackage{graphicx}
\usepackage{gensymb}
\usepackage{amsthm, amsmath, mathtools}
\usepackage[dvipsnames]{xcolor}
\usepackage{enumerate}
\usepackage{mdframed}
\usepackage[most]{tcolorbox}
\usepackage{csquotes}
% https://tex.stackexchange.com/questions/13506/how-to-continue-the-framed-text-box-on-multiple-pages

\tcbuselibrary{theorems}

\newcommand{\R}{\mathbb{R}}
\newcommand{\Z}{\mathbb{Z}}
\newcommand{\N}{\mathbb{N}}
\newcommand{\Q}{\mathbb{Q}}
\newcommand{\C}{\mathbb{C}}
\newcommand{\code}[1]{\texttt{#1}}
\newcommand{\mdiamond}{$\diamondsuit$}
\newcommand{\PowerSet}{\mathcal{P}}
\newcommand{\Mod}[1]{\ (\mathrm{mod}\ #1)}
\DeclareMathOperator{\lcm}{lcm}

%\newtheorem*{theorem}{Theorem}
%\newtheorem*{definition}{Definition}
%\newtheorem*{corollary}{Corollary}
%\newtheorem*{lemma}{Lemma}
\newtheorem*{proposition}{Proposition}


\newtcbtheorem[number within=section]{theorem}{Theorem}
{colback=green!5,colframe=green!35!black,fonttitle=\bfseries}{th}

\newtcbtheorem[number within=section]{definition}{Definition}
{colback=blue!5,colframe=blue!35!black,fonttitle=\bfseries}{def}

\newtcbtheorem[number within=section]{corollary}{Corollary}
{colback=yellow!5,colframe=yellow!35!black,fonttitle=\bfseries}{cor}

\newtcbtheorem[number within=section]{lemma}{Lemma}
{colback=red!5,colframe=red!35!black,fonttitle=\bfseries}{lem}

\newtcbtheorem[number within=section]{example}{Example}
{colback=white!5,colframe=white!35!black,fonttitle=\bfseries}{def}

\newtcbtheorem[number within=section]{note}{Important Note}{
        enhanced,
        sharp corners,
        attach boxed title to top left={
            xshift=-1mm,
            yshift=-5mm,
            yshifttext=-1mm
        },
        top=1.5em,
        colback=white,
        colframe=black,
        fonttitle=\bfseries,
        boxed title style={
            sharp corners,
            size=small,
            colback=red!75!black,
            colframe=red!75!black,
        } 
    }{impnote}
\usepackage[utf8]{inputenc}
\usepackage[english]{babel}
\usepackage{fancyhdr}
\usepackage[hidelinks]{hyperref}

\pagestyle{fancy}
\fancyhf{}
\rhead{CSE 130}
\chead{Wednesday, May 18, 2022}
\lhead{Lecture 22}
\rfoot{\thepage}

\setlength{\parindent}{0pt}

\begin{document}
\section{Type Classes}
Consider the following operator: \code{+}
\begin{itemize}
    \item For \code{Integer}s, we have 
    \begin{verbatim}
        $ 2 + 3
        5\end{verbatim}
    \item For \code{Double}s, we have 
    \begin{verbatim}
        $ 2.9 + 3.5
        6.4 \end{verbatim}
\end{itemize}
But, we can also do things like 
\begin{verbatim}
    $ [2.9, 3.5] == [2.9, 3.5]
    True

    $ ("cat", 10) < ("cat", 20)
    True \end{verbatim}
How does this work? 

\begin{mdframed}[]
    (Quiz.) Which of the following type annotations would work for \code{(+)}?

    \begin{enumerate}[(a)]
        \item \code{(+) :: Int -> Int -> Int}
        \item \code{(+) :: Double -> Double -> Double}
        \item \code{(+) :: a -> a -> a}
        \item Any of the above. 
        \item None of the above.
    \end{enumerate}

    \begin{mdframed}[]
        The answer is \textbf{E}. If we picked A, then we can't add two \code{Double}s; the same idea applies for B. For C, we can't add, for example, two \code{Bool}s.
    \end{mdframed}
\end{mdframed}
We need \textbf{ad-hoc polymorphism}. To do this, we make use of \emph{type classes}.




\subsection{Constrained Types}
We note that 
\begin{verbatim}
    $ :type (+)
    (+) :: (Num a) => a -> a -> a\end{verbatim}
Here, this is saying that \code{(+)} takes in two \code{a} values and returns an \code{a} value, such that \code{a} is an \emph{instance of} the \code{Num} type class\footnote{In terms of Java, we can think of \code{Num} as an interface, so \code{a} would have to implement the \code{Num} interface.}. Then, \code{(Num a) =>} is the \emph{constraint}.

\bigskip 

Let's try to add two \code{Bool} values.
\begin{verbatim}
    $ True + False 
    <interactive>
        No instance for (Num Bool) arising from a use of '+'
        In the expression: True + False
        In an equation for 'it': it = True + False\end{verbatim}
This means that \code{True} and \code{False} are of type \code{Bool}, but that \code{Bool} is not an instance of \code{Num}.

\begin{mdframed}[]
    (Quiz.) What would be a reasonable type for the equality operator? 
    \begin{enumerate}[(a)]
        \item \code{(==) :: a -> a -> a}
        \item \code{(==) :: a -> a -> Bool}
        \item \code{(==) :: (Eq a) => a -> a -> a}
        \item \code{(==) :: (Eq a) => a -> a -> Bool}
        \item None of the above 
    \end{enumerate}

    \begin{mdframed}[]
        The answer is \textbf{D}. Note that one example of something that can't really be compared are \emph{functions}.
    \end{mdframed}
\end{mdframed}

\subsection{What is a Type Class?}
A type class is a \emph{collection of methods} (functions, operators) that must exist for every instance. Some useful type classes in the Haskell standard library are
\begin{itemize}
    \item The \code{Eq} Type Class for \textbf{Equality}.
    \begin{verbatim}
        class Eq a where 
            (==) :: a -> a -> Bool 
            (/=) :: a -> a -> Bool \end{verbatim}
    Note that a type \code{T} is an instance of \code{Eq} if there are two functions 
    \begin{itemize}
        \item \code{(==) :: T -> T -> Bool} that determines if two \code{T} values are equal. 
        \item \code{(/=) :: T -> T -> Bool} that determines if two \code{T} values are not equal. 
    \end{itemize}

    \item The \code{Show} Type Class 
    \begin{verbatim}
        class Show a where 
            show :: a -> String \end{verbatim}
    This type class requires that instances are convertible to \code{String} so that it can be displayed. To see what we mean, note that 
    \begin{verbatim}
    $ 2
    2

    $ show 2
    "2"

    $ show 3.14 
    "3.14"

    $ show (1, "two", ([], [], []))
    "(1,\"two\",([],[],[]))"\end{verbatim}

    \item The \code{Ord} Type Class for \textbf{Order}.
    \begin{verbatim}
        class Eq a => Ord a where 
            (<)     :: a -> a -> Bool
            (<=)    :: a -> a -> Bool
            (>)     :: a -> a -> Bool
            (>=)    :: a -> a -> Bool\end{verbatim}
    Note the \code{Eq a =>}. A type \code{T} is an instance of \code{Ord} if \code{T} is \emph{also} instance of \code{Eq}, and it defines functions for comparing values for inequalities. 

    \bigskip 

    In other words, if \code{T} implements \code{Ord}, then it must also implement \code{Eq} (i.e., \code{Ord} depends on \code{Eq}).
\end{itemize}


\subsection{Creating Type Classes}
Consider the datatype 
\begin{verbatim}
    data Color = Red | Green \end{verbatim}
Let us now add a declaration for \code{Show} on \code{Color}:
\begin{verbatim}
    instance Show Color where 
        show Red        = "Red"
        show Green      = "Green"\end{verbatim}

Let's do the same thing for \code{Eq}:
\begin{verbatim}
    instance Eq Color where 
        (==) Red    Red     = True 
        (==) Green  Green   = True 
        (==) _      _       = False
        (/=) x      y       = not (x == y)\end{verbatim}
This is tedious, and this type isn't very complicated. Indeed, there is a way for us to \emph{automatically} do this, using the \code{deriving} keyword. 
\begin{verbatim}
    data Color = Red | Green 
        deriving (Eq, Show)\end{verbatim}

\end{document}