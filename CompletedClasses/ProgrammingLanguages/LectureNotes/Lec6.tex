\documentclass[letterpaper]{article}
\usepackage[margin=1in]{geometry}
\usepackage[utf8]{inputenc}
\usepackage{textcomp}
\usepackage{amssymb}
\usepackage{natbib}
\usepackage{graphicx}
\usepackage{gensymb}
\usepackage{amsthm, amsmath, mathtools}
\usepackage[dvipsnames]{xcolor}
\usepackage{enumerate}
\usepackage{mdframed}
\usepackage[most]{tcolorbox}
\usepackage{csquotes}
% https://tex.stackexchange.com/questions/13506/how-to-continue-the-framed-text-box-on-multiple-pages

\tcbuselibrary{theorems}

\newcommand{\R}{\mathbb{R}}
\newcommand{\Z}{\mathbb{Z}}
\newcommand{\N}{\mathbb{N}}
\newcommand{\Q}{\mathbb{Q}}
\newcommand{\C}{\mathbb{C}}
\newcommand{\code}[1]{\texttt{#1}}
\newcommand{\mdiamond}{$\diamondsuit$}
\newcommand{\PowerSet}{\mathcal{P}}
\newcommand{\Mod}[1]{\ (\mathrm{mod}\ #1)}
\DeclareMathOperator{\lcm}{lcm}

%\newtheorem*{theorem}{Theorem}
%\newtheorem*{definition}{Definition}
%\newtheorem*{corollary}{Corollary}
%\newtheorem*{lemma}{Lemma}
\newtheorem*{proposition}{Proposition}


\newtcbtheorem[number within=section]{theorem}{Theorem}
{colback=green!5,colframe=green!35!black,fonttitle=\bfseries}{th}

\newtcbtheorem[number within=section]{definition}{Definition}
{colback=blue!5,colframe=blue!35!black,fonttitle=\bfseries}{def}

\newtcbtheorem[number within=section]{corollary}{Corollary}
{colback=yellow!5,colframe=yellow!35!black,fonttitle=\bfseries}{cor}

\newtcbtheorem[number within=section]{lemma}{Lemma}
{colback=red!5,colframe=red!35!black,fonttitle=\bfseries}{lem}

\newtcbtheorem[number within=section]{example}{Example}
{colback=white!5,colframe=white!35!black,fonttitle=\bfseries}{def}

\newtcbtheorem[number within=section]{note}{Important Note}{
        enhanced,
        sharp corners,
        attach boxed title to top left={
            xshift=-1mm,
            yshift=-5mm,
            yshifttext=-1mm
        },
        top=1.5em,
        colback=white,
        colframe=black,
        fonttitle=\bfseries,
        boxed title style={
            sharp corners,
            size=small,
            colback=red!75!black,
            colframe=red!75!black,
        } 
    }{impnote}
\usepackage[utf8]{inputenc}
\usepackage[english]{babel}
\usepackage{fancyhdr}
\usepackage[hidelinks]{hyperref}

\pagestyle{fancy}
\fancyhf{}
\rhead{CSE 130}
\chead{Friday, April 08, 2022}
\lhead{Lecture 6}
\rfoot{\thepage}

\setlength{\parindent}{0pt}

\begin{document}
\section{Lambda Calculus}
\subsection{Programming in Lambda Calculus}
\subsubsection{Recursion}
Suppose we want to implement a function in Lambda Calculus that sums up the natural numbers up to $n$, e.g. $0 + 1 + 2 + \dots + n$.

\bigskip 

In Python, we might do something like: 
\begin{verbatim}
    def sum(n):
        if n == 0:
            return 0
        else 
            return n + sum(n - 1)
\end{verbatim}
In Lambda Calculus, we don't have named functions. The idea, then, is: 
\begin{itemize}
    \item We want to call a function on \code{DEC n}.
    \item And it needs to be the same function.
\end{itemize}
First, we need to pass in the function to call ``recursively'':
\begin{verbatim}
    let STEP = 
        \rec -> \n -> ITE (ISZ n)
                        ZERO 
                        (ADD n (rec (DEC n))) -- Call some rec 
\end{verbatim}
We now need to do something clever to \code{STEP} so that the function is passed as \code{rec} itself. To do this, we want a \textbf{fixpoint combinator} \code{FIX} such that \code{FIX STEP} calls \code{STEP} with itself as the first argument; that is: 
\begin{verbatim}
    FIX STEP 
    =*> STEP (FIX STEP)
\end{verbatim}
Once we have this, we can define 
\begin{verbatim}
    let SUM = FIX STEP 
\end{verbatim}
Then, by the property of \code{FIX}, we have 
\begin{verbatim}
    SUM *> STEP SUM -- (1)
\end{verbatim}
To see this: 
\begin{verbatim}
    eval sum_one: 
        SUM ONE 
        =*> STEP SUM ONE 
        =~> ONE 
\end{verbatim}

\end{document}