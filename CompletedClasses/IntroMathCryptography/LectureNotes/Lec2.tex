\documentclass[letterpaper]{article}
\usepackage[margin=1in]{geometry}
\usepackage[utf8]{inputenc}
\usepackage{textcomp}
\usepackage{amssymb}
\usepackage{natbib}
\usepackage{graphicx}
\usepackage{gensymb}
\usepackage{amsthm, amsmath, mathtools}
\usepackage[dvipsnames]{xcolor}
\usepackage{enumerate}
\usepackage{mdframed}
\usepackage[most]{tcolorbox}
\usepackage{csquotes}
% https://tex.stackexchange.com/questions/13506/how-to-continue-the-framed-text-box-on-multiple-pages

\tcbuselibrary{theorems}

\newcommand{\R}{\mathbb{R}}
\newcommand{\Z}{\mathbb{Z}}
\newcommand{\N}{\mathbb{N}}
\newcommand{\Q}{\mathbb{Q}}
\newcommand{\C}{\mathbb{C}}
\newcommand{\code}[1]{\texttt{#1}}
\newcommand{\mdiamond}{$\diamondsuit$}
\newcommand{\PowerSet}{\mathcal{P}}
\newcommand{\Mod}[1]{\ (\mathrm{mod}\ #1)}
\DeclareMathOperator{\lcm}{lcm}

%\newtheorem*{theorem}{Theorem}
%\newtheorem*{definition}{Definition}
%\newtheorem*{corollary}{Corollary}
%\newtheorem*{lemma}{Lemma}
\newtheorem*{proposition}{Proposition}


\newtcbtheorem[number within=section]{theorem}{Theorem}
{colback=green!5,colframe=green!35!black,fonttitle=\bfseries}{th}

\newtcbtheorem[number within=section]{definition}{Definition}
{colback=blue!5,colframe=blue!35!black,fonttitle=\bfseries}{def}

\newtcbtheorem[number within=section]{corollary}{Corollary}
{colback=yellow!5,colframe=yellow!35!black,fonttitle=\bfseries}{cor}

\newtcbtheorem[number within=section]{lemma}{Lemma}
{colback=red!5,colframe=red!35!black,fonttitle=\bfseries}{lem}

\newtcbtheorem[number within=section]{example}{Example}
{colback=white!5,colframe=white!35!black,fonttitle=\bfseries}{def}

\newtcbtheorem[number within=section]{note}{Important Note}{
        enhanced,
        sharp corners,
        attach boxed title to top left={
            xshift=-1mm,
            yshift=-5mm,
            yshifttext=-1mm
        },
        top=1.5em,
        colback=white,
        colframe=black,
        fonttitle=\bfseries,
        boxed title style={
            sharp corners,
            size=small,
            colback=red!75!black,
            colframe=red!75!black,
        } 
    }{impnote}
\usepackage[utf8]{inputenc}
\usepackage[english]{babel}
\usepackage{fancyhdr}
\usepackage{hyperref}

\pagestyle{fancy}
\fancyhf{}
\rhead{Math 187A}
\chead{Wednesday, January 11, 2023}
\lhead{Lecture 2}
\rfoot{\thepage}

\setlength{\parindent}{0pt}

\begin{document}

\section{Classical Cryptosystems}
(Continued from Lecture 1.)
\subsection{Interlude: Modular Arithmetic}
One fundamental idea in number theory, which is used in cryptography, is modular arithmetic. 

\subsubsection{Quotients and Remainders}

\begin{lemma}{Euclid's Division}{}
    For any integer $a$ and positive integer $n$, there exists a unique pair of integers $q$ and $r$ such that $0 \leq r < n$ and $a = qn + r$. The integers $q$ and $r$ are called the quotient and remainder, respectively. We also write $a \Mod{n}$ to refer to the remainder.
\end{lemma}
For the proof, the deal is that we can keep subtracting, or adding, $n$ from $a$ until we end up in the range $[0, n)$. Therefore, the number of times we had to subtract, or add, $n$ is the \emph{quotient}, and the number in the range $[0, n)$ that we end up with at the end is the \emph{remainder}. 

\begin{mdframed}
    (Example.) Divide $a = 17$ by $n = 5$. Find the quotient and remainder.
    
    \bigskip 

    Using the proof idea, we note that: 
    \begin{itemize}
        \item Subtracting 5 to $a$ once gives us 12. 
        \item Subtracting 5 to $a$ twice gives us 7.
        \item Subtracting 5 to $a$ thrice gives us 2.
    \end{itemize}
    It took us 3 subtractions to get to a number that's in the range $[0, 5)$, so the quotient is $\boxed{3}$ and the remainder is $\boxed{2}$. 
\end{mdframed}
We should note that this is pretty standard when $a \geq 0$. However, for $a < 0$, it might be less familiar, albeit the same process.

\begin{mdframed}
    (Example.) Divide $a = -7$ by $n = 5$. Find the quotient and remainder.

    \bigskip 

    Using the proof idea, we note that: 
    \begin{itemize}
        \item Adding 5 to $a$ once gives us 2.
        \item Adding 5 to $a$ twice gives us 3. 
    \end{itemize}
    It took us 2 additions to get to a number that's in the range $[0, 5)$, so the quotient is $\boxed{-2}$ (because we had to \emph{add}, not subtract) and the remainder is $\boxed{3}$. 
\end{mdframed}

\textbf{Remark:} 
\begin{itemize}
    \item If we have to \textbf{add} $n$ to $a$ $x$ times to get a number that's in the range $[0, n)$, then our final quotient will be negative (that is, $-x$).
    \item If we have to \textbf{subtract} $n$ from $a$ $x$ times to get a number that's in the range $[0, n)$, then our final quotient will be positive (that is, $x$).
\end{itemize}

\begin{mdframed}
    (Exercise.) For each of the following, calculate the quotient and remainder when $a$ is divided by $n$. Do these calculations by hand. 
    \begin{itemize}
        \item $a = 13$, $n = 3$.
        \begin{mdframed}
            We know that $13 / 3 = 4$, and $13 - (3 \cdot 4) = 1 \in [0, 3)$. So, the quotient is \boxed{4} and the remainder is \boxed{1}. 
        \end{mdframed}
        \item $a = 134$, $n = 10$.
        \begin{mdframed}
            We know that $134 / 10 = 13$ and $134 - (10 \cdot 13) = 4 \in [0, 10)$. So, the quotient is \boxed{13} and remainder is \boxed{4}.
        \end{mdframed}
        \item $a = -37$, $n = 10$.
        \begin{mdframed}
            We know that we need to add $n$ to $a$ \textbf{4} times to get a number, $3$, that is in the range $[0, 10)$. To be precise, 
            \[-37 + 10 + 10 + 10 + 10 = -37 + 40 = 3 \in [0, 10).\]
            Therefore, the quotient is \boxed{-4} and the remainder is \boxed{3}. 
        \end{mdframed}
        \item $a = -15$, $n = 60$.
        \begin{mdframed}
            We have to add $n$ to $a$ \textbf{1} time to get $45 \in [0, 60)$. Therefore, the quotient is \boxed{-1} and the remainder is \boxed{45}.
        \end{mdframed}
        \item $a = 13$, $n = 12$.
        \begin{mdframed}
            We know that $13 / 12 = 1$ and $13 - (12 \cdot 1) = 1$. So, the quotient is \boxed{1} and the remainder is \boxed{1}.
        \end{mdframed}
    \end{itemize}
\end{mdframed}

\begin{mdframed}
    (Exercise.) What is $-13 \Mod{5}$?
    \begin{mdframed}
        \[-13 + 5 + 5 + 5 = 2 \in [0, 5),\]
        so the quotient is $-3$ (since we had to perform 3 additions) and the remainder is \boxed{2}. Therefore, 
        \[-13 \Mod{5} = 2.\]
    \end{mdframed}
\end{mdframed}

\begin{proposition}
    Suppose $a$ and $n$ are integers and $n > 0$. All the following statements are equivalent:
    \begin{itemize}
        \item $a \Mod{n} = 0$.
        \item There is no remainder when $a$ is divided by $n$. 
        \item $a$ is a multiple of $n$. 
        \item $a$ is divisible by $n$. 
        \item $n$ is a divisor of $a$. 
        \item $n$ is a factor of $a$. 
        \item $n$ divides $a$ (in notation\footnote{Note that $\vert$ is read as ``\emph{divides}.''}: $n \vert a$).
        \item $a / n$ is an integer.
    \end{itemize}
\end{proposition}

\subsubsection{Congruences}
\begin{definition}{Congruence}{}
    Fix a positive integer $n$. If $a$ and $b$ are integers, we say that ``$a$ is \textbf{congruent} to $b$ mod $n$,'' or that ``$a$ and $b$ are congruent mod $n$,'' if $a$ and $b$ have the same remainder when each is divided by $n$. This can be denoted in symbols as follows: 
    \[a \equiv b \Mod{n}.\] 
\end{definition}
For example, $19 \equiv 7 \Mod{4}$ since $19$ and $7$ both have remainder 3 when divided by 4. Observe also that $19 - 7 = 12$ is a multiple of 4. This can be generalized: 

\begin{lemma}{}{}
    Fix a positive integer $n$. Two integers $a$ and $b$ are congruent mod $n$ if and only if $a - b$ is a multiple of $n$.
\end{lemma}

\begin{proof}
    Divide $a$ and $b$ by $n$ to write $a = q_1 n + r_1$ and $b = q_2 n + r_2$. If \[a \equiv b \Mod{n},\] this by definition means that $r_1 = r_2$ so 
    \[a - b = (q_1 n + r_1) - (q_2 n + r_2) = q_1 n - q_2 n = n(q_1 - q_2).\]
    So, $a - b$ is a multiple of $n$. Conversely, suppose $a - b$ is a multiple of $n$. Then, 
    \[(a - b) - (q_1 - q_2)n = ((q_1 n + r_1) - (q_2 n + r_2)) - (q_1 - q_2)n = r_1 - r_2\]
    is a multiple of $n$. Since $0 \leq r_1, r_2 < n$, however, we must have $|r_1 - r_2| < n$. The only way that $r_1 - r_2$ can be a multiple of $n$ is if $r_1 - r_2 = 0$, i.e., if $r_1 = r_2$. That means $a \equiv b \Mod{n}$. 
\end{proof}

\begin{theorem}{Modular Arithmetic Theorem}{}
    Fix a positive integer $n$. Suppose $a$, $a'$, $b$, $b'$ are integers such that 
    \[a \equiv a' \Mod{n}\]
    \[b \equiv b' \Mod{n}\]
    and $k$ is any positive integer. Then, all of the following are also true: 
    \[a + b \equiv a' + b' \Mod{n}\]
    \[a - b \equiv a' - b' \Mod{n}\]
    \[ab \equiv a'b' \Mod{n}\]
    \[a^k \equiv (a')^k \Mod{n}\]
\end{theorem}

\begin{mdframed}
    (Exercise.) Use the Modular Arithmetic Theorem to quickly calculate the following. 
    \begin{itemize}
        \item $417 \cdot 22 \Mod{10}$.
        \begin{mdframed}
            \begin{equation*}
                \begin{aligned}
                    417 \cdot 22 &\equiv 7 \cdot 2  \\ 
                        &= 14 \\ 
                        &\equiv 4 \Mod{10}.
                \end{aligned}
            \end{equation*}
        \end{mdframed}
        \item $333333 + 666 \Mod{3}$.
        \begin{mdframed}
            \begin{equation*}
                \begin{aligned}
                    333333 + 666 &\equiv 0 + 0 \\ 
                        &\equiv 0 \Mod{3}.
                \end{aligned}
            \end{equation*}
        \end{mdframed}
        \item $7^{202320232023} \Mod{6}$. 
        \begin{mdframed}
            \begin{equation*}
                \begin{aligned}
                    7^{202320232023} &= 7 \cdot 7 \cdot \hdots \cdot 7 \\ 
                        &\equiv 1 \cdot 1 \cdot \hdots \cdot 1 \\ 
                        &= 1 \Mod{6}.
                \end{aligned}
            \end{equation*}
        \end{mdframed}
        \item What is $5^{2023202320232023} \Mod{6}$?
        \begin{mdframed}
            \begin{equation*}
                \begin{aligned}
                    5^{2023202320232023} &= 5 \cdot 5 \cdot \hdots \cdot 5 \\ 
                        &\equiv (-1) \cdot (-1) \cdot \hdots \cdot (-1) \\ 
                        &= (-1)^{2023202320232023} \\ 
                        &\equiv -1 \\ 
                        &\equiv 5 \Mod{6}.
                \end{aligned}
            \end{equation*}
            Therefore, the answer is \boxed{5}.
        \end{mdframed}
    \end{itemize}
\end{mdframed}

\begin{mdframed}
    (Exercise.) Fix positive integers $k$ and $n$. Suppose $a$ and $a'$ are integers such that $a \equiv a' \Mod{n}$. It is not true in general that $k^a \equiv k^{a'} \Mod{n}$. Show this by example: in other words, find $k$, $n$, $a$, and $a'$ such that $a \equiv a' \Mod{n}$ but $k^a \not\equiv k^{a'} \Mod{n}$. 

    \begin{mdframed}
        Let $k = 2$, $n = 5$, $a = 6$, and $a' = 1$ so that 
        \[6 \equiv 1 \Mod{5}.\]
        Then, we note that 
        \[k^a = 2^6 = 64\]
        and 
        \[k^{a'} = 2^1 = 2.\]
        From this, it's clear that 
        \[64 \not\equiv 2 \Mod{5}.\]
    \end{mdframed}
\end{mdframed}

\begin{mdframed}[nobreak=true]
    (Exercise.) \emph{Suppose that the number $273x49y5$, where $x$ and $y$ are unknown digits, is divisible by 495. Find $x$ and $y$.}
    \begin{mdframed}
        We are asked to solve 
        \[273x49y5 \equiv 0 \mod{495}.\]

        We can write $273x49y5$ as 
        \[20000000 + 7000000 + 300000 + 10000x + 4000 + 900 + 10000y + 5.\]
        With this in mind, we have
        \begin{equation*}
            \begin{aligned}
                20000000 &+ 7000000 + 300000 + 10000x + 4000 + 900 + 10y + 5 \\ 
                    &\equiv 20 + 205 + 30 + 100x + 40 + 405 + 10y + 5 \\ 
                    &= 705 + 100x + 10y \\ 
                    &\equiv 210 + 100x + 10y \mod{495}.
            \end{aligned}
        \end{equation*}
        We note that the next multiple of 495 is 990. So, effectively, we want to find some $x$ and $y$ such that $0 \leq x < 10$ and $0 \leq y < 10$ and
        \[210 + 100x + 10y = 990.\]
        This gives us 
        \[100x + 10y = 780.\]
        One obvious solution is $x = 7$ and $y = 8$. 
    \end{mdframed}
\end{mdframed}


\subsubsection{Revisiting the Caesar Cipher}
Suppose we identify the letters $A$ through $Z$ with the numbers 0 through 25. In other words, we have $A \mapsto 0$, $B \mapsto 1$, and so on. Suppose we want to apply the Caesar cipher with a shift of 5 to encrypt the letter $Y$. Consider the following 
\[E(x) = (x + 5) \Mod{26}.\]
We note that $Y$ corresponds to the number 24. Then, it follows that 
\[E(24) = (24 + 5) \Mod{26} = 29 \Mod{26} = 3.\]
The number 3 corresponds to the letter $D$, the desired result. In other words, if we can identify the letters with numbers, the function $E$ is the encryption function of the Caesar cipher with a shift of 5.  

\bigskip 

The decryption function is given by 
\[D(y) = (y - 5) \Mod{26}.\]
So, if we wanted to decrypt the letter $D$, which corresponds to the number 3, then 
\[D(3) = (3 - 5) \Mod{26} = -2 \Mod{26} = 24,\]
which corresponds to $Y$. 

\bigskip 

What we just did is actually a consequence of the Modular Arithmetic Theorem; for a quick little ``proof,'' notice how  
\begin{equation*}
    \begin{aligned}
        D(E(x)) &= D(y) \\
            &\equiv (y - 5) \Mod{26} \\ 
            &\equiv ((x + 5) - 5) \Mod{26} \\ 
            &= x.
    \end{aligned} 
\end{equation*}

\begin{mdframed}
    (Exercise.) Decipher the message below, which was encrypted using a Caesar cipher with a shift of 3 and then using a rectangular transposition with the keyword \code{EARLY}.
    \begin{verbatim}
        DKSSBUIGLDEBXOX\end{verbatim}

    \begin{mdframed}
        To decrypt this message, we need to work backwards: first, use rectangular transposition to undo the first encryption, and then Caesar cipher to undo the second encryption.
        \begin{enumerate}
            \item For the rectangular transposition, note that the keyword has alphabetical ranking \code{21435}, so using the streamlined way discussed earlier, we have 
            \begin{mdframed}
                \begin{verbatim}
12345    21435
DKSSB -> KDSSB
UIGLD -> IULGD
EBXOX -> BEOXX\end{verbatim}
            \end{mdframed}
            Unstacking gives us \code{KDSSBIULGDBEOXX}.

            \item Next, we need to undo the Caesar cipher encryption on the message that we found from the previous step. Since the encryption used a positive shift of 3, undoing it requires us to use a negative shift of 3. This gives us: 
            \begin{mdframed}
                \begin{verbatim}
encrypted   KDSSBIULGDBEOXX
decrypted   HAPPYFRIDAY....\end{verbatim}
            \end{mdframed}
            Note that the last four letters were omitted. In any case, this gives us the decoded message \code{Happy Friday}.
        \end{enumerate}
    \end{mdframed}
\end{mdframed}
\textbf{Remark:} You should not assume that these operations are commutative. That is, if we were to decrypt the message by applying the Caesar cipher first and then the rectangular transposition, as opposed to the reverse order, we may get a different answer!


\end{document}