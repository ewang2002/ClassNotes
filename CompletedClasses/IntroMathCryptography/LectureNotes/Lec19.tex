\documentclass[letterpaper]{article}
\usepackage[margin=1in]{geometry}
\usepackage[utf8]{inputenc}
\usepackage{textcomp}
\usepackage{amssymb}
\usepackage{natbib}
\usepackage{graphicx}
\usepackage{gensymb}
\usepackage{amsthm, amsmath, mathtools}
\usepackage[dvipsnames]{xcolor}
\usepackage{enumerate}
\usepackage{mdframed}
\usepackage[most]{tcolorbox}
\usepackage{csquotes}
% https://tex.stackexchange.com/questions/13506/how-to-continue-the-framed-text-box-on-multiple-pages

\tcbuselibrary{theorems}

\newcommand{\R}{\mathbb{R}}
\newcommand{\Z}{\mathbb{Z}}
\newcommand{\N}{\mathbb{N}}
\newcommand{\Q}{\mathbb{Q}}
\newcommand{\C}{\mathbb{C}}
\newcommand{\code}[1]{\texttt{#1}}
\newcommand{\mdiamond}{$\diamondsuit$}
\newcommand{\PowerSet}{\mathcal{P}}
\newcommand{\Mod}[1]{\ (\mathrm{mod}\ #1)}
\DeclareMathOperator{\lcm}{lcm}

%\newtheorem*{theorem}{Theorem}
%\newtheorem*{definition}{Definition}
%\newtheorem*{corollary}{Corollary}
%\newtheorem*{lemma}{Lemma}
\newtheorem*{proposition}{Proposition}


\newtcbtheorem[number within=section]{theorem}{Theorem}
{colback=green!5,colframe=green!35!black,fonttitle=\bfseries}{th}

\newtcbtheorem[number within=section]{definition}{Definition}
{colback=blue!5,colframe=blue!35!black,fonttitle=\bfseries}{def}

\newtcbtheorem[number within=section]{corollary}{Corollary}
{colback=yellow!5,colframe=yellow!35!black,fonttitle=\bfseries}{cor}

\newtcbtheorem[number within=section]{lemma}{Lemma}
{colback=red!5,colframe=red!35!black,fonttitle=\bfseries}{lem}

\newtcbtheorem[number within=section]{example}{Example}
{colback=white!5,colframe=white!35!black,fonttitle=\bfseries}{def}

\newtcbtheorem[number within=section]{note}{Important Note}{
        enhanced,
        sharp corners,
        attach boxed title to top left={
            xshift=-1mm,
            yshift=-5mm,
            yshifttext=-1mm
        },
        top=1.5em,
        colback=white,
        colframe=black,
        fonttitle=\bfseries,
        boxed title style={
            sharp corners,
            size=small,
            colback=red!75!black,
            colframe=red!75!black,
        } 
    }{impnote}
\usepackage[utf8]{inputenc}
\usepackage[english]{babel}
\usepackage{fancyhdr}
\usepackage{hyperref}
\usepackage{csquotes}
\usepackage{polynom}

\DeclareMathOperator{\ord}{ord}

\pagestyle{fancy}
\fancyhf{}
\rhead{Math 187A}
\chead{Wednesday, March 15, 2023}
\lhead{Lecture 18}
\rfoot{\thepage}

\setlength{\parindent}{0pt}

\begin{document}

\section{Modern Cryptography}
(Continued from previous notes.)

\subsection{Elliptic Curve Diffie-Hellman}
Suppose Alice and Bob publicly agree to fix a prime $p$, an elliptic curve $E$ mod $p$ (specified by integers $a, b$ such that the Weierstrass equation $y^2 = x^3 + ax + b$ is nonsingular mod $p$), and a point $P \in E$. To ensure security, we need for $\ord_{P}(E)$ to be large. The data $(p, E, P)$ is all shared publicly.

\bigskip 

Alice can choose a secret integer $0 \leq k_a < \ord_{E}(P)$ and send Bob $Q_a = k_a P$. She can compute this value quickly using binary multiplication. Similarly, Bob can choose a secret integer $0 \leq k_b < \ord_{E}(P)$ and send Alice $Q_b = k_b P$. Alice computes $R = k_a Q_b$ and Bob computes $R = k_b Q_a \Mod{p}$. The two values of $R$ that Alice and Bob compute are the same since \[k_a Q_b = k_a (k_b P) = k_a k_b P = k_b (k_a P) = k_b Q_a.\]
Thus, Alice and Bob now share a secret point $R$ on the elliptic curve. 

\begin{mdframed}
    (Exercise.) Suppose Alice and Bob publicly agree to use the elliptic curve $y^2 = x^3 + 17 \Mod{p} = 7$ and the point $P = (1, 2)$. 
    \begin{enumerate}[(a)]
        \item Suppose Alice picks the number $k_a = 4$. What is the message $Q_a$ that she sends Bob? 
        \begin{mdframed}
            We know that \[Q_a = k_a P = 4P.\] Given this, we need to compute $4P$. Let's begin with $2P = P + P$. We know that $P = P$ and $y_1 = 2$ is invertible mod 7, so we define 
            \[\lambda = (3(1)^2 + 0)(2(2))^{-1} \Mod{7} = (3)((4)^{-1} \Mod{7}).\]
            Computing the inverse of 4 mod 7 gives us 2, so 
            \[\lambda = 3(2) \Mod{7} = 6 \Mod{7}.\]
            Then, we have 
            \[\nu = 2 - 6(1) = -4 \Mod{7} = 3\]
            \[x_3 = 6^2 - 1 - 1 = 34 \Mod{7} = 6\]
            \[y_3 = 6(6) + 3 = 39 \Mod{7} = 4.\]
            Therefore, we can define $R = (6, 4)$ and thus $P + P = -R = (6, -4 \Mod{7}) = (6, 3)$. Now that we have $2P$, we can compute $4P = 2P + 2P$. We know that $y_1 = 3$ is invertible mod 7, so 
            \[\lambda = (3(6)^2 + 0)(2(3))^{-1} \Mod{7} = (108)(6)^{-1} \Mod{7}.\]
            Computing the inverse of 6 mod 7 gives us 6, so 
            \[\lambda = 108(6) \Mod{7} = 4.\]
            Then, we have 
            \[\nu = 3 - 4(6) \Mod{7} = 0\]
            \[x_3 = 4^2 - 6 - 6 \Mod{7} = 4\]
            \[y_3 = 4(4) + 0 \Mod{7} = 2.\]
            Therefore, we can define $R = (4, 2)$ and thus $2P + 2P = -R = (4, -2 \Mod{7}) = (4, 5)$.
        \end{mdframed}
        \item Suppose Alice receives the point $Q_b = (5, 3)$ from Bob. What is her shared secret with Bob?
        \begin{mdframed}
            We know that 
            \[R = k_a Q_b = k_a (5, 3).\] 
        \end{mdframed}
    \end{enumerate}
\end{mdframed}

\begin{mdframed}
    (Exercise.) Suppose Alice and Bob publicly agree to use the elliptic curve $y^2 = x^3 + 17 \Mod{p} = 7$ and the point $P = (1, 2)$. This point has order 13, which is too small to be secure. Suppose Eve intercepts Alice and Bob's message: she sees that Alice sent Bob $Q_a = (3, 3)$ and that Bob sent Alice $Q_b = (6, 4)$. What is Alice and Bob's shared secret? 
    \begin{mdframed}
        
    \end{mdframed}
\end{mdframed}

\subsection{Interlude: Quadratic Residues}
A familiar feature of the real numbers is that some numbers do not have square roots (e.g., the negatives). The same thing happens when you mod an integer. For example, let $n = 5$. We know that the integer is congruent to 0, 1, 2, 3, or 4 mod 5. This means that any square must be congruent to $0^2 = 0$, $1^2 = 1$, $2^2 = 4$, $3^2 \equiv 4 \Mod{5}$, or $4^2 \equiv 1 \Mod{5}$. In other words, only 0, 1, or 4 have square roots mod 5, and 2 and 3 do not.

\begin{definition}{Quadratic Residue}{}
    Fix a positive integer $n$. We say that an integer $a$ is a \textbf{quadratic residue mod }$n$ if it has a square root mod $n$, i..e, if there exists an integer $x$ such that $x^2 \equiv a \Mod{n}$. 
\end{definition}

\begin{mdframed}
    (Exercise.) Find all quadratic residues mod the following integers. 
    \begin{enumerate}[(a)]
        \item $n = 3$
        \begin{mdframed}
            
        \end{mdframed}
        \item $n = 7$
        \begin{mdframed}
            
        \end{mdframed}
        \item $n = 11$
        \begin{mdframed}
            
        \end{mdframed}
    \end{enumerate}
\end{mdframed}

We'll see below that it will be useful to quickly determine whether an integer $a$ is a quadratic residue mod $a$ prime $p \geq 3$. It turns out that there is a good way to do this; let's introduce the following notation. 
\begin{definition}{Legendre Symbol}{}
    Let $p \geq 3$ be prime. For any integer $a$, define the Legendre symbol $(a/n)$ by 
    \[\left(\frac{a}{n}\right) = \begin{cases}
        0 & \text{if } a \equiv 0 \Mod{p} \\ 
        1 & \text{if } a \not\equiv 0 \Mod{p} \text{ and } a \text{ is a quadratic residue mod } p \\ 
        -1 & \text{if } a \not\equiv 0 \Mod{p} \text{ and } a \text{ is not a quadratic residue mod } p
    \end{cases}\]
\end{definition}
For example, we saw above that 4 is a quadratic residue mod 5, so 
\[\left(\frac{4}{5}\right) = 1\]
and we saw that 2 is not a quadratic residue mod 5, so 
\[\left(\frac{2}{5}\right) = -1.\]
We can now rephrase our goal: we would like a quick way of computing Legendre symbols. This is provided to us by combining binary exponentiation with the following: 
\begin{lemma}{Euler's Criterion}{}
    Let $p \geq 3$ be prime. For any integer $a$, 
    \[\left(\frac{a}{n}\right) \equiv a^{\frac{p - 1}{2}} \Mod{p}.\]
\end{lemma}

Euler's Criterion means that we have an efficient algorithm for determining whether something is a quadratic residue: we simply use binary exponentiation to compute $a^{(p - 1) / 2} \Mod{p}$ and we can read off the answer. 

\begin{mdframed}
    (Example.) Suppose we want to know if $a = 37$ is a quadratic residue mod $p = 97$. We have $(p - 1) / 2 = 96 / 2 = 48$, so we compute $a^{(p - 1) / 2} = 37^{48} \Mod{97}$ using binary exponentiation, and we find that it is $96 \equiv -1 \Mod{97}.$ Euler's Criterion says that 
    \[\left(\frac{37}{97}\right) \equiv 37^{(97 - 1) / 2} = 37^{48} \equiv -1 \Mod{97}.\]
    Therefore, 37 is not a quadratic residue mod 97. 
\end{mdframed}

\begin{mdframed}
    (Exercise.) Use Euler's Criterion to determine whether or not the following integers $a$ are quadratic residues mod $p = 19$. 
    \begin{enumerate}[(a)]
        \item $a = 3$
        \begin{mdframed}
            
        \end{mdframed}
        \item $a = 5$
        \begin{mdframed}
            
        \end{mdframed}
        \item $a = 11$
        \begin{mdframed}
            
        \end{mdframed}
    \end{enumerate}
\end{mdframed}


\subsection{Elliptic Curve Elgamal}
There is a variant of the Elgamal cryptosystem using elliptic curves that can be used to exchange messages, but there is a nontrivial encoding step. To make Elgamal work with elliptic curves, we first need a way to encode a plaintext message as a point on an elliptic curve $E$ mod $p$. 

\bigskip 

For this, there's a probablistic algorithm that encodes plaintext as $x$-coordinate of a plain (but note that not every integer will occur as the $x$-integer of a point on an elliptic curve mod $p$). Specifically, if $E$ is given by $y^2 = x^3 + ax + b$ and if $P = (x, y)$ is a point on the curve, then the $x$-coordinate must have the property that $x^3 + ax + b$ is a quadratic residue mod $p$. 

\subsubsection{The Process}
Suppose Bob wants to receive messages of length $N$. 
\begin{enumerate}
    \item Bob will fix a positive integer $s$. We'll see that, the larger Bob chooses the integer, the higher the probability that encoding will succeed. 
    \item Bob will then choose a prime $p > s26^N$ and an elliptic curve $E$ mod $p$ (defined by integers $a, b$ such that the integral Weierstrass equation $y^2 = x^3 + ax + b$ is nonsingular mod $p$), and a point $P \in E$. 
    \item He then computes $\ord_{E}(P)$. 
    \item Then, Bob chooses a secret integer $0 \leq k < \ord_{E}(P)$ to serve as his private key. He computes $Q = kP$, and this value is part of his public key. 
\end{enumerate}
In other words, Bob will share all of the data $(s, E, P, \ord_{E}(P), Q)$ publicly, and keep only the integer $k$ secret.

\bigskip 

Suppose now that Alice wants to send Bob a message.
\begin{enumerate}
    \item She converts her message into an integer $m$ using the same base 26 idea we used for RSA. 
    \item She will then iterate through values of $r = 0, 1, 2, \hdots, s - 1$ until she finds the first value of $x = sm + r$ such that\footnote{Remember that this is the \textbf{Legendre Symbol}!}
    \[\left(\frac{x^3 + ax + b}{p}\right) \neq -1.\]
    Note that the maximum possible value of $m$ is $26^N - 1$, so 
    \[x = sm + r < s(26^N - 1) + s = s26^N < p\]
    since Bob chose $p$ to be larger than $s26^N$. There is a roughly $1/2$ chance that an integer in $[0, p)$ is not a quadratic residue mod $p$, and here we are iterating through $s$ integers in the range $[0, p)$, so there is a $\left(\frac{1}{2}\right)^s$ chance that $x^3 + ax + b$ is not a quadratic residue for any of the $s$ possible values of $x = sm + r$. If none of the $s$ integers are quadratic residues, encoding fails. However, if Bob chose $s$ to be even moderately large, encoding failure is very unlikely. If encoding does fail, Alice can just modify her message slightly\footnote{Rephrasing slightly or adding a nonsense letter.} and try encoding again. 
    
    \item Once Alice finds a value of $x$ such that $x^3 + ax + b$ is a quadratic residue mod $p$, then there is an integer $y$ such that $y^2 \equiv x^3 + ax + b \Mod{p}$, so the point $M = (x, y)$ is on $E$. This will be the \underline{encoding} of her plaintext.
\end{enumerate}
This is not the ciphertext, but she can now encrypt the encoded message using a process very similar to the Elgamal cryptosystem we discussed earlier. 
\begin{enumerate}
    \item First, Alice chooses an ``ephemeral key'' $h$ such that $0 \leq h < \ord_{E}(P)$.
    \item She computes $S = hQ$, $R_1 = hP$, and $R_2 = M + S$. The pair, $(R_1, R_2)$, is the ciphertext she sends to Bob. 
\end{enumerate}
To decrypt the ciphertext $(R_1, R_2)$, Bob uses his private key $k$ to compute $S = kR_1$. Observe that 
\[kR_1 = k(hP) = khP = h(kP) = hQ,\]
so Bob has found the same value of $S$ that Alice had, even though he does not know the value of the ephemeral key $h$. He can then compute $-S$ by negating the $y$-coordinate, and he then calculates 
\[R_2 - S = R_2 + (-S) = (M + S) + (-S) = M + (S + (-S)) = M + O = M.\]
He has thus recovered Alice's encoded plaintext. 

\bigskip 

Finally, Bob just needs to decode $M$. If $M = (x, y)$, he can extract the first coordinate $x$. The quotient when he divides this by $s$ is the integer $m$ that represents the message in base 26, so he then finishes off by converting back to text using the same process we used for RSA above. 

\subsubsection{Encoding and Decoding}
\begin{mdframed}
    (Exercise.) Suppose Bob's public key has $s = 2$, $p = 97$, $a = 31$, and $b = 20$. The elliptic curve $E$ is then the one given by $y^2 = x^3 + 31x + 20 \Mod{p = 97}$. 
    \begin{enumerate}[(a)]
        \item What is the encoding of the plaintext message \code{B}? Follow the process above to find the corresponding point $M \in E$. 
        \begin{mdframed}
            First, we encode \code{B} into base 26; this gives us $m = 1$. Then, we need to iterate through all $r$ such that $0 \leq r \leq 2 - 1 = 1$. We find that 
            \begin{itemize}
                \item For $r = 0$, we have $x = 2(1) + 0 = 2$ and 
                \begin{equation*}
                    \begin{aligned}
                        \left(\frac{2^3 + 31(2) + 20}{97}\right) &= \left(\frac{90}{97}\right) \\ 
                            &= 90^{\frac{97 - 1}{2}} \Mod{97} \\ 
                            &= 90^{\frac{96}{2}} \Mod{97} \\ 
                            &= 90^{48} \Mod{97}.
                    \end{aligned}
                \end{equation*}
                With this in mind, we find that $90^{48} \equiv 96 \equiv -1 \Mod{97}$, so $r = 0$ is not an option. 

                \item For $r = 1$, we have $x = 2(1) + 1 = 3$ and 
                \begin{equation*}
                    \begin{aligned}
                        \left(\frac{3^3 + 31(3) + 20}{97}\right) &= \left(\frac{140}{97}\right) \\ 
                            &= 140^{\frac{97 - 1}{2}} \Mod{97} \\ 
                            &= 140^{48} \Mod{97} \\ 
                            &= 1 \Mod{97}.
                    \end{aligned}
                \end{equation*}
                Here, we find that $r = 1$ and thus $x = 3$ \emph{is} the option. 
            \end{itemize}
            Now that we have $x = 3$, we can compute \[y^2 \equiv 3^3 + 31(3) + 20 \Mod{97}.\] We find that $y \equiv 25$. Thus, \[M = (3, 25).\]
        \end{mdframed}
        \item Show that the encoding fails for the letter \code{K}. 
        \begin{mdframed}
            Encoding \code{K} gives us $m = 10$. Then, iterating through all $0 \leq r \leq 2 - 1 = 1$, we have 
            \begin{itemize}
                \item For $r = 0$, we have $x = 2(10) + 0 = 20$ and 
                \begin{equation*}
                    \begin{aligned}
                        \left(\frac{20^3 + 31(20) + 20}{97}\right) &= \left(\frac{8640}{97}\right) \\ 
                            &= 8640^{\frac{97 - 1}{2}} \Mod{97} \\ 
                            &= 8640^{48} \Mod{97} \\ 
                            &= 7^{48} \Mod{97} \\ 
                            &= 96 \Mod{97}.
                    \end{aligned}
                \end{equation*}
                This gives us $8640^{48} \equiv 96 \equiv -1 \Mod{97}$, so $r = 0$ is not an option. 

                \item For $r = 1$, we have $x = 2(10) + 1 = 21$ and 
                \begin{equation*}
                    \begin{aligned}
                        \left(\frac{21^3 + 31(21) + 21}{97}\right) &= \left(\frac{9933}{97}\right) \\ 
                            &= 9933^{\frac{97 - 1}{2}} \Mod{97} \\ 
                            &= 9933^{48} \Mod{97} \\ 
                            &= 39^{48} \Mod{97} \\ 
                            &= 96 \Mod{97}.
                    \end{aligned}
                \end{equation*}
                Once again, this gives us $9933^{48} \equiv 96 \equiv -1 \Mod{97}$, so $r = 1$ is not an option. 
            \end{itemize}
            Because we got $-1$ for all valid $r$, encoding is not possible.  
        \end{mdframed}
        \item Follow the process described above to find the plaintext message that results from decoding the point $(25, 30) \in E$.
        \begin{mdframed}
            TODO % ask about this
        \end{mdframed}
    \end{enumerate}
\end{mdframed}


\end{document}