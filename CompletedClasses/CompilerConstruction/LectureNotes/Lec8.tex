\documentclass[letterpaper]{article}
\usepackage[margin=1in]{geometry}
\usepackage[utf8]{inputenc}
\usepackage{textcomp}
\usepackage{amssymb}
\usepackage{natbib}
\usepackage{graphicx}
\usepackage{gensymb}
\usepackage{amsthm, amsmath, mathtools}
\usepackage[dvipsnames]{xcolor}
\usepackage{enumerate}
\usepackage{mdframed}
\usepackage[most]{tcolorbox}
\usepackage{csquotes}
% https://tex.stackexchange.com/questions/13506/how-to-continue-the-framed-text-box-on-multiple-pages

\tcbuselibrary{theorems}

\newcommand{\R}{\mathbb{R}}
\newcommand{\Z}{\mathbb{Z}}
\newcommand{\N}{\mathbb{N}}
\newcommand{\Q}{\mathbb{Q}}
\newcommand{\C}{\mathbb{C}}
\newcommand{\code}[1]{\texttt{#1}}
\newcommand{\mdiamond}{$\diamondsuit$}
\newcommand{\PowerSet}{\mathcal{P}}
\newcommand{\Mod}[1]{\ (\mathrm{mod}\ #1)}
\DeclareMathOperator{\lcm}{lcm}

%\newtheorem*{theorem}{Theorem}
%\newtheorem*{definition}{Definition}
%\newtheorem*{corollary}{Corollary}
%\newtheorem*{lemma}{Lemma}
\newtheorem*{proposition}{Proposition}


\newtcbtheorem[number within=section]{theorem}{Theorem}
{colback=green!5,colframe=green!35!black,fonttitle=\bfseries}{th}

\newtcbtheorem[number within=section]{definition}{Definition}
{colback=blue!5,colframe=blue!35!black,fonttitle=\bfseries}{def}

\newtcbtheorem[number within=section]{corollary}{Corollary}
{colback=yellow!5,colframe=yellow!35!black,fonttitle=\bfseries}{cor}

\newtcbtheorem[number within=section]{lemma}{Lemma}
{colback=red!5,colframe=red!35!black,fonttitle=\bfseries}{lem}

\newtcbtheorem[number within=section]{example}{Example}
{colback=white!5,colframe=white!35!black,fonttitle=\bfseries}{def}

\newtcbtheorem[number within=section]{note}{Important Note}{
        enhanced,
        sharp corners,
        attach boxed title to top left={
            xshift=-1mm,
            yshift=-5mm,
            yshifttext=-1mm
        },
        top=1.5em,
        colback=white,
        colframe=black,
        fonttitle=\bfseries,
        boxed title style={
            sharp corners,
            size=small,
            colback=red!75!black,
            colframe=red!75!black,
        } 
    }{impnote}
\usepackage[utf8]{inputenc}
\usepackage[english]{babel}
\usepackage{fancyhdr}
\usepackage[hidelinks]{hyperref}

\pagestyle{fancy}
\fancyhf{}
\rhead{CSE 131}
\chead{Wednesday, April 19, 2023}
\lhead{Lecture 8}
\rfoot{\thepage}

\setlength{\parindent}{0pt}

\begin{document}
\section{An Overview of Loops, Set, Break, Input, Blocks, and Errors}
In this section, we'll go over a high-level overview of some additional features. 

\subsection{Loops \& Break}
Loop and break expressions are relatively simple; they look something like 
\begin{verbatim}
    (loop <expr>)
    (break <expr>)\end{verbatim}
At its core, a \code{loop} will repeat an expression over and over again until it encounters a \code{break}, in which case the loop terminates.

\subsubsection{Assembly Representation} 
The assembly representation of loops is straightforward:
\begin{verbatim}
    loop: 
                    ; body of loop 
        jmp loop 
    done:\end{verbatim}
Here, we defined two labels: 
\begin{itemize}
    \item \code{loop}, indicating the beginning of the loop. 
    \item \code{done}, indicating the end of the loop (where we should ``break'' out).
\end{itemize}
The idea is that, as long as we aren't breaking out, we will unconditionally jump back to \code{loop}. If we do want to incorporate a \code{break} statement, we can add a jump to that label.

\subsubsection{Labeling}
As is the case with \code{if}-statements, we can have many loops! So, we need to create a unique label for each loop. We can use the \code{new\_label} function from the last section to do this for us. 

\subsubsection{Implementing Break}
To implement \code{break}, the idea is for the compiler to include an additional argument. We can call this argument \code{loop\_label}, which will be an \code{Option<String>} (recall that an \code{Option<T>} will either be a \code{Some(T)} or \code{None}, indicating some or no value, respectively).

\bigskip 

Before we compile the expression associated with the loop, we need to create a unique label. Once we create this label, we can compile the expression, passing that label as our argument for \code{loop\_label}. Because of the recursive nature of the compile function, if we end up inside another loop, compiling its associated expression will result in another label for \emph{that} function call, but not for the current function call. In that sense, we don't need to worry about the possibility of overwriting the break labels.

\subsection{Set}
Set is relatively straightforward: it's analogous to reassignment in most other programming languages. Its syntax looks like 
\begin{verbatim}
    (set! <name> <expr>)\end{verbatim}
Here, we're assigning the result of evaluating \code{<expr>} to the identifier, \code{<name>}. If the identifier doesn't exist, an error should be thrown. 

\subsection{Blocks}
Blocks are just a way of writing more than one statement for an expression. Syntactically, they look like 
\begin{verbatim}
    (block <expr>+)\end{verbatim}
In other words, it takes one or more expressions, and then runs each expression in the order that they appear.

\begin{center}
    \begin{tabular}{p{3in}|p{3in}}
        Rust & Our Language \\ 
        \hline 
        \begin{verbatim}
if x > 10 {
    x = x + y + z;
    y = x - z;
    z = z + 5;
} else {
    x += 10;
}\end{verbatim} 
            & \begin{verbatim}
(if (> x 10) 
    (block 
        (set! x (+ x (+ y z)))
        (set! y (- x z))
        (set! z (+ z 5))
    ) 
    (set! x (+ x 10)))\end{verbatim}
    \end{tabular}
\end{center}

\subsection{Handling Inputs}
Our programming language is now expected to take a single input, which can either be a number or boolean value. Syntactically, the command is just 
\begin{verbatim}
    input \end{verbatim}
In order to handle any input, we need to think about the runtime and, more importantly, the role that \code{start.rs} plays (recall that \code{start.rs} is how we call into our assembly code.)
\begin{verbatim}
    fn parse_input(input: &str) -> i64 {
        if input == "true" {
            0b11
        } else if input == "false" {
            0b01
        } else if let Ok(val) = input.parse::<i64>() {
            (val << 1) as u64
        } else {
            panic!("unsupported input: `{}`", input);
        }
    }

    fn main() {
        let args: Vec<String> = env::args().collect();
        // This is our single input, which by default will be 
        // "false" if none are provided
        let input = if args.len() == 2 { &args[1] } else { "false" };
        // Call our assembly code with the given input.
        let i: i64 = unsafe { our_code_starts_here(parse_input(&input)) };
        // Finally, determine what was returned to us.
        if i & 1 == 0 {
            // Number
            println!("{}", i >> 1);
        } else {
            // Boolean
            println!("{}", i >> 1 == 1);
        }
    }\end{verbatim}
Note that our input will be in the register \code{rdi}. So, we can modify the compiler to simply move \code{rdi} into \code{rax} before using it. 
\begin{verbatim}
    match e {
        ... 
        Expr::Id(s) if s == "input" => "mov rax, rdi"
    }\end{verbatim}
That's it!

\subsection{Errors}
Recall that, in the previous sections, we didn't want things like \code{(+ true 1)} to work. This should cause a \emph{runtime error}. So, how do we invoke a runtime error from our assembly code? 
\begin{itemize}
    \item First, we want to create a function in Rust that our assembly code will call. 
    \begin{verbatim}
    #[export_name = "\x01snek_error"]
    pub extern "C" fn snek_error(errcode: i64) {
        eprintln!("error code {errcode} received.");
        std::process::exit(1);
    }\end{verbatim}
    
    \item Next, in our assembly header, we want to define this function as an external function. 
    \begin{verbatim}
        section .text 
        global our_code_starts_here 
        extern snek_error               ; right here!\end{verbatim}

    \item From there, we can define a \code{throw\_error} label which will handle calling the Rust code. It will look like 
    \begin{verbatim}
        throw_error: 
            mov rdi, 7          ; rdi is the register representing the first arg 
                                ; 7 is some error code we made up 
            push rsp    
            call snek_error     ; calls the snek_error rust function 
            ret \end{verbatim}
    We'll worry about the \code{push} and \code{pop} instruction later. 
\end{itemize}

\end{document}