\documentclass[letterpaper]{article}
\usepackage[margin=1in]{geometry}
\usepackage[utf8]{inputenc}
\usepackage{textcomp}
\usepackage{amssymb}
\usepackage{natbib}
\usepackage{graphicx}
\usepackage{gensymb}
\usepackage{amsthm, amsmath, mathtools}
\usepackage[dvipsnames]{xcolor}
\usepackage{enumerate}
\usepackage{mdframed}
\usepackage[most]{tcolorbox}
\usepackage{csquotes}
% https://tex.stackexchange.com/questions/13506/how-to-continue-the-framed-text-box-on-multiple-pages

\tcbuselibrary{theorems}

\newcommand{\R}{\mathbb{R}}
\newcommand{\Z}{\mathbb{Z}}
\newcommand{\N}{\mathbb{N}}
\newcommand{\Q}{\mathbb{Q}}
\newcommand{\C}{\mathbb{C}}
\newcommand{\code}[1]{\texttt{#1}}
\newcommand{\mdiamond}{$\diamondsuit$}
\newcommand{\PowerSet}{\mathcal{P}}
\newcommand{\Mod}[1]{\ (\mathrm{mod}\ #1)}
\DeclareMathOperator{\lcm}{lcm}

%\newtheorem*{theorem}{Theorem}
%\newtheorem*{definition}{Definition}
%\newtheorem*{corollary}{Corollary}
%\newtheorem*{lemma}{Lemma}
\newtheorem*{proposition}{Proposition}


\newtcbtheorem[number within=section]{theorem}{Theorem}
{colback=green!5,colframe=green!35!black,fonttitle=\bfseries}{th}

\newtcbtheorem[number within=section]{definition}{Definition}
{colback=blue!5,colframe=blue!35!black,fonttitle=\bfseries}{def}

\newtcbtheorem[number within=section]{corollary}{Corollary}
{colback=yellow!5,colframe=yellow!35!black,fonttitle=\bfseries}{cor}

\newtcbtheorem[number within=section]{lemma}{Lemma}
{colback=red!5,colframe=red!35!black,fonttitle=\bfseries}{lem}

\newtcbtheorem[number within=section]{example}{Example}
{colback=white!5,colframe=white!35!black,fonttitle=\bfseries}{def}

\newtcbtheorem[number within=section]{note}{Important Note}{
        enhanced,
        sharp corners,
        attach boxed title to top left={
            xshift=-1mm,
            yshift=-5mm,
            yshifttext=-1mm
        },
        top=1.5em,
        colback=white,
        colframe=black,
        fonttitle=\bfseries,
        boxed title style={
            sharp corners,
            size=small,
            colback=red!75!black,
            colframe=red!75!black,
        } 
    }{impnote}
\usepackage[utf8]{inputenc}
\usepackage[english]{babel}
\usepackage{fancyhdr}
\usepackage[hidelinks]{hyperref}

\pagestyle{fancy}
\fancyhf{}
\rhead{CSE 131}
\chead{Wednesday, May 17, 2023}
\lhead{Lecture 20}
\rfoot{\thepage}

\setlength{\parindent}{0pt}

\begin{document}

\section{Optimization}
In this section, we'll talk more about optimization. 

\begin{definition}{Optimization}{}
    An \textbf{optimization} (for a compiler) is a version that produces programs that evaluate the same answer as the prior version, but are ``better'' on some cost metric.
\end{definition}

\subsection{Examples of Cost Metrics}
Some examples of cost metrics that we might want to improve on include 
\begin{itemize}
    \item \textbf{Time:} How long does it take for the program to run? 
    \item \textbf{Space:} How much process memory does the program use? 
    \item \textbf{Binary Size:} The size of the compiled binary, and the number of instructions.
    \item \textbf{Executed Instructions:} The overall number of instructions executed (compared to the total number of instructions). This is also similar to the number of jumps in the resulting assembly.  
\end{itemize}
\textbf{Remarks:}
\begin{itemize}
    \item The first two metrics -- time and space -- are generally the most important ones. 
    \item We might also care about properties like compile time, extensibility, debuggability, and platform independence, although these are harder to measure.
\end{itemize}

\subsection{High-Level Optimization Suggestions}
Some suggestions for optimizations include 
\begin{itemize}
    \item \textbf{Register Allocation:} Storing values in registers rather than in memory, since access to registers are generaly faster than access to memory. 
    \item \textbf{Dead Code Elimination:} Remove code that the compiler can prove will never run.
    \begin{mdframed}
        (Example.) Consider the following code: 
        \begin{verbatim}
            if false 3 4 \end{verbatim}
        Here, we know for sure the \code{3} will never execute.
    \end{mdframed}
    This also includes things like removing unused variables from compilation. 

    \item \textbf{Constant Folding:} Evaluate ``what you can solve'' in the compiler. 
    \begin{mdframed}
        (Example.) Consider the following code:
        \begin{verbatim}
            (+ (* 2 3) input)\end{verbatim}
        Here, we know that \code{(* 2 3)} should be \code{6}, so this is basically equivalent to 
        \begin{verbatim}
            (+ 6 input)\end{verbatim}
    \end{mdframed}

    \item \textbf{Common Subexpression Elimination:} Eliminate repeated code in favor of a single instance of it. 
    \item \textbf{Memory Packing:} Eliminate unused memory due to alignment (e.g., struct alignment).
    \item \textbf{Loop Unrolling:} We can unroll a loop if we know that it has a constant bound. In other words, essentially hardcode all iterations.
    \begin{mdframed}
        (Example.) Consider the following code: 
        \begin{verbatim}
            (let (x 0) (loop (if (< x 3) (set! x (add1 x)) (break x))))\end{verbatim}
        This is equivalent to  
        \begin{verbatim}
            (let (x 0) (block 
                (set! x (add1 x))
                (set! x (add1 x))
                (set! x (add1 x))
            ))\end{verbatim}
    \end{mdframed}

    \item \textbf{Type-Directed Compilation:} We can remove some code that involves type checking if we know for sure that we're working with the correct types. 
    \begin{mdframed}
        (Example.) Consider the code 
        \begin{verbatim}
            (+ 1 (* input 2))\end{verbatim}
        While type checking is necessary for \code{(* input 2)}, it's probably not necessary for the plus expression since we can assume that both sides are numbers.
    \end{mdframed}

    \item \textbf{Peephole Optimization:} We can remove redundent move operations in the resulting assembly. 
    
    \item \textbf{Inlining Functions:} We can inline function calls, especially if we have a small one. 
    \begin{mdframed}
        (Example.) Consider the following function and resulting code: 
        \begin{verbatim}
            (fun (f x)
                <body>)
            (f 10)\end{verbatim}
        This could be functionally equivalent to  
        \begin{verbatim}
            (let (x 10) <body>)\end{verbatim}
    \end{mdframed}
    Note that we might need to consider things like recursion or other function calls in the function body, since that might prevent us for optimizing. 

    \item \textbf{Instruction Selection:} We can also possibly exploit the structure of our code. 
    \begin{mdframed}
        (Example.) Consider the following code: 
        \begin{verbatim}
            (if (< x 10) ... ...)\end{verbatim}
        Generally, our compiler would put either \code{true} or \code{false} into \code{rax}, and then evaluate \code{rax} when deciding where to jump. However, in this particular code, we can probably just conditionally jump on the spot.  
    \end{mdframed}
\end{itemize}


\end{document}