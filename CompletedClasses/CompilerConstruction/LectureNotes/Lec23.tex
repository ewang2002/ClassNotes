\documentclass[letterpaper]{article}
\usepackage[margin=1in]{geometry}
\usepackage[utf8]{inputenc}
\usepackage{textcomp}
\usepackage{amssymb}
\usepackage{natbib}
\usepackage{graphicx}
\usepackage{gensymb}
\usepackage{amsthm, amsmath, mathtools}
\usepackage{xcolor}
\usepackage{enumerate}
\usepackage{framed}
\usepackage{tcolorbox}
\tcbuselibrary{theorems}

\newcommand{\R}{\mathbb{R}}
\newcommand{\Z}{\mathbb{Z}}
\newcommand{\N}{\mathbb{N}}
\newcommand{\Q}{\mathbb{Q}}
\newcommand{\code}[1]{\texttt{#1}}
\newcommand{\mdiamond}{$\diamondsuit$}

%\newtheorem*{theorem}{Theorem}
%\newtheorem*{definition}{Definition}
\newtheorem*{proposition}{Proposition}
%\newtheorem*{corollary}{Corollary}
%\newtheorem*{lemma}{Lemma}

\newtcbtheorem[number within=section]{theorem}{Theorem}
{colback=green!5,colframe=green!35!black,fonttitle=\bfseries}{def}

\newtcbtheorem[number within=section]{definition}{Definition}
{colback=blue!5,colframe=blue!35!black,fonttitle=\bfseries}{def}

\newtcbtheorem[number within=section]{corollary}{Corollary}
{colback=yellow!5,colframe=yellow!35!black,fonttitle=\bfseries}{def}

\newtcbtheorem[number within=section]{lemma}{Lemma}
{colback=red!5,colframe=red!35!black,fonttitle=\bfseries}{def}
\usepackage[utf8]{inputenc}
\usepackage[english]{babel}
\usepackage{fancyhdr}
\usepackage[hidelinks]{hyperref}

\pagestyle{fancy}
\fancyhf{}
\rhead{CSE 131}
\chead{Wednesday, May 24, 2023}
\lhead{Lecture 23}
\rfoot{\thepage}

\setlength{\parindent}{0pt}

\begin{document}

\section{Optimization (Continued)}

\subsection{Optimization: Register Allocation}

\subsubsection{Restrictions}
One of the main restrictions of the algorithm for register allocation is simply temporary values: what do we do with register allocation for temporary values that don't have any names?

\subsection{Intermediate Representation}
Let's consider the following program:
\begin{verbatim}
    (+ (- 5 input) (* input (if (> 0 input) 1 -1)))\end{verbatim}
Below is a transformed version of the program where every nested expression that would have introduced a temporary is now a \code{let}-bound variable. 
\begin{verbatim}
    (let (tmp 1 (- t input))
        (let (tmp 2 (> 0 input))
            (let (tmp3 (if tmp2 1 -1))
                (let (tmp4 (* input tmp3))
                    (+ tmp1 tmp4)
                )
            )
        )
    )\end{verbatim}
This makes the order of operations very explicit. Notice that it's very clear that we're doing left-to-right operation. This process also makes code generation for operations a lot easier, since everything is already stored on the stack or in a register. 

\begin{mdframed}
    (Example.) Perform the transformation on the function 
    \begin{verbatim}
    (fun (sumsquares x y)
        (+ (* x x) (* y y)))\end{verbatim}

    \begin{mdframed}
        As mentioned, we want to break our complex expression into much simpler types. We can accomplish this by defining any computations into \code{let}-bindings. This gives us 
        \begin{verbatim}
    (fun (sumsquares x y)
        (let (val (* x x))
            (let (val2 (* y y))
                (+ val val2)
            )
        )
    )\end{verbatim}
    \end{mdframed}
\end{mdframed}
This transformation is fairly common, and there is a fairly standard algorithm for this transformation that takes every non-trivial or non-atomic (basically, everything that's not a literal value) and puts them in a \code{let}-binding. 

\subsubsection{Different Grammar Forms}
There are two types of grammars we want to consider. 
\begin{itemize}
    \item \textbf{A-Normal Form:} This is essentially our grammar as is. This cares about scope, binding, order or evaluations, and so on. 
    
    \begin{verbatim}
        <expr> := <number> | <id> | true | false | nil 
            | (+ <expr> <expr>)
            | (- <expr> <expr>)
            | (if <expr> <expr> <expr>)
            | (break <expr>)
            | ... 
            | (let (<id> <expr>) <expr>)
            | ... \end{verbatim}
    \item \textbf{ANF-Restricted Grammar:} This grammar is effectively the transformed grammar; that is, given our A-Normal Form, we can transform it into a more restricted version. We'll denote this as \code{AExpr}.
    \begin{verbatim}
        <val> := <number> | <id> | true | false | nil 
        <expr> := (+ <val> <val>) 
            | (pair <val> <val>) 
            | (if <val> <block> <block>)
            | (break <val>)
        <block> := (let (<id> <expr>) <block>)
            | (loop <block>)
            | (break <block>+)
            | ... 
            | <expr>
            | <val>\end{verbatim}
        This is restricted in the sense that the grammar is broken up into three different groups (productions). You have expressions (\code{<expr>}) that form blocks (\code{<block>}). Blocks have expressions in them that can perform calculations; so, we can think of \code{<expr>} as something that performs a calculation of some type (e.g., binary operations, creating a new pair, and so on). All the arguments to these expressions that do calculations must be primitive values or identifiers.  
\end{itemize}
\textbf{Remark:} Loops are an interesting case to think about here. 

\subsubsection{Going from Normal to Restricted}
How do we create an algorithm that transforms code written under one grammar to code written in the restricted grammar? Note that this is a very standard algorithm, so we'll mainly gain some intuition. One choice to make is whether we want to introduce a new \code{enum} for ANF expressions. For example, is it worth it to introduce a bunch of new \code{enum}s like shown below?
\begin{verbatim}
    enum Val {
        VNum,
        VId(String),
        ...
    }

    enum AExpr {
        APlus(...),
        APair(...),
        ... 
    }

    enum Block {
        BLet(...),
        ...
    }\end{verbatim}

In any case, we can write a few functions to facilitate the conversion process.
\begin{itemize}
    \item \verb|anf_to_val(e: &Expr) -> Val|: Converts an A-Normal Expression Form to a literal value under the ANF-Restricted Expression Form.
    \item \verb|anf_to_expr(e: &Expr) -> AExpr|: Converts an A-Normal Expression Form to a computation expression under the ANF-Restricted Expression Form.
    \item \verb|anf_to_block(e: &Expr) -> Block|: Converts an A-Normal Expression Form to a block expression under the ANF-Restricted Expression Form. 
\end{itemize}

\subsubsection{ANF to Value}
Let's suppose we want to implement 
\begin{verbatim}
    fn anf_to_val(e: &Expr) -> Val {
        match e {
            ...
        }
    }\end{verbatim}

\begin{itemize}
    \item For an expression \code{e} like \code{Number(n)}, we can trivially return \code{VNum(n)}.
    \item For an expression \code{e} like \code{Plus(e1, e2)}, this becomes more complicated. This will involve a nested plus expression, and we should return an identifier that we can use later. Note that \code{e1} and \code{e2} may be complicated expressions, but we want them to be \code{Val}s as well. So, we'll probably need to do some recursive calls to ideally break \code{e1} and \code{e2} down into \code{let}-bindings. 
    \begin{verbatim}
        Plus(e1, e2) => {
            let (v1, b1) = anf_to_val(e1);
            let (v2, b2) = anf_to_val(e2);
            // Assume new_label() is a function that returns a new identifier.
            let new_name = new_label(); 
            // This isn't Rust syntax, but basically we want all the bindings from 
            // b1, all the bindings from b2, and our new binding in this vector.
            (VId(new_name), vec![...b1, ...b2, (new_name, APlus(v1, v2))])
        }\end{verbatim}
    Notice how we're returning a tuple. The first element is the identifier that will store the result of the evaluation of this expression. However, we also need to return list of \code{let}-bindings we need to eventually stick in front of \emph{this} identifier in order for it to work (otherwise, the identifier won't be bound). So, we should modify the function return type: 
    \begin{verbatim}
    fn anf_to_val(e: &Expr) -> (Val, Vec<(String, AExpr)>) { ... }\end{verbatim}
\end{itemize}


% it's about scope, binding, order of evaluation

\end{document}