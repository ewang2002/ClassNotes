\documentclass[letterpaper]{article}
\usepackage[margin=1in]{geometry}
\usepackage[utf8]{inputenc}
\usepackage{textcomp}
\usepackage{amssymb}
\usepackage{natbib}
\usepackage{graphicx}
\usepackage{gensymb}
\usepackage{amsthm, amsmath, mathtools}
\usepackage[dvipsnames]{xcolor}
\usepackage{enumerate}
\usepackage{mdframed}
\usepackage[most]{tcolorbox}
\usepackage{csquotes}
% https://tex.stackexchange.com/questions/13506/how-to-continue-the-framed-text-box-on-multiple-pages

\tcbuselibrary{theorems}

\newcommand{\R}{\mathbb{R}}
\newcommand{\Z}{\mathbb{Z}}
\newcommand{\N}{\mathbb{N}}
\newcommand{\Q}{\mathbb{Q}}
\newcommand{\C}{\mathbb{C}}
\newcommand{\code}[1]{\texttt{#1}}
\newcommand{\mdiamond}{$\diamondsuit$}
\newcommand{\PowerSet}{\mathcal{P}}
\newcommand{\Mod}[1]{\ (\mathrm{mod}\ #1)}
\DeclareMathOperator{\lcm}{lcm}

%\newtheorem*{theorem}{Theorem}
%\newtheorem*{definition}{Definition}
%\newtheorem*{corollary}{Corollary}
%\newtheorem*{lemma}{Lemma}
\newtheorem*{proposition}{Proposition}


\newtcbtheorem[number within=section]{theorem}{Theorem}
{colback=green!5,colframe=green!35!black,fonttitle=\bfseries}{th}

\newtcbtheorem[number within=section]{definition}{Definition}
{colback=blue!5,colframe=blue!35!black,fonttitle=\bfseries}{def}

\newtcbtheorem[number within=section]{corollary}{Corollary}
{colback=yellow!5,colframe=yellow!35!black,fonttitle=\bfseries}{cor}

\newtcbtheorem[number within=section]{lemma}{Lemma}
{colback=red!5,colframe=red!35!black,fonttitle=\bfseries}{lem}

\newtcbtheorem[number within=section]{example}{Example}
{colback=white!5,colframe=white!35!black,fonttitle=\bfseries}{def}

\newtcbtheorem[number within=section]{note}{Important Note}{
        enhanced,
        sharp corners,
        attach boxed title to top left={
            xshift=-1mm,
            yshift=-5mm,
            yshifttext=-1mm
        },
        top=1.5em,
        colback=white,
        colframe=black,
        fonttitle=\bfseries,
        boxed title style={
            sharp corners,
            size=small,
            colback=red!75!black,
            colframe=red!75!black,
        } 
    }{impnote}
\usepackage[utf8]{inputenc}
\usepackage[english]{babel}
\usepackage{fancyhdr}
\usepackage[hidelinks]{hyperref}

\pagestyle{fancy}
\fancyhf{}
\rhead{CSE 131}
\chead{Wednesday, May 03, 2023}
\lhead{Lecture 14}
\rfoot{\thepage}

\setlength{\parindent}{0pt}

\begin{document}

\section{Structured Data: Pairs}
In this section, we'll introduce \textbf{structured data} to our programming language. In particular, we'll introduce \textbf{pairs}, which is essentially a two-element tuple where both elements can be anything -- numbers or pairs.

\subsection{Pair Expressions}
Our language now has the following syntax: 
\begin{verbatim}
    expr := ... | (pair <expr> <expr>) | (fst <expr>) | (snd <expr>) | nil\end{verbatim}
Here, \code{pair} defines a pair of expressions. \code{fst} and \code{snd} returns the first and second element of a pair\footnote{Although \code{fst} and \code{snd} takes any expressions, it expects a pair expression.}, respectively.

\begin{mdframed}
    (Exercise.) What will the following program evaluate to? 
    \begin{verbatim}
        (fun (inc lst)
            (if (= lst nil)
                nil
                (pair (+ (fst lst) 1) (inc (snd lst)))
            )
        )
    
    (inc (pair 70 (pair 800 nil)))\end{verbatim}

    \begin{mdframed}
        The answer is \code{(71, (801, nil))}. In this function, we first check if the given pair is \code{nil}; if it is, return \code{nil}. Otherwise, we create a new pair where the first expression is just the first element of the original pair incremented by 1, and the second element is the result of recursively calling \code{inc} on the second element of the original pair.

        \bigskip 

        We can think of the \code{(snd lst)} as the \emph{rest of the list}. 
    \end{mdframed}
\end{mdframed}

\begin{mdframed}
    (Exercise.) What will the following program evaluate to? 
    \begin{verbatim}
        (fun (sum lst)
            (let (total 0)
                (loop 
                    (if (= lst nil) 
                        (break total) 
                        (block
                            (set! total (+ total (fst lst)))
                            (set! lst (snd lst)) 
                        )
                    )
                )
            )
        )
    
        (sum (pair 70 (pair 800 nil)))\end{verbatim}
    
    \begin{mdframed}
        The answer is 870. This program iterates through each element of the pair, getting its value and adding it to \code{total}. In particular, if we ran the program, we see that 
        \begin{verbatim}        
    Expression                          (fst lst)       Total
    (pair 70 (pair 800 nil))            70              70
    (pair 800 nil)                      800             870 
    nil                                 -               -\end{verbatim}
    \end{mdframed}
\end{mdframed}

\subsection{Representing Pairs}
Recall that we used a tagging system, where we dedicated one bit, to differentiate numbers and booleans. However, with a new type, we need to rethink the tagging system. 

\bigskip 

Our tagging system will now consist of the following:
\begin{itemize}
    \item \textbf{Numbers} will still use \code{0} as its tag value.
    \item \textbf{Booleans} will use \code{11} as its tag value. 
    \begin{itemize}
        \item \code{true} will be represented in binary as \code{111} (7).
        \item \code{false} will be represented in binary as \code{011} (3).
    \end{itemize}
    \item \textbf{Pairs} will use \code{01} as its tag value. 
    \item \textbf{Nil} will use \code{1} as its tag value.
\end{itemize}
With a tagging system in hand, how do we represent pairs themselves? One approach is as follows: 
\begin{itemize}
    \item An idea is to store each of the pair's value as 31-bits. For example, to represent 2 numbers, we would represent the first number as 31 bits, and the next number as another 31 bits, with the tag value being 2 bits. 
    \item \textbf{However}, this won't really work if we have nested pairs. For example, if we have a pair with pairs as its element, then how do we represent this? 
\end{itemize}
Another thing we can think about is heap allocation. 

\subsection{Compiler Design}
As implied, we'll have to allocate pair element on the \textbf{heap} (we'll need to work with the Rust runtime for this). So, our representation is that the pair's value will be a 62-bit address on the heap. 

\bigskip 

How do we know \emph{where} to allocate pair elements on the heap? An idea is to dedicate a register that just keeps track of the current heap location. In our class, we'll use \code{r15} for our purposes. With this in mind, here are a few assumptions we will be making:
\begin{itemize}
    \item \code{r15} is expected to keep growing for now; it's not like \code{rsp} where it can increase or decrease depending on how stack space is used. 
    \item \code{r15} only refers to available memory, never used memory. 
    \item \code{r15} will be 16-byte aligned (it will end with \code{0000})
\end{itemize}
With this in mind, how do we modify our compiler to support pairs? A sketch of an implementation we'll use is as follows: 
\begin{verbatim}
    Pair(e1, e2) => {
        let fst = compile_expr(e1, ...);
        let snd = compile_expr(e2 ...);
        // e1 will be somewhere in [rsp], e2 in rax 
        format!("
            {fst}
            {snd}
            mov [r15 + 8], rax 
            mov rbx, [rsp + offset]
            mov [r15], r15
            mov rax, r15 
            add rax, 1 
            add r15, 16
        ")
    }\end{verbatim}
\textbf{Remarks:} 
\begin{itemize}
    \item We should probably first check and see if we have space left before allocating. We didn't do this part yet.  
    \item Note that we move \code{rax} into \code{[r15 + 8]} (and not \code{[r15]}) because \code{rax} has the value of the \emph{second} element of the pair, not the first. 
    \item \code{mov rax, r15} and \code{add rax, 1} is designed to put the location in heap of the pair's values into \code{rax} and then add 1 to \code{rax} for tagging purposes. Note that we can add \code{1} to \code{rax} like this because we know that \code{r15} will end with \code{0000} (this is one of the assumptions we made). 
    \item \code{add r15, 16} moves \code{r15} by 2 words, thus ensuring that it's always pointing to free memory in the heap. 
\end{itemize}
As one might have suspected, once we execute a pair, it should return the memory location to that pair. 

\end{document}