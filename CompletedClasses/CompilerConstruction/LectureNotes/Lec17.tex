\documentclass[letterpaper]{article}
\usepackage[margin=1in]{geometry}
\usepackage[utf8]{inputenc}
\usepackage{textcomp}
\usepackage{amssymb}
\usepackage{natbib}
\usepackage{graphicx}
\usepackage{gensymb}
\usepackage{amsthm, amsmath, mathtools}
\usepackage{xcolor}
\usepackage{enumerate}
\usepackage{framed}
\usepackage{tcolorbox}
\tcbuselibrary{theorems}

\newcommand{\R}{\mathbb{R}}
\newcommand{\Z}{\mathbb{Z}}
\newcommand{\N}{\mathbb{N}}
\newcommand{\Q}{\mathbb{Q}}
\newcommand{\code}[1]{\texttt{#1}}
\newcommand{\mdiamond}{$\diamondsuit$}

%\newtheorem*{theorem}{Theorem}
%\newtheorem*{definition}{Definition}
\newtheorem*{proposition}{Proposition}
%\newtheorem*{corollary}{Corollary}
%\newtheorem*{lemma}{Lemma}

\newtcbtheorem[number within=section]{theorem}{Theorem}
{colback=green!5,colframe=green!35!black,fonttitle=\bfseries}{def}

\newtcbtheorem[number within=section]{definition}{Definition}
{colback=blue!5,colframe=blue!35!black,fonttitle=\bfseries}{def}

\newtcbtheorem[number within=section]{corollary}{Corollary}
{colback=yellow!5,colframe=yellow!35!black,fonttitle=\bfseries}{def}

\newtcbtheorem[number within=section]{lemma}{Lemma}
{colback=red!5,colframe=red!35!black,fonttitle=\bfseries}{def}
\usepackage[utf8]{inputenc}
\usepackage[english]{babel}
\usepackage{fancyhdr}
\usepackage[hidelinks]{hyperref}

\pagestyle{fancy}
\fancyhf{}
\rhead{CSE 131}
\chead{Wednesday, May 10, 2023}
\lhead{Lecture 17}
\rfoot{\thepage}

\setlength{\parindent}{0pt}

\begin{document}

\section{Structured Data: Pairs (Continued)}
In this section, we'll discuss more about structured data, in particular \textbf{pairs}.

\subsection{Revisiting Print}
Let's suppose we have the following program:
\begin{verbatim}
    (let (p (pair 1 2))
        (block 
            (setfst! p p)
            (print p)
        )
    )\end{verbatim}
One thing we should note is that we have a \textbf{cycle} in the sense that the pair is referring to itself. Therefore, if we tried to \code{print} the pair, we would end up with infinite recursion since we would constantly recurse through the first element of the pair. 

\bigskip 

To fix this, we should consider checking if we've \emph{seen} the pair before. If we've seen it, we can print something indicating that a cycle is detected. Otherwise, we can print out the pair as normal. 
\begin{verbatim}
    fn snek_str(val: i64, seen: &mut Vec<i64>) -> String {
        if val == 7 { "true".to_owned()} 
        else if val == 3 { "false".to_owned() } 
        else if val % 2 == 0 { format!("{}", val >> 1) } 
        else if val == 1 { "nil".to_owned() } 
        else if val & 1 == 1 {
            if seen.contains(&val) { return "...".to_owned() }
            seen.push(val);
            let addr = (val - 1) as *const i64; 
            let fst = unsafe { *addr };
            let snd = unsafe { *addr.offset(1) };
            let v = format!("(pair {} {})", snek_str(fst), snek_str(snd));
            seen.pop();
            v 
        } else { format!("unknown value: {val}") }
    }

    #[export_name = "\x01snek_print"]
    fn snek_print(val: i64) -> i64 {
        println!("{}", snek_str(val, &mut vec![]));
        val 
    }\end{verbatim}

\subsection{A Brief Sketch of Equality}
Given two pairs, how can we check if they are equal? We can use the following Rust implementation, 
\begin{verbatim}
    fn snek_eq_helper(val1: i64, val2: i64, seen: &mut Vec<(i64, i64)>) -> bool {
        if seen.contains(&(val1, val2)) { return true }

        seen.push((val1, val2));
        // continue on.
    }\end{verbatim}
The idea is that if we come across the same two cycles, we can assume that they're equal and return. Otherwise, we can evaluate the pairs as usual.


\end{document}