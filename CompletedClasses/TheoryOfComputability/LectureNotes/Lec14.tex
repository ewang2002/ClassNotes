\documentclass[letterpaper]{article}
\usepackage[margin=1in]{geometry}
\usepackage[utf8]{inputenc}
\usepackage{textcomp}
\usepackage{amssymb}
\usepackage{natbib}
\usepackage{graphicx}
\usepackage{gensymb}
\usepackage{amsthm, amsmath, mathtools}
\usepackage[dvipsnames]{xcolor}
\usepackage{enumerate}
\usepackage{mdframed}
\usepackage[most]{tcolorbox}
\usepackage{csquotes}
% https://tex.stackexchange.com/questions/13506/how-to-continue-the-framed-text-box-on-multiple-pages

\tcbuselibrary{theorems}

\newcommand{\R}{\mathbb{R}}
\newcommand{\Z}{\mathbb{Z}}
\newcommand{\N}{\mathbb{N}}
\newcommand{\Q}{\mathbb{Q}}
\newcommand{\C}{\mathbb{C}}
\newcommand{\code}[1]{\texttt{#1}}
\newcommand{\mdiamond}{$\diamondsuit$}
\newcommand{\PowerSet}{\mathcal{P}}
\newcommand{\Mod}[1]{\ (\mathrm{mod}\ #1)}
\DeclareMathOperator{\lcm}{lcm}

%\newtheorem*{theorem}{Theorem}
%\newtheorem*{definition}{Definition}
%\newtheorem*{corollary}{Corollary}
%\newtheorem*{lemma}{Lemma}
\newtheorem*{proposition}{Proposition}


\newtcbtheorem[number within=section]{theorem}{Theorem}
{colback=green!5,colframe=green!35!black,fonttitle=\bfseries}{th}

\newtcbtheorem[number within=section]{definition}{Definition}
{colback=blue!5,colframe=blue!35!black,fonttitle=\bfseries}{def}

\newtcbtheorem[number within=section]{corollary}{Corollary}
{colback=yellow!5,colframe=yellow!35!black,fonttitle=\bfseries}{cor}

\newtcbtheorem[number within=section]{lemma}{Lemma}
{colback=red!5,colframe=red!35!black,fonttitle=\bfseries}{lem}

\newtcbtheorem[number within=section]{example}{Example}
{colback=white!5,colframe=white!35!black,fonttitle=\bfseries}{def}

\newtcbtheorem[number within=section]{note}{Important Note}{
        enhanced,
        sharp corners,
        attach boxed title to top left={
            xshift=-1mm,
            yshift=-5mm,
            yshifttext=-1mm
        },
        top=1.5em,
        colback=white,
        colframe=black,
        fonttitle=\bfseries,
        boxed title style={
            sharp corners,
            size=small,
            colback=red!75!black,
            colframe=red!75!black,
        } 
    }{impnote}
\usepackage[utf8]{inputenc}
\usepackage[english]{babel}
\usepackage{fancyhdr}
\usepackage[hidelinks]{hyperref}

\pagestyle{fancy}
\fancyhf{}
\rhead{CSE 105}
\chead{Monday, February 28, 2022}
\lhead{Lecture 14}
\rfoot{\thepage}

\setlength{\parindent}{0pt}

\begin{document}

\section{Undecidability (4.2)} 
We will now talk about several computationally unsolvable problems, along with proving that such problems are actually unsolvable.

\subsection{Computational Problems}
We begin by going over some examples of computational problems. 

\subsubsection{Simple Problems: DFAs}
Recall the following examples from the previous notes. 
\begin{itemize}
    \item Check whether a string is accepted by a DFA. 
    \[A_{\text{DFA}} = \{\cyclic{B, w} \mid B \text{ is a DFA over } \Sigma, w \in \Sigma^*, w \in L(B)\}\]

    \item Check whether the language of a DFA is empty. 
    \[E_{\text{DFA}} = \{\cyclic{A} \mid A \text{ is a DFA and } L(A) = \emptyset\}\]
    
    \item Check whether the languages of two DFAs are equal.
    \[EQ_{\text{DFA}} = \{\cyclic{A, B} \mid A \text{ and } B \text{ are DFAs and } L(A) = L(B)\}\]
\end{itemize}
These are \textbf{all} decidable; we did this by constructing algorithms that could solve them. 

\subsubsection{More Complex Problems: Pushdown Automaton}
Consider the following examples.
\begin{itemize}
    \item Check whether a string is accepted by a PDA. 
    \[A_{\text{PDA}} = \{\cyclic{B, w} \mid B \text{ is a PDA over } \Sigma, w \in \Sigma^*, w \in L(B)\}\]

    \item Check whether the language of a PDA is empty. 
    \[E_{\text{PDA}} = \{\cyclic{A} \mid A \text{ is a PDA and } L(A) = \emptyset\}\]
    
    \item Check whether the languages of two PDAs are equal.
    \[EQ_{\text{PDA}} = \{\cyclic{A, B} \mid A \text{ and } B \text{ are PDAs and } L(A) = L(B)\}\]
\end{itemize}
Some of these are decidable, and some are not. We note that, because a PDA has a stack which can be infinite in size, we cannot just simulate a PDA with the infinite because there can be infinitely many options. 

\subsubsection{Complex Problems: Turing Machine}
Consider the following examples.
\begin{itemize}
    \item Check whether a string is accepted by a TM. 
    \[A_{\text{TM}} = \{\cyclic{B, w} \mid B \text{ is a TM over } \Sigma, w \in \Sigma^*, w \in L(B)\}\]

    \item Check whether the language of a TM is empty. 
    \[E_{\text{TM}} = \{\cyclic{A} \mid A \text{ is a TM and } L(A) = \emptyset\}\]
    
    \item Check whether the languages of two TMs are equal.
    \[EQ_{\text{TM}} = \{\cyclic{A, B} \mid A \text{ and } B \text{ are TMs and } L(A) = L(B)\}\]
\end{itemize}
These are all undecidable. There are no algorithms which can solve $A_{\text{TM}}$, for example.

\subsection{\texorpdfstring{$A_{\text{TM}}$}{Checking if TM Accepts String}: An Algorithm}
We will now consider an algorithm for solving $A_{\text{TM}}$. For an input $\cyclic{M, w}$, define the Turing machine $N$ as follows.
\begin{enumerate}
    \item Simulate $M$ on input $w$. 
    \item If $M$ ever enters its accept state, then accept; if $M$ ever enters its reject state, then reject.
\end{enumerate}
Then, we can define
\[L(N) = A_{\text{TM}}\]
and say that $N$ is a \textbf{recognizer} for $A_{\text{TM}}$. However, $N$ is not a \emph{decider} for $A_{\text{TM}}$ because it's possible that the Turing machine $M$ can \emph{loop} (i.e. not accept or reject); if $M$ loops, then so will $N$.

\subsection{Diagonalization Method}
We now talk about the diagonalization method as a way to show that $A_{\text{TM}}$ is undecidable.

\subsubsection{Proof}
\begin{mdframed}[]
    \begin{proof}
        Assume towards a contradiction that $A_{\text{TM}}$ is decidable. Then, call $M_{A_{\text{TM}}}$ the decider for $A_{\text{TM}}$. Then, for every Turing machine $M$ and every string $w$: 
        \begin{itemize}
            \item The computation of $M_{A_{\text{TM}}}$ on $\cyclic{M, w}$ halts and accepts if $w$ is in $L(M)$. 
            \item The computation of $M_{A_{\text{TM}}}$ on $\cyclic{M, w}$ halts and rejects if $w$ is not in $L(M)$.
        \end{itemize}
        For some input $\cyclic{M}$, we can define the Turing machine $D$ as follows.
        \begin{itemize}
            \item Run $M_{A_{\text{TM}}}$ on $\cyclic{M, \cyclic{M}}$. 
            \item If $M_{A_{\text{TM}}}$ accepts, then $D$ rejects. Likewise, if $M_{A_{\text{TM}}}$ rejects, then $D$ accepts.
        \end{itemize}
        Now, suppose we run $D$ on input $\cyclic{D}$. Because $D$ is a decider, then either the computation halts and accepts, or the computation halts and rejects. Now: 
        \begin{itemize}
            \item If $D(\cyclic{D})$ accepts, then $M_{A_{\text{TM}}}$ with $\cyclic{D, \cyclic{D}}$ rejects. So, $\cyclic{D, \cyclic{D}} \notin A_{\text{TM}}$ and so $D(\cyclic{D})$ rejects.
            \item If $D(\cyclic{D})$ rejects, then $M_{A_{\text{TM}}}$ with $\cyclic{D, \cyclic{D}}$ accepts. So, $\cyclic{D, \cyclic{D}} \in A_{\text{TM}}$ and so $D(\cyclic{D})$ accepts.
        \end{itemize}
        This is a contradiction since $D$ cannot accept its own input and cannot reject its own input and cannot halt. 
    \end{proof}
\end{mdframed}

\subsubsection{Another Way of Seeing It}
Recall that we can list all of the Turing machines since they are countable: 
\[M_1, M_2, M_3, \dots\]
Suppose we also have a list of \emph{distinct} inputs, call them: 
\[w_1, w_2, w_3, \dots\]
We can create this two-dimensional table where $A_{i, j}$ is defined to be the cell $[i, j]$, which is the result of running $M_i$ on input $w_j$; that is, the cell will either be 1 if $w_J \in L(M_i)$ and 0 if not.  
\begin{center}
    \begin{tabular}{c|c|c|c|c}
                & $w_1$ & $w_2$ & $w_3$ & \dots \\
        \hline 
        $M_1$   &   1   &   0   &   1   & \\
        \hline 
        $M_2$   &   0   &   0   &   1   & \\
        \hline 
        $M_3$   &       &       &       & \\
        \hline 
        \vdots  &       &       &       &
    \end{tabular}
\end{center}
Note that we made up these entries. Suppose we take the opposite of the diagonals: 
\begin{center}
    \begin{tabular}{c|c|c|c|c}
                & $w_1$ & $w_2$ & $w_3$ & \dots \\
        \hline 
        $M_1$   & \textbf{0} &   0   &   1   & \\
        \hline 
        $M_2$   &   0   & \textbf{1} &   1   & \\
        \hline 
        $M_3$   &       &       &       & \\
        \hline 
        \vdots  &       &       &       &
    \end{tabular}
\end{center}

We claim that there is no Turing machine $M$ such that $M(w_j)$ accepts if and only if $A_{j, j} = 0$ for all $j$. So, if $M_1$ accepted $w_1$, then this Turing machine will reject $w_1$; if $M_2$ rejected $w_2$, then this Turing machine will accept $w_2$. 

\begin{mdframed}[]
    \begin{proof}
        We know that $M \neq M_i$ because $M(w_i) \neq M_{i}(w_i)$.
    \end{proof}
\end{mdframed}
The proof in the previous section is simply what we have here when $w_i = \cyclic{M_i}$. 

\subsection{Summary of this Algorithm}
So, we have shown that this language is not decidable. Note that this language is also recognizable. As a general fact, a language is decidable if and only if it and its complement are both recognizable. Additionally, the complement of $A_{\text{TM}}$ is unrecognizable. 

\subsection{The Halting Problem}
Consider the Halting problem, defined by 
\[\text{HALT} = \{(\cyclic{M}, w) \mid M(w) \text{ hals}\}\]
We can actually make use of the fact that $A_{\text{TM}}$ is undecidable to show that $\text{HALT}_{\text{TM}}$ is undecidable.

\bigskip 

Consider the code: 
\begin{verbatim}
    f(n):
        while n >= 1:
            if n is even: 
                n = n / 2
            else:
                n = 3n + 1
\end{verbatim}
Is $f(n)$ halt for all inputs? We don't know. 

\end{document}