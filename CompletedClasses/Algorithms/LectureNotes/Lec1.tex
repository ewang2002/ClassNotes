\documentclass[letterpaper]{article}
\usepackage[margin=1in]{geometry}
\usepackage[utf8]{inputenc}
\usepackage{textcomp}
\usepackage{amssymb}
\usepackage{natbib}
\usepackage{graphicx}
\usepackage{gensymb}
\usepackage{amsthm, amsmath, mathtools}
\usepackage[dvipsnames]{xcolor}
\usepackage{enumerate}
\usepackage{mdframed}
\usepackage[most]{tcolorbox}
\usepackage{csquotes}
% https://tex.stackexchange.com/questions/13506/how-to-continue-the-framed-text-box-on-multiple-pages

\tcbuselibrary{theorems}

\newcommand{\R}{\mathbb{R}}
\newcommand{\Z}{\mathbb{Z}}
\newcommand{\N}{\mathbb{N}}
\newcommand{\Q}{\mathbb{Q}}
\newcommand{\C}{\mathbb{C}}
\newcommand{\code}[1]{\texttt{#1}}
\newcommand{\mdiamond}{$\diamondsuit$}
\newcommand{\PowerSet}{\mathcal{P}}
\newcommand{\Mod}[1]{\ (\mathrm{mod}\ #1)}
\DeclareMathOperator{\lcm}{lcm}

%\newtheorem*{theorem}{Theorem}
%\newtheorem*{definition}{Definition}
%\newtheorem*{corollary}{Corollary}
%\newtheorem*{lemma}{Lemma}
\newtheorem*{proposition}{Proposition}


\newtcbtheorem[number within=section]{theorem}{Theorem}
{colback=green!5,colframe=green!35!black,fonttitle=\bfseries}{th}

\newtcbtheorem[number within=section]{definition}{Definition}
{colback=blue!5,colframe=blue!35!black,fonttitle=\bfseries}{def}

\newtcbtheorem[number within=section]{corollary}{Corollary}
{colback=yellow!5,colframe=yellow!35!black,fonttitle=\bfseries}{cor}

\newtcbtheorem[number within=section]{lemma}{Lemma}
{colback=red!5,colframe=red!35!black,fonttitle=\bfseries}{lem}

\newtcbtheorem[number within=section]{example}{Example}
{colback=white!5,colframe=white!35!black,fonttitle=\bfseries}{def}

\newtcbtheorem[number within=section]{note}{Important Note}{
        enhanced,
        sharp corners,
        attach boxed title to top left={
            xshift=-1mm,
            yshift=-5mm,
            yshifttext=-1mm
        },
        top=1.5em,
        colback=white,
        colframe=black,
        fonttitle=\bfseries,
        boxed title style={
            sharp corners,
            size=small,
            colback=red!75!black,
            colframe=red!75!black,
        } 
    }{impnote}
\usepackage[utf8]{inputenc}
\usepackage[english]{babel}
\usepackage{fancyhdr}
\usepackage[hidelinks]{hyperref}

\pagestyle{fancy}
\fancyhf{}
\rhead{CSE 101}
\chead{Monday, January 03, 2022}
\lhead{Lecture 1}
\rfoot{\thepage}

\setlength{\parindent}{0pt}

\begin{document}

\section{Problem: Fibonacci Numbers}
\begin{definition}{Fibonacci Numbers}{}
    The \textbf{Fibonacci numbers} are the sequence defined by: 
    \[F_0 = F_1 = 1\]
    \[F_n = F_{n - 1} + F_{n - 2} \text{ for } n \geq 2\]
\end{definition}

\subsection{Naive Algorithm}
There is an easy recursive algorithm:
\begin{verbatim}
    Fib(n)
        if n <= 1
            Return 1
        Else 
            Return Fib(n - 1) + Fib(n - 2)
\end{verbatim}

\subsection{Runtime}
Let $T(n)$ be the number of lines of code $\code{Fib}(n)$ needs to execute.
\begin{itemize}
    \item The \code{if} and \code{else} statements make up 2 lines of code. 
    \item Otherwise, we need to run $1 + T(n - 1) + T(n - 2)$ lines.
\end{itemize}
So: 
\[T(n) = \begin{cases}
    2 & n \leq 1 \\ 
    T(n - 1) + T(n - 2) + 3 & \text{Otherwise}
\end{cases}\]
Here, if we want to compute $\code{Fib}(100)$, the runtime would be: 
\[T(100) \approx 2.87 \cdot 10^{21}\]
At a billion lines of code per second, this would take over 90,000 years to run. 

\subsection{Why So Slow?}
This algorithm, in particular, has too many (redundant) recursive calls. For example:
\[f(5) = f(4) + f(3)\]
\[f(4) = f(3) + f(2)\]
\[f(3) = f(2) + f(1)\]
Already, there are some repeated calls. How can we make this more efficient? 

\subsection{Improving this Algorithm}
First, we avoid recomputing things. When we do this by hand, we often already have the last two numbers. So, we can use an array to store the $n$th Fibonacci number. 

\subsection{Improved Algorithm}
If we were to simulate finding the Fibonacci numbers by hand, we would have an algorithm similar to:
\begin{verbatim}
    Fib2(n)
        Initialize A[0..n]
        A[0] = A[1] = 1
        For k = 2 to n
            A[k] = A[k - 1] + A[k - 2]
        Return A[n]
\end{verbatim}
In the first two lines (initializing the array and setting initial values), there are 2 lines. In the \code{for} loop, we run $2(n - 1)$ lines. Finally, we run 1 line of code. So, we have the final result of: 
\[T(n) = 2n + 1\]
With this new algorithm, we have $T(100) = 201$. So, this is easily runnable on almost any computer.

\bigskip 

Essentially, the power of algorithms is: 
\begin{mdframed}
    Sometimes, the right algorithm is the difference between something working and not finishing in your lifetime. 
\end{mdframed}


\end{document}
