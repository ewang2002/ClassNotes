\documentclass[letterpaper]{article}
\usepackage[margin=1in]{geometry}
\usepackage[utf8]{inputenc}
\usepackage{textcomp}
\usepackage{amssymb}
\usepackage{natbib}
\usepackage{graphicx}
\usepackage{gensymb}
\usepackage{amsthm, amsmath, mathtools}
\usepackage[dvipsnames]{xcolor}
\usepackage{enumerate}
\usepackage{mdframed}
\usepackage[most]{tcolorbox}
\usepackage{csquotes}
% https://tex.stackexchange.com/questions/13506/how-to-continue-the-framed-text-box-on-multiple-pages

\tcbuselibrary{theorems}

\newcommand{\R}{\mathbb{R}}
\newcommand{\Z}{\mathbb{Z}}
\newcommand{\N}{\mathbb{N}}
\newcommand{\Q}{\mathbb{Q}}
\newcommand{\C}{\mathbb{C}}
\newcommand{\code}[1]{\texttt{#1}}
\newcommand{\mdiamond}{$\diamondsuit$}
\newcommand{\PowerSet}{\mathcal{P}}
\newcommand{\Mod}[1]{\ (\mathrm{mod}\ #1)}
\DeclareMathOperator{\lcm}{lcm}

%\newtheorem*{theorem}{Theorem}
%\newtheorem*{definition}{Definition}
%\newtheorem*{corollary}{Corollary}
%\newtheorem*{lemma}{Lemma}
\newtheorem*{proposition}{Proposition}


\newtcbtheorem[number within=section]{theorem}{Theorem}
{colback=green!5,colframe=green!35!black,fonttitle=\bfseries}{th}

\newtcbtheorem[number within=section]{definition}{Definition}
{colback=blue!5,colframe=blue!35!black,fonttitle=\bfseries}{def}

\newtcbtheorem[number within=section]{corollary}{Corollary}
{colback=yellow!5,colframe=yellow!35!black,fonttitle=\bfseries}{cor}

\newtcbtheorem[number within=section]{lemma}{Lemma}
{colback=red!5,colframe=red!35!black,fonttitle=\bfseries}{lem}

\newtcbtheorem[number within=section]{example}{Example}
{colback=white!5,colframe=white!35!black,fonttitle=\bfseries}{def}

\newtcbtheorem[number within=section]{note}{Important Note}{
        enhanced,
        sharp corners,
        attach boxed title to top left={
            xshift=-1mm,
            yshift=-5mm,
            yshifttext=-1mm
        },
        top=1.5em,
        colback=white,
        colframe=black,
        fonttitle=\bfseries,
        boxed title style={
            sharp corners,
            size=small,
            colback=red!75!black,
            colframe=red!75!black,
        } 
    }{impnote}
\usepackage[utf8]{inputenc}
\usepackage[english]{babel}
\usepackage{fancyhdr}
\usepackage[hidelinks]{hyperref}

\pagestyle{fancy}
\fancyhf{}
\rhead{CSE 101}
\chead{Wednesday, February 16, 2022}
\lhead{Lecture 17}
\rfoot{\thepage}

\setlength{\parindent}{0pt}

\begin{document}

\section{Greedy Algorithm}
We continue our discussion of greedy algorithms.

\subsection{Problem: Minimum Spanning Trees}
Given a weighted graph $G$, find a minimum spanning tree of $G$. 

\subsubsection{Trees and Cuts}
\begin{lemma}{}{}
    Let $G$ be a weighted graph, and let $C$ be a cut (i.e. a partition of the vertices of $G$ into two sets). 
    \begin{itemize}
        \item Let $e$ be a lowest weight edge crossing $C$. Then, there exists a MST of $G$ that contains $e$. 
        \item Let $e$ be the unique lowest weight edge crossing $C$. Then, every MST of $G$ contains $e$. 
    \end{itemize}
\end{lemma}

\begin{mdframed}[]
    \begin{proof}
        Suppose we have a cut $C$ which splits the vertices $V$ into two subsets $V_1$ and $V_2$. Let $T$ be a MST; then, $T$'s edges will have to cross this cut at some point. If $e \in T$, we're done. Otherwise, we can add the edge to $T$ such that it crosses the cut. This creates a cycle $R$, which implies that $R$ contains some other $e'$ that crosses the cut $C$. Then, let
        \[T' = T \cup \{e\} \setminus \{e'\}\]
        Then, the weight of $T'$ is given by 
        \[w(T') = w(T) + w(e) - w(e')\]
        but since $e$ is a minimum weight edge, and the weight of $e$ must be no more than $e'$, so it follows that 
        \[w(T') \leq w(T)\] 
        Additionally, if $e$ is a unique minimum weight edge, then $w(T') < w(T)$, which is a contradiction as this implies that $T$ wasn't a minimum spanning tree. So, we are done. 
    \end{proof}
\end{mdframed}

\subsubsection{Prim's Algorithm}
Prim's algorithm relies on the above lemma. Let $b(v)$ be the lightest weight edge (for vertices that we have not reached) that we have discovered that allows us to reach $v$ from the vertices that we have reached. 
\begin{verbatim}
    Prims(G):
        T = {}
        b(v) = inf 
        b(s) = 0
        Insert all V's into priority queue Q 
        while Q not empty:
            u = DeleteMin(Q)
            If u != s:
                Add (u, prev(u)) to T 
            For all (u, v) in E: 
                if v in Q and b(v) > l(u, v):
                    b(v) = l(u, v)
                    DecreaseKey(v)
                    prev(v) = u 
        return T 
\end{verbatim}
Basically, what's going on is that: 
\begin{itemize}
    \item For all $v$ in the priority queue (which we haven't reached), $b(v)$ will store the cheapest length of an edge that connects it to some vertex that is already connected to $s$, and \code{prev(v)} will tell you the other end of that best edge.
    \item So, find the vertex in $Q$ with the smallest with the smallest value of $b$, which is the lightest edge which connects some vertex in $u$ which has been connected to $s$ to some vertex that hasn't. We will add that edge to the tree, and then we need to do some updates. 
\end{itemize} 
This algorithm looks nearly like Dijkstra's algorithm, and in fact has the same runtime as Dijkstra's algorithm. That is, $\BigO(|V|\log(|V|) + |E|)$.

\end{document}