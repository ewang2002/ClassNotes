\documentclass[letterpaper]{article}
\usepackage[margin=1in]{geometry}
\usepackage[utf8]{inputenc}
\usepackage{textcomp}
\usepackage{amssymb}
\usepackage{natbib}
\usepackage{graphicx}
\usepackage{gensymb}
\usepackage{amsthm, amsmath, mathtools}
\usepackage[dvipsnames]{xcolor}
\usepackage{enumerate}
\usepackage{mdframed}
\usepackage[most]{tcolorbox}
\usepackage{csquotes}
% https://tex.stackexchange.com/questions/13506/how-to-continue-the-framed-text-box-on-multiple-pages

\tcbuselibrary{theorems}

\newcommand{\R}{\mathbb{R}}
\newcommand{\Z}{\mathbb{Z}}
\newcommand{\N}{\mathbb{N}}
\newcommand{\Q}{\mathbb{Q}}
\newcommand{\C}{\mathbb{C}}
\newcommand{\code}[1]{\texttt{#1}}
\newcommand{\mdiamond}{$\diamondsuit$}
\newcommand{\PowerSet}{\mathcal{P}}
\newcommand{\Mod}[1]{\ (\mathrm{mod}\ #1)}
\DeclareMathOperator{\lcm}{lcm}

%\newtheorem*{theorem}{Theorem}
%\newtheorem*{definition}{Definition}
%\newtheorem*{corollary}{Corollary}
%\newtheorem*{lemma}{Lemma}
\newtheorem*{proposition}{Proposition}


\newtcbtheorem[number within=section]{theorem}{Theorem}
{colback=green!5,colframe=green!35!black,fonttitle=\bfseries}{th}

\newtcbtheorem[number within=section]{definition}{Definition}
{colback=blue!5,colframe=blue!35!black,fonttitle=\bfseries}{def}

\newtcbtheorem[number within=section]{corollary}{Corollary}
{colback=yellow!5,colframe=yellow!35!black,fonttitle=\bfseries}{cor}

\newtcbtheorem[number within=section]{lemma}{Lemma}
{colback=red!5,colframe=red!35!black,fonttitle=\bfseries}{lem}

\newtcbtheorem[number within=section]{example}{Example}
{colback=white!5,colframe=white!35!black,fonttitle=\bfseries}{def}

\newtcbtheorem[number within=section]{note}{Important Note}{
        enhanced,
        sharp corners,
        attach boxed title to top left={
            xshift=-1mm,
            yshift=-5mm,
            yshifttext=-1mm
        },
        top=1.5em,
        colback=white,
        colframe=black,
        fonttitle=\bfseries,
        boxed title style={
            sharp corners,
            size=small,
            colback=red!75!black,
            colframe=red!75!black,
        } 
    }{impnote}
\usepackage[utf8]{inputenc}
\usepackage[english]{babel}
\usepackage{fancyhdr}
\usepackage[hidelinks]{hyperref}

\pagestyle{fancy}
\fancyhf{}
\rhead{MATH 180A}
\chead{Wednesday, April 20, 2022}
\lhead{Lecture 10}
\rfoot{\thepage}

\setlength{\parindent}{0pt}

\begin{document}

\section{Conditional Probability: Continuous Case}
Conditional probability is slightly different in the ``world'' of continuous random variables. Most of the same definitions carry over from the discrete case, but since we're dealing with \emph{PDFs}\footnote{Remember, PDFs are not probabilities; they are densities. They allow us to get a probability by integrating, but they're not probabilities.} $f(x)$ instead of PMFs $p(x) = \PR(X = x)$ (which are probabilities, unlike PDFs).

\bigskip 

Recall that if $\PR(\omega)$ is a discrete probability distribution and $\PR(B) > 0$, then the distribution $\PR(\omega | B) = \frac{\PR(\omega)}{\PR(B)}$ is a probability distribution on $B$.
\begin{definition}{}{}
    Let $X$ be a continuous random variable with PDF $f$. Suppose that $B$ is an event with $\PR(B) > 0$. Then, the conditional PDF of $X$, given $B$, is 
    \[f(x | B) = \begin{cases}
        \frac{f(x)}{\PR(B)} & x \in B \\ 
        0 & x \notin B
    \end{cases}.\]
\end{definition}
We note that $f(x) \geq 0$ and 
\[\int_{-\infty}^{\infty} f(x | B) dx = \frac{1}{\PR(B)} \int_{B} f(x) dx = \frac{1}{\PR(B)} \PR(B) = 1.\]
Note that the integration over a region gives us the probability of being in that region, i.e. integrating $f(x)$ over the region $B$ gives us just $\PR(B)$. Therefore, $f(x | B)$ is indeed a probability density on $B$. 

\begin{mdframed}[]
    (Example.) Recall the spinner example. The location of the spinner, when it eventually comes to rest, is uniform on the unit circle; thus, 
    \[f(x) = 1 \text{ for } x \in [0, 1).\]
    Suppose that the spinner comes to rest in the upper-half of the circle, i.e. in $[0, 1/2]$. What is the (conditional) probability with which it lands in the region $[1/6, 2/3]$?

    \begin{mdframed}[]
        The probability of $B$ occurring (the event that the spinner lands in the upper-half of the circle) is $\frac{1}{2}$ since it is uniform from $[0, 1)$.

        \bigskip 

        Normally, we would integrate from $\frac{1}{6}$ to $\frac{2}{3}$. However, since we know that the spinner lands in the region $[0, 1/2]$, we know that there is no density \emph{outside} of this region. Since $x$ is outside of the region for $\frac{1}{2} \leq x \leq \frac{2}{3}$, we don't need to consider it. Instead, we only need to consider the region $[1/6, 1/2]$. Hence,
        \[\int_{\frac{1}{6}}^{\frac{1}{2}} \frac{1}{1/2} dx = \frac{2/6}{1/2} = \frac{2}{3}.\]
    \end{mdframed}
\end{mdframed}

Recall that the CDF of a continuous random variable is the function 
\[F(x) = \PR(X \leq x) = \int_{-\infty}^{x} f(u) du.\]

\begin{definition}{}{}
    A sequence of random variables $X_1, \dots, X_n$ is said to be \textbf{mutually independent} if their joint CDF is the product of the individual CDF. That is, if 
    \[F(x_1, \dots, x_n) = \PR(X_1 \leq x_1, \dots, X_n \leq x_n) = \prod_{i = 1}^{n} F_{i}(x_i)\]
    for \emph{all} $x_1, \dots, x_n$. 
\end{definition}
By calculus, we can find the joint PDF of a sequence of continuous random variables by differentating: 
\[f(x_1, \dots, x_n) = \frac{\partial^n}{\partial x_1, \dots, \partial x_n} F(x_1, \dots, x_n).\]
From this, we have the following theorem: 
\begin{theorem}{}{}
    $X_1, \dots, X_n$ are mutually independent if and only if their joint PDF is the product of the individual PDFs, $f(x_1, \dots, x_n) = \prod_{i = 1}^{n} f_{i}(x_i)$. 
\end{theorem}
Additionally, consider the following theorem: 
\begin{theorem}{}{}
    If $X_1, \dots, X_n$ are mutually independent, and $f$ is a continuous function, then $f(X_1), \dots, f(X_n)$ are mutually independent.
\end{theorem}

\begin{mdframed}[]
    (Example.) Suppose that $X_1$ and $X_2$ are independent exponential random variables with rates $\lambda_1$ and $\lambda_2$. Find the PDF of $M = \min\{X_1, X_2\}$. 

    \begin{mdframed}[]
        Recall that if $X$ is an exponential rate $\lambda$ random variable if and only if its survival function is 
        \[S(x) = 1 - F(x) = P(X > x) = e^{-\lambda x}.\]
        Since $X_1, X_2$ are independent, their joint PDF is the product (from differentating)
        \[f(x_1, x_2) = f_{1}(x_1) f_{2}(x_2) = \lambda_1 e^{-\lambda_1 x_1} \cdot \lambda_2 e^{-\lambda_2 x_2}.\]
        Also, note that $M = \min\{X_1, X_2\} > m$ if and only if $X_1 > m$ and $X_2 > m$. Hence,
        \begin{equation*}
            \begin{aligned}
                \PR(M > m) &= \int_{m}^{\infty} \int_{m}^{\infty} f(x_1, x_2) dx_1 dx_2 \\ 
                    &= \left(\int_{m}^{\infty} \lambda_1 e^{-\lambda_1 x_1} dx_1\right) \left(\int_{m}^{\infty} \lambda_2 e^{-\lambda_2 x_2} dx_2\right) \\ 
                    &= \PR(X_1 > m) \PR(X_2 > m) \\ 
                    &= e^{-\lambda_1 m} e^{-\lambda_2 m} \\ 
                    &= e^{-(\lambda_1 + \lambda_2)m}.
            \end{aligned}
        \end{equation*}
        Therefore, $M = \min\{X_1, X_2\}$ is an exponential random variable with rate $\lambda_1 + \lambda_2$ (the sum of the rates of $X_1$ and $X_2$). 
    \end{mdframed}
\end{mdframed}

\subsection{Memoryless Property of Exponential Variables}
\begin{theorem}{}{}
    Suppose $X$ is exponential with rate $\lambda$. Then, $X$ has no memory, in the sense that for any $t, s > 0$, 
    \[\PR(X > s + t | X > s) = \PR(X > t).\]
\end{theorem}
That is, given the \emph{survival} of $X$ to time $s$, it then behaves like a ``brand new'' exponential rate $\lambda$ random variable. 

\begin{mdframed}[]
    \begin{proof}
        Note that $\PR(X > x) = e^{-\lambda x}$ for any $x > 0$. Hence, 
        \[\PR(X > s + t | X > s) = \int_{s + t}^{\infty} \frac{\lambda e^{-\lambda x}}{e^{-\lambda s}} dx = e^{\lambda s} \PR(X > s + t) = e^{\lambda s} e^{-\lambda (s + t)} = e^{-\lambda t}.\]
        Hence, this is equal to $\PR(X > t)$ as claimed. 
    \end{proof}
\end{mdframed}

\begin{mdframed}[]
    (Example: Beta Distribution.) The $\text{Beta}(\alpha, \beta)$ random variable, with parameters $\alpha, \beta > 0$, has PDF 
    \[f(x) = B(\alpha, \beta, x) = \frac{x^{\alpha - 1}(1 - x)^{\beta - 1}}{B(\alpha, \beta)}\]
    for $x \in [0, 1]$ (and $f(x) = 0$ otherwise), where 
    \[B(\alpha, \beta) = \int_{0}^{1} x^{\alpha - 1}(1 - x)^{\beta - 1} dx\]
    is the beta integral. Now, when $\alpha, \beta$ are integers, then 
    \[B(\alpha, \beta) = \frac{(\alpha - 1)!(\beta - 1)!}{(\alpha + \beta - 1)!}.\] 
    Note that when $\alpha = \beta = 1$, then $B(\alpha, \beta, x) = 1$. Thus, $\text{Beta}(1, 1)$ is uniform on $[0, 1]$. 
\end{mdframed}
The beta distribution is used in \textbf{Bayesian statistics}. 
\begin{mdframed}[]
    (Example: Drug Testing.) Suppose that a drug has an unknown probability $X$ of being effective. Therefore, the first time it is administered, it would be quite natural to assume that the distribution of $X$ is uniform on $[0, 1]$. However, as time goes on, and more data is collected, we might want to update this distribution. For example, if it is successful in all except 6 of the first 100 trials, a uniform distribution would no longer seem appropriate. 

    \bigskip 

    Suppose that we given this drug to $n$ patients. Assuming independence, we can model this as a series of $n$ Bernoulli trials with (unknown) success probability $X$. If $X = x$, then the probability that the $i$ trials will be successful is the binomial probability 
    \[b(n, x, i) = \binom{n}{i} x^i (1 - x)^{n - i}.\]
    Hence, the conditional PMF is $p(i | x) = b(n, x, i)$. 

    \bigskip 

    If $X$ is uniform on $[0, 1]$, which has PDF of 1 on $[0, 1]$, then the $i$ of the $n$ trials will be successful with probability
    \[\int_0^1 1 \cdot p(i | x) dx = \int_0^1 b(n, x, i)dx = \binom{n}{i} \int_0^1 x^i (1 - x)^{n - i} dx.\]
    We note that 
    \[\int_0^1 x^i (1 - x)^{n - i} dx = B(i + 1, n - i + 1).\]
    Hence, the PMF is given by 
    \[p(i) = \binom{n}{i} B(i + 1, n - i + 1).\]
    Now, still assuming that $X$ is uniform, note that the joint distribution is given by $f(x, i) = p(i | x) \cdot 1 = b(n, x, i)$. So, the conditional PDF is 
    \[f(x | i) = \frac{f(x, i)}{p(i)} = \frac{b(n, x, i)}{\binom{n}{i} B(i + 1, n - i + 1)} = \frac{x^i (1 - x)^{n - i}}{B(i + 1, n - i + 1)},\]
    which is the PDF of a $\text{Beta}(i + 1, n - i + 1)$ random variable.

    \bigskip 

    In fact, we could continue to update this distribution as we go along and continue to collect more data. So, for the first patient, we have seen 0 successes and 0 failures (no data yet). So, we assume that the probability of success on the first patient is distributed as a $\text{Beta}(1, 1) = \text{Uniform}[0, 1]$ random variable. 
\end{mdframed}


\end{document}