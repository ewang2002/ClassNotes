\documentclass[letterpaper]{article}
\usepackage[margin=1in]{geometry}
\usepackage[utf8]{inputenc}
\usepackage{textcomp}
\usepackage{amssymb}
\usepackage{natbib}
\usepackage{graphicx}
\usepackage{gensymb}
\usepackage{amsthm, amsmath, mathtools}
\usepackage[dvipsnames]{xcolor}
\usepackage{enumerate}
\usepackage{mdframed}
\usepackage[most]{tcolorbox}
\usepackage{csquotes}
% https://tex.stackexchange.com/questions/13506/how-to-continue-the-framed-text-box-on-multiple-pages

\tcbuselibrary{theorems}

\newcommand{\R}{\mathbb{R}}
\newcommand{\Z}{\mathbb{Z}}
\newcommand{\N}{\mathbb{N}}
\newcommand{\Q}{\mathbb{Q}}
\newcommand{\C}{\mathbb{C}}
\newcommand{\code}[1]{\texttt{#1}}
\newcommand{\mdiamond}{$\diamondsuit$}
\newcommand{\PowerSet}{\mathcal{P}}
\newcommand{\Mod}[1]{\ (\mathrm{mod}\ #1)}
\DeclareMathOperator{\lcm}{lcm}

%\newtheorem*{theorem}{Theorem}
%\newtheorem*{definition}{Definition}
%\newtheorem*{corollary}{Corollary}
%\newtheorem*{lemma}{Lemma}
\newtheorem*{proposition}{Proposition}


\newtcbtheorem[number within=section]{theorem}{Theorem}
{colback=green!5,colframe=green!35!black,fonttitle=\bfseries}{th}

\newtcbtheorem[number within=section]{definition}{Definition}
{colback=blue!5,colframe=blue!35!black,fonttitle=\bfseries}{def}

\newtcbtheorem[number within=section]{corollary}{Corollary}
{colback=yellow!5,colframe=yellow!35!black,fonttitle=\bfseries}{cor}

\newtcbtheorem[number within=section]{lemma}{Lemma}
{colback=red!5,colframe=red!35!black,fonttitle=\bfseries}{lem}

\newtcbtheorem[number within=section]{example}{Example}
{colback=white!5,colframe=white!35!black,fonttitle=\bfseries}{def}

\newtcbtheorem[number within=section]{note}{Important Note}{
        enhanced,
        sharp corners,
        attach boxed title to top left={
            xshift=-1mm,
            yshift=-5mm,
            yshifttext=-1mm
        },
        top=1.5em,
        colback=white,
        colframe=black,
        fonttitle=\bfseries,
        boxed title style={
            sharp corners,
            size=small,
            colback=red!75!black,
            colframe=red!75!black,
        } 
    }{impnote}
\usepackage[utf8]{inputenc}
\usepackage[english]{babel}
\usepackage{fancyhdr}
\usepackage[hidelinks]{hyperref}

\pagestyle{fancy}
\fancyhf{}
\rhead{MATH 180A}
\chead{Friday, May 6, 2022}
\lhead{Lecture 17}
\rfoot{\thepage}

\setlength{\parindent}{0pt}

\begin{document}

\section{Sums of Random Variables}
We will now work towards the Law of Large Numbers and the Central Limit Theorem. Before we do this, we need to first talk about sums of random variables.

\subsection{Discrete Case}
\begin{theorem}{}{}
    Suppose that $X$ and $Y$ are independent discrete random variables with PMFs $p_X$ and $p_Y$. Then, the PMF of their sum $X + Y$ is the \textbf{convolution} of $p_X$ and $p_Y$. That is, 
    \[p_{X + Y}(z) = \sum_x p_{X}(x) p_{Y}(z - x).\]
\end{theorem}
\textbf{Remark:} We want to find the probability that $X + Y = z$. To do this, we can take the sum of all possible values $X$ can take. Then, $X$ will take on some value and $Y$ will take the rest of the value $z - x$. 

\bigskip 

More generally, if $X_1, \dots, X_n$ are independent, then the PMF for their sum 
\[S_n = \sum_{i = 1}^{n} X_i\]
is the \textbf{$n$-fold convolution}
\[p_{S_n}(z) = \sum_{x_1 + \dots + x_n = z} \left(\prod_{i = 1}^{n} p_{i}(x_i)\right).\]
Alternatively, we note that (this is often useful for an induction proof)
\[p_{S_n}(z) = \sum_{x} p_{S_{n - 1}} p_{n}(z - x)\]
is the convolution of $p_{S_{n - 1}}$ and $p_n$.

\begin{mdframed}[]
    (Example.) Let $X_1, X_2, \dots$ be the result of independent dice rolls. Let $S_2$ be the sum of the first \emph{two} rolls. To find $\{S_2 = 5\}$, we note that 
    \begin{center}
        \begin{tabular}{||c|c||}
            \hline 
            Roll 1 ($x$) & Roll 2 ($5 - x$) \\ 
            \hline \hline 
            1 & 4 \\ 
            2 & 3 \\ 
            3 & 2 \\ 
            4 & 1 \\ 
            \hline 
        \end{tabular}
    \end{center}
    Then, 
    \[\PR(S_2 = 5) = \sum_{x} p_{1}(x) p_{2}(5 - x) = \sum_{x = 1}^{4} \frac{1}{6} \frac{1}{6} = \frac{4}{36} = \frac{1}{9}.\]
\end{mdframed}

\begin{mdframed}[]
    (Example.) Let's suppose that we now want to find $\{S_3 = 4\}$. There are two ways to do this.
    \begin{itemize}
        \item \underline{Approach 1:} We note that 
        \[\PR(S_3 = 4) = \sum_{x_1 + x_2 + x_3 = 4} \frac{1}{6^3} = \frac{3}{6^3},\]
        since the only possibilities are $\{112, 121, 211\}$.

        \item \underline{Approach 2:} We also note that 
        \begin{equation*}
            \begin{aligned}
                \PR(S_3 = 4) &= \sum_{x} \PR(S_2 = x) \PR(X_3 = 4 - x) \\ 
                    &= \sum_{x = 2}^{3} \PR(S_2 = x) \PR(X_3 = 4 - x) \\ 
                    &= \frac{1}{6^2} \frac{1}{6} + \frac{2}{6^2} \frac{1}{6} \\ 
                    &= \frac{3}{6^3}.
            \end{aligned}
        \end{equation*}
        To see how we got this, note that $S_2$ represents the sum of the first two rolls. The minimum value $S_2$ can take is 2 (since the minimum value each die has is 1). The maximum value $S_2$ can take is 3 (since we need to account for the third roll as well). So, we have: 
        \begin{center}
            \begin{tabular}{||c|c||}
                \hline 
                Roll 1 \& 2 ($S_2 = x$) & Roll 3 ($4 - x$) \\ 
                \hline \hline 
                2 & 2 \\ 
                3 & 1 \\ 
                \hline 
            \end{tabular}
        \end{center}
    \end{itemize}
\end{mdframed}

\begin{mdframed}[]
    (Example.) Recall the convolution of $k$ independent Geometric RVs is a Negative Binomial RV (the number of trials until the $k$th ``success.'') What is the convolution of two independent Binomial RVs with the same probability parameter $p$? 

    \begin{mdframed}[]
        Recall that a Binomial random variable with parameters $n$ and $p$ is the distribution of the number of successes in a sequence of $n$ independent experiments, where each experiment is a Bernoulli trial. 

        \bigskip 

        If $X$ is a Binomial random variable with parameters $n$ and $p$, then we can represent it like 
        \[X = B_1 + B_2 + \dots + B_n.\]
        Likewise, if $Y$ is a Binomial random variable with parameters $m$ and $p$, then 
        \[Y = B'_1 + B'_2 + \dots + B'_m.\]
        Thus, the convolution is given by 
        \[X + Y = B_1 + B_2 + \dots + B_n + B'_1 + B'_2 + \dots + B'_m.\]
        Notice that this is also a Binomial random variable with parameters $n + m$ and $p$.
    \end{mdframed}
\end{mdframed}

\subsection{Continuous Case}
The continuous case is very similar to the discrete case, except we make use of integration. 

\begin{theorem}{}{}
    Suppose that $X$ and $Y$ are \textbf{independent} continuous RVs with PDFs $f_X$ and $f_Y$. Then, the PDF of their sum $X + Y$ is the \textbf{convolution} of $f_X$ and $f_Y$. That is, 
    \[f_{X + Y}(z) = \int_{-\infty}^{\infty} f_{X}(x) f_{Y}(z - x) dx.\]
    We can generalize this to sums 
    \[S_n = \sum_{i = 1}^{n} X_i\]
    of independent RVs, as before.
\end{theorem}

\begin{mdframed}[]
    (Example.) Recall the example on the sum $S = M + N$ of two independent Uniform[0, 1] RVs. We found that 
    \[f(s) = \begin{cases}
        s & s \in [0, 1] \\ 
        2 - s & s \in (1, 2] \\ 
        0 & \text{Otherwise}
    \end{cases}.\]
    It is somewhat easier, although essentially equivalent, to do this with convolutions. To do this, note that 
    \[f(s) = \int_{-\infty}^{\infty} f_{M}(u) f_{N}(s - u) du.\]
    Note that $f_{M}(u) f_{N}(s - u) = 1$ if and only if $0 \leq u, s - u \leq 1$ if and only if $u \in [0, 1] \cap [s - 1, s]$. Therefore,
    \[f(s) = \min\{1, s\} - \max\{0, s - 1\} = \begin{cases}
        s & s \in [0, 1] \\ 
        2 - s & s \in (1, 2] \\ 
        0 & \text{Otherwise}
    \end{cases},\] 
    as expected.
\end{mdframed}

\subsection{Normal Random Variables}
The sum of independent Normal RVs is still Normal. Moreover, we add the means and add the variances. 

\begin{theorem}{}{}
    Suppose that $X_1, \dots, X_n$ are independent Normal RVs with means $\mu_i$ and variances $\sigma_i^2$. Then, their sum 
    \[S_n = \sum_{i = 1}^{n} X_i\]
    is normal with mean 
    \[\mu = \sum_{i = 1}^{n} \mu_i\]
    and 
    \[\sigma^2 = \sum_{i = 1}^{n} \sigma_{i}^2.\]
\end{theorem}
\textbf{Remark:} Note that $S_n$ having this sum and variance comes from LoE and the fact that they are independent (so we can add the variances). 

% cauchy stuff past 44min

\end{document}