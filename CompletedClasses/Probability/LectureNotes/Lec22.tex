\documentclass[letterpaper]{article}
\usepackage[margin=1in]{geometry}
\usepackage[utf8]{inputenc}
\usepackage{textcomp}
\usepackage{amssymb}
\usepackage{natbib}
\usepackage{graphicx}
\usepackage{gensymb}
\usepackage{amsthm, amsmath, mathtools}
\usepackage[dvipsnames]{xcolor}
\usepackage{enumerate}
\usepackage{mdframed}
\usepackage[most]{tcolorbox}
\usepackage{csquotes}
% https://tex.stackexchange.com/questions/13506/how-to-continue-the-framed-text-box-on-multiple-pages

\tcbuselibrary{theorems}

\newcommand{\R}{\mathbb{R}}
\newcommand{\Z}{\mathbb{Z}}
\newcommand{\N}{\mathbb{N}}
\newcommand{\Q}{\mathbb{Q}}
\newcommand{\C}{\mathbb{C}}
\newcommand{\code}[1]{\texttt{#1}}
\newcommand{\mdiamond}{$\diamondsuit$}
\newcommand{\PowerSet}{\mathcal{P}}
\newcommand{\Mod}[1]{\ (\mathrm{mod}\ #1)}
\DeclareMathOperator{\lcm}{lcm}

%\newtheorem*{theorem}{Theorem}
%\newtheorem*{definition}{Definition}
%\newtheorem*{corollary}{Corollary}
%\newtheorem*{lemma}{Lemma}
\newtheorem*{proposition}{Proposition}


\newtcbtheorem[number within=section]{theorem}{Theorem}
{colback=green!5,colframe=green!35!black,fonttitle=\bfseries}{th}

\newtcbtheorem[number within=section]{definition}{Definition}
{colback=blue!5,colframe=blue!35!black,fonttitle=\bfseries}{def}

\newtcbtheorem[number within=section]{corollary}{Corollary}
{colback=yellow!5,colframe=yellow!35!black,fonttitle=\bfseries}{cor}

\newtcbtheorem[number within=section]{lemma}{Lemma}
{colback=red!5,colframe=red!35!black,fonttitle=\bfseries}{lem}

\newtcbtheorem[number within=section]{example}{Example}
{colback=white!5,colframe=white!35!black,fonttitle=\bfseries}{def}

\newtcbtheorem[number within=section]{note}{Important Note}{
        enhanced,
        sharp corners,
        attach boxed title to top left={
            xshift=-1mm,
            yshift=-5mm,
            yshifttext=-1mm
        },
        top=1.5em,
        colback=white,
        colframe=black,
        fonttitle=\bfseries,
        boxed title style={
            sharp corners,
            size=small,
            colback=red!75!black,
            colframe=red!75!black,
        } 
    }{impnote}
\usepackage[utf8]{inputenc}
\usepackage[english]{babel}
\usepackage{fancyhdr}
\usepackage[hidelinks]{hyperref}

\pagestyle{fancy}
\fancyhf{}
\rhead{MATH 180A}
\chead{Wednesday, May 18, 2022}
\lhead{Lecture 22}
\rfoot{\thepage}

\setlength{\parindent}{0pt}

\begin{document}

\section{Generating Functions}
We will first begin by covering the \textbf{discrete distributions}. 

\bigskip 

Recall that the mean $\mathbb{E}(X)$ and variance 
\[\text{Var}(X) = \mathbb{E}(X^2) = [\mathbb{E}(X)]^2\]
of a random variable provides useful information about the ``shape'' of a distribution. Note that these quantities only involve the \textbf{first moment} $\mathbb{E}(X)$ and \textbf{second moment} $\mathbb{E}(X^2)$ of $X$. But, what about higher moments? 

\begin{definition}{}{}
    $\mu_n = \mathbb{E}(X^n)$ is called the $n$th moment of the $X$.
\end{definition}
Recall that LotUS tells us how to compute this, using the PMF/PDF of $X$; in particular, 
\[\mathbb{E}(X^n) = \sum_x x^{n} p(x)\]
and 
\[\mathbb{E}(X^n) = \int_{-\infty}^{\infty} x^n f(x) dx.\]

\begin{definition}{}{}
    The \textbf{moment generating function (MGF)} of $X$ is the function 
    \[g(t) = \mathbb{E}(e^{tX}).\]
\end{definition}
\textbf{Remark:} Note that this is the expectation of the random variable $Y = e^{tX}$. 

\bigskip 

Recall the Taylor Series of $e^x$ is given by 
\[e^x = \sum_{n = 0}^{\infty} \frac{x^n}{n!}.\]
Hence, 
\[e^{tX} = \sum_{n = 0}^{\infty} \frac{X^n t^n}{n!}.\]
Therefore, by Linearity of Expectation, we have 
\[g(t) = \mathbb{E}(e^{tX}) = \sum_{n = 0}^{\infty} \mathbb{E}\left(\frac{X^n t^n}{n!}\right) = \sum_{n = 0}^{\infty} \mathbb{E}(X^n) \frac{t^n}{n!} = \sum_{n = 0}^{\infty} \frac{\mu_n t^n}{n!}.\]
\begin{theorem}{}{}
    The MGF \[g(t) = \mathbb{E}(e^{tX}) = \sum_{n = 0}^{\infty} \frac{\mu_n t^n}{n!}.\] Hence, for each $n \geq 1$, we have that \[\frac{d^n}{dt^n} g(0) = \mu_n.\]
\end{theorem}
\textbf{Remark:} We note that the MGF here contains all the information about its moments, which in many cases is enough to uniquely determine a distribution.

\begin{mdframed}[]
    (Example.) A Binomial($n, p$) RV $X$ has MGF 
    \[g(t) = \sum_{k = 0}^{n} e^{tk} \binom{n}{k} p^k q^{n - k} = \sum_{k = 0}^{n} \binom{n}{k} (pe^t)^k q^{n - k} = (pe^t + q)^n,\]
    where $q = 1 - p$.

    \bigskip 

    If can be checked that $g'(0) = np$ and $g''(0) = n(n - 1)p^2 + np$. Hence, as we already know, $\mathbb{E}(X) = np$. Additionally, we know that 
    \begin{equation*}
        \begin{aligned}
            \text{Var}(X) &= \mathbb{E}(X^2) - [\mathbb{E}(X)]^2 \\ 
                &= [n(n - 1)p^2 + np] - (np)^2 \\ 
                &= -np^2 + np \\ 
                &= np(1 - p) \\ 
                &= npq.
        \end{aligned}
    \end{equation*}
\end{mdframed}
\textbf{Remark:} There are some technical points to consider.
\begin{itemize}
    \item Note that $g(0) = 1$. 
    \item The function (series) $g(t)$ may not exist. That is, the series $g(t) = \mathbb{E}(e^{tX}) = \sum_{n = 0}^{\infty} \frac{\mu_n t^n}{n!}$ may not converge in a neighborhood of 0. 
\end{itemize}

\begin{theorem}{}{}
    Suppose that $X$ is a RV that only takes a finite number of possible values. That is, \[\sum_{i = 1}^{n} \PR(X = x_i) = 1,\] for some $x_1, \dots, x_n$. Then, the distribution of $X$ is uniquely determined by its MGF. That is, if some other RV $Y$ has the same MGF, then it has the same distribution as $X$. 
\end{theorem}

\begin{theorem}{}{}
    If the MGFs $M_{X}(t)$ and $M_{Y}(t)$ exists, and for some $\epsilon > 0$, $M_{X}(t) = M_{Y}(t)$ for all $t \in (-\epsilon, \epsilon)$, then the CDFs $F_{X}(t) = F_{Y}(t)$. That is, $X$ and $Y$ have the same distribution.
\end{theorem}
\textbf{Remark:} That is, if (they might not) the MGFs exist and are equal in some (perhaps very small) open neighborhood of 0, then they have the same distribution.

\subsection{Important Properties of the MGF}
Here are some useful properties.
\begin{itemize}
    \item Linearity: 
    \[g_{aX + b}(t) = e^{bt} g_{X}(at),\]
    since $\mathbb{E}(e^{t(aX + b)}) = e^{tb} \mathbb{E}(e^{(at)X}).$

    \item Independence: If $X$ and $Y$ are independent, then
    \[g_{X + Y}(t) = g_{X}(t) g_{Y}(t).\]
    More generally, if $X_1, \dots, X_n$ are independent, then the MGF of their sum \[S_n = \sum_{i = 1}^{n} X_i\] is \[g_{S_n}(t) = \prod_{i = 1}^{n} g_{X_i}(t).\]
    Hence, if $X_1, \dots, X_n$ are IID with common MGF $g(t)$, then 
    \[g_{S_n}(t) = [g(t)]^n.\]
\end{itemize}

\begin{mdframed}[]
    (Example.) A Bernoulli($p$) RV has MGF 
    \[g(t) = \mathbb{E}(e^{tX}) = e^{t(1)}p + e^{t(0)} q = e^t p + q.\]
    This is because a Bernoulli random variable can only take two values: 1 or 0. Since a Binomial is just a sum of these Bernoulli random variables, the MGF of a Binomial($n, p$) has MGF \[(e^t p + q)^n.\]
\end{mdframed}

\end{document}