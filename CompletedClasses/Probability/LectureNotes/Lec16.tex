\documentclass[letterpaper]{article}
\usepackage[margin=1in]{geometry}
\usepackage[utf8]{inputenc}
\usepackage{textcomp}
\usepackage{amssymb}
\usepackage{natbib}
\usepackage{graphicx}
\usepackage{gensymb}
\usepackage{amsthm, amsmath, mathtools}
\usepackage[dvipsnames]{xcolor}
\usepackage{enumerate}
\usepackage{mdframed}
\usepackage[most]{tcolorbox}
\usepackage{csquotes}
% https://tex.stackexchange.com/questions/13506/how-to-continue-the-framed-text-box-on-multiple-pages

\tcbuselibrary{theorems}

\newcommand{\R}{\mathbb{R}}
\newcommand{\Z}{\mathbb{Z}}
\newcommand{\N}{\mathbb{N}}
\newcommand{\Q}{\mathbb{Q}}
\newcommand{\C}{\mathbb{C}}
\newcommand{\code}[1]{\texttt{#1}}
\newcommand{\mdiamond}{$\diamondsuit$}
\newcommand{\PowerSet}{\mathcal{P}}
\newcommand{\Mod}[1]{\ (\mathrm{mod}\ #1)}
\DeclareMathOperator{\lcm}{lcm}

%\newtheorem*{theorem}{Theorem}
%\newtheorem*{definition}{Definition}
%\newtheorem*{corollary}{Corollary}
%\newtheorem*{lemma}{Lemma}
\newtheorem*{proposition}{Proposition}


\newtcbtheorem[number within=section]{theorem}{Theorem}
{colback=green!5,colframe=green!35!black,fonttitle=\bfseries}{th}

\newtcbtheorem[number within=section]{definition}{Definition}
{colback=blue!5,colframe=blue!35!black,fonttitle=\bfseries}{def}

\newtcbtheorem[number within=section]{corollary}{Corollary}
{colback=yellow!5,colframe=yellow!35!black,fonttitle=\bfseries}{cor}

\newtcbtheorem[number within=section]{lemma}{Lemma}
{colback=red!5,colframe=red!35!black,fonttitle=\bfseries}{lem}

\newtcbtheorem[number within=section]{example}{Example}
{colback=white!5,colframe=white!35!black,fonttitle=\bfseries}{def}

\newtcbtheorem[number within=section]{note}{Important Note}{
        enhanced,
        sharp corners,
        attach boxed title to top left={
            xshift=-1mm,
            yshift=-5mm,
            yshifttext=-1mm
        },
        top=1.5em,
        colback=white,
        colframe=black,
        fonttitle=\bfseries,
        boxed title style={
            sharp corners,
            size=small,
            colback=red!75!black,
            colframe=red!75!black,
        } 
    }{impnote}
\usepackage[utf8]{inputenc}
\usepackage[english]{babel}
\usepackage{fancyhdr}
\usepackage[hidelinks]{hyperref}

\pagestyle{fancy}
\fancyhf{}
\rhead{MATH 180A}
\chead{Wednesday, May 4, 2022}
\lhead{Lecture 16}
\rfoot{\thepage}

\setlength{\parindent}{0pt}

\begin{document}

\section{Expected Value and Variance}
\subsection{Conditional Expectation}
Recall that if $X$ is a discrete random variable with PMF $p$, and $B$ is an event with $\PR(B) > 0$, then 
\[p(x | B) = \frac{p(x)}{\PR(B)}\]
is a probability distribution on $B$. This is the PMF of the random variable $X$, given $B$. 

\begin{definition}{}{}
    Let $X$ be a random variable with PMF $p$. Suppose that $\PR(B) > 0$. Then, the conditional expectation of $X$ given $B$ is 
    \[\mathbb{E}(X | B) = \sum_{x} xp(x | B).\]    
\end{definition}
\textbf{Remark:} The situation is similar in the continuous case, but we instead have a conditional PDF 
\[f(x | B) = \frac{f(x)}{\PR(B)}\]
and the conditional expectation is given by  
\[\mathbb{E}(X | B) = \int_{-\infty}^{\infty} xp(x | B) dx.\]  


\subsubsection{Law of Total Expectation}
Just like how there is a Law of Total Probability, there is also a Law of Total Expectation. 
\begin{theorem}{Law of Total Expectation}{}
    Let $X$ be a random variable on sample space $\Omega$. Suppose that $B_1, \dots, B_n$ is a partition $\Omega$. Then, 
    \[\mathbb{E}(X) = \sum_{i = 1}^{n} \mathbb{E}(X | B_i) \PR(B_i).\]
\end{theorem}
This is useful because, often, $\mathbb{E}(X)$ is sometimes difficult to find directly. However, if we condition on a well-chosen $B_i$, then it becomes manageable. 

\begin{mdframed}[]
    (Example.) In the gambling game ``craps,'' a player makes a bet and then rolls a pair of dice. If the sum is 7 or 11, the player wins. If it is 2, 3, or 12, the player loses. If the sum is any other number $s$, the player continues to roll until either another $s$ (they win) or 7 (they lose) occurs (7 is lucky the first time). Now, let $R$ be the number of rolls in a single game of craps.
    \begin{enumerate}
        \item Find $\mathbb{E}(R)$. 
    \end{enumerate}

    \begin{mdframed}[]
        \begin{enumerate}
            \item By the Law of Total Expectation, we have 
            \[\mathbb{E}(R) = \sum_{x = 2}^{12} \mathbb{E}(R | X = x)\PR(X = x),\]
            where $X$ is the initial sum. Note that if 
            \[x \in \{2, 3, 7, 11, 12\},\]
            then \[\mathbb{E}(R | X = x) = 1\]
            since the game is immediately over if we get one of those numbers. In particular,
            \begin{itemize}
                \item There is 1 way to get a 2 (11).
                \item There are 2 ways to get a 3 (12, 21).
                \item There are 6 ways to get a 7 (16, 61, 25, 52, 43, 34).
                \item There are 2 ways to get a 11 (56, 65).
                \item There is 1 way to get a 12 (66).
            \end{itemize}
            Hence, 
            \[\sum_{x \in \{2, 3, 7, 11, 12\}} \mathbb{E}(R | X = x) \PR(X = x) = \frac{12}{36}.\]
            Now, if \[x \in \{4, 5, 6, 8, 9, 10\},\] then we can use a similar argument to the one above. For example, when $x = 4$, we have 3 ways to get 4 (13, 31, 22). This gives us \[\PR(X = 4) = \frac{3}{36}.\] There are also 6 ways to get 7 (16, 61, 25, 52, 43, 34). Therefore, the number of rolls $R$, given that the initial sum is $X = 4$, is distributed as $1 + G$, where $G$ is a geometric random variable (where the success is defined by rolling a 4 or 7) with success probability $p = \frac{9}{36}$. Note that the 1 is there because of the initial roll of the 4, but then we have to keep rolling. Thus, we get  
            \[\mathbb{E}(R | X = 4)\PR(X = 4) = \left(1 + \frac{36}{9}\right) \frac{3}{36}.\]
            To see how we got $\left(1 + \frac{36}{9}\right)$, recall that $E(X) = \frac{1}{p}$ if $X$ is Geometric($p$). So, by Linearity of Expectation, we get that the expected value of 1 (so it's 1) plus the expected value of the geometric (so it's $36 / 9$).

            \bigskip 

            So, by similar reasoning, we get 
            \begin{equation*}
                \begin{aligned}
                    \mathbb{E}(R) = \frac{12}{36} &+ \left(1 + \frac{36}{9}\right) \frac{3}{36} + \left(1 + \frac{36}{10}\right) \frac{4}{36} + \left(1 + \frac{36}{11}\right) \frac{5}{36} \\ 
                        &+ \left(1 + \frac{36}{11}\right) \frac{5}{36} + \left(1 + \frac{36}{10}\right) \frac{4}{36} + \left(1 + \frac{36}{9}\right)\frac{3}{36} = \frac{557}{165} \approx 3.375
                \end{aligned}
            \end{equation*}
            So, on average, we expect a little bit more than 3 and $1/3$ rolls of the dice in each of te game of craps. 
        \end{enumerate}
    \end{mdframed}
\end{mdframed}


\subsubsection{Martingales}
Informally, we can think of martingales as random processes that encapsulate the idea of a \emph{fair game}.
\begin{definition}{}{}
    Let $(X_0, X_1, X_2, \dots)$ be a sequence of random variables and $\phi$ a function. Let $M_n = \phi(X_n)$. The sequence $(M_0, M_1, M_2, \dots)$ is called a \textbf{martingale} (MG) with respect to $(X_0, X_1, X_2, \dots)$ if, for any $n$ and any $x_0, x_1, \dots, x_n$, we have that 
    \[\mathbb{E}(M_{n + 1} - M_{n} | X_{n} = x_n, \dots, X_0 = x_0) = 0.\]
\end{definition}
\textbf{Remark:} If we think of $M_n$ as the total winnings after $n$ bets by a gambler, then $(M_0, M_1, M_2, \dots)$ is a ``fair game'' in the sense that neither the gambler nor the ``house'' has an advantage. In other words, after the $(n + 1)$th game, we would expect no additional gain (hence why this is equal to 0).

\begin{theorem}{}{}
    For any random variables $X$ and $Y$, we have that 
    \[\mathbb{E}[\mathbb{E}(X | Y)] = \mathbb{E}(X).\]
\end{theorem}
\textbf{Remarks:} 
\begin{itemize}
    \item Recall that $\mathbb{E}(M_{n + 1} | X_n, \dots, X_0) = M_n$ for a MG. Then, taking expectations on both sides, it follows that $\mathbb{E}(M_{n + 1}) = \mathbb{E}(M_n)$ for all $n$. Hence, for all (deterministic) times $n \geq 0$, we have that $\mathbb{E}(M_n) = \mathbb{E}(M_0)$. 
    \item These are a powerful tool, which can lead to quick or slick proofs of things that would be computationally challenging otherwise. 
\end{itemize}

\begin{mdframed}[]
    (Example.) Recall, from Lecture 1, that Peter and Paul keeps flipping a coin. If it is ``Heads,'' then Peter wins \$1. Otherwise, Peter loses \$1. Let $X_i$ be Peter's winning after the $i$th bet. Note that $X_0 = 0$.

    \bigskip 

    To see that this is a MG, note that the series of flips are independent. Hence, for any $n$ and any $x_1, \dots, x_n$, we have 
    \[\mathbb{E}(X_{n + 1} - X_{n} | X_n = x_n, \dots, X_1 = x_1) = (1)\frac{1}{2} + (-1) \frac{1}{2} = 0.\]
\end{mdframed}
Recall that $\mathbb{E}(M_n) = \mathbb{E}(M_0)$ for all (deterministic) times $n \geq 0$. However, this is not necessarily true for \emph{random} $T$. Hence, we restrict to a special class of random times. 

\begin{definition}{}{}
    A time $T$ is a \textbf{stopping time} (ST) if and only if, for any $n$, to know if $T = n$, we only need to know the values of $X_0, X_1, \dots, X_n$. That is, we do not need any information about the future after time $n$. 
\end{definition}
For example, the first time we visit 0 is a ST, but the last time we visit 0 is not (in order for us to know if time $n$ is 0, we need to know the full history).

\begin{theorem}{The Optional Stopping Theorem (OST)}{}
    If $(M_0, M_1, \dots)$ is a MG and $T$ is a ST, then $\mathbb{E}(M_T) = \mathbb{E}(M_0)$ if the following conditions are satisfied: 
    \begin{enumerate}
        \item $M_n$ is bounded until time $T$, and 
        \item $\PR(T < \infty) = 1$. 
    \end{enumerate}
\end{theorem}

\begin{mdframed}[]
    (Example: Gambler's Ruin). Suppose that Peter currently has \$1. Furthermore, suppose that they play instead with a coin that comes up Heads with probability $p \neq 1/2$. What is the probability $\PR(J)$ that Peter wins the ``Jackpot'' ($\$N$) before going ``bust'' (\$0)?

    \begin{mdframed}[]
        For this biased RW, we know that \[M_n = (q / p)^{X_n}\] is a MG, where $X_n$ is Peter's winnings. Note that here \[\phi(x) = (q / p)^x.\] Then, 
        \begin{equation*}
            \begin{aligned}
                \mathbb{E}(M_{n + 1} - M_{n} | X_n = x_n, \dots, X_1 = x_1) &= (q / p)^{x_n} (q / p - 1)p + (q / p)^{x_n} (p / q - 1)q \\ 
                    &= (q / p)^{x_n}[(q - p) + (p - q)] \\ 
                    &= 0.
            \end{aligned}
        \end{equation*}
        Now, let $T$ be the first time that $X_n \in \{0, N\}$. By the OST, we know that \[q / p = \mathbb{E}(M_T) = 1 \cdot \PR(J^C) + (q / p)^N \cdot \PR(J).\]
        Hence, \[\PR(J) = \frac{(q / p) - q}{(q / p)^N - 1}.\]
    \end{mdframed}
\end{mdframed}

% 50:00

\end{document}