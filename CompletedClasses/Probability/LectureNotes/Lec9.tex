\documentclass[letterpaper]{article}
\usepackage[margin=1in]{geometry}
\usepackage[utf8]{inputenc}
\usepackage{textcomp}
\usepackage{amssymb}
\usepackage{natbib}
\usepackage{graphicx}
\usepackage{gensymb}
\usepackage{amsthm, amsmath, mathtools}
\usepackage{xcolor}
\usepackage{enumerate}
\usepackage{framed}
\usepackage{tcolorbox}
\tcbuselibrary{theorems}

\newcommand{\R}{\mathbb{R}}
\newcommand{\Z}{\mathbb{Z}}
\newcommand{\N}{\mathbb{N}}
\newcommand{\Q}{\mathbb{Q}}
\newcommand{\code}[1]{\texttt{#1}}
\newcommand{\mdiamond}{$\diamondsuit$}

%\newtheorem*{theorem}{Theorem}
%\newtheorem*{definition}{Definition}
\newtheorem*{proposition}{Proposition}
%\newtheorem*{corollary}{Corollary}
%\newtheorem*{lemma}{Lemma}

\newtcbtheorem[number within=section]{theorem}{Theorem}
{colback=green!5,colframe=green!35!black,fonttitle=\bfseries}{def}

\newtcbtheorem[number within=section]{definition}{Definition}
{colback=blue!5,colframe=blue!35!black,fonttitle=\bfseries}{def}

\newtcbtheorem[number within=section]{corollary}{Corollary}
{colback=yellow!5,colframe=yellow!35!black,fonttitle=\bfseries}{def}

\newtcbtheorem[number within=section]{lemma}{Lemma}
{colback=red!5,colframe=red!35!black,fonttitle=\bfseries}{def}
\usepackage[utf8]{inputenc}
\usepackage[english]{babel}
\usepackage{fancyhdr}
\usepackage[hidelinks]{hyperref}

\pagestyle{fancy}
\fancyhf{}
\rhead{MATH 180A}
\chead{Monday, April 18, 2022}
\lhead{Lecture 9}
\rfoot{\thepage}

\setlength{\parindent}{0pt}

\begin{document}

\section{Conditional Probability}
\subsection{Bayes' Formula}
Bayes' Formula is a powerful -- and very famous -- application of the conditional probability formula. Often, it can be difficult to calculate a conditional probability $\PR(A | B)$ of interest. However, the other way around, $\PR(B | A)$, could be easier to find. Bayes' Formula gives us a way of finding $\PR(A | B)$, provided that we know $\PR(A)$, $\PR(B)$, and $\PR(B | A)$. Recall that the conditional probability formula is given by 
\[\PR(A | B) = \frac{\PR(A \cap B)}{\PR(B)}.\]
We also know that 
\[\PR(B | A) = \frac{\PR(A \cap B)}{\PR(A)}.\]
So, solving for $\PR(A \cap B)$, we have 
\[\PR(A \cap B) = \PR(A | B) \PR(B) \qquad \PR(A \cap B) = \PR(B | A) \PR(A).\]
Then, setting these terms equal to each other, we have 
\[\PR(A | B) \PR(B) = \PR(B | A) \PR(A).\]
Thus, we get 
\[\boxed{\PR(A | B) = \frac{\PR(A) \PR(B | A)}{\PR(B)}}.\]
It is often useful to apply the conditional Law of Total Probability in the denominator; that is,
\[\PR(B) = \PR(A) \PR(B | A) + \PR(A^C) \PR(B | A^C).\]
So, sometimes, Bayes' Rule is stated like so:
\[\boxed{\PR(A | B) = \frac{\PR(A) \PR(B | A)}{\PR(A) \PR(B | A) + \PR(A^C) \PR(B | A^C)}}.\]
Now, there is a general version of Bayes' Rule; suppose that $B \subset \Omega$ is an event and $A_1, \dots, A_n$ partitions the sample space $\Omega$. Then, for each $1 \leq j \leq n$, we have 
\[\boxed{\PR(A_j | B) = \frac{\PR(A_j) \PR(B | A_j)}{\sum_{i = 1}^{n} \PR(A_i) \PR(B | A_i)}}.\]
Here\footnote{In this course, we aren't expected to memorize these.}, 
\begin{itemize}
    \item The $\PR(A_j)$ are called \textbf{prior probabilities}.
    \item The $\PR(A_j | B)$ are called \textbf{posterior probabilities}.
    \item The events $A_j$ are called \textbf{hypotheses}.
    \item The event $B$ is called the \textbf{evidence}.
\end{itemize}
Often, we want to see which hypothesis is more likely given the evidence.

\begin{mdframed}[]
    (Example.) Suppose that a doctor gives a patient a test for cancer. Before the test, all we know is that, on average, 1 in every 1000 women develop this cancer. According to the manufacturer, this test is $99\%$ accurate at detecting the cancer, when it is there. However, there is a $5\%$ chance it will show positive when the cancer is not there (i.e. there is a $1\%$ chance of a false negative, and a $5\%$ chance of a false positive). To make things more clear:
    \begin{itemize}
        \item If she has the cancer, there's a $1\%$ chance the test will incorrectly say negative, and a $99\%$ chance the test will correctly say positive. 
        \item If she does not have the cancer, there's a $5\%$ chance the test will incorrectly say positive.
    \end{itemize}
    Now, suppose that the patient tests positive. \emph{With what probability does she actually have the cancer?}

    \begin{mdframed}[]
        Let $C$ be the event that she has this cancer. Let $P$ be the event that this test is positive. We want to find $\PR(C | P)$. We know that 
        \[\PR(C) = 0.001 \qquad \PR(P | C) = 0.99 \qquad \PR(P | C^C) = 0.05.\]
        By Bayes' Rule, we have 
        \[\PR(C | P) = \frac{\PR(C) \PR(P | C)}{\PR(C) \PR(P | C) + \PR(C^C) \PR(P | C^C)} = \frac{0.001(0.99)}{0.001(0.99) + 0.999(0.05)} \approx 1.94\%.\]
    \end{mdframed}
\end{mdframed}
\textbf{Remark:} Given that the test is supposed to be $99\%$ accurate and that the patient tested positive, one would think that the patient has the cancer; so, a near $2\%$ probability that the patient has cancer is quite surprising. However, there are a few reasons why this may be the case.
\begin{enumerate}
    \item Even though the test is fairly accurate when you do have the cancer, when you don't have the cancer there is a decent chance ($5\%$) that it will be incorrect.
    \item This cancer is \emph{extremely} rare; it's so rare that there is a much better chance that this test will incorrectly read positive than if you actually have cancer. 
\end{enumerate}
Now, without any additional information (i.e. without the test), we could only assume that the patient had this cancer with probability $0.1\%$. After testing positive, this probability increases by quite a bit, but it is still only $1.94\%$. This example demonstrates that it is very difficult to design an accurate test for a rare disease; in this example, the probability of a false positive is much more likely than the patient actually having cancer.

% tree diagram 24:00

\end{document}