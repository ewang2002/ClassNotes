\documentclass[letterpaper]{article}
\usepackage[margin=1in]{geometry}
\usepackage[utf8]{inputenc}
\usepackage{textcomp}
\usepackage{amssymb}
\usepackage{natbib}
\usepackage{graphicx}
\usepackage{gensymb}
\usepackage{amsthm, amsmath, mathtools}
\usepackage{xcolor}
\usepackage{enumerate}
\usepackage{framed}
\usepackage{tcolorbox}
\tcbuselibrary{theorems}

\newcommand{\R}{\mathbb{R}}
\newcommand{\Z}{\mathbb{Z}}
\newcommand{\N}{\mathbb{N}}
\newcommand{\Q}{\mathbb{Q}}
\newcommand{\code}[1]{\texttt{#1}}
\newcommand{\mdiamond}{$\diamondsuit$}

%\newtheorem*{theorem}{Theorem}
%\newtheorem*{definition}{Definition}
\newtheorem*{proposition}{Proposition}
%\newtheorem*{corollary}{Corollary}
%\newtheorem*{lemma}{Lemma}

\newtcbtheorem[number within=section]{theorem}{Theorem}
{colback=green!5,colframe=green!35!black,fonttitle=\bfseries}{def}

\newtcbtheorem[number within=section]{definition}{Definition}
{colback=blue!5,colframe=blue!35!black,fonttitle=\bfseries}{def}

\newtcbtheorem[number within=section]{corollary}{Corollary}
{colback=yellow!5,colframe=yellow!35!black,fonttitle=\bfseries}{def}

\newtcbtheorem[number within=section]{lemma}{Lemma}
{colback=red!5,colframe=red!35!black,fonttitle=\bfseries}{def}
\usepackage[utf8]{inputenc}
\usepackage[english]{babel}
\usepackage{fancyhdr}
\usepackage[hidelinks]{hyperref}

\pagestyle{fancy}
\fancyhf{}
\rhead{Math 103B}
\chead{Monday, January 10, 2022}
\lhead{Lecture 4}
\rfoot{\thepage}

\setlength{\parindent}{0pt}

\begin{document}

\section{Characteristic of a Ring}
Consider the ring $\Z_{3}[i]$, with the elements:
\[\{0, 1, 2, i, 1 + i, 2 + i, 2i, 1 + 2i, 2 + 2i\}\]
For any element $x$ in this ring, we have: 
\[3x = x + x + x = 0\]
For example:
\begin{itemize}
    \item $2i + 2i + 2i = 6i = 0i = 0$
    \item $(1 + 2i) + (1 + 2i) + (1 + 2i) = 3 + 6i = 0$
    \item And so on.
\end{itemize}
Similarly, in the ring $\{0, 3, 6, 9\} \subset \Z_{12}$, we have, for all $x$: 
\[4x = x + x + x + x = 0\]

\subsection{Characteristic of a Ring}
\begin{definition}{Characteristic of a Ring}{}
    The \textbf{characteristic} of a ring $R$ is the least positive integer $n$ such that $nx = 0$ for all $x \in R$. If no such integer exists, we say that $R$ has characteristic 0. The characteristic of $R$ is denoted by $\ch{R}$.    
\end{definition}
So, for example, the ring of integers $\Z$ has characteristic 0 and $\Z_n$ has characteristic $n$. For example, consider $\Z_{3} = \{0, 1, 2\}$. Then, we know that:
\[3x = x + x + x = 0 \qquad \forall x\]
So the characteristic of $\Z_{3}$ is \boxed{3}. Now, consider $\Z_{6}$. We know that:
\[6x = x + x + x + x + x + x = 0 \qquad \forall x\]
So, its characteristic is \boxed{6}. As a final example, $\{0\}$ has characteristic \boxed{1}.

\subsection{Characteristic of a Ring with Unity}
Occasionally, we might have more complicated rings where the above theorem may be hard to apply. 
\begin{theorem}{Characteristic of a Ring with Unity}{}
    Let $R$ be a ring with unity 1. If 1 has infinite order under addition, then the characteristic of $R$ is 0. If 1 has order $n$ under addition, then the characteristic of $R$ is $n$. 
\end{theorem}
\textbf{Remark:} Here, suppose $(\R, +)$ is a group. Then, we say that $x \in \R$ has an additive order $n$ if $nx = 0$ and $n$ is the smallest positive number with this property. 

\begin{mdframed}[]
    \begin{proof}
        Suppose 1 has infinite order. Then, there is no positive integer $n$ such that $n \cdot 1 = 0$, so $R$ must have characteristic 0. Now, let's suppose that 1 does have additive order $n$. Then, we know that $n \cdot 1 = 0$ and $n$ is the least positive integer with this property. So, for any $x \in R$, we have: 
        \begin{equation*}
            \begin{aligned}
                n \cdot x &= \overbrace{x + x + \dots + x}^{n \text{ times}} \\ 
                    &= \overbrace{1x + 1x + \dots + 1x}^{n \text{ times}} \\ 
                    &= (\overbrace{1 + 1 + \dots + 1}^{n \text{ times}})x \\ 
                    &= (n \cdot 1)x = 0x = 0
            \end{aligned}
        \end{equation*}
        So, $R$ has characteristic $n$.
    \end{proof}
\end{mdframed}
For example, take $R = \Z / 6\Z \oplus \Z / 4\Z \oplus \Z / 10\Z$.
\begin{enumerate}
    \item \underline{Does this ring have unity?} Each member of this direct product ring has 1, so the unity would be $(1, 1, 1) \in R$. 
    \item \underline{What is the characteristic of $R$?} The characteristic order of $R$ is the additive order of $(1, 1, 1) \in R$. Well, we have that: 
    \[n(1, 1, 1) = (n1, n1, n1)\]
    Consider the first element in the pair. When is $n1 \equiv 0 \Mod{6}$? This is when $6 | n$, or: 
    \[n \in \{6, 12, 18, 24, \dots\}\]
    For the third element in the pair, we need to know when $n1 \equiv 0 \Mod{10}$. This is when $10 | n$, or: 
    \[n \in \{10, 20, 30, \dots\}\]
    Here, it's clear that the answer is $\lcm(6, 4, 10) = 60$. 
\end{enumerate}


\begin{theorem}{Characteristic of an Integral Domain}{}
    The characteristic of an integral domain is 0 or prime.
\end{theorem}

\begin{mdframed}[]
    \begin{proof}
        It suffices to consider the additive order of 1. Suppose towards a contradiction that 1 has composite order $n$ and $1 < s$ and $t < n$ such that $n = st$. Then, we know that: 
        \[0 = n1 = (st)1 = s(t1) = (s1)(t1)\]
        But, $1 < s$ and $t < n$, so by minimality of $n$ being the order of 1, it must be that $s1, t1 \neq 0$ and are thus zero-divisors. But, this is a contradiction.
    \end{proof}
\end{mdframed}

\subsubsection{Problem 1: Field}
Suppose $F$ is a field of order $3^n$. Show that $\ch{F} = 3$. 

\begin{mdframed}[]
    \begin{proof}
        % F has unity so we can use this
        We know that $\ch{F}$ has order $1$. Well, the order of 1 divides the order of $F$. The order of $F$ is $3^n$, so the order of 1 is potentially $1, 3^1, 3^2, \dots, 3^n$. But, the characteristic of an integral domain is prime, so the answer must be 3. 
    \end{proof}
\end{mdframed}

\subsubsection{Problem 2: Field}
Find the characteristic of $\Z / 4\Z \oplus 4\Z$. 

\begin{mdframed}[]
    \begin{proof}
        We know that $\ch{4\Z} = 0$, so $\ch{\Z / 4\Z \oplus 4\Z} = 0$. If we pick $(1, 4) \in \Z / 4\Z \oplus 4\Z$, then for any positive integer $n$:
        \[n(1, 4) = (n, 4n) \neq (0, 0)\]
        So, we can't find one.
    \end{proof}
\end{mdframed}

\subsection{Summary of Rings}
\begin{center}
    \begin{tabular}{c|c|c}
        \textbf{Ring}   & \textbf{Characteristic} & \textbf{Integral Domain?} \\ 
        \hline 
        $\Z$            & 0                       & Yes \\ 
        $M_{2}(\Z)$     & 0                       & No \\ 
        $\Z \oplus \Z$  & 0                       & No \\ 
        $\F_p$ ($\Z / p\Z$) & $p$                 & Yes \\ 
        $\F_p \oplus \F_p$ & $p$                  & No \\ 
        $\F_{p}[x]$     & $p$                     & Yes \\ 
        $\Z / n\Z[i]$   & $n$                     & $\begin{cases}
            \text{No} & n \text{ not prime.} \\ 
            \text{Maybe} & n \text{ prime.}
        \end{cases}$
    \end{tabular}
\end{center}

\end{document}