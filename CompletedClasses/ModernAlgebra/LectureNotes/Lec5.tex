\documentclass[letterpaper]{article}
\usepackage[margin=1in]{geometry}
\usepackage[utf8]{inputenc}
\usepackage{textcomp}
\usepackage{amssymb}
\usepackage{natbib}
\usepackage{graphicx}
\usepackage{gensymb}
\usepackage{amsthm, amsmath, mathtools}
\usepackage{xcolor}
\usepackage{enumerate}
\usepackage{framed}
\usepackage{tcolorbox}
\tcbuselibrary{theorems}

\newcommand{\R}{\mathbb{R}}
\newcommand{\Z}{\mathbb{Z}}
\newcommand{\N}{\mathbb{N}}
\newcommand{\Q}{\mathbb{Q}}
\newcommand{\code}[1]{\texttt{#1}}
\newcommand{\mdiamond}{$\diamondsuit$}

%\newtheorem*{theorem}{Theorem}
%\newtheorem*{definition}{Definition}
\newtheorem*{proposition}{Proposition}
%\newtheorem*{corollary}{Corollary}
%\newtheorem*{lemma}{Lemma}

\newtcbtheorem[number within=section]{theorem}{Theorem}
{colback=green!5,colframe=green!35!black,fonttitle=\bfseries}{def}

\newtcbtheorem[number within=section]{definition}{Definition}
{colback=blue!5,colframe=blue!35!black,fonttitle=\bfseries}{def}

\newtcbtheorem[number within=section]{corollary}{Corollary}
{colback=yellow!5,colframe=yellow!35!black,fonttitle=\bfseries}{def}

\newtcbtheorem[number within=section]{lemma}{Lemma}
{colback=red!5,colframe=red!35!black,fonttitle=\bfseries}{def}
\usepackage[utf8]{inputenc}
\usepackage[english]{babel}
\usepackage{fancyhdr}
\usepackage[hidelinks]{hyperref}

\pagestyle{fancy}
\fancyhf{}
\rhead{Math 103B}
\chead{Wednesday, January 12, 2022}
\lhead{Lecture 5}
\rfoot{\thepage}

\setlength{\parindent}{0pt}

\begin{document}

\section{Quotient Rings}
Recall that if $H$ is a \emph{normal} subgroup of $G$, then there exists a quotient group $G / H$ defined by: 
\[G / H = \{gH \mid g \in G\}\]
Where the operation of the quotient group is: 
\[(g_1 H)(g_2 H) = (g_1 g_2) H\]

\subsection{Ideals}
\begin{definition}{Ideal}{}
    A subring $A$ of a ring $R$ is called a (two-sided) \textbf{ideal} of $R$ if for every element $r \in R$ and every $a \in A$ then: 
    \[ra \in A \text{ and } ar \in A\]
    That is, $rA = \{ra \mid a \in A\} \subseteq A$ and $Ar \subseteq A$. 
\end{definition}

\begin{definition}{Proper Ideal}{}
    An ideal $A$ is called \textbf{proper} if $A \subset R$. 
\end{definition}

\subsubsection{Example 1: Even Integers}
$2\Z \subseteq \Z$ is an ideal. Suppose that there is some integer $r \in \Z$ and $a \in 2\Z$. Then, $a = 2k$ for some $k \in \Z$ so that $ra = r \cdot 2k = 2(rk) \in 2\Z$. Note that this also extends to any $n\Z$; that is, $n\Z$ is an ideal. 

\subsubsection{Example 2: Trivial Subring}
$\{0\} \subseteq R$ is a trivial ideal because $r \{0\} = \{0\} r = \{0\}$.

\subsubsection{Example 3: Integers/Rationals}
$\Z \subseteq \Q$ is \emph{not} an ideal. Take $r = \frac{1}{2} \in \Q$ and $a = 1 \in \Z$. Then:
\[ra = \frac{1}{2} (1) = \frac{1}{2} \notin \Z\]

\subsection{Ideal Test}
\begin{theorem}{Ideal Test}{}
    A nonempty subset $A \subseteq R$ is an ideal if and only if: 
    \begin{enumerate}
        \item $a, b \in A \implies a - b \in A$.
        \item $a \in A, r \in R \implies ra, ar \in R$.
    \end{enumerate}
\end{theorem}

\begin{mdframed}[]
    \begin{proof}
        This is similar to the subring test.
    \end{proof}
\end{mdframed}

\subsection{Principal Ideal}
If $R$ is a commutative ring with unity, then the \underline{principal ideal} generated by $a \in R$ is:
\[\cyclic{a} = (a) = \{ra \mid r \in R\}\]

\begin{mdframed}[]
    \begin{proof}
        Pick two elements $ra, sa \in \cyclic{a}$. Then, $ra - sa = (r - s)a \in \cyclic{a}$. Likewise, if $r \in R$, then $sa \in \cyclic{a}$ so:
        \[(sa)r = r(sa) = (rs)a \in \cyclic{a}\]
        So, we are done. 
    \end{proof}
\end{mdframed}

\subsubsection{Example 4: Integers}
Consider $R = \Z$ with $\cyclic{n} = n\Z$. This is a principal ideal. 

\subsubsection{Example 5: Polynomials}
If $R = \R[x]$, then:
\begin{equation*}
    \begin{aligned}
        \cyclic{x} &= \{f(x)x \mid f(x) \in \R[x]\} \\ 
            &= \{\text{Polynomials divisible by } x\} \\
            &= \{f(x) \in \R[x] \mid f(0) = 0\}
    \end{aligned}
\end{equation*}

\subsubsection{Example 6: Ring of Unity}
The ideal generated by $a_1, a_2, \dots, a_n \in R$, where $R$ is a commutative ring of unity, is:
\[\cyclic{a_1, a_2, \dots, a_n} = \{r_1 a_1 + r_2 a_2 + \dots + r_n a_n \mid r_1, \dots, r_n \in R\}\]

\subsubsection{Example 7: Two Elements}
Consider $\cyclic{2, x} \subseteq \Z[x]$. This is defined by: 
\[\{f(x) \in \Z[x] \mid f(0) \text{ is even.}\}\]


\subsubsection{Problem 1: Ideal}
Consider $I = \cyclic{a_1, a_2, \dots, a_n} = \{r_1 a_1 + \dots + r_n a_n \mid r_i \in R\}$. Show that this is an ideal. 

\begin{mdframed}[]
    \begin{proof}
        We use the ideal test. 
        \begin{enumerate}
            \item If we take two elements $a, b \in I$, then it's trivial to see that $a - b \in I$.
            \item Likewise, if we took $a \in I$ and $r \in R$, then it's trivial to see that $ra \in I$. 
        \end{enumerate}
        So, we are done. 
    \end{proof}
\end{mdframed}

\subsection{Quotient Groups}
\begin{definition}{Quotient Group}{}
    Let $I \subseteq R$ be an ideal of $R$. Then, the \textbf{quotient ring} (or factor ring) is the set of \emph{cosets}
    \[R / I = \{r + I \mid r \in R\}\]
    with the operations
    \[(r + I) + (s + I) = (r + s) + I\]
    \[(r + I)(s + I) = (rs) + I\]
\end{definition}

\begin{proposition}
    $R / I$ is a ring. 
\end{proposition}

\begin{mdframed}[]
    \begin{proof}
        \begin{itemize}
            \item For addition, we know that $(R, +)$ is an abelian group. This implies that $(I, +)$ is a normal subgroup of $(R, +)$, so $(R / I, +)$ is a group.
            \item For multiplication, suppose $r + I = r' + I$ and $s + I = s' + I$, i.e.
            \[r = r' + a \text{ and } s = s' + b \text{ for some } a, b \in I\]
            Then, $(rs) = (r' + a)(s' + b) = r's' + r'b + as' + ab$. Note that $r'b, as', ab$ all belong to the ideal. So $r's' + r'b + as' + ab \in r's' + I$.
        \end{itemize}
        And, we are done. 
    \end{proof}
\end{mdframed}


\subsubsection{Example 1: Integers Modulo 5}
Consider $\Z / 5\Z = \{0 + 5\Z, 1 + 5\Z, 2 + 5\Z, 3 + 5\Z, 4 + 5\Z\}$. We know that $5\Z \subseteq \Z$ is an ideal. 

\subsubsection{Example 2: Polynomial Ideal}
Consider $\R[x] / \cyclic{x^2 + 1}$. This ring is ``isomorphic'' to $\C$. By identifying $x + \cyclic{x^2 + 1} \in \R[x] / \cyclic{x^2 + 1}$ as $i \in \C$, then:
\[(x + \cyclic{x^2 + 1})^2 = x^2 + \cyclic{x^2 + 1} = x^2 + -(x^2 + 1) + \cyclic{x^2 + 1} = -1 + \cyclic{x^2 + 1}\]
We can also see this through polynomial long division. There is a unique way to write $f(x) = g(x)q(x) + r(x)$ with $\text{deg } r(x) < \text{deg } g(x)$. From this, we can tell that: 
\[f(x) + \cyclic{x^2 + 1} = (x^2 + 1)q(x) + (a + bx) + \cyclic{x^2 + 1} = (a + bx) + \cyclic{x^2 + 1}\]

\subsubsection{Example 3: Gaussian Integers}
Take $\Z[i] / \cyclic{2 - i}$. We claim that this is ``isomorphic'' to $\Z / 5\Z$. It turns out:
\[\Z[i] / \cyclic{2 - i} = \{0 + \cyclic{2 - i}, 1 + \cyclic{2 - i}, 2 + \cyclic{2 - i}, 3 + \cyclic{2 - i}, 4 + \cyclic{2 - i}\}\] 
Consider that $2 + \cyclic{2 - i} = i + \cyclic{2 - i}$ because $2 - i \in \cyclic{2 - i}$. Then:
\begin{equation*}
    \begin{aligned}
        2^2 &+ \cyclic{2 - i} = i^2 + \cyclic{2 - i} \\ 
            &\implies 4 + \cyclic{2 - i} = -1 + \cyclic{2 - i} \\ 
            &\implies 5 \in \cyclic{2 - i}
    \end{aligned}
\end{equation*}
Thus, $a + bi + \cyclic{2 - i} = a + 2b + \cyclic{2 - i} = r + \cyclic{2 - i}$ for $0 \leq r < 5$ such that $a + 2b = 5q + r$. Now, how do we know that these cosets are distinct? It suffices to show that $1 + \cyclic{2 - i}$ has additive order 5. So: 
\[5(1 + \cyclic{2 - i}) = 5 + \cyclic{2 - i} = 0 + \cyclic{2 - i}\]
Where the last step is due to $5 \in \cyclic{2 - i}$. This tells us that the additive order of $1 + \cyclic{2 - i}$ divides 5. This implies that the order is either 1 or 5. If the order is 5, we are done since this implies that there are 5 distinct cosets. Otherwise, suppose towards a contradiction that $1 + \cyclic{2 - i} \in \Z[i] / \cyclic{2 - i}$ has additive order 1. In this case:
\begin{equation*}
    \begin{aligned}
        1(1 &+ \cyclic{2 - i}) = 0 + \cyclic{2 - i} \\ 
            &\implies 1 \in \cyclic{2 - i} = \{(2 - i)r \mid r \in \Z[i]\}\\ 
            &\implies 1 = (2 - i)(a + bi) \text{ for some } a, b \in \Z \\
            &\implies 1 = 2a + 2bi - ai + b \\ 
            &\implies 1 + 0i = (2a + b) + (2b - a)i \\ 
            &\implies \begin{cases}
                1 = 2a + b \\ 
                0 = 2b - a 
            \end{cases}  \\ 
            &\implies a = \frac{1}{5} \text{ and } \frac{2}{5}
    \end{aligned}
\end{equation*}
However, $a, b \in \Z$ so we have a contradiction and so $1 + \cyclic{2 - i}$ must have additive order 5. 

\end{document}