\documentclass[letterpaper]{article}
\usepackage[margin=1in]{geometry}
\usepackage[utf8]{inputenc}
\usepackage{textcomp}
\usepackage{amssymb}
\usepackage{natbib}
\usepackage{graphicx}
\usepackage{gensymb}
\usepackage{amsthm, amsmath, mathtools}
\usepackage[dvipsnames]{xcolor}
\usepackage{enumerate}
\usepackage{mdframed}
\usepackage[most]{tcolorbox}
\usepackage{csquotes}
% https://tex.stackexchange.com/questions/13506/how-to-continue-the-framed-text-box-on-multiple-pages

\tcbuselibrary{theorems}

\newcommand{\R}{\mathbb{R}}
\newcommand{\Z}{\mathbb{Z}}
\newcommand{\N}{\mathbb{N}}
\newcommand{\Q}{\mathbb{Q}}
\newcommand{\C}{\mathbb{C}}
\newcommand{\code}[1]{\texttt{#1}}
\newcommand{\mdiamond}{$\diamondsuit$}
\newcommand{\PowerSet}{\mathcal{P}}
\newcommand{\Mod}[1]{\ (\mathrm{mod}\ #1)}
\DeclareMathOperator{\lcm}{lcm}

%\newtheorem*{theorem}{Theorem}
%\newtheorem*{definition}{Definition}
%\newtheorem*{corollary}{Corollary}
%\newtheorem*{lemma}{Lemma}
\newtheorem*{proposition}{Proposition}


\newtcbtheorem[number within=section]{theorem}{Theorem}
{colback=green!5,colframe=green!35!black,fonttitle=\bfseries}{th}

\newtcbtheorem[number within=section]{definition}{Definition}
{colback=blue!5,colframe=blue!35!black,fonttitle=\bfseries}{def}

\newtcbtheorem[number within=section]{corollary}{Corollary}
{colback=yellow!5,colframe=yellow!35!black,fonttitle=\bfseries}{cor}

\newtcbtheorem[number within=section]{lemma}{Lemma}
{colback=red!5,colframe=red!35!black,fonttitle=\bfseries}{lem}

\newtcbtheorem[number within=section]{example}{Example}
{colback=white!5,colframe=white!35!black,fonttitle=\bfseries}{def}

\newtcbtheorem[number within=section]{note}{Important Note}{
        enhanced,
        sharp corners,
        attach boxed title to top left={
            xshift=-1mm,
            yshift=-5mm,
            yshifttext=-1mm
        },
        top=1.5em,
        colback=white,
        colframe=black,
        fonttitle=\bfseries,
        boxed title style={
            sharp corners,
            size=small,
            colback=red!75!black,
            colframe=red!75!black,
        } 
    }{impnote}
\usepackage[utf8]{inputenc}
\usepackage[english]{babel}
\usepackage{fancyhdr}
\usepackage[hidelinks]{hyperref}

\pagestyle{fancy}
\fancyhf{}
\rhead{Math 103B}
\chead{Friday, March 4, 2022}
\lhead{Lecture 23}
\rfoot{\thepage}

\setlength{\parindent}{0pt}

\begin{document}

\section{Extension Fields}
We continue our discussion on extension fields. 


\subsection{More on Extension Fields}
\begin{corollary}{}{}
    If $\alpha, \beta \in E$, which are both roots of an irreducible polynomial $p(x) \in F[x]$, then $F(\alpha) \cong F(\beta)$. 
\end{corollary}

\begin{mdframed}[]
    \begin{proof}
        We know that $F(\alpha) \cong F[x] / \cyclic{p(x)} \cong F(\beta)$. So, we're done. 
    \end{proof}
\end{mdframed}

\subsubsection{Example 1: Polynomials}
Consider $x^3 - 2 \in \Q[x]$. By Eisenstein's criterion, this is irreducible. Although there are no roots in $\Q$, we can find roots in other places. In particular, looking at the complex and real numbers, we know that a root is $\sqrt[3]{2}$. Now, 
\[(x^3 - 2) = (x - \sqrt[2]{3})q(x)\]
where $q(x)$ is quadratic. The other roots, then, are 
\[\left(\frac{-1 + \sqrt{-3}}{2}\right) \sqrt[3]{2}, \left(\frac{-1 - \sqrt{-3}}{2}\right) \sqrt[3]{2}\]
If we let $\zeta_3 = \frac{-1 + \sqrt{-3}}{2}$, then we know that $\zeta_3 \in \C$ such that 
\[(\zeta_3)^3 = 1\]
We have 
\[\Q(\sqrt[3]{2}) \cong \Q(\zeta_3 \sqrt[3]{2}) \cong \Q(\zeta_{3}^2 \sqrt[3]{2})\]
Now, notice that  
\[\Q(\sqrt[3]{2}) = \{a + b 2^{\frac{1}{3}} + c2^{\frac{2}{3}} \mid a, b, c \in \Q\} \subseteq \R\]
\[\Q(\zeta_3 \sqrt[3]{2}) = \{a + b \zeta_3 2^{\frac{1}{3}} + c \zeta_3 2^{\frac{2}{3}} \mid a, b, c \in \Q\} \not\subseteq \R\]

\subsubsection{Example 2: Pi}
Consider $\pi \in \R$, and suppose we look at $\Q(\pi)$. We note that $\pi$ is not a root of \emph{any} nonzero polynomial in $\Q[x]$. This kind of number is called \emph{transcendental} over $\Q$.


\subsection{Splitting Field}
\begin{definition}{Splitting Field}{}
    Let $E$ be an extension field of $F$, and let $f(x) \in F[x]$. We say that $f(x)$ \emph{splits} in $E$ if 
    \[f(x) = a(x - \alpha_1)(x - \alpha_2) \dots (x - \alpha_n)\]
    for $a \in F$, $\alpha_i \in E$ for $1 \leq i \leq n$. We call $E$ a \textbf{splitting field} for $f(x)$ over $F$ if $E = F(\alpha_1, \alpha_2, \dots, \alpha_n)$. 
\end{definition}

\begin{theorem}{}{}
    Let $F$ be a field and $f(x) \in F[x]$ a nonconstant polynomial. Then, there exists a splitting field for $f(x)$ over $F[x]$. 
\end{theorem}

\begin{mdframed}[]
    \begin{proof}
        We use induction on the degree of $f(x)$. 
        \begin{itemize}
            \item \underline{Base Case:} For $\deg f(x) = 1$, we have $f(x) = ax + b$. A polynomial of degree 1 will have one root; in this case, it's $-\frac{b}{a}$. So, this should already be split. So,
            \[f(x) = ax + b = a\left(x - \left(-\frac{b}{a}\right)\right)\]
            splits in $F$. 

            \item \underline{Inductive Step:} Suppose that if $\deg g(x) = n - 1$, then $g(x)$ has a splitting field over $F$. Suppose $\deg f(x) = n$. There exists a field extension $E$ in which $f(x)$ has a root $\alpha \in E$. This implies that 
            \[f(x) = (x - \alpha) g(x)\]
            for $g(x) \in E[x]$. By the inductive hypothesis, there exists a splitting field $K$ for $g(x)$ over $E$. This implies that, for $a \in E$, $\alpha_1, \dots, \alpha_n \in K$ and so 
            \[f(x) = a(x - \alpha_1)(x - \alpha_2) \dots (x - \alpha_n) = ax^n + \dots\]
            but $a \in F$. Thus, $f(x)$ splits in $K$. So, $F(\alpha_1, \alpha_2, \dots, \alpha_n) \subseteq K$ is a splitting field. 
        \end{itemize}
        So, we are done. 
    \end{proof}
\end{mdframed}


\subsubsection{Example 1: Polynomials}
$x^3 - 2$ does not split over $\Q$ because it's irreducible. It does not split over $\Q(\sqrt[3]{2})$ because it does not contain the other two roots. 

\bigskip 

A splitting field is $\Q(\sqrt[3]{2}, \zeta_3 \sqrt[3]{2}, \zeta_{3}^2 \sqrt[3]{2})$. This is the same thing as writing $\Q(\sqrt[3]{2}, \zeta_3)$. This is because 
\begin{itemize}
    \item They both contain $\Q$. 
    \item They both contain $\sqrt[3]{2}$. 
    \item Since $\Q(\sqrt[3]{2}, \zeta_3)$ contains both $\sqrt[3]{2}$ and $\zeta_3$, and since it's a field, it must be closed under multiplication.
\end{itemize}

\subsection{Even More on Extension Fields}
\begin{theorem}{}{}
    Let $F$ be a field, $p(x) \in F[x]$ a irreducible polynomial, and an isomorphism 
    \[\varphi: F \mapsto F'\]
    Then, if $\alpha$ is a root of $p(x)$ and $\beta$ is a root $\varphi(p(x))$, then $F(\alpha) \cong F'(\beta)$. 
\end{theorem}

\begin{mdframed}[]
    \begin{proof}
        \[F(\alpha) \xrightarrow{\sim{}} F[x] / \cyclic{p(x)} \xrightarrow{\varphi} F'[x] / \cyclic{\varphi(p(x))} \xrightarrow{\sim{}} F'(\beta)\]
        So 
        \[\varphi(a_n x^n + \dots + a_0 + \cyclic{p(x)}) = \varphi(a_n) x^n + \dots + \varphi(a_0) + \cyclic{\varphi(p(x))}\]
        And we are done. 
    \end{proof}
\end{mdframed}

\begin{theorem}{}{}
    Let $\varphi: F \mapsto F'$ be an isomorphism of fields, $f(x) \in F[x]$. If $E$ is a splitting field for $f(x)$ over $F$ and $E'$ is a splitting field for $\varphi(f(x))$ over $F'$, then there is an isomorphism $E \cong E'$ that agrees with $\varphi$ on $F$. 
\end{theorem}

\begin{corollary}{}{}
    Any two splitting fields of $f(x) \in F[x]$ over $F$ are isomorphic. 
\end{corollary}

\begin{mdframed}[]
    \begin{proof}
        Let $F' = F$. We can define $\varphi: F \mapsto F$ the identity function by $a \mapsto a$. THen, we can apply the theorem.
    \end{proof}
\end{mdframed}

\end{document}