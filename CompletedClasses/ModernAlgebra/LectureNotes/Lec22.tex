\documentclass[letterpaper]{article}
\usepackage[margin=1in]{geometry}
\usepackage[utf8]{inputenc}
\usepackage{textcomp}
\usepackage{amssymb}
\usepackage{natbib}
\usepackage{graphicx}
\usepackage{gensymb}
\usepackage{amsthm, amsmath, mathtools}
\usepackage{xcolor}
\usepackage{enumerate}
\usepackage{framed}
\usepackage{tcolorbox}
\tcbuselibrary{theorems}

\newcommand{\R}{\mathbb{R}}
\newcommand{\Z}{\mathbb{Z}}
\newcommand{\N}{\mathbb{N}}
\newcommand{\Q}{\mathbb{Q}}
\newcommand{\code}[1]{\texttt{#1}}
\newcommand{\mdiamond}{$\diamondsuit$}

%\newtheorem*{theorem}{Theorem}
%\newtheorem*{definition}{Definition}
\newtheorem*{proposition}{Proposition}
%\newtheorem*{corollary}{Corollary}
%\newtheorem*{lemma}{Lemma}

\newtcbtheorem[number within=section]{theorem}{Theorem}
{colback=green!5,colframe=green!35!black,fonttitle=\bfseries}{def}

\newtcbtheorem[number within=section]{definition}{Definition}
{colback=blue!5,colframe=blue!35!black,fonttitle=\bfseries}{def}

\newtcbtheorem[number within=section]{corollary}{Corollary}
{colback=yellow!5,colframe=yellow!35!black,fonttitle=\bfseries}{def}

\newtcbtheorem[number within=section]{lemma}{Lemma}
{colback=red!5,colframe=red!35!black,fonttitle=\bfseries}{def}
\usepackage[utf8]{inputenc}
\usepackage[english]{babel}
\usepackage{fancyhdr}
\usepackage[hidelinks]{hyperref}

\pagestyle{fancy}
\fancyhf{}
\rhead{Math 103B}
\chead{Wednesday, March 2, 2022}
\lhead{Lecture 22}
\rfoot{\thepage}

\setlength{\parindent}{0pt}

\begin{document}

\section{Extension Fields}
We now talk about extension fields. 

\subsection{Definition of an Extension Field}
\begin{definition}{Extension Field}{}
    A field $E$ is an \textbf{extension field} of a field $F$ if $F \subseteq E$ and $F$ is a field under the same operations as $E$. 
\end{definition}
\textbf{Remark:} Note that we say that $F$ is a subfield of $E$. Alternatively, we can now say that $E$ is a extension field of $F$. 

\subsubsection{Example 1: Extension Fields of the Rational Numbers}
Consider $\Q \subseteq \R \subseteq \C \cong \R[x] / \cyclic{x^2 + 1}$. We call these the extension field of the rational numbers. 

\subsubsection{Example 2: Quadratic Extension Fields}
Consider $Q \subseteq \Q[\sqrt{2}] = \{a + b\sqrt{2} \mid a, b \in \Q\} \cong \Q[x] / \cyclic{x^2 - 2}$.

\subsubsection{Example 3: Extensions of Finite Fields}
Consider $\F_{3} \subseteq \F_{3}[i] \cong \F_{3}[x] / \cyclic{x^2 + 1}$. 

\subsection{Fundamental Theorem of Field Extensions}
\begin{theorem}{Fundamental Theorem  of Field Extensions}{}
    Let $F$ be a field, and $f(x) \in F[x]$. Then, there exists an extension field $E$ of $F$ in which $f(x)$ has a zero. 
\end{theorem}
\textbf{Remarks:}
\begin{itemize}
    \item The complex numbers is a field extension of the real numbers in which the polynomial $x^2 + 1$ has a root. 
    \item $\Q[\sqrt{2}]$ is a field extension of $\Q$ in which $x^2 - 2$ has a root. 
    \item $\F_{3}[i]$ is a field extension of $\F_3$ in which $x^2 + 1$ has a root. 
\end{itemize}
\textbf{Note:} The extension is not just $F[x] / \cyclic{f(x)}$. This is because if $f(x)$ is reducible, then $F[x] / \cyclic{f(x)}$ is not a field.

\textbf{Fact:} Every polynomial $f(x) \in \C[x]$ of degree $> 0$ has a root in the complex numbers, known as \emph{algebraic closure}.

\begin{mdframed}[]
    \begin{proof}
        $F[x]$ is a UFD, so choose an irreducible polynomial $p(x) \in F[x]$ with $\deg p(x) > 0$ such that $p(x) | f(x)$. Then, $F[x] / \cyclic{p(x)}$ is a field. Now, we show that this is an extension field. Consider the mapping 
        \[\varphi: F \mapsto E\]
        sending 
        \[a \mapsto a + \cyclic{p(x)}\]
        This is clearly injective by $\cyclic{p(x)}$ having no non-zero constants. Additionally, the kernal is trivial. So, by the First Isomorphism Theorem, $F \cong \varphi(F) \subseteq E$. 

        \bigskip 

        Now, let $x + \cyclic{p(x)} \in E$. If $f(x) = a_n x^n + \dots + a_0$, then 
        \begin{equation*}
            \begin{aligned}
                f(x + \cyclic{p(x)}) &= a_n (x + \cyclic{p(x)})^n + \dots + a_1 (x + \cyclic{p(x)}) + a_0 \\ 
                    &= a_n (x^n + \cyclic{p(x)}) + \dots \\ 
                    &= a_n x^n + \dots + a_1 x + a_0 + \cyclic{p(x)} \\ 
                    &= f(x) + \cyclic{p(x)} 
            \end{aligned}
        \end{equation*}
        But, because $p(x) | f(x)$, this implies that $f(x) \in \cyclic{p(x)}$. This further implies that 
        \[f(x) + \cyclic{p(x)} = 0 + \cyclic{p(x)}\]
        so, we are done. 
    \end{proof}
\end{mdframed}

\subsubsection{Example 1: Polynomial}
Consider the polynomial $x^4 + x^3 + x + 2 \in \F_{3}[x]$. We can factor this into the polynomial
\[(x^2 + 1)(x^2 + x + 2)\]
This is contained in the fields $\F_{3}[x] / \cyclic{x^2 + 1}$ or $\F_{3}[x] / \cyclic{x^2 + x + 2}$.

\subsubsection{Example 2: Polynomial}
Consider the polynomial $3x^8 + 2x^6 + 4x + 14 \in \Q[x]$. This is irreducible by Eisenstein's criterion. So, we have the extension field 
\[\Q \subseteq \Q[x] / \cyclic{3x^8 + 2x^6 + 4x + 14}\]
which has a root of $f(x)$. 


\subsection{More on Extension Fields}
\begin{definition}{}{}
    Let $E$ be an extension field of $F$, and let $\alpha_1, \alpha_2, \dots, \alpha_n \in E$. Then, we define $F(\alpha_1, \alpha_2, \dots, \alpha_n)$ to be the smallest subfield of $E$ containing $F, \alpha_1, \alpha_2, \dots, \alpha_n$. 
\end{definition}

\begin{theorem}{}{}
    Let $E$ be an extension field of $F$, $\alpha \in E$ which is a root of the irreducible polynomial $p(x) \in F[x]$. Then, 
    \[F(\alpha) \cong F[x] / \cyclic{p(x)}\]
\end{theorem}

\begin{mdframed}[]
    \begin{proof}
        Consider the homomorphism 
        \[\varphi: F[x] \mapsto E\]
        defined by 
        \[f(x) \mapsto f(\alpha)\]
        We make the claim that $\ker \varphi = \cyclic{p(x)}$. We note that $\varphi(p(x)) = p(\alpha) = 0$ by definition. This implies that 
        \[\cyclic{p(x)} \subseteq \ker \varphi \neq F[x]\]
        We note that $\cyclic{p(x)}$ is maximal. This implies that $\ker \varphi = \cyclic{p(x)}$. 

        \bigskip 

        The First Isomorphism Theorem says that $F[x] / \cyclic{p(x)} \cong \im \varphi \subseteq E$. We now make the claim that $\im \varphi = F(\alpha)$. To see this, we note that 
        \[\alpha = \varphi(x) \implies \alpha \in \im \varphi\]
        but if $a \in F$ is constant, then 
        \[a = \varphi(a) \implies F \subseteq \im \varphi\]
        By the First Isomorphism Theorem, $\im \varphi$ is a field. This implies that 
        \[F(\alpha) \subseteq \im \varphi\]
        To show the other side, take some $y \in \im \varphi$. Then, this means taht 
        \[y = \varphi(a_n x^n + \dots + a_1 x + a_0) = a_n \alpha^n + \dots + a_1 \alpha + a_0\]
        We know that $\alpha \in F(\alpha)$ and $F \subseteq F(\alpha) \implies a_0, a_1, \dots, a_n \in F(\alpha)$. This implies that $y \in F(\alpha)$ by closure, implying that $\im \varphi \subseteq F(\alpha)$. 
    \end{proof}
\end{mdframed}

\subsubsection{Example 1}
Consider $\Q\left(5^{\frac{1}{4}}\right)$. We know that the polynomial $x^4 - 5$ has the root $5^{\frac{1}{4}}$. By Eisenstein's criterion, this polynomial is irreducible. So 
\[\Q\left(5^{\frac{1}{4}}\right) \cong \Q[x] / \cyclic{x^4 - 5}\]
where we can define the isomorphism by 
\[5^{\frac{1}{4}} \mapsto x + \cyclic{x^4 - 5}\]
We note that $\Q[x] / \cyclic{x^4 - 5}$ is 4-dimensional. The basis is given by 
\[\left\{1, x, x^2, x^3\right\}\]
So, it follows that 
\[\Q[x] / \cyclic{x^4 - 5} = \left\{a + bx + cx^2 + dx^3 + \cyclic{x^4 - 5} \mid a, b, c, d \in \Q\right\}\]
But, we can map this to 
\[\left\{a + b 5^{\frac{1}{4}} + c 5^{\frac{2}{4}} + d 5^{\frac{3}{4}}\right\}\]

\end{document}