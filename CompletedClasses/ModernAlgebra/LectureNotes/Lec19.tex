\documentclass[letterpaper]{article}
\usepackage[margin=1in]{geometry}
\usepackage[utf8]{inputenc}
\usepackage{textcomp}
\usepackage{amssymb}
\usepackage{natbib}
\usepackage{graphicx}
\usepackage{gensymb}
\usepackage{amsthm, amsmath, mathtools}
\usepackage[dvipsnames]{xcolor}
\usepackage{enumerate}
\usepackage{mdframed}
\usepackage[most]{tcolorbox}
\usepackage{csquotes}
% https://tex.stackexchange.com/questions/13506/how-to-continue-the-framed-text-box-on-multiple-pages

\tcbuselibrary{theorems}

\newcommand{\R}{\mathbb{R}}
\newcommand{\Z}{\mathbb{Z}}
\newcommand{\N}{\mathbb{N}}
\newcommand{\Q}{\mathbb{Q}}
\newcommand{\C}{\mathbb{C}}
\newcommand{\code}[1]{\texttt{#1}}
\newcommand{\mdiamond}{$\diamondsuit$}
\newcommand{\PowerSet}{\mathcal{P}}
\newcommand{\Mod}[1]{\ (\mathrm{mod}\ #1)}
\DeclareMathOperator{\lcm}{lcm}

%\newtheorem*{theorem}{Theorem}
%\newtheorem*{definition}{Definition}
%\newtheorem*{corollary}{Corollary}
%\newtheorem*{lemma}{Lemma}
\newtheorem*{proposition}{Proposition}


\newtcbtheorem[number within=section]{theorem}{Theorem}
{colback=green!5,colframe=green!35!black,fonttitle=\bfseries}{th}

\newtcbtheorem[number within=section]{definition}{Definition}
{colback=blue!5,colframe=blue!35!black,fonttitle=\bfseries}{def}

\newtcbtheorem[number within=section]{corollary}{Corollary}
{colback=yellow!5,colframe=yellow!35!black,fonttitle=\bfseries}{cor}

\newtcbtheorem[number within=section]{lemma}{Lemma}
{colback=red!5,colframe=red!35!black,fonttitle=\bfseries}{lem}

\newtcbtheorem[number within=section]{example}{Example}
{colback=white!5,colframe=white!35!black,fonttitle=\bfseries}{def}

\newtcbtheorem[number within=section]{note}{Important Note}{
        enhanced,
        sharp corners,
        attach boxed title to top left={
            xshift=-1mm,
            yshift=-5mm,
            yshifttext=-1mm
        },
        top=1.5em,
        colback=white,
        colframe=black,
        fonttitle=\bfseries,
        boxed title style={
            sharp corners,
            size=small,
            colback=red!75!black,
            colframe=red!75!black,
        } 
    }{impnote}
\usepackage[utf8]{inputenc}
\usepackage[english]{babel}
\usepackage{fancyhdr}
\usepackage[hidelinks]{hyperref}

\pagestyle{fancy}
\fancyhf{}
\rhead{Math 103B}
\chead{Wednesday, February 16, 2022}
\lhead{Lecture 19}
\rfoot{\thepage}

\setlength{\parindent}{0pt}

\begin{document}

\section{Divisibility in Integral Domains}
We continue again with our discussion on this. 

\subsection{Euclidian Domain}
\begin{definition}{}{}
    An integral domain is called a \textbf{Euclidian domain} (ED) if there exists a function 
    \[d: D \setminus \{0\} \mapsto \Z_{\geq 0}\]
    such that
    \begin{enumerate}
        \item $d(a) \leq d(ab)$ for all $a, b \in D \setminus \{0\}$.
        \item If $a, b \in D$ with $b \neq 0$, then there exist $q, r \in D$ such that
        \[a = bq + r\]
        and either $r = 0$ or $d(r) < d(b)$. 
    \end{enumerate}
\end{definition}
\textbf{Remark:} This is basically just long division. In other words, this is an ``abstraction'' of when long division exists. 

\subsubsection{Example 1: The Integers}
The integers, $\Z$, are an Euclidian domain with $d(a) = |a|$. 
\begin{enumerate}
    \item If $a, b \in \Z \setminus \{0\}$, then $|ab| = |a||b| \geq 1$. This implies that 
    \[d(ab) \geq d(a)\]
    \item Long division, specifically if $a, b \in \Z$ with $b \neq 0$, then 
    \[a = bq + r\]
    and 
    \[r = 0 \text{ or } 0 < r < |b|\]
    i.e. $d(r) < d(b)$. 
\end{enumerate}

\subsubsection{Example 2: A Field}
Consider $\F[x]$, where $\F$ is a field. Then, this is an Euclidian domain with $d(f(x)) = \deg f(x)$. 
\begin{enumerate}
    \item $\deg(g(x)f(x)) = \deg g(x) + \deg f(x) \geq \deg g(x)$. This implies that 
    \[d(f(x) g(x)) \geq d(g(x))\]

    \item This is just long division. If $f(x), g(x) \in \F[x]$, then there exists $q(x), r(x) \in F[x]$ such that 
    \[f(x) = g(x)q(x) + r(x)\]
    and 
    \[r(x) = 0 \text{ or } \deg r(x) < \deg g(x) \implies d(r(x)) < d(g(x))\]
\end{enumerate}

\subsubsection{Example 3: Gaussian Integers}
Consider $\Z[i]$ with $d(a + bi) = a^2 + b^2$. 
\begin{enumerate}
    \item $d(xy) = d(x)d(y) \geq d(x)$ by $a^2 + b^2 \geq 1$ if $a + bi \neq 0 + 0i$. 
    \item Let $x, y \in \Z[i]$ and $y \neq 0$. Then 
    \[xy^{-1} \in \Q[i] \supseteq \Z[i]\]
    We can then write $xy^{-1} = s + ti$ for $s, t \in \Q$. In other words, an integer part plus a small fractional part. Let $m$ be the closest integer to $s$ and let $n$ be the closest integer to $t$ such that 
    \[|s - m| \leq \frac{1}{2}\]
    \[|t - n| \leq \frac{1}{2}\]
    So now we have 
    \[xy^{-1} = m + ni + (s - m) + (t - n)i\]
    so that 
    \[x = \underbrace{(m + ni)y}_{\in \Z[i]} + ((s - m) + (t - n)i)y\]
    But we know that 
    \[((s - m) + (t - n)i)y = x - (m + ni)y \in \Z[i]\]
    Either 
    \[((s - m) + (t - n)i)y = 0\]
    or 
    \begin{equation*}
        \begin{aligned}
            d(((s - m) + (t - n)i)y) &= d((s - m) + (t - n)i) d(y) \\ 
                &= ((s - m)^2 + (t - n)^2) d(y) \\ 
                &\leq \left(\left(\frac{1}{2}\right)^2 + \left(\frac{1}{2}\right)^2\right) d(y) \\ 
                &= \frac{1}{2} d(y) < d(y)
        \end{aligned}
    \end{equation*}
\end{enumerate}

\subsection{Euclidian Domain}
\begin{theorem}{}{}
    Every Euclidian domain is a PID. 
\end{theorem}
\begin{mdframed}[]
    \begin{proof}
        Let $D$ be an ED and $I \subseteq D$ a non-zero ideal. Choose $a \in I \setminus \{0\}$ with $d(a)$ minimal. If $b \in I$ then $b = aq + r$ with $r = 0$ or $d(r) < d(a)$. But, $r = b - aq \in I$ so by minimality, $r = 0$ or $d(r) \geq d(a)$, implying that $r = 0$ and thus $I = \cyclic{a}$. 
    \end{proof}
\end{mdframed}

\textbf{Remarks:}
\begin{itemize}
    \item $\Z[i]$ is a PID and so a UFD.
    \item $\Z[\sqrt{-3}]$ is not a PID and so not a ED. In particular, $\Z[\sqrt{-3}]$ with $N(a + b\sqrt{-3}) = a^2 + 3b^2$ fails the properties of ED. 
    \item From this theorem, we now know that an integral domain being an ED implies that it is also a PID which implies that it is also a UFD; that is: 
    \[\text{ED} \implies \text{PID} \implies \text{UFD}\]
    The converse is \emph{not} true. 
    \item There exists a UFD which is not a PID. 
\end{itemize}

\begin{theorem}{}{}
    If $D$ is a UFD, then so is $D[x]$. 
\end{theorem}
\textbf{Remark:} In particular, $\Z[x]$ is a UFD and \emph{not} a PID. 

\end{document}