\documentclass[letterpaper]{article}
\usepackage[margin=1in]{geometry}
\usepackage[utf8]{inputenc}
\usepackage{textcomp}
\usepackage{amssymb}
\usepackage{natbib}
\usepackage{graphicx}
\usepackage{gensymb}
\usepackage{amsthm, amsmath, mathtools}
\usepackage[dvipsnames]{xcolor}
\usepackage{enumerate}
\usepackage{mdframed}
\usepackage[most]{tcolorbox}
\usepackage{csquotes}
% https://tex.stackexchange.com/questions/13506/how-to-continue-the-framed-text-box-on-multiple-pages

\tcbuselibrary{theorems}

\newcommand{\R}{\mathbb{R}}
\newcommand{\Z}{\mathbb{Z}}
\newcommand{\N}{\mathbb{N}}
\newcommand{\Q}{\mathbb{Q}}
\newcommand{\C}{\mathbb{C}}
\newcommand{\code}[1]{\texttt{#1}}
\newcommand{\mdiamond}{$\diamondsuit$}
\newcommand{\PowerSet}{\mathcal{P}}
\newcommand{\Mod}[1]{\ (\mathrm{mod}\ #1)}
\DeclareMathOperator{\lcm}{lcm}

%\newtheorem*{theorem}{Theorem}
%\newtheorem*{definition}{Definition}
%\newtheorem*{corollary}{Corollary}
%\newtheorem*{lemma}{Lemma}
\newtheorem*{proposition}{Proposition}


\newtcbtheorem[number within=section]{theorem}{Theorem}
{colback=green!5,colframe=green!35!black,fonttitle=\bfseries}{th}

\newtcbtheorem[number within=section]{definition}{Definition}
{colback=blue!5,colframe=blue!35!black,fonttitle=\bfseries}{def}

\newtcbtheorem[number within=section]{corollary}{Corollary}
{colback=yellow!5,colframe=yellow!35!black,fonttitle=\bfseries}{cor}

\newtcbtheorem[number within=section]{lemma}{Lemma}
{colback=red!5,colframe=red!35!black,fonttitle=\bfseries}{lem}

\newtcbtheorem[number within=section]{example}{Example}
{colback=white!5,colframe=white!35!black,fonttitle=\bfseries}{def}

\newtcbtheorem[number within=section]{note}{Important Note}{
        enhanced,
        sharp corners,
        attach boxed title to top left={
            xshift=-1mm,
            yshift=-5mm,
            yshifttext=-1mm
        },
        top=1.5em,
        colback=white,
        colframe=black,
        fonttitle=\bfseries,
        boxed title style={
            sharp corners,
            size=small,
            colback=red!75!black,
            colframe=red!75!black,
        } 
    }{impnote}
\usepackage[utf8]{inputenc}
\usepackage[english]{babel}
\usepackage{fancyhdr}
\usepackage[hidelinks]{hyperref}

\pagestyle{fancy}
\fancyhf{}
\rhead{Math 103B}
\chead{Friday, January 07, 2022}
\lhead{Lecture 3}
\rfoot{\thepage}

\setlength{\parindent}{0pt}

\begin{document}

\section{Integral Domains}
Recall that rings do not have multiplicative cancellation. That is, $ab = ac$ does not imply that $b = c$. However, there are exceptions to this rule.

\begin{definition}{Zero-Divisors}{}
    A \textbf{zero-divisor} is a nonzero element $a$ of a commutative ring $R$ such that there is a nonzero element $b \in R$ with $ab = 0$. 
\end{definition}
For example, $2 \in \Z / 4 \Z$ is a zero divisor. That is: 
\[2 \cdot 2 \equiv 0 \Mod{4}\]
Another example is $M_{2}(\R)$. Take $A = \begin{bmatrix}
    0 & 0 \\ 1 & 0 
\end{bmatrix}$ and $B = \begin{bmatrix}
    0 & 0 \\ 0 & 1
\end{bmatrix}$. Then:
\[A \cdot B = \begin{bmatrix}
    0 & 0 \\ 0 & 0
\end{bmatrix}\] 

\begin{definition}{Integral Domain}{}
    An \textbf{integral domain} is a commutative ring with unity and no zero-divisors.
\end{definition}
\textbf{Remarks:}
\begin{itemize}
    \item Recall that a ring $R$ has \textbf{unity} if $1 \in R$ is a multiplicative identity; that is, $1a = a1 = a$. 
    \item Essentially, in an integral domain, a product is 0 only when one of the facts is 0. That is, $ab = 0$ only when $a = 0$ or $b = 0$. 
\end{itemize}

\subsection{Examples}
Here are some examples of integral domains. 

\subsubsection{Example 1: The Integers}
The ring of integers is an integral domain. 

\subsubsection{Example 2: Gaussian Integers}
The ring of Gaussian integers $Z[i] = \{a + bi \mid a, b \in \Z\}$ is an integral domain. 

\subsubsection{Example 3: Ring of Polynomials}
The ring $\Z[x]$ of polynomials with integer coefficients is an integral domain. 

\subsubsection{Example 4: Square Root 2}
The ring $\Z[\sqrt{2}] = \{a + b\sqrt{2} \mid a, b \in \Z\}$ is an integral domain. 

\subsubsection{Example 5: Modulo Prime Integers}
The ring $\Z / p\Z$ of integers modulo $a$ prime $p$ is an integral domain. This is because:
\[ab \equiv 0 \Mod{p} \iff p | ab \implies p | a \text{ or } p | b \implies a \equiv 0 \Mod{p} \text{ or } b \equiv 0 \Mod{p}\] 

\subsubsection{Non-Example 1: Modulo Integers}
The ring $\Z / n\Z$ of integers modulo $n$ is not an integral domain when $n$ is not prime. If we write $n = ab$, then $1 < a$ and $b < n$ implies that $ab \equiv 0 \Mod{b}$. 
% TODO clarify 

\subsubsection{Non-Example 2: Matrices}
The ring $M_{2}(\Z)$ of $2 \times 2$ matrices over the integers is not an integral domain. 

\subsubsection{Non-Example 3: Direct Product}
$\Z \oplus \Z$ is not an integral domain. The reason why is because, for:
\[\Z \oplus \Z = \{(x, y) \mid x, y \in \Z\}\]
Take $(0, 1) \in \Z \oplus \Z$ and $(1, 0) \in \Z \oplus \Z$. Then:
\[(0, 1) \cdot (1, 0) = (0, 0)\]


\subsection{Properties of Integral Domains}
\begin{theorem}{Cancellation}{}
    Let $a$, $b$, and $c$ belong to an integral domain. If $ab = ac$, then: 
    \[a = 0 \text{ or } b = c\]
\end{theorem}
\begin{mdframed}[]
    \begin{proof}
        From $ab = ac$, we know that $ab - ac = 0$. Then, we know that $a(b - c) = 0$. There are two cases to consider:
        \begin{itemize}
            \item If $a \neq 0$, it follows that $b - c = 0$.
            \item Otherwise, $a = 0$ and it's trivial.
        \end{itemize}
        So, we are done. 
    \end{proof}
\end{mdframed}

\section{Fields}

\begin{definition}{Field}{}
    A \textbf{field} is a commutative ring with unity in which every nonzero element is a unit (i.e. every nonzero element has a multiplicative inverse).
\end{definition}
\textbf{Remarks:} 
\begin{itemize}
    \item To verify that every field is an integral domain, observe that if $a$ and $b$ belong to a field with $a \neq 0$ and $ab = 0$, we can multiply both sides of the last expression by $a^{-1}$ to obtain $b = 0$.
    \item In other words, you can never have an $x$ such that $0x = 1$. 
\end{itemize}

\subsection{Examples of Fields}
Here are some examples and non-examples of fields.

\subsubsection{Example 1: Infinite Sets}
$\R$, $\C$, and $\Q$ are all fields. 

\subsubsection{Non-Example 1: Integers}
$\Z$ is not a field because 2 does not have a multiplicative inverse. 

\subsubsection{Example 2: Matrices}
$M_{2}(\R)$ is not a field because not all matrcies have an inverse. 

\subsubsection{Non-Example 2: Polynomials}
$\R[x]$ is not a field. This is because not all functions have a \emph{polynomial} inverse. For example, the inverse of $x + 3$ is $\frac{1}{x + 3}$, which isn't a polynomial. However, $\R[x]$ is an integral domain. 

\subsection{Properties of Fields}

\begin{theorem}{}{}
    A finite integral domain is a field.
\end{theorem}

\begin{mdframed}[]
    \begin{proof}
        Let $R$ be a finite integral domain and suppose $a \in R \setminus \{0\}$. Consider the set: 
        \[\{a, a^2, a^3, a^4, \dots\} \subseteq R\]
        Because $R$ is finite, there must be some overlap, i.e. there exists two integers $j < i$ such that $a^j = a^i$. This implies that $a^j = a^{i - j} a^j$. Since we have an integral domain, we can perform multiplicative cancellation; so: 
        \[1 = a^{i - j} \text{ for } i - j \geq 1\]
        Then, $i - j - 1 \geq 0$ with $(a)(a^{i - j - 1}) = a^{i - j} = 1$. So, $a^{-1} = a^{i - j - 1}$ is a multiplicative inverse of $a$. 
    \end{proof}
\end{mdframed}

\begin{corollary}{}{}
    For every prime $p$, $\Z / p \Z$, the ring of integers modulo $p$ is a field.
\end{corollary}
\textbf{Remark:} This is often denoted $\F_p$ in this context. 

\bigskip 

Some other examples are: 
\begin{itemize}
    \item Fields with 9 elements: $\F_{3}[i] = \{0, 1, 2, i, 1 + i, 2 + i, 2i, 1 + 2i, 2 + 2i\}$. Recall that $i^2 = -1 \equiv 2 \Mod{3}$. 
    \item Fields witih 4 elements: $\{0, 1, a, b\}$. 
    \begin{center}
        \begin{tabular}{c|c c c c}
            $+$ & 0 & 1 & $a$ & $b$ \\  
            \hline 
            0   & 0 & 1 & $a$ & $b$ \\ 
            1   & 1 & 0 & $b$ & $a$ \\
            $a$ & $a$ & $b$ & 0 & 1 \\ 
            $b$ & $b$ & $a$ & 1 & 0
        \end{tabular}

        \begin{tabular}{c|c c c c}
            $\cdot$ & 0 & 1 & $a$ & $b$ \\ 
            \hline 
            0   & 0 & 0 & 0 & 0 \\ 
            1   & 0 & 1 & $a$ & $b$ \\
            $a$ & 0 & $a$ & $b$ & 1 \\ 
            $b$ & 0 & $b$ & 1 & $a$
        \end{tabular}
    \end{center}
    \item $\Q[\sqrt{5}] = \{a + b\sqrt{5} \mid a, b \in \Q\} \subseteq \R$ is a field. First, to show that it's a field, we need to show that every nonzero element has multiplicative inverses. Suppose $a + b\sqrt{5} \neq 0$. Then: 
    \[a + b\sqrt{5} \neq 0 \iff b\sqrt{5} \neq a \iff (a, b) \neq (0, 0)\]
    In other words, $a, b$ are not both zero. Note that since $\sqrt{5}$ is irrational, $\sqrt{5} \neq \frac{a}{b}$. So, $\frac{1}{a + b\sqrt{5}} \in \R$ by $a + b\sqrt{5}$ is not zero and $\R$ is a field. 
\end{itemize}



\end{document}