\documentclass[letterpaper]{article}
\usepackage[margin=1in]{geometry}
\usepackage[utf8]{inputenc}
\usepackage{textcomp}
\usepackage{amssymb}
\usepackage{natbib}
\usepackage{graphicx}
\usepackage{gensymb}
\usepackage{amsthm, amsmath, mathtools}
\usepackage[dvipsnames]{xcolor}
\usepackage{enumerate}
\usepackage{mdframed}
\usepackage[most]{tcolorbox}
\usepackage{csquotes}
% https://tex.stackexchange.com/questions/13506/how-to-continue-the-framed-text-box-on-multiple-pages

\tcbuselibrary{theorems}

\newcommand{\R}{\mathbb{R}}
\newcommand{\Z}{\mathbb{Z}}
\newcommand{\N}{\mathbb{N}}
\newcommand{\Q}{\mathbb{Q}}
\newcommand{\C}{\mathbb{C}}
\newcommand{\code}[1]{\texttt{#1}}
\newcommand{\mdiamond}{$\diamondsuit$}
\newcommand{\PowerSet}{\mathcal{P}}
\newcommand{\Mod}[1]{\ (\mathrm{mod}\ #1)}
\DeclareMathOperator{\lcm}{lcm}

%\newtheorem*{theorem}{Theorem}
%\newtheorem*{definition}{Definition}
%\newtheorem*{corollary}{Corollary}
%\newtheorem*{lemma}{Lemma}
\newtheorem*{proposition}{Proposition}


\newtcbtheorem[number within=section]{theorem}{Theorem}
{colback=green!5,colframe=green!35!black,fonttitle=\bfseries}{th}

\newtcbtheorem[number within=section]{definition}{Definition}
{colback=blue!5,colframe=blue!35!black,fonttitle=\bfseries}{def}

\newtcbtheorem[number within=section]{corollary}{Corollary}
{colback=yellow!5,colframe=yellow!35!black,fonttitle=\bfseries}{cor}

\newtcbtheorem[number within=section]{lemma}{Lemma}
{colback=red!5,colframe=red!35!black,fonttitle=\bfseries}{lem}

\newtcbtheorem[number within=section]{example}{Example}
{colback=white!5,colframe=white!35!black,fonttitle=\bfseries}{def}

\newtcbtheorem[number within=section]{note}{Important Note}{
        enhanced,
        sharp corners,
        attach boxed title to top left={
            xshift=-1mm,
            yshift=-5mm,
            yshifttext=-1mm
        },
        top=1.5em,
        colback=white,
        colframe=black,
        fonttitle=\bfseries,
        boxed title style={
            sharp corners,
            size=small,
            colback=red!75!black,
            colframe=red!75!black,
        } 
    }{impnote}
\usepackage[utf8]{inputenc}
\usepackage[english]{babel}
\usepackage{fancyhdr}
\usepackage[hidelinks]{hyperref}

\pagestyle{fancy}
\fancyhf{}
\rhead{Math 103B}
\chead{Wednesday, January 19, 2022}
\lhead{Lecture 7}
\rfoot{\thepage}

\setlength{\parindent}{0pt}

\begin{document}

\section{The Quotient Field}

\begin{theorem}{}{}
    If $D$ is an integral domain (a ring with no zero divisors; it's a ring with multiplicative cancellation), then there exists a field $\F$ that contains $D$ as a subring. 
\end{theorem}
Here are some examples: 
\begin{enumerate}
    \item Consider $\Z \subseteq \Q$. $\Z$ is an integral domain while $\Q$ is a field. We know that the integers look like: 
    \[\Z = \{\dots, -3, -2, -1, 0, 1, 2, 3, \dots\}\]
    We can define the rationals like so: 
    \[\Q := \left\{\frac{a}{b} \mid a \in \Z, b \in \Z \setminus \{0\}\right\}\]
\end{enumerate}

\begin{definition}{}{}
    If $D$ is an integral domain, we can define:
    \[S = \{(a, b) \mid a, b \in D, b \neq 0\}\]
    We can define an equivalence relation on $S$ by $(a, b) \sim (c, d)$ if and only if $ad = bc$. Then, we can write $F = \frac{S}{\sim}$ and:
    \[\frac{a}{b} = [(a, b)] = \{(c, d) \in S \mid (a, b) \sim (c, d)\}\]
    Here, we use the operation: 
    \[\frac{a}{b} + \frac{c}{d} = \frac{ad + bc}{bd}\]
    \[\frac{a}{b} \frac{c}{d} = \frac{ac}{bd}\]
    We call $F$ the field of fractions or the field of quotients of $D$. 
\end{definition}
\textbf{Remarks:} 
\begin{itemize}
    \item In $\Q$, we know that $\frac{2}{4} = \frac{1}{2}$. So, we can say that $(2, 4)$ is ``equal'' to $(1, 2)$.
    \item The idea is that $\frac{a}{b} = cd \iff ad = bc$. 
\end{itemize} 

\subsection{Equivalence Relation}
We say that $(a, b) \sim (c, d)$ if and only if $ad = bc$.
\begin{itemize}
    \item \underline{Reflexive:} $(a, b) \sim (a, b)$ because $ab = ba$ as $D$ is commutative.
    \item \underline{Symmetric:} $(a, b) \sim (c, d) \implies ad = bc \implies cb = da \implies (c, d) \sim (a, b)$. 
    \item \underline{Transitive:} $(a, b) \sim (c, d)$ and $(c, d) \sim (e, f) \implies ad = bc$ and $cf = de$. Then, $adf = bcf \implies adf = bde \implies daf = dbe \implies af = be$ since $D$ is an integral domain. This tells us that $(a, b) \sim (e, f)$ as expected. 
\end{itemize}
Thus, this equivalence relation is well-defined as a set. 

\subsubsection{Addition Well-Defined}
Note that $\frac{a}{b} + \frac{c}{d} = \frac{ad + bc}{bd}$. Remember that, depending on the representation $(a, b)$ of $\frac{a}{b}$, we might get the same values. For example, $\frac{1}{2} = \frac{2}{4}$. So, suppose $(a, b) \sim (a', b')$ and $(c, d) \sim (c', d')$. Then: 
\begin{equation*}
    \begin{aligned}
        (ad + bc)(b' d') &= adb'd' + bcb'd' \\ 
            &= (ab')dd' + (cd')bb' && \text{Ring is commutative} \\ 
            &= (a'b)dd' + (c'd)(bb') && \text{By the equivalence relation} \\ 
            &= (a'd' + c'b')(bd)
    \end{aligned}
\end{equation*}
Thus, $\frac{ad + bc}{bd} = \frac{a'd' + c'b'}{b'd'}$. Finally, if $\frac{a}{b}, \frac{c}{d} \in F$, then $b, d \neq 0$. This implies that $bd \neq 0$ since $D$ is an integral domain. This tells us that $\frac{ad + bc}{bd} \in F$. 

\subsubsection{Addition Commutative}
Here, we have: 
\[\frac{a}{b} + \frac{c}{d} = \frac{ad + bc}{bd} = \frac{cb + da}{db} = \frac{c}{d} + \frac{a}{b}\]

\subsubsection{Addition Associative}
This is similar to above.

\subsubsection{Additive Identity}
The identity is $\frac{0}{1} = \frac{0}{a} \in F$ for all $a \neq 0$. This is because:
\[\frac{0}{1} + \frac{a}{b} = \frac{0 \cdot b + 1 \cdot a}{1 \cdot b} = \frac{a}{b}\] 

\subsubsection{Additive Inverse}
For an element $\frac{a}{b}$, its inverse is $\frac{-a}{b}$. This is because: 
\[\frac{a}{b} + \frac{-a}{b} = \frac{ab + b(-a)}{b^2} = \frac{ab - ab}{b^2} = \frac{0}{b^2} = \frac{0}{1}\]

\subsubsection{Multiplication Well-Defined}
Let $(a, b) \sim (a', b')$ and $(c, d) \sim (c', d')$. Then:
\[acb'd' = (ab')(cd') = (ba')(dc') = (a'c')(bd) \implies \frac{ac}{bd} = \frac{a'c'}{b'd'}\]
Also, $\frac{a}{b}, \frac{c}{d} \in F$ so $b, d \neq 0$ and thus $bd \neq 0$ since $D$ is an integral domain. Thus, $\frac{ac}{bd} \in F$. 

\subsubsection{Multiplication Associative}
\[\left(\frac{a}{b} \frac{c}{d}\right) = \frac{a}{b} \left(\frac{c}{d} \frac{e}{f}\right)\]

\subsubsection{Multiplication Commutative}
\[\frac{a}{b} \frac{c}{d} = \frac{ac}{bd} = \frac{ca}{db} = \frac{c}{d} \frac{a}{b}\]

\subsubsection{Multiplication Unity}
The unity is $\frac{1}{1} \in F$. This is because: 
\[\frac{1}{1} \frac{a}{b} = \frac{1a}{1b} = \frac{a}{b}\]

\subsubsection{Multiplicative Inverses}
If $\frac{a}{b} \neq \frac{0}{1}$, then $\left(\frac{a}{b}\right)^{-1} = \frac{b}{a}$. Note that: 
\[\frac{a}{b} \neq \frac{0}{1} \implies a1 \neq b0\]
In other words, $a \neq 0$ and thus $\frac{b}{a} \in F$. Thus: 
\[\frac{a}{b} \frac{b}{a} = \frac{ab}{ab} = \frac{1}{1}\]

\subsubsection{Multiplication Distributive}
This is left as an exercise.

\subsection{Subring}
How is $D$ a subring of $F$? In $\Q$, we write $\frac{2}{1}$ as $2$. Well: 
\[a \in D \mapsto \frac{a}{1} \in F\]
In other words, we have a homomorphism.

\subsection{Examples of Fields of Fractions}
Here are some examples.
\begin{enumerate}
    \item $\Z \mapsto \Q$. 
    \item $\R[x] \mapsto \R(x) = \left\{\frac{f(x)}{g(x)} \mid f, g \in \R[x], g \neq 0\right\}$.
    \item $\F_{p}[x] \mapsto \F_{p}(x) = \left\{\frac{f(x)}{g(x)} \mid f, g \in \F_{p}[x], g \neq 0\right\}$. Note that $\F_{p}(x)$ has infinite size and has characteristic $p$. Additionally, $x + 1 \in \F_{p}[x]$ has no multiplicative inverse.
\end{enumerate}

\end{document}