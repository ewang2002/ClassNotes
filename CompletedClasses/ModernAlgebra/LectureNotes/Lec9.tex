\documentclass[letterpaper]{article}
\usepackage[margin=1in]{geometry}
\usepackage[utf8]{inputenc}
\usepackage{textcomp}
\usepackage{amssymb}
\usepackage{natbib}
\usepackage{graphicx}
\usepackage{gensymb}
\usepackage{amsthm, amsmath, mathtools}
\usepackage[dvipsnames]{xcolor}
\usepackage{enumerate}
\usepackage{mdframed}
\usepackage[most]{tcolorbox}
\usepackage{csquotes}
% https://tex.stackexchange.com/questions/13506/how-to-continue-the-framed-text-box-on-multiple-pages

\tcbuselibrary{theorems}

\newcommand{\R}{\mathbb{R}}
\newcommand{\Z}{\mathbb{Z}}
\newcommand{\N}{\mathbb{N}}
\newcommand{\Q}{\mathbb{Q}}
\newcommand{\C}{\mathbb{C}}
\newcommand{\code}[1]{\texttt{#1}}
\newcommand{\mdiamond}{$\diamondsuit$}
\newcommand{\PowerSet}{\mathcal{P}}
\newcommand{\Mod}[1]{\ (\mathrm{mod}\ #1)}
\DeclareMathOperator{\lcm}{lcm}

%\newtheorem*{theorem}{Theorem}
%\newtheorem*{definition}{Definition}
%\newtheorem*{corollary}{Corollary}
%\newtheorem*{lemma}{Lemma}
\newtheorem*{proposition}{Proposition}


\newtcbtheorem[number within=section]{theorem}{Theorem}
{colback=green!5,colframe=green!35!black,fonttitle=\bfseries}{th}

\newtcbtheorem[number within=section]{definition}{Definition}
{colback=blue!5,colframe=blue!35!black,fonttitle=\bfseries}{def}

\newtcbtheorem[number within=section]{corollary}{Corollary}
{colback=yellow!5,colframe=yellow!35!black,fonttitle=\bfseries}{cor}

\newtcbtheorem[number within=section]{lemma}{Lemma}
{colback=red!5,colframe=red!35!black,fonttitle=\bfseries}{lem}

\newtcbtheorem[number within=section]{example}{Example}
{colback=white!5,colframe=white!35!black,fonttitle=\bfseries}{def}

\newtcbtheorem[number within=section]{note}{Important Note}{
        enhanced,
        sharp corners,
        attach boxed title to top left={
            xshift=-1mm,
            yshift=-5mm,
            yshifttext=-1mm
        },
        top=1.5em,
        colback=white,
        colframe=black,
        fonttitle=\bfseries,
        boxed title style={
            sharp corners,
            size=small,
            colback=red!75!black,
            colframe=red!75!black,
        } 
    }{impnote}
\usepackage[utf8]{inputenc}
\usepackage[english]{babel}
\usepackage{fancyhdr}
\usepackage[hidelinks]{hyperref}

\pagestyle{fancy}
\fancyhf{}
\rhead{Math 103B}
\chead{Monday, January 24, 2022}
\lhead{Lecture 9}
\rfoot{\thepage}

\setlength{\parindent}{0pt}

\begin{document}


\section{Ring Homomorphisms}
\begin{theorem}{}{}
    Let $\varphi: R \mapsto S$ be a ring homomorphism. Then, $\ker \varphi = \{r \in R \mid \varphi(r) = 0\}$ is an ideal of $R$. 
\end{theorem}

\begin{mdframed}[]
    \begin{proof}
        If $a, b \in \ker \varphi$, then $\varphi(a - b) = \varphi(a) - \varphi(b) = 0 - 0 = 0$, which implies that $a - b \in \ker \varphi$. Now, if we check $a \in \ker \varphi$ and $r \in R$, then $\varphi(ra) = \varphi(r) \varphi(a) = \varphi(r) 0 = 0$. Therefore, $ra \in \ker \varphi$. Thus, $\ker \varphi$ is an ideal by the ideal test. 
    \end{proof}
\end{mdframed}


\subsection{First Isomorphism Theorem}
\begin{theorem}{First Isomorphism Theorem}{}
    Let $\varphi: R \mapsto S$ be a ring homomorphism. Then, the map
    \[\overline{\varphi} = R / \ker\varphi \mapsto \varphi(R)\]
    defined by the mapping
    \[r + \ker\varphi \mapsto \varphi(r)\]
    is an isomorphism.
\end{theorem}

\begin{mdframed}[]
    \begin{proof}
        We already know that $\overline{\varphi}: R / \ker \varphi \mapsto \varphi(R)$ is an isomorphism of additive groups; in particular,
        \[(R / \ker \varphi, +) \mapsto (\varphi(R), +)\]
        by the First Isomorphism Theorem for groups. Thus, it suffices to check that:
        \[\overline{\varphi}(xy) = \overline{\varphi}(x) \overline{\varphi}(y)\]
        So, it suffices to check:
        \begin{equation*}
            \begin{aligned}
                \overline{\varphi}((r + \ker\varphi)(s + \ker\varphi)) &= \overline{\varphi}(rs + \ker\varphi) \\ 
                    &= \varphi(rs) \\ 
                    &= \varphi(r)\varphi(s) \\ 
                    &= \overline{\varphi}(r + \ker\varphi) \overline{\varphi}(s + \ker\varphi)
            \end{aligned}
        \end{equation*}
        And so we are done. 
    \end{proof}
\end{mdframed}
\textbf{Remark:} If $I \subseteq R$ is an ideal, then $I = \ker q$ where $q: R \mapsto R / I$, defined by the mapping $r \mapsto r + I$, is the quotient homomorphism.

\subsection{Examples}

\begin{enumerate}
    \item Consider the homomorphism $\varphi: \Z[x] \mapsto \Z$ defined by the mapping $f(x) \mapsto f(0)$. $\varphi$ is a surjective\footnote{If $a \in \Z$, then $(x + a) \xrightarrow{\varphi} 0 + a = a$} homomorphism. By the First Isomorphism Theorem:
    \[\Z[x] / \ker\varphi \cong \Z\]
    Here, we define $\ker\varphi = \{a_1 x + a_2 x^2 + \dots + a_n x^n \mid a_i \in \Z\}$ because $f(0)$ is a constant term. However, we can factor $x$ out to get:
    \[\ker\varphi = \{x(a_1 + a_2 x^1 + \dots + a_n x^{n - 1}) \mid a_i \in \Z\} = \cyclic{x}\]
    And so it follows that:
    \[\boxed{\Z[x] / \cyclic{x} \cong \Z}\]


    \item Consider the homomorphism $\varphi: \R[x] \mapsto \C$ defined by the mapping $f(x) \mapsto f(i)$. $\varphi$ is surjective because $f(a + bx) = a + bi$ for any $a, b \in \R$. We also know that $x^2 + 1 \in \ker \varphi$ by $i^2 + 1 = 0$. This implies that: 
    \[\cyclic{x^2 + 1} \subseteq \ker \varphi \subset \R[x]\]
    \textbf{Fact:} $\cyclic{x^2 + 1}$ is maximal, which implies that $\cyclic{x^2 + 1} = \ker \varphi$.
    \begin{mdframed}[]
        \begin{proof}
            (Of fact.) We prove that $\R[x] / I$ for $I = \cyclic{x^2 + 1}$ is a field for any $a + bx + I$ with $a, b$ not both zero, then $(a + b + I)^{-1} = \frac{a - bx}{a^2 + b^2} + I$.
        \end{proof}
    \end{mdframed}
    Therefore, $\boxed{\R[x] / \cyclic{x^2 + 1} \cong \C}$ by the First Isomorphism Theorem. 
\end{enumerate}

\subsection{Rings with Unity}
\begin{proposition}
    If $R$ has unity, then $\varphi: \Z \mapsto \R$ defined by
    \[\varphi(n) = n \cdot 1 = \begin{cases}
        \underbrace{1 + \dots + 1}_{n \text{ times}} & n > 0 \\ 
        0 & n = 0 \\ 
        \underbrace{-1 - 1 - \dots - 1}_{-n \text{ times}} & n < 0
    \end{cases}\]
    is a homomorphism.
\end{proposition}

\begin{mdframed}[]
    \begin{proof}
        Left as an exercise.
    \end{proof}
\end{mdframed}


\begin{proposition}
    If $R$ is a ring with unity, then:
    \begin{enumerate}[(a)]
        \item If $\ch R = n > 0$, then $R$ contains a subring isomorphic to $\Z / n\Z$.
        \item If $\ch R = 0$, then $R$ contains a subring isomorphic to $\Z$.
    \end{enumerate}
\end{proposition}

\begin{mdframed}[]
    \begin{proof}
        Let $\varphi: \Z \mapsto \R$ with $\varphi(n) = n \cdot 1$. Then, $\ch R$ is the additive order of 1. This implies that if $\ch R = n > 0$, then $\ker \varphi = n\Z$ so $\Z / n\Z \cong \varphi(\Z) \subseteq R$ by the First Isomorphism Theorem. Likewise, if $\ch R = 0$, then $\ker \varphi = \{0\}$ and it follows that $\Z \cong \varphi(\Z) \subseteq R$ by the First Isomorphism Theorem. 
    \end{proof}
\end{mdframed}

\subsection{Fields}

\begin{definition}{Prime Subfield}{}
    The subfield of a field isomorphic to $\F_p$ or $\Q$ is called the \textbf{prime subfield}.
\end{definition}

\begin{theorem}{}{}
    \begin{itemize}
        \item If $F$ is a field of characteristic $p$, then $F$ contains a subfield isomorphic to $\F_p$.
        \item If $F$ is a field of characteristic 0, then $F$ contains a subfield isomorphic to $\Q$. 
    \end{itemize} 
\end{theorem}

\begin{mdframed}[]
    \begin{proof}
        We prove both parts.
        \begin{itemize}
            \item By the previous proposition, $\ch F = p$ implies that the subring is isomorphic to $\Z / p\Z = \F_p$.
            \item $\ch F = 0$ implies that the subring is isomorphic to $\Z$, given by 
            \[\varphi: \Z \mapsto F\]
            which sends $n \mapsto n \cdot 1$. Consider the set 
            \[T = \{ab^{-1} \mid a, b \in \varphi(\Z), b \neq 0\} \subseteq F\]
            We claim that $T$ is a subring isomorphic to $\Q$. 
            \begin{proof}
                Define $\overline{\varphi}: \Q \mapsto F$ by $\overline{\varphi}(a / b) = \varphi(a) \varphi(b)^{-1}$. 

                \begin{itemize}
                    \item \underline{Well-Defined:} This is well-defined since $\frac{a}{b} = \frac{c}{d}$ if and only if $ad = bc$, which then implies that $\varphi(a) \varphi(d) = \varphi(b) \varphi(c)$ for $\varphi: \Z \mapsto F$. This implies that $\varphi(a) \varphi(b)^{-1} = \varphi(c) \varphi(d)^{-1}$, which again implies that $\overline{\varphi}(a / b) = \overline{\varphi}(c / d)$.
                    \item \underline{Homomorphism:} Addition is left as an exercise. For multiplication, see lecture. 
                \end{itemize}
                So, we are done. 
            \end{proof}
        \end{itemize}
        And so on (need to come back).
    \end{proof}
\end{mdframed}

\textbf{Remark:} If $F$ is a field and $I \subseteq F$ is an ideal, then $I = \{0\}$ or $I = F$.

\begin{mdframed}[]
    \begin{proof}
        $F / \{0\} \cong F$ is a field. Thus, $\{0\}$ is a maximal ideal of $F$. This implies that, for all ideals $I$ with $\{0\} \subseteq I \subseteq F$, $I = \{0\}$ or $I = F$. The fact that all ideals satisfy $\{0\} \subseteq I \subseteq F$ concludes the proof.
    \end{proof}
\end{mdframed}


\end{document}