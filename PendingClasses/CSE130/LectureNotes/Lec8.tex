\documentclass[letterpaper]{article}
\usepackage[margin=1in]{geometry}
\usepackage[utf8]{inputenc}
\usepackage{textcomp}
\usepackage{amssymb}
\usepackage{natbib}
\usepackage{graphicx}
\usepackage{gensymb}
\usepackage{amsthm, amsmath, mathtools}
\usepackage{xcolor}
\usepackage{enumerate}
\usepackage{framed}
\usepackage{tcolorbox}
\tcbuselibrary{theorems}

\newcommand{\R}{\mathbb{R}}
\newcommand{\Z}{\mathbb{Z}}
\newcommand{\N}{\mathbb{N}}
\newcommand{\Q}{\mathbb{Q}}
\newcommand{\code}[1]{\texttt{#1}}
\newcommand{\mdiamond}{$\diamondsuit$}

%\newtheorem*{theorem}{Theorem}
%\newtheorem*{definition}{Definition}
\newtheorem*{proposition}{Proposition}
%\newtheorem*{corollary}{Corollary}
%\newtheorem*{lemma}{Lemma}

\newtcbtheorem[number within=section]{theorem}{Theorem}
{colback=green!5,colframe=green!35!black,fonttitle=\bfseries}{def}

\newtcbtheorem[number within=section]{definition}{Definition}
{colback=blue!5,colframe=blue!35!black,fonttitle=\bfseries}{def}

\newtcbtheorem[number within=section]{corollary}{Corollary}
{colback=yellow!5,colframe=yellow!35!black,fonttitle=\bfseries}{def}

\newtcbtheorem[number within=section]{lemma}{Lemma}
{colback=red!5,colframe=red!35!black,fonttitle=\bfseries}{def}
\usepackage[utf8]{inputenc}
\usepackage[english]{babel}
\usepackage{fancyhdr}
\usepackage[hidelinks]{hyperref}

\pagestyle{fancy}
\fancyhf{}
\rhead{CSE 130}
\chead{Wednesday, April 13, 2022}
\lhead{Lecture 8}
\rfoot{\thepage}

\setlength{\parindent}{0pt}

\begin{document}

\section{Haskell: An Introduction}
\textbf{Haskell} is essentially Lambda Calculus plus
\begin{itemize}
    \item Better syntax. 
    \item Types.
    \item Built-in features like primitives (booleans, numbers, etc.), records, lists, recurison, and more.
\end{itemize}

\subsection{Types}
In Haskell, every expression either \textbf{has a type} or is \textbf{ill-typed} and rejected statically (at compile-time, before execution begins). This is similar to Java (which is statically typed), and differs from languages like Python (which is dynamically typed).

\subsubsection{Type Annotations}
While the Haskell compiler can infer types, you should still annotate your bindings with their types using \code{::}, like so: 
\begin{verbatim}
    aBoolean :: Bool 
    aBoolean = True 

    message :: String 
    message = if aBoolean
                then "True!"
                else "False."

    rating :: Int 
    rating = if aBoolean
                then 10 
                else 0
\end{verbatim}
Note that we can use the GHCi command \code{:t} to inspect types. For example, 
\begin{verbatim}
    > :t if x then 'a' else 'b'             -- Char 
\end{verbatim}

\subsubsection{Function Types}
Functions have arrow types. That is, 
\begin{itemize}
    \item \code{\bsttt{x} -> e} has type \code{A -> B}. 
    \item In other words, \code{e} has type \code{B} if \code{x} has type \code{A}.
\end{itemize}
For example, 
\begin{verbatim}
    > :t \b -> if b then 'a' else 'n'       -- Bool -> Char 
\end{verbatim}
Note that we can also have functions with mulitple parameters; that is:
\begin{verbatim}
    pair :: String -> (String -> (Bool -> String))  
    pair :: String -> String -> Bool -> String      -- Same as above. 
    pair x y b = if b then x else y                 -- Definition of function.
\end{verbatim}

\begin{mdframed}[]
    (Quiz.) With \code{pair :: String -> String -> Bool -> String}, what would GGHCi say to 
    \begin{verbatim}
        > :t pair "apple" "orange"
    \end{verbatim}

    \begin{enumerate}[(a)]
        \item Syntax error 
        \item The term is ill-typed. 
        \item \code{String}
        \item \code{Bool -> String}
        \item \code{String -> String -> Bool -> String}
    \end{enumerate}

    \begin{mdframed}[]
        The answer is \textbf{D}. Recall that the annotation given is just syntactic sugar for 
        \begin{verbatim}
            pair :: String -> (String -> (Bool -> String)).\end{verbatim}
        So, if we pass in two \code{String}s, we get back a \code{Bool -> String}.
    \end{mdframed}
\end{mdframed}

Like with general variables, function bindings should be annotated. 



\subsection{Lists}
A list is either: 
\begin{itemize}
    \item An empty list
    \begin{verbatim}
        []              -- pronounced "nil"\end{verbatim}

    \item Or, a head element attached to a tail list.
    \begin{verbatim}
        x:xs            -- pronounced "x cons xs"\end{verbatim}
    \textbf{Remarks:}
    \begin{itemize}
        \item Note that this is pronounced \code{cons} for historical element. 
        \item The head element is just the first element. The \emph{tail} list is everything after the head element (not the last element).
    \end{itemize}
\end{itemize}

\subsubsection{Example List Declarations}
\begin{verbatim}
    []                              -- A list with 0 elements. 

    1:[]                            -- A list with 1 element. 
                                    -- This is essentially just a cons of a head 
                                    -- which is the number 1, followed by an empty 
                                    -- list.

    (:) 1 []                        -- Same thing as above. Any infix operator 'op' in 
                                    -- Haskell can be transformed into a regular 
                                    -- function which can be called in infix notation
                                    -- by putting it in parentheses ('op'). 

    1:(2:(3:(4:[])))                -- A list with 4 elements. 

    1:2:3:4:[]                      -- Same thing (cons, :, is right associative).

    [1, 2, 3, 4]                    -- Same thing (syntactic sugar for 
                                    -- 1:(2:(3:(4:[]))) ).
\end{verbatim}
\textbf{Remark:} With regards to the infix operator in the third example, this is not just exclusive to lists. For example, \code{2 + 3} can be equivalently written as \boxed{\code{(+) 2 3}}. Likewise, \code{cmpSquare 2 3} can be equivalently written\footnote{Using backticks to denote the function name.} as \boxed{\code{2 `cmpSquare` 3}}.

\subsubsection{Constructors and Values}
\code{[]} and \code{(:)} are known as the list \textbf{constructors}. We've seen constructors before; for example, 
\begin{itemize}
    \item \code{True} and \code{False} are both \code{Bool} constructors. 
    \item \code{0}, \code{1}, \code{2}, and so can be \emph{thought of} as \code{Int} constructors.
    \item These constructors didn't take any parameters; so, we call then \emph{values}.
\end{itemize}
In general, a \textbf{value} is a constructor applied to other values. The list examples above are list \emph{values}. 

\end{document}