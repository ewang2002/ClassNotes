\documentclass[letterpaper]{article}
\usepackage[margin=1in]{geometry}
\usepackage[utf8]{inputenc}
\usepackage{textcomp}
\usepackage{amssymb}
\usepackage{natbib}
\usepackage{graphicx}
\usepackage{gensymb}
\usepackage{amsthm, amsmath, mathtools}
\usepackage[dvipsnames]{xcolor}
\usepackage{enumerate}
\usepackage{mdframed}
\usepackage[most]{tcolorbox}
\usepackage{csquotes}
% https://tex.stackexchange.com/questions/13506/how-to-continue-the-framed-text-box-on-multiple-pages

\tcbuselibrary{theorems}

\newcommand{\R}{\mathbb{R}}
\newcommand{\Z}{\mathbb{Z}}
\newcommand{\N}{\mathbb{N}}
\newcommand{\Q}{\mathbb{Q}}
\newcommand{\C}{\mathbb{C}}
\newcommand{\code}[1]{\texttt{#1}}
\newcommand{\mdiamond}{$\diamondsuit$}
\newcommand{\PowerSet}{\mathcal{P}}
\newcommand{\Mod}[1]{\ (\mathrm{mod}\ #1)}
\DeclareMathOperator{\lcm}{lcm}

%\newtheorem*{theorem}{Theorem}
%\newtheorem*{definition}{Definition}
%\newtheorem*{corollary}{Corollary}
%\newtheorem*{lemma}{Lemma}
\newtheorem*{proposition}{Proposition}


\newtcbtheorem[number within=section]{theorem}{Theorem}
{colback=green!5,colframe=green!35!black,fonttitle=\bfseries}{th}

\newtcbtheorem[number within=section]{definition}{Definition}
{colback=blue!5,colframe=blue!35!black,fonttitle=\bfseries}{def}

\newtcbtheorem[number within=section]{corollary}{Corollary}
{colback=yellow!5,colframe=yellow!35!black,fonttitle=\bfseries}{cor}

\newtcbtheorem[number within=section]{lemma}{Lemma}
{colback=red!5,colframe=red!35!black,fonttitle=\bfseries}{lem}

\newtcbtheorem[number within=section]{example}{Example}
{colback=white!5,colframe=white!35!black,fonttitle=\bfseries}{def}

\newtcbtheorem[number within=section]{note}{Important Note}{
        enhanced,
        sharp corners,
        attach boxed title to top left={
            xshift=-1mm,
            yshift=-5mm,
            yshifttext=-1mm
        },
        top=1.5em,
        colback=white,
        colframe=black,
        fonttitle=\bfseries,
        boxed title style={
            sharp corners,
            size=small,
            colback=red!75!black,
            colframe=red!75!black,
        } 
    }{impnote}
\usepackage[utf8]{inputenc}
\usepackage[english]{babel}
\usepackage{fancyhdr}
\usepackage[hidelinks]{hyperref}

\pagestyle{fancy}
\fancyhf{}
\rhead{CSE 130}
\chead{Monday, May 16, 2022}
\lhead{Lecture 21}
\rfoot{\thepage}

\setlength{\parindent}{0pt}

\begin{document}
\section{Lexing and Parsing}
\subsection{Running the Parser}

Recall that there were two issues. 
\begin{itemize}
    \item Wrong Precedence
    \begin{verbatim}
    $ evalString [] "2 * 5 + 5"
    20\end{verbatim}
    This is invalid, and is due to the fact that our grammar is \textbf{ambiguous} -- there are multiple ways to parse the string \code{2 * 5 + 5}; one way is correct (\code{(2 * 5) + 5}) and one way is incorrect (\code{2 * (5 + 5)}). Thus, we want to tell \code{happy} that \code{*} has higher precedence than \code{+}.

    \item Wrong Associativity
    \begin{verbatim}
    $ evalString [] "2 - 1 - 1"
    2\end{verbatim}
    So, we also need to tell \code{happy} that \code{-} is \textbf{left-associative}.     
\end{itemize}
Therefore, we need to tell \code{happy} about precedence and associativity. 

\subsubsection{Solution 1: Grammar Factoring}
We can split the \code{AExpr} non-terminal into multiple ``levels.''
\begin{verbatim}
    Aexpr : Aexpr '+' Aexpr
          | Aexpr '-' Aexpr
          | Aexpr2

    Aexpr2 : Aexpr2 '*' Aexpr2
          | Aexpr2 '/' Aexpr2
          | Aexpr3

    Aexpr3 : TNUM
          | ID
          | '(' Aexpr ')'\end{verbatim}
Note that \code{AExpr} is the most general term. Then, \code{AExpr2} is slightly more specific; it only has expressions relating to multiplication and division, and numbers/identifiers. \code{AExpr3} is the most strict, with only numbers/identifiers. Intuitively, \code{AExpr2} will ``bind tighter'' than \code{AExpr}. We also have \code{AExpr3}, which is the ``tightest.''

\bigskip 

This fixes the issue with \code{2 * 5 + 5}, but fails to parse \code{5 + 5}.

\begin{mdframed}[]
    (Quiz.) With this new grammar, can we parse \code{2 - 1 - 1} the wrong way?

    \begin{enumerate}[(a)]
        \item Yes.
        \item No.
    \end{enumerate}

    \begin{mdframed}[]
        The answer is \textbf{B}. There are still multiple ways to parse \code{2 - 1 - 1}.
    \end{mdframed}
\end{mdframed}
How do we fix this? One way to do so is to disallow the right-hand side of a minus to be a minus. 

\begin{verbatim}
    Aexpr : Aexpr '+' Aexpr2
          | Aexpr '-' Aexpr2
          | Aexpr2

    Aexpr2 : Aexpr2 '*' Aexpr3
          | Aexpr2 '/' Aexpr3
          | Aexpr3

    Aexpr3 : TNUM
          | ID
          | '(' Aexpr ')'\end{verbatim}

\subsubsection{Solution 2: Parser Directives}
We can just use a special syntax for the parser generator; in \code{happy}, this is the syntax\footnote{Other parser generators may use other syntax.}: 
\begin{verbatim}
    %left '+' '-'
    %left '*' '/'\end{verbatim}
This means that 
\begin{itemize}
    \item All our operations are left-associative. 
    \item Operators on the lower line have higher precedence.
\end{itemize}
Note that operations like \emph{applications}, which is left-associative, do not have an operator; thus, you still need to worry about things like these. 




\end{document}