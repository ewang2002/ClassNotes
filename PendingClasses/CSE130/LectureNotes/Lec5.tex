\documentclass[letterpaper]{article}
\usepackage[margin=1in]{geometry}
\usepackage[utf8]{inputenc}
\usepackage{textcomp}
\usepackage{amssymb}
\usepackage{natbib}
\usepackage{graphicx}
\usepackage{gensymb}
\usepackage{amsthm, amsmath, mathtools}
\usepackage[dvipsnames]{xcolor}
\usepackage{enumerate}
\usepackage{mdframed}
\usepackage[most]{tcolorbox}
\usepackage{csquotes}
% https://tex.stackexchange.com/questions/13506/how-to-continue-the-framed-text-box-on-multiple-pages

\tcbuselibrary{theorems}

\newcommand{\R}{\mathbb{R}}
\newcommand{\Z}{\mathbb{Z}}
\newcommand{\N}{\mathbb{N}}
\newcommand{\Q}{\mathbb{Q}}
\newcommand{\C}{\mathbb{C}}
\newcommand{\code}[1]{\texttt{#1}}
\newcommand{\mdiamond}{$\diamondsuit$}
\newcommand{\PowerSet}{\mathcal{P}}
\newcommand{\Mod}[1]{\ (\mathrm{mod}\ #1)}
\DeclareMathOperator{\lcm}{lcm}

%\newtheorem*{theorem}{Theorem}
%\newtheorem*{definition}{Definition}
%\newtheorem*{corollary}{Corollary}
%\newtheorem*{lemma}{Lemma}
\newtheorem*{proposition}{Proposition}


\newtcbtheorem[number within=section]{theorem}{Theorem}
{colback=green!5,colframe=green!35!black,fonttitle=\bfseries}{th}

\newtcbtheorem[number within=section]{definition}{Definition}
{colback=blue!5,colframe=blue!35!black,fonttitle=\bfseries}{def}

\newtcbtheorem[number within=section]{corollary}{Corollary}
{colback=yellow!5,colframe=yellow!35!black,fonttitle=\bfseries}{cor}

\newtcbtheorem[number within=section]{lemma}{Lemma}
{colback=red!5,colframe=red!35!black,fonttitle=\bfseries}{lem}

\newtcbtheorem[number within=section]{example}{Example}
{colback=white!5,colframe=white!35!black,fonttitle=\bfseries}{def}

\newtcbtheorem[number within=section]{note}{Important Note}{
        enhanced,
        sharp corners,
        attach boxed title to top left={
            xshift=-1mm,
            yshift=-5mm,
            yshifttext=-1mm
        },
        top=1.5em,
        colback=white,
        colframe=black,
        fonttitle=\bfseries,
        boxed title style={
            sharp corners,
            size=small,
            colback=red!75!black,
            colframe=red!75!black,
        } 
    }{impnote}
\usepackage[utf8]{inputenc}
\usepackage[english]{babel}
\usepackage{fancyhdr}
\usepackage[hidelinks]{hyperref}

\pagestyle{fancy}
\fancyhf{}
\rhead{CSE 130}
\chead{Wednesday, April 06, 2022}
\lhead{Lecture 5}
\rfoot{\thepage}

\setlength{\parindent}{0pt}

\begin{document}

\section{Lambda Calculus}

\subsection{Programming in Lambda Calculus}
How do we encode features like
\begin{itemize}
    \item Booleans
    \item Records
    \item Numbers
    \item Recursion
\end{itemize}
and more? 

\subsubsection{Booleans}
How can we encode Boolean values, like \code{TRUE} or \code{FALSE}, as functions? 

\bigskip 

To answer this question, we ask another one: what \emph{do} we do with a Boolean \code{b}? Making a binary choice is one: 
\begin{verbatim}
    if b then E1 else E2
\end{verbatim}
We want to define three functions 
\begin{verbatim}
    let TRUE = ??? 
    let FALSE = ??? 
    let ITE = \b x y -> ???         -- if b then x else y
\end{verbatim}
such that 
\begin{verbatim}
    ITE TRUE apple banana =~> apple 
    ITE FALSE apple banana =~> banana
\end{verbatim}
Our implementation is as follows: 
\begin{verbatim}
    let TRUE = \x y -> x        -- Returns the first argument
    let FALSE = \x y -> y       -- Returns the second argument 
    let ITE = \b x y -> b x y   -- Applies condition to branch
\end{verbatim}
To see how this works, suppose we want to evaluate \code{ITE TRUE egg ham}, which should resolve to \code{egg}. We have:  
\begin{verbatim}
    eval ite_true: 
        ITE TRUE egg ham
        =d> (\b x y -> b x y) TRUE egg ham      -- Expand ITE 
        =b> (\x y -> TRUE x y) egg ham          -- Beta-step on TRUE 
        =b> (\y -> TRUE egg y) ham              -- Beta-step on egg 
        =b> TRUE egg ham                        -- Beta-step on ham 
        =d> (\x y -> x) egg ham                 -- Expand TRUE 
        =b> (\y -> egg)                         -- Beta-step on egg 
        =b> egg                                 -- Beta-step on ham
\end{verbatim}

\subsubsection{Boolean Operators}
Now that we have \code{TRUE}, \code{FALSE}, and \code{ITE}, we can define other Boolean operators like: 
\begin{verbatim}
    let NOT = \b     -> ??? 
    let AND = \b1 b2 -> ???
    let OR  = \b1 b2 -> ???
\end{verbatim}
Recall that: 
\[\text{NOT}(b) = \begin{cases}
    \code{FALSE} & b \text{ is \code{TRUE}} \\ 
    \code{TRUE} & b \text{ is \code{FALSE}}
\end{cases}\]
\[\text{AND}(b_1, b_2) = \begin{cases}
    \code{TRUE} & b_1 \text{ is \code{TRUE} and } b_2 \text{ is \code{TRUE}} \\ 
    \code{FALSE} & \text{Otherwise}
\end{cases}\]
\[\text{OR}(b_1, b_2) = \begin{cases}
    \code{TRUE} & b_1 \text{ is \code{TRUE} or } b_2 \text{ is \code{TRUE}} \\ 
    \code{FALSE} & \text{Otherwise}
\end{cases}\]

The implementation is as follows:
\begin{verbatim}
    let NOT = \b        -> ITE b FALSE TRUE 
                        -> b FALSE TRUE 

    let AND = \b1 b2    -> ITE b1 (ITE b2 TRUE FALSE) FALSE 
                        -> ITE b1 b2 FALSE
                        -> b1 b2 FALSE  

    let OR = \b1 b2     -> ITE b1 TRUE b2
                        -> b1 TRUE b2 
\end{verbatim}

\subsubsection{Records}
A record (tuple) is a way to bundle multiple values together and then access them. The simplest kind of a record is a \textbf{pair}, which holds two values. In particular, a pair can: 
\begin{itemize}
    \item Pack two items into a pair. 
    \item Get the first item. 
    \item Get the second item. 
\end{itemize}

We need to implement the following functions: 
\begin{verbatim}
    let MKPAIR = \x y -> ???        -- Makes a pair with elements x and y
    let FST    = \p   -> ???        -- Returns the first element of the pair p 
    let SND    = \p   -> ???        -- Returns the second element of the pair p 
\end{verbatim}
The functions work like so: 
\begin{verbatim}
    FST (MKPAIR apple banana) =~> apple 
    SND (MKPAIR apple banana) =~> banana 
\end{verbatim}
One thing to notice is that we can use a \emph{boolean} value to indicate whether we want the first or second element. So, creating a pair is the same thing as creating a function which returns value $x$ or $y$ based on whether \code{TRUE} or \code{FALSE} is passed in.
\begin{verbatim}
    let MKPAIR = \x y -> (\b -> ITE b x y)
    let FST    = \p -> p TRUE               -- Returns the first element 
    let SND    = \p -> p FALSE              -- Returns the second element
\end{verbatim}
Now, suppose we want to make a triple $(x, y, z)$. The idea is that we can make use of two pairs, like so: $((x, y), z)$. We can define our implementation like so: 
\begin{verbatim}
    let MKTRIPLE = \x y z -> MKPAIR (MKPAIR x y) z
    let FST3     = \t     -> FST (FST t)
    let SND3     = \t     -> SND (FST t)                -- Apply FST to t first 
                                                        -- and then apply SND to FST t
    let TRD3     = \t     -> SND t
\end{verbatim}
Alternatively, if we have $(x, (y, z))$, our implementation will be: 
\begin{verbatim}
    let MKTRIPLE = \x y z -> MKPAIR x (MKPAIR y z)
    let FST3     = \t     -> FST t
    let SND3     = \t     -> FST (SND t)
    let TRD3     = \t     -> SND (SND t)
\end{verbatim}

\subsubsection{Numbers}
Let us start with natural numbers $\{0, 1, 2, \dots\}$. What can we do with natural numbers? 
\begin{itemize}
    \item We can count them: \code{0}, \code{inc}. 
    \item Arithmetic: \code{dec}, \code{+}, \code{-}, \code{*}.
    \item Comparisons: \code{==}, \code{<=}, etc. 
\end{itemize}
We need to define the following: 
\begin{itemize}
    \item A family of \textbf{numerals}: \code{ZERO}, \code{ONE}, \code{TWO}, \code{THREE}.
    \item Arithmetic functions: \code{INC}, \code{DEC}, \code{ADD}, \code{SUB}, \code{MULT}.
    \item Comparisons: \code{IS\_ZERO}, \code{EQ}.
\end{itemize}
Where they respect all regular laws of arithmetic; for example: 
\begin{verbatim}
    IS_ZERO ZERO        =~> TRUE 
    IS_ZERO (INC ZERO)  =~> FALSE 
    INC ONE             =~> TWO 
        .
        . 
        .
\end{verbatim}
How do we implement numerals? We can define a numeral as the number of times we can apply a function. In particular, we define a \textbf{church numeral} as a number $N$ which is encoded as an combinator that calls a function on an argument $N$ times. 
\begin{verbatim}
    let ZERO  = \f x -> x
    let ONE   = \f x -> f x
    let TWO   = \f x -> f (f x)
    let THREE = \f x -> f (f (f x))
    let FOUR  = \f x -> f (f (f (f x)))
    let FIVE  = \f x -> f (f (f (f (f x))))
    let SIX   = \f x -> f (f (f (f (f (f x)))))
        .
        .
        .
\end{verbatim} 

\end{document}