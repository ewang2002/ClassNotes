\documentclass[letterpaper]{article}
\usepackage[margin=1in]{geometry}
\usepackage[utf8]{inputenc}
\usepackage{textcomp}
\usepackage{amssymb}
\usepackage{natbib}
\usepackage{graphicx}
\usepackage{gensymb}
\usepackage{amsthm, amsmath, mathtools}
\usepackage[dvipsnames]{xcolor}
\usepackage{enumerate}
\usepackage{mdframed}
\usepackage[most]{tcolorbox}
\usepackage{csquotes}
% https://tex.stackexchange.com/questions/13506/how-to-continue-the-framed-text-box-on-multiple-pages

\tcbuselibrary{theorems}

\newcommand{\R}{\mathbb{R}}
\newcommand{\Z}{\mathbb{Z}}
\newcommand{\N}{\mathbb{N}}
\newcommand{\Q}{\mathbb{Q}}
\newcommand{\C}{\mathbb{C}}
\newcommand{\code}[1]{\texttt{#1}}
\newcommand{\mdiamond}{$\diamondsuit$}
\newcommand{\PowerSet}{\mathcal{P}}
\newcommand{\Mod}[1]{\ (\mathrm{mod}\ #1)}
\DeclareMathOperator{\lcm}{lcm}

%\newtheorem*{theorem}{Theorem}
%\newtheorem*{definition}{Definition}
%\newtheorem*{corollary}{Corollary}
%\newtheorem*{lemma}{Lemma}
\newtheorem*{proposition}{Proposition}


\newtcbtheorem[number within=section]{theorem}{Theorem}
{colback=green!5,colframe=green!35!black,fonttitle=\bfseries}{th}

\newtcbtheorem[number within=section]{definition}{Definition}
{colback=blue!5,colframe=blue!35!black,fonttitle=\bfseries}{def}

\newtcbtheorem[number within=section]{corollary}{Corollary}
{colback=yellow!5,colframe=yellow!35!black,fonttitle=\bfseries}{cor}

\newtcbtheorem[number within=section]{lemma}{Lemma}
{colback=red!5,colframe=red!35!black,fonttitle=\bfseries}{lem}

\newtcbtheorem[number within=section]{example}{Example}
{colback=white!5,colframe=white!35!black,fonttitle=\bfseries}{def}

\newtcbtheorem[number within=section]{note}{Important Note}{
        enhanced,
        sharp corners,
        attach boxed title to top left={
            xshift=-1mm,
            yshift=-5mm,
            yshifttext=-1mm
        },
        top=1.5em,
        colback=white,
        colframe=black,
        fonttitle=\bfseries,
        boxed title style={
            sharp corners,
            size=small,
            colback=red!75!black,
            colframe=red!75!black,
        } 
    }{impnote}
\usepackage[utf8]{inputenc}
\usepackage[english]{babel}
\usepackage{fancyhdr}
\usepackage[hidelinks]{hyperref}

\pagestyle{fancy}
\fancyhf{}
\rhead{CSE 130}
\chead{Wednesday, May 25, 2022}
\lhead{Lecture 25}
\rfoot{\thepage}

\setlength{\parindent}{0pt}

\begin{document}

\section{Monads}
\subsection{Functors}
\subsubsection{Class for Mapping}
Recall 
\begin{verbatim}
    eval :: Expr -> Result Int 
    eval (Num n)            = Value n 
    eval (Plus e1 e2)       = 
        case eval e1 of 
            Error err1  -> Error err1 
            Value v1    -> case eval e2 of 
                            Error err2  -> Error err2 
                            Value v1    -> Value (v1 + v2)
    
    eval (Div e1 e2)        = 
        case eval e1 of 
            Error err1  -> Error err1 
            Value v1    -> case eval e2 of 
                            Error err2  -> Error err2 
                            Value v1    -> if v2 == 0 
                                            then Error ("DBZ: " ++ show e2)
                                            else Value (v1 `div` v2)\end{verbatim}
Notice how there is a common pattern. Our goal is to refactor the common pattern so we can avoid repeating it. In this case, the pattern of interest is 
\begin{verbatim}
    case eval e2 of 
        Error err2  -> Error err2 
        Value v1    -> ... \end{verbatim}
We always have some \code{case ... of}, where the \code{...} is of type \code{Result}. If we get an \code{Error}, then we just return that \code{Error} If we get a \code{Value}, we will return that \code{Value}. So, how do we abstract this into a pattern? 
\begin{verbatim}
    bind :: Result a -> (a -> Result b) -> Result b
    bind res process = case res of 
        Error msg   -> Error msg
        Value v     -> process v\end{verbatim}
This can be refactored to:
\begin{verbatim}
    bind :: Result a -> (a -> Result b) -> Result b 
    bind (Error msg) _          = Error msg 
    bind (Value v)   process    = process v\end{verbatim}
Finally, we can make this into an infix operator: 
\begin{verbatim}
    (>>=) :: Result a -> (a -> Result b) -> Result b 
    (>>=) (Error msg) _          = Error msg 
    (>>=) (Value v)   process    = process v\end{verbatim}
Rewriting this to look more natural, we have 
\begin{verbatim}
    (>>=) :: Result a -> (a -> Result b) -> Result b
    (Error msg) >>= _       = Error msg
    (Value v)   >>= process = process v\end{verbatim}
So, \code{>>=} takes two inputs: 
\begin{itemize}
    \item \code{Result a}: The result of the first evaluation.
    \item \code{a -> Result b}: In case the first evaluation produced a value, what to do \emph{next} with the value.
\end{itemize}

\begin{mdframed}
    (Quiz.) With \code{>>=} defined as before, what does the following evaluate to? 
    \begin{verbatim}
        eval (Num 5) >>= \v -> Value (v + 1)\end{verbatim}

    \begin{enumerate}[(a)]
        \item Type Error. 
        \item \code{5}
        \item \code{Value 5}
        \item \code{Value 6}
        \item \code{Error msg}
    \end{enumerate}

    \begin{mdframed}
        The answer is \textbf{D}. Recall that our \code{Result} is defined by 
        \begin{verbatim}
            data Result a
                = Error String 
                | Value a\end{verbatim}
        So, \code{eval (Num 5)} will give us back a \code{Value 5}. Since this is not an error, we can extract \code{5} from \code{Value 5}, and then pass \code{5} into the function to get \code{Value 5 + 1 = Value 6}.
    \end{mdframed}
\end{mdframed}

\begin{mdframed}[]
    (Quiz.) With \code{>>=} defined as before, what does the following evaluate to? 
    \begin{verbatim}
        eval (Error "nope") >> \v -> Value (v + 1)\end{verbatim}
    \begin{enumerate}[(a)]
        \item Type Error. 
        \item \code{5}
        \item \code{Value 5}
        \item \code{Value 6}
        \item \code{Error "nope"}
    \end{enumerate}
    
    \begin{mdframed}[]
        The answer is \textbf{E}. Because we have an error \code{Error "nope"}, we can immediately use the \code{Error} pattern.
    \end{mdframed}
\end{mdframed}
So, with this new function, we can do 
\begin{verbatim}
    eval :: Expr -> Result Int
    eval (Num n)      = Value n
    eval (Plus e1 e2) = eval e1 >>= \v1 ->
                        eval e2 >>= \v2 ->
                        Value (v1 + v2)
    eval (Div e1 e2)  = eval e1 >>= \v1 ->
                        eval e2 >>= \v2 ->
                        if v2 == 0
                          then Error ("DBZ: " ++ show e2)
                          else Value (v1 `div` v2)\end{verbatim}

\subsection{Monads}
Note that \code{>>=}, like \code{fmap} or \code{show} or \code{==}, can be useful across many types, not just \code{Result}. Let us define a type class for it. 

\begin{verbatim}
    class Monad m where 
        (>>=)   :: m a -> (a -> m b) -> m b         -- Bind 
        return  :: a -> m a                         -- Return \end{verbatim}

\subsubsection{Monad Instance for Result}
Now, let's make \code{Result} an instance of \code{Monad}.

\begin{verbatim}
    instance Monad Result where 
        (>>=)                   :: Result a -> (a -> Result b) -> Rseult b 
        (Error msg) >>= _       = Error msg
        (Value v)   >>= process = process v
        
        return :: a -> Result a
        return v = Value v\end{verbatim}

With this in mind, we can simplify our \code{eval} further.
\begin{verbatim}
    eval :: Expr -> Result Int
    eval (Num n)      = return n
    eval (Plus e1 e2) = do v1 <- eval e1
                           v2 <- eval e2
                           return (v1 + v2)
    eval (Div e1 e2)  = do v1 <- eval e1
                           v2 <- eval e2
                           if v2 == 0
                                then Error ("DBZ: " ++ show e2)
                                else return (v1 `div` v2)\end{verbatim}
Note that \code{>>=} is so useful that there is a special syntax for it; known as a \code{do} block, instead of writing 
\begin{verbatim}
    e1 >>= \v1 ->
    e2 >>= \v2 ->
    e3 >>= \v3 ->
    e\end{verbatim}
we can just write 
\begin{verbatim}
    do  v1 <- e1 
        v2 <- e2 
        v3 <- e3 
        e\end{verbatim}
Therefore, we can simplify our \code{eval} to 
\begin{verbatim}
    eval :: Expr -> Result Int
    eval (Num n)      = return n
    eval (Plus e1 e2) = do v1 <- eval e1
                           v2 <- eval e2
                           return (v1 + v2)
    eval (Div e1 e2)  = do v1 <- eval e1
                           v2 <- eval e2
                           if v2 == 0
                                then Error ("DBZ: " ++ show e2)
                                else return (v1 `div` v2)\end{verbatim}
\subsubsection{Either Monad}
Knowing that error handling is a common task, instead of defining our own \code{Result} type, we can use \code{Either} from the Haskell standard library. So, 
\begin{verbatim}
    data Either a b
        = Left  a       -- Something has gone wrong. 
        | Right b       -- Everything has gone right. \end{verbatim}
Since \code{Either} is already an instance of \code{Monad}, we do not need to define our own \code{>>=}.


\end{document}