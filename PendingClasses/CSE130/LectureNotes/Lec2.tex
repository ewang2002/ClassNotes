\documentclass[letterpaper]{article}
\usepackage[margin=1in]{geometry}
\usepackage[utf8]{inputenc}
\usepackage{textcomp}
\usepackage{amssymb}
\usepackage{natbib}
\usepackage{graphicx}
\usepackage{gensymb}
\usepackage{amsthm, amsmath, mathtools}
\usepackage{xcolor}
\usepackage{enumerate}
\usepackage{framed}
\usepackage{tcolorbox}
\tcbuselibrary{theorems}

\newcommand{\R}{\mathbb{R}}
\newcommand{\Z}{\mathbb{Z}}
\newcommand{\N}{\mathbb{N}}
\newcommand{\Q}{\mathbb{Q}}
\newcommand{\code}[1]{\texttt{#1}}
\newcommand{\mdiamond}{$\diamondsuit$}

%\newtheorem*{theorem}{Theorem}
%\newtheorem*{definition}{Definition}
\newtheorem*{proposition}{Proposition}
%\newtheorem*{corollary}{Corollary}
%\newtheorem*{lemma}{Lemma}

\newtcbtheorem[number within=section]{theorem}{Theorem}
{colback=green!5,colframe=green!35!black,fonttitle=\bfseries}{def}

\newtcbtheorem[number within=section]{definition}{Definition}
{colback=blue!5,colframe=blue!35!black,fonttitle=\bfseries}{def}

\newtcbtheorem[number within=section]{corollary}{Corollary}
{colback=yellow!5,colframe=yellow!35!black,fonttitle=\bfseries}{def}

\newtcbtheorem[number within=section]{lemma}{Lemma}
{colback=red!5,colframe=red!35!black,fonttitle=\bfseries}{def}
\usepackage[utf8]{inputenc}
\usepackage[english]{babel}
\usepackage{fancyhdr}
\usepackage[hidelinks]{hyperref}
\newcommand{\bsttt}[1]{\string\ \hspace{-4pt}#1{}}

\pagestyle{fancy}
\fancyhf{}
\rhead{CSE 130}
\chead{Wednesday, March 30, 2022}
\lhead{Lecture 2}
\rfoot{\thepage}

\setlength{\parindent}{0pt}

\begin{document}

\section{Lambda Calculus}
Most programming languages have modern features like 
\begin{itemize}
    \item Assignments
    \item Types
    \item Conditions
    \item Loops
    \item Classes
    \item And so on\dots
\end{itemize}

However the smallest universal language doesn't need \emph{any of these} at all. \textbf{What is the smallest universal language?}

\subsection{The Smallest Universal Language}

\subsubsection{What is Computable?}
Before the 1930s, therw was the informal notion of an \emph{effectively calculable} function. That is, it can be computed by a human with pen and paper, followed by an algorithm. 

\bigskip 

\subsubsection{Formalization of a Language}
\textbf{Alan Turing} introduced the \emph{Turing Machine}, which is an infinite tape with some symbols, a head that can read from and write to the tape. To actually interact with a Turing machine, it has a state machine which has a bunch of states, along with a transition function, which tells it how to move around and what to do. The programming language is essentially the transition function of the Turing machine. 

\bigskip 

\textbf{Alonzo Church} came up with the \emph{Lambda Calculus}, which is (in some sense) simpler than a Turing machine. 

\bigskip 

\textbf{Peter Landin} used the Lambda Calculus to formalize the notion of a programming language. Lambda Calculus was influential in the creation of many modern programming languages, especially functional programming languages like Haskell. 

\subsection{The Lambda Calculus}
It has one feature: \emph{functions}. It does not have assignments, primitive types, control flow, recursion, etc. It literally only has \emph{functions}. Specifically, you can
\begin{itemize}
    \item define a function. 
    \item call a function. 
\end{itemize}

\subsubsection{Describing a Programming Language}
We're interested in two things: 
\begin{itemize}
    \item Syntax: what do programs look like? We use formal grammars (context-free grammars) to explain the syntax of a programming language.
    \item Semantics: what do programs mean? Specifically, \emph{operational semantics}, or the idea of how programs execute step-by-step. 
\end{itemize}

\subsubsection{Syntax}
We have one syntaxtical category: expressions (also called $\lambda$-terms).
\[\underbrace{\code{E}}_{\text{Expression}}\]
What can this expression expand to? 
\[\code{E ::= } \underbrace{\code{x}}_{\text{Variable}} \code{ | } \overbrace{\code{\bsttt{x} -> E}}^{\text{Abstraction}} \code{ | } \underbrace{\code{E1 E2}}_{\text{Application}}\]
Note that \code{x} can be any variable, e.g. \code{y} or \code{z} or even something like \code{apple}. 

\bigskip 

We should think of variables as mathematical variables (immutable). It does not change its value over time; it's like a variable in math where all it does is holds its value. 

\bigskip 

We can think of \code{x -> E} in the following mathematical way: 
\[f(x) = E\]
This is, specifically, a nameless function that takes in input $x$ and returns an expression $E$. 

\bigskip 

For calling functions, in math, we do something like $f(5)$. In Lambda Calculus, we do \code{f 5}. 


\subsubsection{Examples}
Consider the following program. 
\begin{verbatim}
    apple
\end{verbatim}
This does nothing, but is a syntaxtically value program in Lambda Calculus.

\bigskip 

Consider the following program. 
\begin{verbatim}
    apple banana
\end{verbatim}
This is an \emph{application} of the variable \code{apple} to the variable \code{banana}.

\bigskip 

Consider the following program.
\begin{verbatim}
    \x -> x
\end{verbatim}
The identity function, which says that \emph{for any \code{x}, compute \code{x}}. This is the first program which is meaningful for us; it is a \emph{very} important function. 

\bigskip 

Consider the following program.
\begin{verbatim}
    (\x -> x) apple
\end{verbatim}
This (whole program) is an application of the identity function to the variable \code{apple}. The result of this program will be \code{apple}. Note that \code{()} is not in the grammar that we described above; the grammar is known as an \emph{abstract syntax}, which simply describes what the programming language should have. In other words, the parenthesis are ignored. Note that if we have 
\begin{verbatim}
    \x -> x apple 
\end{verbatim}
which is a different program. Here, \code{x apple} is the body. 

\bigskip 

Consider the following program.
\begin{verbatim}
    \x -> (\y -> y)
\end{verbatim}
Here, we introduce another variable \code{y}. This takes one argument \code{x}, completely ignores \code{x}, and returns the identity function. Comparing this to the previous example, all we're doing is changing the name of the formal parameter. 

\bigskip 

Consider the following program.
\begin{verbatim}
    \f -> f (\x -> x)
\end{verbatim}
All this does is takes a argument \code{f}, and applies that argument to the identity function. 

\subsubsection{Two Input Arguments}
Suppose you wanted a function that takes arguments \code{x} and \code{y} and returns \code{y}? 

\bigskip 

Consider the following program.
\begin{verbatim}
    \x -> (\y -> y)
\end{verbatim}
Here, this function returns the identity function. This is the same thing as a function that takes two arguments and returns the second one. 

\subsubsection{Applying Function to Two Arguments}
For example, how do we apply 
\begin{verbatim}
    \x -> (\y -> y)
\end{verbatim}
to \code{apple} and \code{banana}?

\bigskip 

We can do something like 
\begin{verbatim}
    ((\x -> (\y -> y)) apple) banana 
\end{verbatim}
This first applies \code{apple} and then applies to \code{banana}. 


\subsubsection{Syntatical Sugar}

\begin{itemize}
    \item Instead of 
    \begin{verbatim}
        \x -> (\y -> (\z -> E))
    \end{verbatim}
    we can write 
    \begin{verbatim}
        \x -> \y -> \z -> E
    \end{verbatim}

    \item Instead of 
    \begin{verbatim}
        \x -> \y -> \z -> E
    \end{verbatim}
    we can write 
    \begin{verbatim}
        \x y z -> E
    \end{verbatim}

    \item Instead of 
    \begin{verbatim}
        (((E1 E2) E3) E4)
    \end{verbatim}
    we can write 
    \begin{verbatim}
        E1 E2 E3 E4
    \end{verbatim}
\end{itemize}





\end{document}