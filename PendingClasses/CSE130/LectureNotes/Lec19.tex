\documentclass[letterpaper]{article}
\usepackage[margin=1in]{geometry}
\usepackage[utf8]{inputenc}
\usepackage{textcomp}
\usepackage{amssymb}
\usepackage{natbib}
\usepackage{graphicx}
\usepackage{gensymb}
\usepackage{amsthm, amsmath, mathtools}
\usepackage{xcolor}
\usepackage{enumerate}
\usepackage{framed}
\usepackage{tcolorbox}
\tcbuselibrary{theorems}

\newcommand{\R}{\mathbb{R}}
\newcommand{\Z}{\mathbb{Z}}
\newcommand{\N}{\mathbb{N}}
\newcommand{\Q}{\mathbb{Q}}
\newcommand{\code}[1]{\texttt{#1}}
\newcommand{\mdiamond}{$\diamondsuit$}

%\newtheorem*{theorem}{Theorem}
%\newtheorem*{definition}{Definition}
\newtheorem*{proposition}{Proposition}
%\newtheorem*{corollary}{Corollary}
%\newtheorem*{lemma}{Lemma}

\newtcbtheorem[number within=section]{theorem}{Theorem}
{colback=green!5,colframe=green!35!black,fonttitle=\bfseries}{def}

\newtcbtheorem[number within=section]{definition}{Definition}
{colback=blue!5,colframe=blue!35!black,fonttitle=\bfseries}{def}

\newtcbtheorem[number within=section]{corollary}{Corollary}
{colback=yellow!5,colframe=yellow!35!black,fonttitle=\bfseries}{def}

\newtcbtheorem[number within=section]{lemma}{Lemma}
{colback=red!5,colframe=red!35!black,fonttitle=\bfseries}{def}
\usepackage[utf8]{inputenc}
\usepackage[english]{babel}
\usepackage{fancyhdr}
\usepackage[hidelinks]{hyperref}

\pagestyle{fancy}
\fancyhf{}
\rhead{CSE 130}
\chead{Wednesday, May 11, 2022}
\lhead{Lecture 19}
\rfoot{\thepage}

\setlength{\parindent}{0pt}

\begin{document}
\section{Environments and Closures}
\subsection{Nano: Functions}
\subsubsection{Evaluating Applications}
Our \code{eval} function uses dynamic binding, but we want static binding. So, we instead want to ``preserve'' the environment at the point of the function definition. In essense, to implement static scoping, we need to 
\begin{enumerate}
    \item Freeze the environment in the function's value at \emph{definition}.
    \item Use the frozen environment, instead of the current environment, to evaluate the body at \emph{call}.
\end{enumerate}
So, in our example, we would have: 
\begin{verbatim}
    let c = 1                   -- []
    in                          -- ["c" := 1]
        let inc = \x -> x + c   -- 
        in                      -- ["inc" := <fro, x, x + c>, "c" := 1]
                                --      where fro = ["c" := 1]
            let c = 100         -- 
            in                  -- ["c" := 100, "inc" := <x, x + c>, "c" := 1]
                inc 10 \end{verbatim}
        So, what ends up happening is that, when we evaluate \code{inc 10}, we have our \emph{extended environment}, which extends the \emph{frozen environment} instead of the general environment:
        \begin{verbatim}
            ["x" := 10, "c" := 1]\end{verbatim}
This gives us the desired answer of 11. 


\subsubsection{Function Values as Closures \& Evaluating Function Calls}
To implement lexical scoping, we want to represent function values as closures. A closure is essentially a lambda abstraction which has a formal, body, \emph{and} environment at function definition. Our updated grammar is then given by 
\begin{verbatim}
    v ::= n 
        | <env, x, e>        -- frozen environment + formal + body\end{verbatim}
where \code{env} is defined by 
\begin{verbatim}
    env ::= [] 
            | (x := v) : env\end{verbatim}
Hence, our updated \code{Value} representation is 
\begin{verbatim}
    dataValue = VNum Int 
                | VClos Env Id Expr \end{verbatim}
How do we modify our \code{eval} for \code{Lam}? Likewise, how do we modify our \code{eval} for \code{App}? (HW4\footnote{See lecture notes for hints.}.)

\subsubsection{Advanced Features of Functions}
In particular, what can \code{eval} support? Can it support the following? 
\begin{itemize}
    \item Returning functions from functions (partial application). 
    \item Functions taking functions as arguments (higher-order functions). 
    \item Recursion.
\end{itemize}

\begin{mdframed}[]
    (Quiz.) What should the following evaluate to? 
    \begin{verbatim}
        let add = \x y -> x + y
        in
            let add1 = add 1
            in
                let add10 = add 10
                in
                    add1 100 + add10 1000\end{verbatim}
    
    \begin{enumerate}[(a)]
        \item Runtime error. 
        \item \code{1102}
        \item \code{1120}
        \item \code{1111}
    \end{enumerate}

    \begin{mdframed}[]
        The answer is \textbf{D}. Note that \code{add1} takes in an argument and adds 1 to it. \code{add10} takes in an argument and adds \code{10}. So, \code{add1 100} should give us \code{101} and \code{add10 1000} should give us \code{1010}, so the final answer is \code{1111}.
    \end{mdframed}
\end{mdframed}
In particular, partial application is supported.

\begin{mdframed}[]
    (Quiz.) What should the following evaluate to?
    \begin{verbatim}
        let inc = \x -> x + 1
        in
            let doTwice = \f -> (\x -> f (f x))
            in
                doTwice inc 10 \end{verbatim}
    \begin{enumerate}[(a)]
        \item Runtime error. 
        \item \code{11}
        \item \code{12}
    \end{enumerate}

    \begin{mdframed}[]
        The answer is \textbf{C}.
    \end{mdframed}
\end{mdframed}
So, higher-order functions have been achieved.

\begin{mdframed}[]
    (Quiz.) What does this evaluate to? 
    \begin{verbatim}
        let f = \n -> n * f (n - 1) 
        in
            f 5\end{verbatim}
    \begin{enumerate}[(a)]
        \item \code{120}
        \item Evaluation does not terminate 
        \item Error: unbound variable \code{f}
    \end{enumerate}

    \begin{mdframed}[]
        The answer is \textbf{C}; when we call \code{f} recursively, we do not have \code{f} in our environment.
    \end{mdframed}
\end{mdframed}









\newpage 
\section{Lexing and Parsing}
Here, we won't actually be using Nano. Rather, we will be using an even simpler language -- a calculator. Its AST representation can be defined by 
\begin{verbatim}
    data Aexpr = AConst  Int
        | AVar    Id
        | APlus   Aexpr Aexpr
        | AMinus  Aexpr Aexpr
        | AMul    Aexpr Aexpr\end{verbatim}
Its evaluator function can be defined by 
\begin{verbatim}
    eval :: Env -> Aexpr -> Value
    ...\end{verbatim}
Its implementation is very similar to Nano. Thus, using the evaluator:
\begin{verbatim}
    -> eval [] (APlus (AConst 2) (AConst 6))
    8

    -> eval [("x", 16), ("y", 10)] (AMinus (AVar "x") (AVar "y"))
    6\end{verbatim}
Notice that this is very tedious; most of us are used to writing programs as text. How do we do this? In particular, we want to write a function that \emph{parses} the AST for us. 
\begin{verbatim}
    parse :: String -> Aexpr \end{verbatim}
We will incorporate a two-step strategy. In particular, consider the sentence: \emph{He ate a bagel}. How do we read it? 
\begin{itemize}
    \item First, we split the sentence into words: \code{["He", "ate", "a", "bagel"]}.
    \item Then, we relate the words to each other. \code{He} is the subject, \code{ate} is the verb, and so on. 
\end{itemize}
We can do the same thing when ``reading'' programs. 

\subsection{Step 1: Lexing}
We want to first convert a string to a bunch of tokens. A string is simply a list of characters: 
\begin{verbatim}
    229 + 98 * x2\end{verbatim}
The idea is to aggregate characters that ``belong together'' into \textbf{tokens}, or the ``words'' of the program. So, 
\begin{verbatim}
    229 PLUS 98 TIMES x2\end{verbatim}
Often, we distinguish tokens of different kinds based on their format: 
\begin{itemize}
    \item All numbers: integer constant. 
    \item Alphanumerica: starts with a letter: identifier. 
    \item \code{+}: Plus operator.
    \item And so on.
\end{itemize}

\subsection{Step 2: Parsing}
Now, we want to convert the tokens to AST. This is usually difficult, so we make use of a set of tools known as \textbf{parser generators}. 
\begin{itemize}
    \item Given the description of the token format, generates a lexer. 
    \item Given the description of the grammar, generates a parser. 
\end{itemize}
We will be using parser generators, so we only care about how to describe the token format and the grammar. 

\end{document}