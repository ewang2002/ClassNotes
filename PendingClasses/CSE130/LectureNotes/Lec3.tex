\documentclass[letterpaper]{article}
\usepackage[margin=1in]{geometry}
\usepackage[utf8]{inputenc}
\usepackage{textcomp}
\usepackage{amssymb}
\usepackage{natbib}
\usepackage{graphicx}
\usepackage{gensymb}
\usepackage{amsthm, amsmath, mathtools}
\usepackage{xcolor}
\usepackage{enumerate}
\usepackage{framed}
\usepackage{tcolorbox}
\tcbuselibrary{theorems}

\newcommand{\R}{\mathbb{R}}
\newcommand{\Z}{\mathbb{Z}}
\newcommand{\N}{\mathbb{N}}
\newcommand{\Q}{\mathbb{Q}}
\newcommand{\code}[1]{\texttt{#1}}
\newcommand{\mdiamond}{$\diamondsuit$}

%\newtheorem*{theorem}{Theorem}
%\newtheorem*{definition}{Definition}
\newtheorem*{proposition}{Proposition}
%\newtheorem*{corollary}{Corollary}
%\newtheorem*{lemma}{Lemma}

\newtcbtheorem[number within=section]{theorem}{Theorem}
{colback=green!5,colframe=green!35!black,fonttitle=\bfseries}{def}

\newtcbtheorem[number within=section]{definition}{Definition}
{colback=blue!5,colframe=blue!35!black,fonttitle=\bfseries}{def}

\newtcbtheorem[number within=section]{corollary}{Corollary}
{colback=yellow!5,colframe=yellow!35!black,fonttitle=\bfseries}{def}

\newtcbtheorem[number within=section]{lemma}{Lemma}
{colback=red!5,colframe=red!35!black,fonttitle=\bfseries}{def}
\usepackage[utf8]{inputenc}
\usepackage[english]{babel}
\usepackage{fancyhdr}
\usepackage[hidelinks]{hyperref}

\pagestyle{fancy}
\fancyhf{}
\rhead{CSE 130}
\chead{Friday, April 01, 2022}
\lhead{Lecture 3}
\rfoot{\thepage}

\setlength{\parindent}{0pt}

\begin{document}

\section{The Lambda Calculus (Continued)}

\subsection{Semantics: What Programs Mean}
In particular, how do you run or execute a $\lambda$-term? 

\bigskip 

Unlike programming languages like Java or Python, where code runs sequentially, one can \emph{think} of running a Lambda Calculus program as middle-school algebra.  
\begin{verbatim}
    (x + 2) * (3x - 1)
        => 3x^2 - x + 6x - 2    -- multiply polynomials
        => 3x^2 + 5x - 2        -- add monomials
                                -- no more rules apply
\end{verbatim}
Essentially, rewrite step-by-step following simple rules until no more rules apply. 

\subsection{Scope of Variables}
The scope of a variable is simply where, in a program, the variable is visible. In the expression \code{\bsttt{x} -> E}:
\begin{itemize}
    \item \code{x} is the newly introduced variable. 
    \item \code{E} is the \textbf{scope} of \code{x}.
    \item any occurrence of \code{x} in \code{\bsttt{x} -> E} is \textbf{bound} (by the \textbf{binder} \code{\bsttt{x}}). 
\end{itemize}

For example, \code{x} is bound in:
\begin{verbatim}
    \x -> x 
    \x -> (\y -> x)     -- The 'x' in the abstraction is bound by \x 
                        -- Same thing as: \x y -> x
\end{verbatim}

We say that an occurrence of \code{x} in \code{E} is \textbf{free} if it's not bound by an enclosing abstraction. For example: 
\begin{verbatim}
    x y                     -- No binders at all 
    \y -> x y               -- No \x binder 
    (\x -> \y -> y) x       -- x is outside of the scope of the \x binder
                            -- It's not "the same" 'x'
                            -- The outer x is not in the abstraction 
\end{verbatim}

Consider the expression \code{(\bsttt{x} -> x) x}. Here, there are two occurrences\footnote{A use of the variable.} of \code{x}; The \code{\bsttt{x}} is a parameter (i.e. declaration) of which the second \code{x} is the first occurrence. The third \code{x} is another occurrence. So, the very last \code{x} is a free occurrence and the second \code{x} (the one after the arrow) is bound to \code{\bsttt{x}}. 

\subsubsection{Free Variables}
A variable \code{x} is \textbf{free} in \code{E} if there exists a free occurrence of \code{x} in \code{E}. We can formally define the set of all free variables in a term using this function: 
\[FV(\code{Exp}) = \begin{cases}
    \{\code{x}\} & \text{If } \code{Exp = x} \\ 
    FV(\code{E}) \setminus \{\code{x}\} & \text{If } \code{Exp = \bsttt{x} -> E} \\ 
    FV(\code{E1}) \cup FV(\code{E2}) & \text{If } \code{Exp = E1 E2} 
\end{cases}\]
Here, \code{Exp} is some arbitrary expression. 

%We can formally define the set of all free variables in a term like so: 
%\begin{verbatim}
%    FV(x)       =   {x}
%    FV(\x -> E) =   FV(E) \ {x}
%    FV(E1 E2)   =   FV(E1) u FV(E2)
%\end{verbatim}

\subsubsection{Closed Expressions}
If \code{E} has no free variables, it is said to be \textbf{closed}. Closed expressions are also called \textbf{combinators}. 

\bigskip 

The shortest closed expression is just 
\[\code{\bsttt{x} -> x}\]
since \code{x} is bound to by \code{\bsttt{x}}. 



\subsection{The Rewrite Rules of Lambda Calculus}

\subsubsection{Semantics: Beta Step}
$\beta$-step is known as calling a function, or \textbf{function calls}. Formally, we write this as 
\[\code{(\bsttt{x} -> E1) E2 =b> E1[x := E2]}\]
where \code{E1[x := E2]} means that \code{E1} with all \emph{free} occurrences of \code{x} is replaced with \code{E2}. Effectively, the computation is done by search-and-replace: 
\begin{itemize}
    \item If you see an abstraction applied to an argument, take the body of the abstraction and replace all free occurrences of the formal by that argument. 
\end{itemize}
Thus, the above formal definition is saying that \code{(\bsttt{x} -> E1) E2} $\beta$-steps to \code{E1[x := E2]}.

\bigskip 

A few other examples are: 
\begin{itemize}
    \item Consider \code{(\bsttt{f} -> f (\bsttt{x} -> x)) (give apple)}. The idea is: 
    \begin{itemize}
        \item We take the body of the abstraction, i.e. \code{f (\bsttt{x} -> x)}.
        \item We replace all instances of \code{f} (the formal parameter) in the body with \code{(give apple)}.
        \item This gives us \code{(give apple) (\bsttt{x} -> x)}.
    \end{itemize}

    \item Consider \code{(\bsttt{x} -> (\bsttt{y} -> y)) apple}. The idea is: 
    \begin{itemize}
        \item We take the body of the abstraction, i.e. \code{(\bsttt{y} -> y)}.
        \item We replace all instances of \code{x}, the formal parameter, in the body with \code{apple}.
        \item This gives us \code{\bsttt{y} -> y} (nothing has changed).  
    \end{itemize}

%    \item Consider \code{(\bsttt{x} -> x (\bsttt{x} -> x)) apple}. The idea is: 
 %   \begin{itemize}
5        \item We take the body of the abstraction, i.e. \code{x (\bsttt{x} -> x)}. Note that there is another abstraction with the same parameter name as the original formal parameter name. 
%        \item x
%    \end{itemize}

    \item \code{(\bsttt{x} -> x) y} becomes \code{y}.
    \item \code{(\bsttt{a} b c -> b) d} becomes \code{(\bsttt{b} c -> b)}.
    \item \code{(\bsttt{b} c -> b) e} becomes \code{(\bsttt{c} -> e)}.
    \item \code{(\bsttt{a} b c -> b) d e} becomes \code{(\bsttt{c} -> e)}.
    \item \code{(\bsttt{x} y z -> z y x) y z x} cannot be reduced; instead, we need to use what's known as an $\alpha$-step. 
\end{itemize}

\subsubsection{Alpha Step}
$\alpha$-step is known as \textbf{renaming formals}. That is, we can rename a formal parameter and replace all its occurrences in the body.

\bigskip 

For example: 
\begin{verbatim}
    (\x y z -> z y x) y z x 
    =a> (\x a z -> z a x) y z x
\end{verbatim}



\end{document}