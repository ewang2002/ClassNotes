\documentclass[letterpaper]{article}
\usepackage[margin=1in]{geometry}
\usepackage[utf8]{inputenc}
\usepackage{textcomp}
\usepackage{amssymb}
\usepackage{natbib}
\usepackage{graphicx}
\usepackage{gensymb}
\usepackage{amsthm, amsmath, mathtools}
\usepackage[dvipsnames]{xcolor}
\usepackage{enumerate}
\usepackage{mdframed}
\usepackage[most]{tcolorbox}
\usepackage{csquotes}
% https://tex.stackexchange.com/questions/13506/how-to-continue-the-framed-text-box-on-multiple-pages

\tcbuselibrary{theorems}

\newcommand{\R}{\mathbb{R}}
\newcommand{\Z}{\mathbb{Z}}
\newcommand{\N}{\mathbb{N}}
\newcommand{\Q}{\mathbb{Q}}
\newcommand{\C}{\mathbb{C}}
\newcommand{\code}[1]{\texttt{#1}}
\newcommand{\mdiamond}{$\diamondsuit$}
\newcommand{\PowerSet}{\mathcal{P}}
\newcommand{\Mod}[1]{\ (\mathrm{mod}\ #1)}
\DeclareMathOperator{\lcm}{lcm}

%\newtheorem*{theorem}{Theorem}
%\newtheorem*{definition}{Definition}
%\newtheorem*{corollary}{Corollary}
%\newtheorem*{lemma}{Lemma}
\newtheorem*{proposition}{Proposition}


\newtcbtheorem[number within=section]{theorem}{Theorem}
{colback=green!5,colframe=green!35!black,fonttitle=\bfseries}{th}

\newtcbtheorem[number within=section]{definition}{Definition}
{colback=blue!5,colframe=blue!35!black,fonttitle=\bfseries}{def}

\newtcbtheorem[number within=section]{corollary}{Corollary}
{colback=yellow!5,colframe=yellow!35!black,fonttitle=\bfseries}{cor}

\newtcbtheorem[number within=section]{lemma}{Lemma}
{colback=red!5,colframe=red!35!black,fonttitle=\bfseries}{lem}

\newtcbtheorem[number within=section]{example}{Example}
{colback=white!5,colframe=white!35!black,fonttitle=\bfseries}{def}

\newtcbtheorem[number within=section]{note}{Important Note}{
        enhanced,
        sharp corners,
        attach boxed title to top left={
            xshift=-1mm,
            yshift=-5mm,
            yshifttext=-1mm
        },
        top=1.5em,
        colback=white,
        colframe=black,
        fonttitle=\bfseries,
        boxed title style={
            sharp corners,
            size=small,
            colback=red!75!black,
            colframe=red!75!black,
        } 
    }{impnote}
\usepackage[utf8]{inputenc}
\usepackage[english]{babel}
\usepackage{fancyhdr}
\usepackage[hidelinks]{hyperref}

\pagestyle{fancy}
\fancyhf{}
\rhead{CSE 130}
\chead{Wednesday, May 4, 2022}
\lhead{Lecture 16}
\rfoot{\thepage}

\setlength{\parindent}{0pt}

\begin{document}
\section{Higher-Order Functions}
\subsection{Tail-Recursive Versions}
Let's write a tail-recursive version of \code{sum}. In particular,
\begin{verbatim}
    sumTR :: [Int] -> Int 
    sumTR xs = helper 0 xs 
        where 
            helper :: Int -> [Int] -> Int 
            helper acc []       = acc 
            helper acc (x:xs)   = helper (acc + x) xs \end{verbatim}

Let us now write a tail-recursive \code{cat} function. Note that this is very similar to what we have above. Its implementation is
\begin{verbatim}
    catTR :: [String] -> String 
    catTR xs = helper "" xs 
        where 
            helper :: String -> [String] -> String 
            helper acc []       = acc 
            helper acc (x:xs)   = helper (acc ++ x) xs \end{verbatim}

Note that there is an apparent pattern here, which can be extracted to the \emph{fold-left} pattern: 
\begin{verbatim}
    foldl :: (b -> a -> b) -> b -> [a] -> b
    foldl f b xs = helper b xs 
        where 
            helper acc []       = acc 
            helper acc (x:xs)   = helper (f acc x) xs \end{verbatim}
In general, the pattern is: 
\begin{itemize}
    \item Use a helper function with an extra accumulator argument.
    \item To compute the new accumulator, combine the current accumulator with the head using some binary operation.
\end{itemize}

\begin{mdframed}[nobreak=true]
    (Quiz.) What does this evaluate to? 
    \begin{verbatim}
        foldl f b []     = b
        foldl f b (x:xs) = foldl f (f b x) xs

        quiz = foldl (:) [] [1,2,3]\end{verbatim}
    \begin{enumerate}[(a)]
        \item Type error.
        \item \code{[1, 2, 3]}
        \item \code{[3, 2, 1]}
        \item \code{[[3], [2], [1]]}
        \item \code{[[1], [2], [3]]}
    \end{enumerate}

    \begin{mdframed}[]
        The answer is \textbf{A}. Note that \code{a} is an \code{Int} and \code{b} is an \code{[Int]}. So, our accumulator function is of type \code{[Int] -> Int -> [Int]}. But, keep in mind that \code{(:)} (the cons operator) takes an \code{Int} followed by a \code{[Int]}. This is a type error since the accumulator function types disagree. 
    \end{mdframed}
\end{mdframed}

\begin{mdframed}[nobreak=true]
    What does this evaluate to? 
    \begin{verbatim}
        foldl f b []     = b
        foldl f b (x:xs) = foldl f (f b x) xs

        quiz = foldl (\xs x -> x : xs) [] [1,2,3]
    \end{verbatim}

    \begin{enumerate}[(a)]
        \item Type error. 
        \item \code{[1,2,3]}
        \item \code{[3,2,1]}
        \item \code{[[3], [2], [1]]}
        \item \code{[[1],[2],[3]]}
    \end{enumerate}

    \begin{mdframed}[]
        The answer is \textbf{C}. To see why this is the case, consider the following work: 
        \begin{verbatim}
    foldl f []                           [1,2,3]
        ==> foldl f (1 : [])               [2,3]
        ==> foldl f (2 : (1 : []))           [3]
        ==> foldl f (3 : (2 : (1 : [])))      []
        ==> 3 : (2 : (1 : []))
        = [3,2,1]\end{verbatim}
    \end{mdframed}
\end{mdframed}

\subsubsection{Fold Left vs. Right}
Too see the difference between the two fold functions, consider the following: 
\begin{verbatim}
    foldl f b [x1, x2, x3]  ==> f (f (f b x1) x2) x3  -- Left
    foldr f b [x1, x2, x3]  ==> f x1 (f x2 (f x3 b))  -- Right\end{verbatim}
As an example, we have: 
\begin{verbatim}
    foldl (+) 0 [1, 2, 3]  ==> ((0 + 1) + 2) + 3  -- Left
    foldr (+) 0 [1, 2, 3]  ==> 1 + (2 + (3 + 0))  -- Right\end{verbatim}
As for their types: 
\begin{verbatim}    
    foldl :: (b -> a -> b) -> b -> [a] -> b  -- Left
    foldr :: (a -> b -> b) -> b -> [a] -> b  -- Right\end{verbatim}


\subsection{Useful Higher-Order Functions}
Consider the function:
\begin{verbatim}
    foldl (\xs x -> x : xs) [] [1,2,3]\end{verbatim}
This is the same thing as:
\begin{verbatim}
    foldl (flip (:)) [] [1,2,3]\end{verbatim}
Its type signature is given by: 
\begin{verbatim}
    flip :: (a -> b -> c) -> (b -> a -> c)\end{verbatim}

There is also the \emph{compose} function. So, instead of writing 
\begin{verbatim}
    map (\x -> f (g x)) ys\end{verbatim}
we can write 
\begin{verbatim}
    map (f . g) ys\end{verbatim}
Its type signature is given by: 
\begin{verbatim}
    --     f           g           f . g
    (.) :: (a -> b) -> (c -> a) -> (c -> b)\end{verbatim}



\newpage 
\section{Environments and Closures}
We will now begin the process of \emph{implementing} a functional language. In this section, we will discuss how to evaluate a program given its abstract syntax tree (AST), and also prove properties about our interpreter.

\bigskip 

We will implement the Nano programming language. Its features include
\begin{enumerate}
    \item Arithmetic
    \item Variables
    \item Let-bindings
    \item functions
    \item Recursion
\end{enumerate}
Generally, the idea is, given a string containing the program, it will be converted to its AST (abstract syntax tree) form\footnote{This process is known as parsing.}. From there, it can be evaluated to the desired result. 

\subsection{Nano: Arithmetic}
A grammar of arithmetic expressions can be represented like so: 
\begin{verbatim}
    e :: n 
        | e1 + e2 
        | e1 - e2 
        | e1 * e2\end{verbatim}
We can represent this by the following datatype: 
\begin{verbatim}
    data Expr = Num Int 
            | Add Expr Expr 
            | Sub Expr Expr 
            | Mul Expr Expr\end{verbatim}
We can represent arithmetic values as a type: 
\begin{verbatim}
    type Value = Int \end{verbatim}

\subsubsection{Evaluating Arithmetic Expressions}
We can now write a Haskell function to evaluate an expression. 
\begin{verbatim}
    eval :: Expr -> Value
    eval (Num n)     = n
    eval (Add e1 e2) = eval e1 + eval e2
    eval (Sub e1 e2) = eval e1 - eval e2
    eval (Mul e1 e2) = eval e1 * eval e2\end{verbatim}
However, we can refactor this. 

\subsubsection{Alternative Representation}
Rather than writing out each operation (e.g. \code{Add}, \code{Sub}, and so on), thus repeating ourselves, we can extract that into a datatype itself.
\begin{verbatim}
    data Binop = Add | Sub | Mul 
    data Expr = Num Int 
            | Bin Binop Expr Expr \end{verbatim}

Hence, we can structure the \code{eval} code like so: 
\begin{verbatim}
    eval :: Expr -> Value
    eval (Num n)            = n 
    eval (Bin op e1 e2)     = evalOp op (eval e1) (eval e2)\end{verbatim}
Here, we made use of an \code{evalOp} helper function.

\begin{mdframed}[]
    (Quiz.) Consider the evaluator for the alternative representation.
    \begin{verbatim}
    eval :: Expr -> Value
    eval (Num n)        = n
    eval (Bin op e1 e2) = evalOp op (eval e1) (eval e2)\end{verbatim} 

    What is a suitable type for \code{evalOp}? 
    \begin{enumerate}[(a)]
        \item \code{Binop -> Value}
        \item \code{Binop -> Value -> Value -> Value}
        \item \code{Binop -> Expr -> Expr -> Value}
        \item \code{BInop -> Expr -> Expr -> Expr}
        \item \code{Binop -> Expr -> Value}
    \end{enumerate}

    \begin{mdframed}[]
        The answer is \textbf{B}. Note that \code{eval} returns a \code{Value}, so it follows that \code{(eval e1)} and \code{(eval e2)} both returns \code{Value}. Finally, the helper function itself is supposed to return a \code{Helper} since we're using the helper function to evaluate \code{eval}, and \code{eval} again returns \code{Value}.
    \end{mdframed}
\end{mdframed}

Now that we know the type of \code{evalOp}, we can declare it. 
\begin{verbatim}
    evalOp :: Binop -> Value -> Value -> Value 
    evalOp Add v1 v2    = v1 + v2 
    evalOp Sub v1 v2    = v1 - v2 
    evalOp Mult v1 v2   = v1 * v2 \end{verbatim}
Note that a shorter way to do this is: 
\begin{verbatim}
    evalOp Add  = (+)
    evalOp Sub  = (-)
    evalOp Mult = (*)\end{verbatim}

\end{document}