\documentclass[letterpaper]{article}
\usepackage[margin=1in]{geometry}
\usepackage[utf8]{inputenc}
\usepackage{textcomp}
\usepackage{amssymb}
\usepackage{natbib}
\usepackage{graphicx}
\usepackage{gensymb}
\usepackage{amsthm, amsmath, mathtools}
\usepackage[dvipsnames]{xcolor}
\usepackage{enumerate}
\usepackage{mdframed}
\usepackage[most]{tcolorbox}
\usepackage{csquotes}
% https://tex.stackexchange.com/questions/13506/how-to-continue-the-framed-text-box-on-multiple-pages

\tcbuselibrary{theorems}

\newcommand{\R}{\mathbb{R}}
\newcommand{\Z}{\mathbb{Z}}
\newcommand{\N}{\mathbb{N}}
\newcommand{\Q}{\mathbb{Q}}
\newcommand{\C}{\mathbb{C}}
\newcommand{\code}[1]{\texttt{#1}}
\newcommand{\mdiamond}{$\diamondsuit$}
\newcommand{\PowerSet}{\mathcal{P}}
\newcommand{\Mod}[1]{\ (\mathrm{mod}\ #1)}
\DeclareMathOperator{\lcm}{lcm}

%\newtheorem*{theorem}{Theorem}
%\newtheorem*{definition}{Definition}
%\newtheorem*{corollary}{Corollary}
%\newtheorem*{lemma}{Lemma}
\newtheorem*{proposition}{Proposition}


\newtcbtheorem[number within=section]{theorem}{Theorem}
{colback=green!5,colframe=green!35!black,fonttitle=\bfseries}{th}

\newtcbtheorem[number within=section]{definition}{Definition}
{colback=blue!5,colframe=blue!35!black,fonttitle=\bfseries}{def}

\newtcbtheorem[number within=section]{corollary}{Corollary}
{colback=yellow!5,colframe=yellow!35!black,fonttitle=\bfseries}{cor}

\newtcbtheorem[number within=section]{lemma}{Lemma}
{colback=red!5,colframe=red!35!black,fonttitle=\bfseries}{lem}

\newtcbtheorem[number within=section]{example}{Example}
{colback=white!5,colframe=white!35!black,fonttitle=\bfseries}{def}

\newtcbtheorem[number within=section]{note}{Important Note}{
        enhanced,
        sharp corners,
        attach boxed title to top left={
            xshift=-1mm,
            yshift=-5mm,
            yshifttext=-1mm
        },
        top=1.5em,
        colback=white,
        colframe=black,
        fonttitle=\bfseries,
        boxed title style={
            sharp corners,
            size=small,
            colback=red!75!black,
            colframe=red!75!black,
        } 
    }{impnote}
\usepackage[utf8]{inputenc}
\usepackage[english]{babel}
\usepackage{fancyhdr}
\usepackage[hidelinks]{hyperref}

\pagestyle{fancy}
\fancyhf{}
\rhead{CSE 130}
\chead{Friday, May 6, 2022}
\lhead{Lecture 17}
\rfoot{\thepage}

\setlength{\parindent}{0pt}

\begin{document}
\section{Environments and Closures}
\subsection{Nano: Variables}
We now need to add variables. Hence, we modify the grammar like so: 
\begin{verbatim}
    e :: n 
        | e1 + e2 
        | e1 - e2 
        | e1 * e2 
        | x             -- New \end{verbatim}
This can be represented by the datatype\footnote{We don't plan on introducing tyoe checking here.}: 
\begin{verbatim}
    type Id = String 
    data Expr = Num Int             -- Number 
        | Bin Binop Expr Expr       -- Binary Expression
        | Var Id                    -- Variable \end{verbatim}

We now need to extend the evaluation function.

\begin{mdframed}[]
    (Quiz.) What should the following expression evaluate to? 
    \begin{verbatim}
        x + 1\end{verbatim}
    \begin{enumerate}[(a)]
        \item \code{0}
        \item \code{1}
        \item Runtime Error.
    \end{enumerate}

    \begin{mdframed}[]
        The answer is \textbf{C}. We don't know what the value of \code{x} is.
    \end{mdframed}
\end{mdframed}
Clearly, variables aren't useful unless we can somehow map variable names to values. 

\subsubsection{Environment}
An expression is evaluated in an \textbf{environment}. It's like a phone book that maps variables to values. 
\begin{verbatim}
    ["x" := 0, "y" := 12, ...]\end{verbatim}

We can represent an environment using the following type: 
\begin{verbatim}
    type Env = [(Id, Value)]\end{verbatim}

\subsubsection{Evaluation in an Environment}
We can write 
\begin{verbatim}
    eval env expr => value \end{verbatim}
to mean that evaluating \code{expr} in the \emph{environment} \code{env} should return \code{value}.

\begin{mdframed}[]
    (Quiz.) What should the result of the following code be? 
    \begin{verbatim}
        eval ["x" := 0, "y" := 12, ...] (x + 1)\end{verbatim}
    
    \begin{enumerate}[(a)]
        \item \code{0}
        \item \code{1}
        \item Runtime Error.
    \end{enumerate}

    \begin{mdframed}[]
        The answer is \textbf{B}.
    \end{mdframed}
\end{mdframed}

To evaluate a variable, we can just look up its value in the environment.

\subsubsection{Evaluating Variables}
We now need to update our evaluation function to take the environment as an argument. 
\begin{verbatim}
    eval :: Env -> Expr -> Value 
    eval env (Num n)            = n
    eval env (Binop op e1 e2)   = evalOp op (eval env e1) (eval env e2)
    eval env (Var x)            = lookup x env \end{verbatim}

Now that we have variables, we now need to find some way of \emph{adding} variables to the environment. In other words, how do variables get into the environment? 

\subsection{Nano: Let-Bindings}
We now need to add \code{let}-bindings. Our grammar needs to be updated:
\begin{verbatim}
    e :: n 
        | e1 + e2 
        | e1 - e2 
        | e1 * e2 
        | x 
        | let x = e1 in e2 \end{verbatim}

For example, if our environment is \code{[]} and our expression is \code{let x = 2 + 3 in x * 2}, then we would end up with \code{10}. Notice that \code{x} isn't in our environment; rather, we introduced \code{x}  through a let-binding. Hence, we need to extend the representation of expressions, or the datatype. 
\begin{verbatim}
    data Expr = Num Int             -- Number 
            | Bin Binop Expr Expr   -- Binary Expression
            | Var Id                -- Variable 
            | Let Id Expr Expr      -- Let-binding \end{verbatim}

But, how do we extend the \code{eval} function to account for let-bindings? 

\begin{mdframed}[]
    (Quiz.) What should this evaluate to? 
    \begin{verbatim}
        let x = 5
        in 
            x + 1\end{verbatim}
    \begin{enumerate}[(a)]
        \item \code{1}
        \item \code{5}
        \item \code{6}
        \item Error: unbound variable \code{x}
        \item Error: unbound variable \code{y}
    \end{enumerate}

    \begin{mdframed}[]
        The answer is \textbf{C}. \code{x} is bound to the value \code{5}, so \code{5 + 1} gives us \code{6}.
    \end{mdframed}
\end{mdframed}

\begin{mdframed}[]
    (Quiz.) What should the following evaluate to? 
    \begin{verbatim}
        let x = 5
        in 
            let y = x + 1
            in 
                x * y \end{verbatim}
    \begin{enumerate}[(a)]
        \item \code{5}
        \item \code{6}
        \item \code{30}
        \item Error: unbound variable \code{x}
        \item Error: unbound variable \code{y}
    \end{enumerate}

    \begin{mdframed}[]
        The answer is \textbf{C}. Once again, we've bound \code{x} to 5, then bound \code{y} to \code{5 + 1}. Thus, we get the value \code{5 * (6 + 1)}, which is 30.
    \end{mdframed}
\end{mdframed}

\begin{mdframed}[]
    (Quiz.) What should the following evaluate to? 
    \begin{verbatim}
        let x = 0 
        in
            (let x = 100 
            in
                x + 1
            ) + x\end{verbatim}
    \begin{enumerate}[(a)]
        \item \code{1}
        \item \code{101}
        \item \code{201}
        \item \code{2}
        \item Error: multiple definitions of \code{x}.
    \end{enumerate}

    \begin{mdframed}[]
        The answer is \textbf{B}. Here, we note that the inner \code{x} is shadowing the outer \code{x}. Hence, the inner \code{x + 1} is \code{101}. 
    \end{mdframed}
\end{mdframed}

\subsubsection{Principle: Static (Lexical) Scoping}
Every variable use (occurrence) gets its value from its most \emph{local definition} (binding). In a pure language, the value never changes once defined, thus it's easier to tell by looking at a program where the variable's value came from. 

\subsubsection{Implementing Lexical Scoping}
How would we implement this? 

\begin{mdframed}[]
    (Example.) Consider 
    \begin{verbatim}
        let x = 5
        in 
            x + 1\end{verbatim}
    Note that its environment is given by 
    \begin{verbatim}
        let x = 5       -- []
        in              -- | [x := 5]
            x + 1       -- | \end{verbatim}
\end{mdframed}

\begin{mdframed}[]
    (Example.) Consider
    \begin{verbatim}
        let x = 5           -- []
        in                  -- | [x := 5]
            let y = x + 1   -- | 
            in              -- | | [y := 6, x := 5]
                x * y       -- | |\end{verbatim}
\end{mdframed}

\begin{mdframed}[]
    (Example.) Consider 
    \begin{verbatim}
        let x = 0           -- []
        in                  -- | [x := 0]
            (let x = 100    -- | [x := 0]
            in              -- | | [x := 100, x := 0]
                x + 1       -- | | 
            ) + x           -- |\end{verbatim}

 \end{mdframed}

\subsubsection{Evaluating \code{let} Expressions}
To evaluate \code{let x = e1 in e2} in \code{env}, we need to do the following. 
\begin{enumerate}
    \item Evaluate \code{e1} in \code{env} to \code{val}.
    \item \emph{Extend} \code{env} with a mapping \code{["x" := val]}.
    \item Evaluate \code{e2} in this extended environment.
\end{enumerate}
So, we can now extend the \code{eval} function like so: 
\begin{verbatim}
    eval :: Env -> Expr -> Value
    eval env (Num n)          = n
    eval env (Bin op e1 e2)   = evalOp op (eval e1) (eval e2)
    eval env (Var x)          = lookup x env
    eval env (Let x e1 e2)    = eval env' e2
        where
            val  = eval env e1
            env' = (x, val) : env    
\end{verbatim}

\end{document}