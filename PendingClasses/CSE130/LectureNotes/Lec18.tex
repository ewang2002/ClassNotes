\documentclass[letterpaper]{article}
\usepackage[margin=1in]{geometry}
\usepackage[utf8]{inputenc}
\usepackage{textcomp}
\usepackage{amssymb}
\usepackage{natbib}
\usepackage{graphicx}
\usepackage{gensymb}
\usepackage{amsthm, amsmath, mathtools}
\usepackage{xcolor}
\usepackage{enumerate}
\usepackage{framed}
\usepackage{tcolorbox}
\tcbuselibrary{theorems}

\newcommand{\R}{\mathbb{R}}
\newcommand{\Z}{\mathbb{Z}}
\newcommand{\N}{\mathbb{N}}
\newcommand{\Q}{\mathbb{Q}}
\newcommand{\code}[1]{\texttt{#1}}
\newcommand{\mdiamond}{$\diamondsuit$}

%\newtheorem*{theorem}{Theorem}
%\newtheorem*{definition}{Definition}
\newtheorem*{proposition}{Proposition}
%\newtheorem*{corollary}{Corollary}
%\newtheorem*{lemma}{Lemma}

\newtcbtheorem[number within=section]{theorem}{Theorem}
{colback=green!5,colframe=green!35!black,fonttitle=\bfseries}{def}

\newtcbtheorem[number within=section]{definition}{Definition}
{colback=blue!5,colframe=blue!35!black,fonttitle=\bfseries}{def}

\newtcbtheorem[number within=section]{corollary}{Corollary}
{colback=yellow!5,colframe=yellow!35!black,fonttitle=\bfseries}{def}

\newtcbtheorem[number within=section]{lemma}{Lemma}
{colback=red!5,colframe=red!35!black,fonttitle=\bfseries}{def}
\usepackage[utf8]{inputenc}
\usepackage[english]{babel}
\usepackage{fancyhdr}
\usepackage[hidelinks]{hyperref}

\pagestyle{fancy}
\fancyhf{}
\rhead{CSE 130}
\chead{Monday, May 09, 2022}
\lhead{Lecture 18}
\rfoot{\thepage}

\setlength{\parindent}{0pt}

\begin{document}
\section{Environments and Closures}
\subsection{Nano: Functions}
We now want to add functions. In particular, we want to add 
\begin{itemize}
    \item Lambda abstractions (i.e., function definitions).
    \item Applications (i.e., function calls).
\end{itemize}
Our grammar would look something like
\begin{verbatim}
    e :: n 
        | e1 + e2 
        | e1 - e2 
        | e1 * e2 
        | x 
        | let x = e1 in e2
        | \x -> e               -- abstraction
        | e1 e2                 -- application \end{verbatim}

\begin{mdframed}[]
    (Quiz.) What should this evaluate to? 
    \begin{verbatim}
        let inc = \x -> x + 1 
        in 
            inc 10 \end{verbatim}
    \begin{enumerate}[(a)]
        \item Undefined variable \code{x}
        \item Undefined variable \code{inc}
        \item \code{1}
        \item \code{10}
        \item \code{11}
    \end{enumerate}

    \begin{mdframed}[]
        The answer is \textbf{E}. The idea is that we're binding \code{inc 10} to the \code{let}-binding, so \code{inc 10} would call \code{inc}, which would return \code{11}.
    \end{mdframed}
\end{mdframed}
How would we represent functions? 
\begin{verbatim}
    data Expr = Num Int             -- n
            | Bin Binop Expr Expr   -- e1 op e2 
            | Var Id                -- x
            | Let Id Expr Expr      -- let x = e1 in e2 
            | Lam ???               -- \x -> e 
            | App ???               -- e1 e2\end{verbatim}

For the lambda expression, we would have \code{Lam Id Expr}, since we want a new variable with an expression. For application, we would have \code{App Expr Expr} since we want two expressions. Thus, our final definition would be 
\begin{verbatim}
    data Expr = Num Int             -- n
            | Bin Binop Expr Expr   -- e1 op e2 
            | Var Id                -- x
            | Let Id Expr Expr      -- let x = e1 in e2 
            | Lam Id Expr           -- \x -> e 
            | App Expr Expr         -- e1 e2\end{verbatim}

\begin{mdframed}[]
    (Example.) Suppose we want to represent the expression 
    \begin{verbatim}
        let inc = \x -> x + 1 
        in 
            inc 10 \end{verbatim}
    using our definition of \code{Expr} above. This can be done like so: 
    \begin{verbatim}
        fun1 = Let "inc"
                (Lam "x" (Bin Add (Var "x") (Num 1)))
                (App (Var "inc") (Num 10))\end{verbatim}
\end{mdframed}

We now want to implement functions in our \code{eval}.
\begin{verbatim}
    eval :: Env -> Expr -> Value
    eval env (Num n)          = n
    eval env (Bin op e1 e2)   = evalOp op (eval e1) (eval e2)
    eval env (Var x)          = lookup x env
    eval env (Let x e1 e2)    = eval env' e2
        where
            val  = eval env e1
            env' = (x, val) : env
    eval env (Lam x e)        = ??? 
    eval env (App e1 e2)      = ???\end{verbatim}
Recall that, when we have a \code{let}-body, we put a binding of the variable to its definition that it evaluates to in the environment. So far, all of the values that we've been using are \emph{integers}; however, we cannot store our function as an integer. 

\subsubsection{Rethinking Values}
Until now, we said that a program evaluates to an integer, or fails. However, programs like 
\begin{verbatim}
    \x -> x + 1
    => Increment Function

    let f = \x y -> x + y 
    in 
        f 1
    => Increment Function\end{verbatim}
will not work, since these are all \emph{functions}. Therefore, we want a program that evaluates to an \textbf{integer or a function}, or fails. Thus, we need to change our definition of values. 
\begin{verbatim}
    data Value = VNum Int 
                | VFun Id Expr \end{verbatim}
So, the value of a function is its code. Hence, our grammar for values is defined by 
\begin{verbatim}
    v ::= n 
        | <x, e>        -- formal + body\end{verbatim}
We can now try to implement this. But, note that we changed \code{Value}, so it follows that we need to make some adjustments in our code. 
\begin{verbatim}
    eval :: Env -> Expr -> Value
    eval env (Num n)          = VNum n                              -- Changed This 
    ...\end{verbatim}
Likewise, for \code{evalOp} (the helper function), we need to do:
\begin{verbatim}
    evalOp :: Binop -> Value -> Value -> Value 
    evalOp Add  (VNum n1) (VNum n2)    = VNum (v1 + v2) 
    evalOp Sub  (VNum n1) (VNum n2)    = VNum (v1 - v2)
    evalOp Mult (VNum n1) (VNum n2)    = VNum (v1 * v2) 
    evalOp _    _         _            = error "Unsupported operation" \end{verbatim}
This way, we can compute any values of type \code{VNum} while throwing an error if we end up with a \code{VFun} somehow. 


\subsubsection{Evaluating Lambdas}
Now, we will deal with how to evaluating a lambda. 
\begin{verbatim}
    ...
    eval env (Lam x e)        = ??? \end{verbatim}

A lambda should just evaluate to itself. So, we have:
\begin{verbatim}
    ...
    eval env (Lam x e)        =  VFun x e\end{verbatim}

\subsubsection{Evaluating Applications}
Now, we need to deal with the application case. To motivate this, consider again 
\begin{verbatim}
    let inc = \x -> x + 1 
    in 
        inc 10 \end{verbatim}
To evaluate \code{inc 10}, we want to 
\begin{enumerate}
    \item Evaluate \code{inc}, get \code{<x, x + 1>}
    \item Evaluate \code{10}, get \code{10}
    \item Evaluate \code{x + 1} in an environment \emph{extneded} with \code{[x := 10]}
\end{enumerate}

Thus, 
\begin{verbatim}
    ... 
    eval env (App e1 e2)      = eval env' body 
        where 
            -- You should do pattern matching in a helper function (like with 'evalOp'),
            -- esp. since you can handle any possible errors yourself instead of getting 
            -- a cryptic error message. 
            (VFun x body)   = eval env e1 
            v2              = eval env e2 
            env'            = (x, v2) : env\end{verbatim}

\begin{mdframed}[]
    (Quiz.) What should this evaluate to? 
    \begin{verbatim}
        let c = 1
        in 
            let inc = \x -> x + c 
            in 
                inc 10 \end{verbatim}
    \begin{enumerate}[(a)]
        \item Undefined variable \code{x}
        \item Undefined variable \code{c}
        \item \code{1}
        \item \code{10}
        \item \code{11}
    \end{enumerate}

    \begin{mdframed}[]
        The answer is \textbf{E}.
    \end{mdframed}
\end{mdframed}
\textbf{Remark:} This example is slightly more involved than previous examples since we introduce the function definition appears after a variable definition. Although this example is still simple, the next example will be slightly more complicated.

\begin{mdframed}[]
    (Quiz.) What should this evaluate to? 
    \begin{verbatim}
        let c = 1
        in 
            let inc = \x -> x + c 
            in 
                let c = 100
                in 
                    inc 10 \end{verbatim}
    \begin{enumerate}[(a)]
        \item Error: multiple definitions of \code{c}
        \item \code{11}
        \item \code{110}
    \end{enumerate}

    \begin{mdframed}[]
        The answer is \textbf{B}. By the time we define the \code{inc} function, there is only one \code{c} in our environment (when \code{c} is \code{1}).s

        \bigskip 

        The answer is not A because we allow multiple definitions.
    \end{mdframed}
\end{mdframed}
\textbf{Remark:} This is a classic example of static vs. dynamic scoping. 
\begin{itemize}
    \item For static (or lexical) scoping, which is what this example highlights, each occurrence of a variable refers to the most recent binding in the program text. The definition of each variable is unique and known statically. And, finally, it guarantees referential transparency\footnote{The same expression must always evaluate to the same value. In particular, a function must always return the same output for a given input.}. 
    \item For dynamic scoping, each occurrence of a variable refers to the most \textbf{recent} binding during program execution. Thus, we can't tell where a variable is defined just by looking at the function body. Hence, this violates referential transparency. 
\end{itemize}

\begin{mdframed}[]
    (Quiz.) Which scoping does our \code{eval} function, specifically \code{Lam} and \code{App}, implement?
    \begin{enumerate}[(a)]
        \item Static 
        \item Dynamic 
        \item Neither 
    \end{enumerate} 

    \begin{mdframed}[]
        The answer is \textbf{B}. To see why this is the case, consider the following example:
        \begin{verbatim}
    let c = 1                   -- []
    in                          -- ["c" := 1]
        let inc = \x -> x + c   -- 
        in                      -- ["inc" := <x, x + c>, "c" := 1]
            let c = 100         --
            in                  -- ["c" := 100, "inc" := <x, x + c>, "c" := 1]
                inc 10 \end{verbatim}
        So, what ends up happening is that, when we evaluate \code{inc 10}, we shave our \emph{extended environment}
        \begin{verbatim}
            ["x" := 10, "c" := 100, "inc" := <x, x + c>, "c" := 1]\end{verbatim}
        and then the first \code{c} will be found first (so \code{c} is 100).
    \end{mdframed}
\end{mdframed}

\end{document}