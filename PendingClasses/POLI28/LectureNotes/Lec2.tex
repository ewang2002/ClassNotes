\documentclass[letterpaper]{article}
\usepackage[margin=1in]{geometry}
\usepackage[utf8]{inputenc}
\usepackage{textcomp}
\usepackage{amssymb}
\usepackage{natbib}
\usepackage{graphicx}
\usepackage{gensymb}
\usepackage{amsthm, amsmath, mathtools}
\usepackage{xcolor}
\usepackage{enumerate}
\usepackage{framed}
\usepackage{tcolorbox}
\tcbuselibrary{theorems}

\newcommand{\R}{\mathbb{R}}
\newcommand{\Z}{\mathbb{Z}}
\newcommand{\N}{\mathbb{N}}
\newcommand{\Q}{\mathbb{Q}}
\newcommand{\code}[1]{\texttt{#1}}
\newcommand{\mdiamond}{$\diamondsuit$}

%\newtheorem*{theorem}{Theorem}
%\newtheorem*{definition}{Definition}
\newtheorem*{proposition}{Proposition}
%\newtheorem*{corollary}{Corollary}
%\newtheorem*{lemma}{Lemma}

\newtcbtheorem[number within=section]{theorem}{Theorem}
{colback=green!5,colframe=green!35!black,fonttitle=\bfseries}{def}

\newtcbtheorem[number within=section]{definition}{Definition}
{colback=blue!5,colframe=blue!35!black,fonttitle=\bfseries}{def}

\newtcbtheorem[number within=section]{corollary}{Corollary}
{colback=yellow!5,colframe=yellow!35!black,fonttitle=\bfseries}{def}

\newtcbtheorem[number within=section]{lemma}{Lemma}
{colback=red!5,colframe=red!35!black,fonttitle=\bfseries}{def}
\usepackage[utf8]{inputenc}
\usepackage[english]{babel}
\usepackage{fancyhdr}
\usepackage[hidelinks]{hyperref}

\pagestyle{fancy}
\fancyhf{}
\rhead{POLI 28}
\chead{Wednesday, March 30, 2022}
\lhead{Lecture 2}
\rfoot{\thepage}

\setlength{\parindent}{0pt}

\begin{document}

\section{Mar. 30: Plato and the Variety of Dissent}
% 81 battery

\subsection{The Apology}
Socrates was accused of \textbf{heresy} and \textbf{corrupting the youth of Athens}. In particular, Socrates went up to people and asked them to explain some concept; then, after the person explains said concept, Socrates proceeds to tell them that they're wrong (they have no idea what they're talking about). Socrates didn't really directly talk about his own views, but more so tore down other people's views. 

\bigskip 

While the older people found him to be very annoying and dangerous, the younger people found him to be relatable; this, in turn, caused the older people to believe that he was corrupting the youth. 

\bigskip 

The Oracle claimed that he was the wisest of men, but didn't recognize how that was the case. In reality, a lot of people thought they knew more than they actually do; Socrates believed that he knew less than what he actually did. 


\subsection{The Crito}
Crito offers Socrates several reasons why he morally ought to be escape. In particular, this is because Socrates needed to be convinced of the reasons why he should escape. 
\begin{itemize}
    \item Socrates' death will reflect poorly on his friends. The public at large will see that Crito has the means to help Socrates, but chose to not.
    \item He should not be concerned with the financial costs of escape. In other words, the cost is negligible when considering that Socrates should actually try to escape.
    \item If Socrates stays, he will aid is enemies wronging him unjustly. In other words, if Socrates stays and dies, Crito will believe that he is complicit in his death. 
    \item He will abandon his children and leave them without a father. By bringing children into the world, Socrates is morally responsible for caring for them. 
\end{itemize}
We note that Socrates will argue against Crito's arguments. Socrates responds to each of Crito's arguments like so: 
\begin{itemize}
    \item We should not be concerned with the popular opinion, only with the opinion of the wise. In other words, if you were sick, you wouldn't ask the world what you should do; rather, you should ask a doctor -- or someone who is professional -- for advice. For him, the same thing is true for morality. 
    \item In breaking the \emph{Law of Athens}, he would do it great often, almost similar to striking a parent. He is treating the law as of it was a person; this city treated him well (up to to the point of imprisonment), raised him well, etc. So, by breaking the law, it's like harming the person, or specifically the law. So, rather than breaking the law, he must instead convince the law to allow him to leave.
    \item By choosing to live in Athens, Socrates believes he signed a \emph{social contact} to obey the laws. That is, by living in some particular city, you're \emph{choosing} to abide by the laws.
    \item If he disliked the laws, he was free to move to another city with different laws. Relating this back to the previous point, he's essentially binding himself to the city that he chose to stay in.  
\end{itemize}

\subsection{Lessons from Plato}
Plato offers little room for political dissent. In particular: 
\begin{itemize}
    \item Plato's conception of consent seems to rely on free movement and limitless options. How accurate is this today? If a state were to say that no protests were allowed, then you can't really do much to protest since the state is effectively right. 
    \item Plato's personification of the Law gives it prominence due to how much it benefits us. How much does the law benefit us? If we are not adequately benefited, would this argument hold sway? 
    \item There seems to be no restriction on legal dissent. 
    \item How justified is Plato's dissent against the execution of Socrates by his own lights? 
\end{itemize}

\subsection{What is Dissent?}
Dissent can be defined by so: 
\begin{definition}{Dissent}{}
    To hold or express opinions that are at variance with those commonly or officially held. 
\end{definition}

From the old French \emph{dissentir}, which is from the Latin \emph{dissentire}, which means: To differ in sentiments, disagree, be at odds, contradict, etc. 

\subsubsection{How is it Done?}
\begin{itemize}
    \item Marches/Protests. 
    \item Speeches.
    \item Sit-ins. 
    \item Petitions. 
    \item Voting. 
    \item Revolutions.
    \item Boycotts. 
    \item Strikes. 
    \item Property destruction. 
    \item Individual violence.
    \item Art.  
\end{itemize}
We will primarily focus on student dissent. 

\subsubsection{Student Dissent}
\begin{itemize}
    \item \textbf{Greensboro Sit-Ins:} For historical content, this was in protest to the practice of segregation in both the southern states and implicitly in the northern states. 

    \bigskip 
    
    Four students sat in at the Woolworth lunch counter -- specifically, the one that only served white people -- on Feb. 1, 1960, and \emph{refused} to leave without being served. The following day, 20 students joined the sit-in. The sit-in movement spread across the country with thousands of participants. 

    \item \textbf{UC Berkeley:} Led by Mario Savio, this lasted between 1964-1965. The primary focus was the suppression of political speech on campus, and was tied to the Vietnam war and Civil Rights movement. 

    \item \textbf{Tiananmen Square Protests:} In 1989, pro-democracy students led protests in Tiananmen Square. The protests lasted between April 15 and July 4, when troops and tanks moved into Tiananmen Square to dispel the protests. The troops fired on those in their path, with the estimated deaths falling between a few hundred and a few thousand.

    \item \textbf{Iranian Student Protests:} In 1999, a group of students protested the closing of \emph{Salam} by the government, a reformist newspaper at Tehran University. In response, a paramilitary group attacked the students, resulting in three killed, hundreds injured, and thousands disappearing.

    \item \textbf{Hong Kong Student Protests for Democracy:} In 1897, Hong Kong was established as a British colony. However, in 1997, the control reverted back to China, who then imposed the ``One country two systems'' policy. In 2014, the Chinese Communist Party proposed electoral reforms which would allow them to 'pre-screen' any political candidates for elections in Hong Kong. In other words, this would give the government the ability to tell you who you could vote for.

    \item \textbf{Iguala Mass Kidnapping:} In 2014, students from the Raul Isidro Burgos Teachers College went to Iguala to procure buses for an upcoming march in Mexico City. 43 students were kidnapped by local forces and killed. 
\end{itemize}

\subsection{Self-Immolation}
This is the act of setting oneself on fire in protest. It was widely used to protest the Vietnam War and Iranian revolution.

\end{document}