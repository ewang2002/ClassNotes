\documentclass[letterpaper]{article}
\usepackage[margin=1in]{geometry}
\usepackage[utf8]{inputenc}
\usepackage{textcomp}
\usepackage{amssymb}
\usepackage{natbib}
\usepackage{graphicx}
\usepackage{gensymb}
\usepackage{amsthm, amsmath, mathtools}
\usepackage[dvipsnames]{xcolor}
\usepackage{enumerate}
\usepackage{mdframed}
\usepackage[most]{tcolorbox}
\usepackage{csquotes}
% https://tex.stackexchange.com/questions/13506/how-to-continue-the-framed-text-box-on-multiple-pages

\tcbuselibrary{theorems}

\newcommand{\R}{\mathbb{R}}
\newcommand{\Z}{\mathbb{Z}}
\newcommand{\N}{\mathbb{N}}
\newcommand{\Q}{\mathbb{Q}}
\newcommand{\C}{\mathbb{C}}
\newcommand{\code}[1]{\texttt{#1}}
\newcommand{\mdiamond}{$\diamondsuit$}
\newcommand{\PowerSet}{\mathcal{P}}
\newcommand{\Mod}[1]{\ (\mathrm{mod}\ #1)}
\DeclareMathOperator{\lcm}{lcm}

%\newtheorem*{theorem}{Theorem}
%\newtheorem*{definition}{Definition}
%\newtheorem*{corollary}{Corollary}
%\newtheorem*{lemma}{Lemma}
\newtheorem*{proposition}{Proposition}


\newtcbtheorem[number within=section]{theorem}{Theorem}
{colback=green!5,colframe=green!35!black,fonttitle=\bfseries}{th}

\newtcbtheorem[number within=section]{definition}{Definition}
{colback=blue!5,colframe=blue!35!black,fonttitle=\bfseries}{def}

\newtcbtheorem[number within=section]{corollary}{Corollary}
{colback=yellow!5,colframe=yellow!35!black,fonttitle=\bfseries}{cor}

\newtcbtheorem[number within=section]{lemma}{Lemma}
{colback=red!5,colframe=red!35!black,fonttitle=\bfseries}{lem}

\newtcbtheorem[number within=section]{example}{Example}
{colback=white!5,colframe=white!35!black,fonttitle=\bfseries}{def}

\newtcbtheorem[number within=section]{note}{Important Note}{
        enhanced,
        sharp corners,
        attach boxed title to top left={
            xshift=-1mm,
            yshift=-5mm,
            yshifttext=-1mm
        },
        top=1.5em,
        colback=white,
        colframe=black,
        fonttitle=\bfseries,
        boxed title style={
            sharp corners,
            size=small,
            colback=red!75!black,
            colframe=red!75!black,
        } 
    }{impnote}
\usepackage[utf8]{inputenc}
\usepackage[english]{babel}
\usepackage{fancyhdr}
\usepackage[hidelinks]{hyperref}

\pagestyle{fancy}
\fancyhf{}
\rhead{POLI 28}
\chead{Monday, April 11, 2022}
\lhead{Lecture 5}
\rfoot{\thepage}

\setlength{\parindent}{0pt}

\begin{document}

\section{Theories of Civil Disobedience}
\begin{mdframed}[]
    \begin{mdframed}[]
        \emph{Civil disobedience is a public, nonviolent, conscience yet political at contrary to law usually done with the aim of bringing about change in the law or politices of the government.}
    \end{mdframed}
    \emph{John Rawls}
\end{mdframed}
Broadly speaking, there are 4 different elements in order to constitute civil disobedience (according to Rawls): 
\begin{enumerate}
    \item Be public. If it's private, then you really aren't doing anything.
    \item Be nonviolent. You aren't there to harm people (that's not civil disobedience).
    \item Be political. It cannot be something done out of self-interest; specifically, it cannot \emph{purely} be out of self-interest.
    \item Be contrary to law. It's not a law-abiding action.
\end{enumerate}

\subsection{Two Categories}
There are two categories of civil disobedience.
\begin{enumerate}
    \item The law being protested is the law being violated. This was the case for \textbf{Susan B. Anthony}, where the law she broke was the law that she was protesting (women not being able to vote).
    
    \item The law being protested is different from the law being violated. For example, you break one law, which isn't the same law as the one that's being protested. There could be some reasons for this -- punishment, really hard to break the law, etc. 
    
    \bigskip
    
    For example, suppose you oppose the increase of nuclear weapons in the United States. How would you break that law? So, you might decide to break a law where you aren't allowed to protest in a military site. 
\end{enumerate}

\begin{mdframed}[]
    (Example: The Jury.) Suppose there is a case in which a law unjustly (to your mind) allows for the death penalty. You have been selected for a jury for that case, and the judge asks you whether you are willing to administer the death penalty. You have three options: 
    \begin{enumerate}
        \item Lie to the judge\footnote{You are technically under oath.}; say that you are open to the death penalty and use your position to thwart conviction. 
        \item Tell the truth; be recused from the jury and risk the administration of the death penalty. 
        \item Potentially convict the person regardless of your personal reviews. 
    \end{enumerate}
    Consider the following questions. 
    \begin{itemize}
        \item What would you do?
        \item And, if someone were to choose action (a), would that be civil disobedience?
        
        \bigskip 

        We should note that this may or may not be civil disobedience. It certainly is nonviolent, it certainly is political, and in some sense it is contrary to law. However, this action may not be publicized.
    \end{itemize}
\end{mdframed}

\subsection{Henry David Thoreau}
Thoreau was a transcendentalist, most prominent for his book \emph{Walden}. He was also widely known for \emph{Civil Disobedience}.

\subsubsection{Civil Disobedience}
He begins with political philosophy. He argues that government is only necessary because people act unethically; ideally, a government would not exist. Note that, for practical purposes, he isn't demanding anarchy but, rather, a better government. 

\bigskip 

He maintains that it is immoral to participate in unjust organizations and, given the presence of slavery, the United States is such an organization. For this reason, it is immoral to participate in the US government. He claims that the biggest obstacle to abolitionism is the complicit, inactive northerner. 

\bigskip 

Essentially, he says that the abolitionists up north are also complicit due to them paying taxes to the government which is very much supporting slavery.
\begin{mdframed}[]
    \begin{mdframed}[]
        \emph{Under a government which governs unjustly, the true place for a just man is prison.}
    \end{mdframed}
    - Thoreau
\end{mdframed}

He also critiqued voting as a form of dissent. By definition, it can only succeed once dissent is no longer needed, since a majority of the country already supports the cause being dissented for. 

\bigskip 

How immoral does something need to be to warrant civil disobedience? Thoreau doesn't really say anything about this. 

\subsection{John Rawls}
According to Rawls, civil disobedience occurs when participants in a \textbf{nearly-just} society violate laws to draw attention to the injustices and to \textbf{communicate} opposition to then. 

\bigskip 

Suppose you're in a society where there are no violations of justice. Then, is it morally permissible to engage in civil disobedience in said society? Rawls implies no. Now, suppose you're in a society that was so unjust that violent dissent is morally acceptable. Rawls still implies that civil disobedience is morally allowed. 

\bigskip 

The \textbf{communicative} aspects of civil disobedience has two components:
\begin{enumerate}
    \item Backward Looking: highlighting previous injustices. 
    \item Forward Looking: attempt to prevent future injustices. Essentially seeking some sort of a policy change. 
\end{enumerate}
He also presents three conditions in which civil disobedience is justified: 
\begin{itemize}
    \item The principles at issues are clearly principles of justice. You can harm the state's laws if the issues you are addressing are issues in society. 
    \item Appeals to justice within the scope of the law have been attempted and ineffective. In other words, you need to make an attempt to get the issues resolved by working with the state law.
    \item If necessary, there is coordination with other groups experiencing injustice. 
\end{itemize}

\end{document}