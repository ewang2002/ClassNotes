\documentclass[letterpaper]{article}
\usepackage[margin=1in]{geometry}
\usepackage[utf8]{inputenc}
\usepackage{textcomp}
\usepackage{amssymb}
\usepackage{natbib}
\usepackage{graphicx}
\usepackage{gensymb}
\usepackage{amsthm, amsmath, mathtools}
\usepackage{xcolor}
\usepackage{enumerate}
\usepackage{framed}
\usepackage{tcolorbox}
\tcbuselibrary{theorems}

\newcommand{\R}{\mathbb{R}}
\newcommand{\Z}{\mathbb{Z}}
\newcommand{\N}{\mathbb{N}}
\newcommand{\Q}{\mathbb{Q}}
\newcommand{\code}[1]{\texttt{#1}}
\newcommand{\mdiamond}{$\diamondsuit$}

%\newtheorem*{theorem}{Theorem}
%\newtheorem*{definition}{Definition}
\newtheorem*{proposition}{Proposition}
%\newtheorem*{corollary}{Corollary}
%\newtheorem*{lemma}{Lemma}

\newtcbtheorem[number within=section]{theorem}{Theorem}
{colback=green!5,colframe=green!35!black,fonttitle=\bfseries}{def}

\newtcbtheorem[number within=section]{definition}{Definition}
{colback=blue!5,colframe=blue!35!black,fonttitle=\bfseries}{def}

\newtcbtheorem[number within=section]{corollary}{Corollary}
{colback=yellow!5,colframe=yellow!35!black,fonttitle=\bfseries}{def}

\newtcbtheorem[number within=section]{lemma}{Lemma}
{colback=red!5,colframe=red!35!black,fonttitle=\bfseries}{def}
\usepackage[utf8]{inputenc}
\usepackage[english]{babel}
\usepackage{fancyhdr}
\usepackage[hidelinks]{hyperref}

\pagestyle{fancy}
\fancyhf{}
\rhead{POLI 28}
\chead{Wednesday, April 06, 2022}
\lhead{Lecture 4}
\rfoot{\thepage}

\setlength{\parindent}{0pt}

\begin{document}

\section{The History of Democracy}
How did we arrive at democracy as a form of government? 

\subsection{Early Democracy}
There are some scattered evidence of democracy in the early days. 

\bigskip 

Arguably, the first democracies occurred when small tribes made group decisions prior to the agricultural revolution. There is also scattered evidence of democratic decisisons in Mesopotamia and the Indian subcontinent. 

\subsubsection{The Seeds of Mycenae}
We begin by talking about Ancient Greece. Democracy started off in Ancient Greece, although it wasn't a true democracy since many groups of people (women, slaves, etc.) could not vote. 

\bigskip 

The Mycenaean civilization (1750 BCE - 1050 BCE) was relatively ordinary. In particular:
\begin{itemize}
    \item There was moderate, but not exceptional, wealth. 
    \item Power was centralized among a few individuals.
    \item This civilization had a written language, but a predominantly illiterate population. This is due to the many symbols associated with this language. 
\end{itemize}
For unknown reasons, the civilization collapsed in the 12th century BCE; the city-states were destroyed without explanation. 

\subsubsection{The (Greek) Dark Ages}
A certain amount of loss occurred in this period. In particular:
\begin{itemize}
    \item A population where some people could read or write became a population where no one could read or write. 
    \item Technological advances were lost. 
\end{itemize}
That being said, this was also a period of relative equality; there was so little money that even the aristocracy had little wealth. 

\bigskip

Athens was ruled by 9 Archons. 6 functioned primarily as judges while three (Eponymous Archon, Polemarch, Archon Basileus) were in charge of domestic, foreign, and religious affairs, respectively. Since these judges were effectively running the country, this was known as a \textbf{kritocracy}.

\bigskip 

One of the first set of written laws was the \textbf{Code of Draco} (620 BC). It was said that this set of law was written in blood as these laws were more like a set of punishment -- it was very \emph{draconian}, hence the name. This was seen as one step towards democracy. \emph{Note} that this also implies that, during this time period, some literacy came back. 

\subsubsection{Athens in Crisis}
Athens reached a point of crisis when large numbers of its citizens were debted into slavery. Solon was appointed \emph{sole Archon} and charged with enacting reforms on society. 
\begin{enumerate}
    \item Repealed (most of) the Code of Draco. 
    \item Freed slaves who had been Athenian citizens. 
    \item He did \emph{not} redistribute wealth or abolish Archonships. He allowed them to keep their money. 
    \item Established the \textbf{Athenian Assembly}.
\end{enumerate}
The Athenian Assembly was open to all male citizens of Athens. You would have the ability to vote or veto laws that were put into place. 

\subsubsection{Tyranny and Democracy}
Shortly after Solon left Athens, Peisistratos enacted a coup and ruled Athens. He targeted the aristocracy, reduced their power, and redistributed much of their wealth. After his death, the Assembly had political control. 

\bigskip 

The resulting \emph{democracy} in Athens was limited and unstable. Athenians attempted to use democracy as a weapon in political conflicts. 

\subsubsection{Roman Democracy}
Roman democracy began gradually, with the nobility gradually ceding voting rights to (free) male citizens. 

\bigskip 

The collapse of Roman democracy was foreshadowed by Gaius Marius's decision to allow individuals to control private armies and the Senate's inability to govern a large region effectively. 

\subsubsection{Democracy and Islam}
The Prophet Muhammad (570 - 632) was both a religious and political leader. In addition to founding the religion of Islam, he united Arabia under a single government. 

\bigskip 

The Quran does not dictate how political leaders are to be chosen, but says that the affairs of the faithful are to be decided by mutual consultation amongst themselves. 

\bigskip 

When Muhammad died, Abu Bakr succeeded him via election, though future leaders were determined \textbf{hereditarily}.

\bigskip 

Al-Mansur (714 - 775) ordered the translation of many Greek texts, including the Platonic and Aristotelian critques of democracy. We note that, in the Islamic world, these critques were taken seriously; in particular, writings by people like Plato, who critqued democracy for essentially being a tool to elect populist leaders instead of the knowledgeable, were taken seriously.

\subsubsection{Religious Crisis in Europe}
There was one main religion in Europe; other religions were prosecuted or else shunned. Essentially, 16th century witnessed religious, moral, and political crisis in Europe. 

\bigskip 

Martin Luther's 95 theses triggered the Protestant Reformation, and the Pope's refusal to allow Henry VIII to divorce Catherine of Aragon led to the formation of the Anglican Church. 

\subsubsection{English Civil War}
This religious crisis essentially led to the civil war. Here, this occurred between the royalists and parliamentarians. It culminated with the execution of King Charles I. 

\bigskip 

Thomas Hobbes (1588 - 1679) supported the monarchy, but nevertheless held that legitimacy arose from the people, not God. Hobbes believed that the people themselves made democracy legit. 

\bigskip 

John Locke followed Hobbes; Locke believed very much in democracy, as opposed to Hobbes, who believed more in the monarchy. Locke believed very strongly in the right to life, liberty, and right to property. Note that the Declaration of Independence was based on this (life, liberty, and pursuit of happiness). 

\subsubsection{The Iroquois Confederacy}
The Iroquois Confederacy was formed by 1450 by the \textbf{The Great Peacemaker} to unit five nations (Mohawk, Onondaga, Oneida, Cayuga, Seneca) in present-day New York, Pennsylvania, and Canada. After 1722, the Tuscarora nation was added as well. 

\bigskip 

The constitution of the Confederacy was known as the \textbf{Great Law of France}. The Iroquois Confederacy had two bodies of government, similar to what we have in the United States. 

\subsubsection{American Democracy}
The American Democracy was founded with elements from Greek, Roman, English, and Iroquois civilizations. At first, women and slaves could not vote. By modern standards, America was not a democracy. Many political scientists believed that America was really only a democracy after the Voting Rights Act of 1965. 

\subsubsection{French Revolution}
This revolution itself had moments of nominal democracy. In practice, nearly all elections were faked in order to support whichever group held power at the time. Therefore, it wasn't a democracy, but it was the result of the revolution which caused the unstability of the monarchy. 

\subsubsection{Haitian Revolution}
This was the most successful revolution of the time, and occurred during the same time as the French Revolution. In this revolution, Black Haitians were liberated from slavery and French colonial rule. Dessalines declared Haiti a free republic for Haitians -- though his troops referred to him as \emph{Emperor of Life}. After his death, Haiti was split into a monarchy and a republic.

\subsection{Facism and Colonialism}
European colonialism began in the 16th century. By 1914, Europe had effective political control over 84\% of the global population. The rise of facism in Europe upended democracies and monarchies. After World War II, many colonial empires collapsed; democracy and communism both took hold. After the fall of the Soviet Union, democracy became the predominantly acceptable version of government. 

\subsubsection{Democracy Under Threat}
As of 2017, 57\% of countries are democracies of some kind. 13\% are autocracies and 28\% have elements of both democracy and autocracy. Note that $\frac{1}{3}$ of countries are considered to be less democratic than they used to be, including the United States.

\end{document}