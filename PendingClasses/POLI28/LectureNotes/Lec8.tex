\documentclass[letterpaper]{article}
\usepackage[margin=1in]{geometry}
\usepackage[utf8]{inputenc}
\usepackage{textcomp}
\usepackage{amssymb}
\usepackage{natbib}
\usepackage{graphicx}
\usepackage{gensymb}
\usepackage{amsthm, amsmath, mathtools}
\usepackage[dvipsnames]{xcolor}
\usepackage{enumerate}
\usepackage{mdframed}
\usepackage[most]{tcolorbox}
\usepackage{csquotes}
% https://tex.stackexchange.com/questions/13506/how-to-continue-the-framed-text-box-on-multiple-pages

\tcbuselibrary{theorems}

\newcommand{\R}{\mathbb{R}}
\newcommand{\Z}{\mathbb{Z}}
\newcommand{\N}{\mathbb{N}}
\newcommand{\Q}{\mathbb{Q}}
\newcommand{\C}{\mathbb{C}}
\newcommand{\code}[1]{\texttt{#1}}
\newcommand{\mdiamond}{$\diamondsuit$}
\newcommand{\PowerSet}{\mathcal{P}}
\newcommand{\Mod}[1]{\ (\mathrm{mod}\ #1)}
\DeclareMathOperator{\lcm}{lcm}

%\newtheorem*{theorem}{Theorem}
%\newtheorem*{definition}{Definition}
%\newtheorem*{corollary}{Corollary}
%\newtheorem*{lemma}{Lemma}
\newtheorem*{proposition}{Proposition}


\newtcbtheorem[number within=section]{theorem}{Theorem}
{colback=green!5,colframe=green!35!black,fonttitle=\bfseries}{th}

\newtcbtheorem[number within=section]{definition}{Definition}
{colback=blue!5,colframe=blue!35!black,fonttitle=\bfseries}{def}

\newtcbtheorem[number within=section]{corollary}{Corollary}
{colback=yellow!5,colframe=yellow!35!black,fonttitle=\bfseries}{cor}

\newtcbtheorem[number within=section]{lemma}{Lemma}
{colback=red!5,colframe=red!35!black,fonttitle=\bfseries}{lem}

\newtcbtheorem[number within=section]{example}{Example}
{colback=white!5,colframe=white!35!black,fonttitle=\bfseries}{def}

\newtcbtheorem[number within=section]{note}{Important Note}{
        enhanced,
        sharp corners,
        attach boxed title to top left={
            xshift=-1mm,
            yshift=-5mm,
            yshifttext=-1mm
        },
        top=1.5em,
        colback=white,
        colframe=black,
        fonttitle=\bfseries,
        boxed title style={
            sharp corners,
            size=small,
            colback=red!75!black,
            colframe=red!75!black,
        } 
    }{impnote}
\usepackage[utf8]{inputenc}
\usepackage[english]{babel}
\usepackage{fancyhdr}
\usepackage[hidelinks]{hyperref}

\pagestyle{fancy}
\fancyhf{}
\rhead{POLI 28}
\chead{Wednesday, April 20, 2022}
\lhead{Lecture 8}
\rfoot{\thepage}

\setlength{\parindent}{0pt}

\begin{document}

\section{Uncivil Obedience}
We begin with an example of this. 
\begin{mdframed}[]
    (Example: Work to Rule.) Work to rule is one (common) type of uncivil obedience. This is a process in which workers comply with rules to such an extent that it prevents a company or government's functioning.
    \begin{itemize}
        \item \underline{2012 collective bargaining between American Airlines and its pilots:} The pilots didn't agree with the airline regarding pay and benefits, so they started (obsessively) complying with every single regulation and rule that they theoretically were supposed to follow before, causing hours of delays.
        \item \underline{British Rail in during negotiations in 1968:} Workers were in a contract dispute, so they started to follow every rules and regulations that they were theoretically supposed to follow. 
        \item \underline{Taxi drivers in protest of Uber in Paris, 2015:} The taxi industry has numerous rules to begin with (compared to Uber's policies), and there were numerous rules governing taxis in general; all of this caused the taxi industry to be significantly less competitive than Uber drivers. So, they protested this by complying with every rule/regulation obsessively, thus causing lots of traffic. 
    \end{itemize} 
\end{mdframed}
Some reasons why uncivil obedience exists are because: 
\begin{itemize}
    \item If workers go on strike, they may not get paid. If they show up to work, they will get paid. 
    \item Workers on strike may get fired. Workers that show up will most likely not get fired. 
\end{itemize}

\subsection{What is Uncivil Obedience?}
\begin{center}
    \begin{tabular}{p{3in}|p{3in}}
        \textbf{Civil Disobedience} & \textbf{Uncivil Obedience} \\ 
        \hline 
        The public and peaceful \textbf{violation} of the law in order to protest laws and policies believed to be unjust. & The public and peaceful \textbf{compliance} with law in order to protest laws and policies believed to be unjust.
    \end{tabular}
\end{center} 
More specifically, uncivil obedience consists of the following elements:  
\begin{enumerate}
    \item \textbf{Conscientiousness}: a deliberate, normative act or coordinated acts. In other words, you need to believe that what you're doing is morally right\footnote{You do not necessarily need to be ``right.''}. If this is done \emph{purely} out of self-interest does not count as uncivil obedience. 
    \item \textbf{Communicativeness:} that communicates criticism of law or policy. You need to communicate with the people you are protesting against. Plus, even though they vocally deny it\footnote{For example, they fear they may get fired for protesting.}, their actions may say otherwise.
    \item \textbf{Reformist Intent:} With a significant purpose of changing or disrupting that law or policy. That is, your intent is to change some policy that you are protesting against. 
    \item \textbf{Legality:} In confirmity with all applicable positive law. In other words, if you're violating the law, you do not satisfy this requirement. 
    \item \textbf{Legal Provocation:} In a manner that calls attention to its own formal legality, while departing from prevailing expectations about how the law will be followed or applied. In other words, you're following the law/policy in a way that calls said law/policy to question (e.g. shows that said law/policy is ridiculous).
\end{enumerate}

\begin{mdframed}[]
    (Example: Speed Limits.) In 1993, California motorists protested the reduction in speed limit (from 60 to 55 mph) by driving at the speed limit along the highway. So, these motorists protested by driving no faster than 55 mph, causing massive delays for everyone else and annoyed many drivers. 
\end{mdframed}


\begin{mdframed}[]
    (Example: Sexist Dress Code.) In Boston in the 1960's, philosophy professor Elizabeth Anscombe attended a fine-dining restaurant in Boston, and was informed that they refused to seat women wearing pants. 

    \bigskip 

    She responded by removing her pants and demanding to be seated. 
\end{mdframed}

\begin{mdframed}[]
    (Example: Marriage and Divorce.) In 1980, Congress enacted a \emph{marriage tax}\footnote{Essentially, married couples paid more in taxes than unmarried couples.}; it was cheaper for some couples to live together unmarried than for them to be offficially wed. 

    \bigskip 

    Angela and David Boyter protested by divorcing every December and remarrying each January in order to file taxes as unmarried people.
\end{mdframed}


\subsection{Two Methods of Uncivil Obedience}
There are two methods of uncivil obedience. 
\begin{center}
    \begin{tabular}{p{3in}|p{3in}}
        \textbf{Obsessive Compliance} & \textbf{Reliance on Options} \\ 
        \hline 
        \begin{itemize}
            \item Complies with a law more than is normal, to an extent that it undermines the law itself or surrounding practices.
            \item An example is following the speed limit.
        \end{itemize} & \begin{itemize}
            \item Avails oneself of options (which are legally permitted) in order to hinder a system which relies upon those who do not use such options. 
            \item An example is divorcing and then remarrying to avoid paying more in taxes.
        \end{itemize}
    \end{tabular}
\end{center}

\subsection{Rules \& Standards}
There are some differences between rules and standards. 
\begin{center}
    \begin{tabular}{p{3in}|p{3in}}
        \textbf{Rules} & \textbf{Standards} \\ 
        \hline
        \begin{itemize}
            \item A rule is a precise, clear-cut restriction on behavior.
            \item For example, ``do not drive faster than 55 mph.''
            \item Often, there is an amount of arbitrariness in order to faciliate precision.
        \end{itemize} & \begin{itemize}
            \item A standard is a looser and broader restriction with room for reasonable disagreement. 
            \item For example, ``drive at a reasonable speed.''
            \item Often, standards build in norms; for example, any driver who violates the reasonable speed norm is clearly reckless and unsafe. 
        \end{itemize}
    \end{tabular}
\end{center}
Uncivil obedience often lives in the gap between rules and standards. In other words, by complying with the rules, you may end up violating some standards. 

\subsection{Restricting Uncivil Obedience}
What kind of things can restrict uncivil obedience?
\begin{itemize}
    \item Shift from rules to standards. If you get rid of the rules, it can make it much harder to engage in uncivil obedience.
    \item Introduce additional rules that eliminate previously-available options. For example, if people started paying large debts in pennies, a rule could be put in place that disallows large debt to be paid with pennies. 
    \item Introduce additional rules prohibiting ``abuse of right.''
    \item Appeal to group dynamics. If you start violating some norms in a close group dynamic, there may be larger consequences. 
\end{itemize}

\subsection{Morality of Uncivil Obedience}
Those who believe that societal norms ought to be obeyed in gneral have an initial reason to object to uncivil obedience; however, those who merely believe that laws ought to be obeyed have no such reasons. 

\bigskip

However, some might argue that uncivil obedience highlights the absurdity and arbitrariness of law, and therefore threatens to undermine the legal system more broadly. 

\bigskip 

Additionally, uncivil obedience might undermine the principles guiding the laws, rather than the laws themselves; in particular, principles like honesty and integrity.

\bigskip 

Often, civil disobedience is defended when other formal methods have failed, especially by those who lack social power. Notably, those who enact noncivil obedience are sometimes government agents who have power.

\end{document}