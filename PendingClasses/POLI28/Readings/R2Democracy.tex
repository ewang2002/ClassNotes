\documentclass[letterpaper]{article}
\usepackage[margin=1in]{geometry}
\usepackage[utf8]{inputenc}
\usepackage{textcomp}
\usepackage{amssymb}
\usepackage{natbib}
\usepackage{graphicx}
\usepackage{gensymb}
\usepackage{amsthm, amsmath, mathtools}
\usepackage[dvipsnames]{xcolor}
\usepackage{enumerate}
\usepackage{mdframed}
\usepackage[most]{tcolorbox}
\usepackage{csquotes}
% https://tex.stackexchange.com/questions/13506/how-to-continue-the-framed-text-box-on-multiple-pages

\tcbuselibrary{theorems}

\newcommand{\R}{\mathbb{R}}
\newcommand{\Z}{\mathbb{Z}}
\newcommand{\N}{\mathbb{N}}
\newcommand{\Q}{\mathbb{Q}}
\newcommand{\C}{\mathbb{C}}
\newcommand{\code}[1]{\texttt{#1}}
\newcommand{\mdiamond}{$\diamondsuit$}
\newcommand{\PowerSet}{\mathcal{P}}
\newcommand{\Mod}[1]{\ (\mathrm{mod}\ #1)}
\DeclareMathOperator{\lcm}{lcm}

%\newtheorem*{theorem}{Theorem}
%\newtheorem*{definition}{Definition}
%\newtheorem*{corollary}{Corollary}
%\newtheorem*{lemma}{Lemma}
\newtheorem*{proposition}{Proposition}


\newtcbtheorem[number within=section]{theorem}{Theorem}
{colback=green!5,colframe=green!35!black,fonttitle=\bfseries}{th}

\newtcbtheorem[number within=section]{definition}{Definition}
{colback=blue!5,colframe=blue!35!black,fonttitle=\bfseries}{def}

\newtcbtheorem[number within=section]{corollary}{Corollary}
{colback=yellow!5,colframe=yellow!35!black,fonttitle=\bfseries}{cor}

\newtcbtheorem[number within=section]{lemma}{Lemma}
{colback=red!5,colframe=red!35!black,fonttitle=\bfseries}{lem}

\newtcbtheorem[number within=section]{example}{Example}
{colback=white!5,colframe=white!35!black,fonttitle=\bfseries}{def}

\newtcbtheorem[number within=section]{note}{Important Note}{
        enhanced,
        sharp corners,
        attach boxed title to top left={
            xshift=-1mm,
            yshift=-5mm,
            yshifttext=-1mm
        },
        top=1.5em,
        colback=white,
        colframe=black,
        fonttitle=\bfseries,
        boxed title style={
            sharp corners,
            size=small,
            colback=red!75!black,
            colframe=red!75!black,
        } 
    }{impnote}
\usepackage[utf8]{inputenc}
\usepackage[english]{babel}
\usepackage{fancyhdr}
\usepackage[hidelinks]{hyperref}

\pagestyle{fancy}
\fancyhf{}
\rhead{POLI 28}
\chead{Monday, April 04, 2022}
\lhead{Reading 2}
\rfoot{\thepage}

\setlength{\parindent}{0pt}

\begin{document}

\section{Democracy - Instrumental vs. Non-Instrumental Value}
At an introductory level, this essay is not denying that voting has instrumental\footnote{It's a means to something; in this case, democracy can produce many good things, hence having instrumental value.} value; however, like shopping, voting has noninstrumental as well as instrumental value. 

\bigskip 

In shopping, instrumental value is this idea that we can acquire goods that we desire. Noninstrumental value is this idea that we enjoy surveying the options that we're given and choosing what we want to buy. 

\bigskip 

In particular, its (shopping) noninstrumental value is conditional on its instrumental value. The same can be applied to democratic participation, in particular voting. 

\subsection{Values of Democracy}
Many citizens enjoy this idea of voting. In particular, they enjoy the idea that they can essentially govern themselves. 

\bigskip 

To see the noninstrumental values of democracy, we need to alter our understanding of democracy in general. We consider what John Stuart Mill and John Dewey have to say. Both agree that: 
\begin{itemize}
    \item Democracy is more than a set of governing practices; it is a \textbf{culture of way of life of a community} defined by equality of membership, reciprocal cooperation, and mutual respect and sympathy and located in civic society. 
    \item Voting is just one mode of democratic self-expression among many others that constitute a democratic way of life.
\end{itemize}
Mill states that: 
\begin{itemize}
    \item Democratic participation is a way of life that unites two higher pleasures -- sympathy and autonomy. 
\end{itemize}
Dewey states that:
\begin{itemize}
    \item It is the exercise of practical intelligence in discovering and implementing collective solutions to shared problems, which is the basic function of community life. 
\end{itemize}

\subsection*{Democracy as a Way of Life}
It can be understood on three levels.
\begin{itemize}
    \item \textbf{Membership organization:} Democracy involves universal and equal citizenship of all the permanent members of a society who live under the jurisdiction of a state.
    \begin{itemize}
        \item The idea is that those who are subject to the laws of government should have a say in its operations, either directly (participatory democracy) or indirectly (election of representatives).
        \item All permanent members of a society should be entitled to the status of citizens, not subjects, with the rights to vote upon reaching adulthood. 
    \end{itemize}
    \item \textbf{Mode of government:} Democracy is government by discussion among equals.
    \begin{itemize}
        \item Walter Bagehot defined democracy as the \emph{government by discussion}. The author would add that democracy is government by discussion \emph{among equals}. 
        \item In a democracy, there is but one class of citizens; no one is second-class, and no permanent member is excluded from access to citizenship. 
    \end{itemize}
    \item \textbf{Culture:} Democracy consists in the freewheeling cooperative interaction of citizens from all walks of life on terms of equality in civil society.
    \begin{itemize}
        \item Democracy requires citizens from different walks of life to talk to one another about matters of common interest, to determine what issues warrant collective action, etc. 
    \end{itemize}
\end{itemize}


\subsection{The Values of a Democratic Way of Life}
A few things to stress. 
\begin{itemize}
    \item First, democracy embodies relations of mutual respect and equality, which are required as a matter of right.
    \item Second, democracy helps avoid some of the evils of undemocratic ways of life. It helps secure individuals against abuse, neglect, subordination, and pariah status. It also protects against the corruption of character of those who occupy privileged positions in society. 
    \item Third, democratic ways of life realize the shared goods of sympathy and autonomy.
    \item Fourth, democracy is a mode of collective learning.
\end{itemize}

\end{document}