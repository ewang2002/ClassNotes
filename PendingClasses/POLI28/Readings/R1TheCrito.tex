\documentclass[letterpaper]{article}
\usepackage[margin=1in]{geometry}
\usepackage[utf8]{inputenc}
\usepackage{textcomp}
\usepackage{amssymb}
\usepackage{natbib}
\usepackage{graphicx}
\usepackage{gensymb}
\usepackage{amsthm, amsmath, mathtools}
\usepackage[dvipsnames]{xcolor}
\usepackage{enumerate}
\usepackage{mdframed}
\usepackage[most]{tcolorbox}
\usepackage{csquotes}
% https://tex.stackexchange.com/questions/13506/how-to-continue-the-framed-text-box-on-multiple-pages

\tcbuselibrary{theorems}

\newcommand{\R}{\mathbb{R}}
\newcommand{\Z}{\mathbb{Z}}
\newcommand{\N}{\mathbb{N}}
\newcommand{\Q}{\mathbb{Q}}
\newcommand{\C}{\mathbb{C}}
\newcommand{\code}[1]{\texttt{#1}}
\newcommand{\mdiamond}{$\diamondsuit$}
\newcommand{\PowerSet}{\mathcal{P}}
\newcommand{\Mod}[1]{\ (\mathrm{mod}\ #1)}
\DeclareMathOperator{\lcm}{lcm}

%\newtheorem*{theorem}{Theorem}
%\newtheorem*{definition}{Definition}
%\newtheorem*{corollary}{Corollary}
%\newtheorem*{lemma}{Lemma}
\newtheorem*{proposition}{Proposition}


\newtcbtheorem[number within=section]{theorem}{Theorem}
{colback=green!5,colframe=green!35!black,fonttitle=\bfseries}{th}

\newtcbtheorem[number within=section]{definition}{Definition}
{colback=blue!5,colframe=blue!35!black,fonttitle=\bfseries}{def}

\newtcbtheorem[number within=section]{corollary}{Corollary}
{colback=yellow!5,colframe=yellow!35!black,fonttitle=\bfseries}{cor}

\newtcbtheorem[number within=section]{lemma}{Lemma}
{colback=red!5,colframe=red!35!black,fonttitle=\bfseries}{lem}

\newtcbtheorem[number within=section]{example}{Example}
{colback=white!5,colframe=white!35!black,fonttitle=\bfseries}{def}

\newtcbtheorem[number within=section]{note}{Important Note}{
        enhanced,
        sharp corners,
        attach boxed title to top left={
            xshift=-1mm,
            yshift=-5mm,
            yshifttext=-1mm
        },
        top=1.5em,
        colback=white,
        colframe=black,
        fonttitle=\bfseries,
        boxed title style={
            sharp corners,
            size=small,
            colback=red!75!black,
            colframe=red!75!black,
        } 
    }{impnote}
\usepackage[utf8]{inputenc}
\usepackage[english]{babel}
\usepackage{fancyhdr}
\usepackage[hidelinks]{hyperref}

\pagestyle{fancy}
\fancyhf{}
\rhead{POLI 28}
\chead{Wednesday, March 30, 2022}
\lhead{Reading 1}
\rfoot{\thepage}

\setlength{\parindent}{0pt}

\begin{document}

\section{The Crito}
This was a conversation between Socrates (denoted ``So'') and Crito (denoted ``Cr''). 

\begin{itemize}
    \item It looks like Socrates is in prison awaiting death (i.e. death penalty).
    \begin{mdframed}[]
        \begin{quotation}
            Cr: \emph{\dots It's clear from this that it will arrive today, and you will have to end your life tomorrow, Socrates.}

            \dots 

            So: \emph{I will tell you. I must be put to death sometime the day after the ship arrives?}
        \end{quotation}
        From \emph{The Crito}, 43c
    \end{mdframed}
    Here, it seems like Socrates has accepted his death. 

    \item Crito gives Socrates the opportunity to be saved by helping Socrates escape prison. 
    \begin{mdframed}[]
        \begin{quotation}
            Cr: \dots \emph{But, my supernatural Socrates, even now listen to me and be saved. I think that if you die it won't just be one misfortune. Apart from being separated from the kind of friend the like of which I will never find again, many people, moreover, who do not know me and you well will think that I could have saved you if I were willing to spend the money, but that I didn't care to. And wouldn't this indeed be the most shameful reputation, that I would seem to value money above friends? For the many will not believe that it was you yourself who refused to leave here, even though we were urging you to.}
        \end{quotation}
        From \emph{The Crito}, 44c
    \end{mdframed}
    Crito states that, if he didn't save Socrates, there would be multiple misfortunes. Not only would Crito lose a friend, but other people (who didn't know them well) would think that Crito didn't try hard enough to save Socrates. Essentially, while part of Crito's justification is that he would lose a friend, another part of Crito's justification is his reputation. 

    \item Socrates responds by telling Crito to not care too much about other people's opinions. 
    \begin{mdframed}[]
        \begin{quotation}
            So: \emph{But why should we, blessed Crito, care so much about the opinion of the many?}

            Cr: \emph{But surely you see, Socrates, that we must pay attention to the opinion of the many, too. The present circumstances make it clear that the many can inflict not just the least of evils but practically the greatest, \textbf{when one has been slandered amongst them}}
        \end{quotation}
        From \emph{The Crito}, 44c
    \end{mdframed}
    Note that Crito is implying that Socrates was wrongfully imprisoned. 

    \item Crito states that, if he were to help Socrates escape, Socrates should not worry about the potential suffering that his friends would have to go through (with regards to Socrates' escape).
    \begin{mdframed}[]
        \begin{quotation}
            Cr: \emph{Well, let's leave that there. But tell me this, Socrates. You're not worried, are you, about me and your other friends, how, if you were to leave here, the informers would make trouble for us, about how we stole you away from here, and we would be compelled either to give up all our property or a good deal of money, or suffer some other punishment at their hands? If you have any such fear, let it go, because it is our obligation to run this risk in saving you and even greater ones if necessary. So trust me and do not refuse.}
        \end{quotation}
        From \emph{The Crito}, 45a
    \end{mdframed}
    Here, Crito is saying that Socrates should not worry about whatever retaliation his friends may have to deal with due to Socrates escaping. Specifically, Crito says that it is an obligation for him to help his friend -- Socrates -- escape. This is a strong obligation, especially since Crito says that he -- and Socrates' friends -- are willing to give up their money and property to make this escape a reality. 

    \begin{mdframed}[]
        \begin{quotation}
            So: \emph{I certainly am worried about these things, Crito, and lots of others too.}
        \end{quotation}
        From \emph{The Crito}, 45a
    \end{mdframed}
    Note that Socrates still worries, despite Crito's assurances.

    \begin{mdframed}[]
        \begin{quotation}
            Cr: \emph{Well don't fear them. Indeed, some people only need to be given a little silver and they're willing to rescue you and get you out of here. And on top of that, don't you see how cheap those informers are and that we wouldn't need to spend a lot of money on them? My money is at your disposal, and is, I think, sufficient. Furthermore, even if, because of some concern for me, you think you shouldn't spend my money, there are these visitors here who are prepared to spend theirs. One of them has brought enough silver for this very purpose, Simmias of Thebes, and Kebes too is willing, and very many others. So, as I say, don't give up on saving yourself because you are uneasy about these things.}
        \end{quotation}
        From \emph{The Crito}, 45b
    \end{mdframed}
    Once again, Crito assures Socrates, saying that there are many people willing to help Socrates escape. More specifically, there are people willing to spend their own money to help Socrates escape. 

    \begin{mdframed}[]
        \begin{quotation}
            Cr: \emph{And don't let what you said in the court get to you, that you wouldn't know what to do with yourself as an exile. In many places, wherever you go, they would welcome you. And if you want to go to Thessaly, I have some friends there who will think highly of you and provide you with safety, so that no one in Thessaly will harass you.}
        \end{quotation}
        From \emph{The Crito}, 45c
    \end{mdframed}
    Specifically, Crito is making the point that, if Socrates escapes, Socrates' friends will help ensure that Socrates lives well in exile, that Socrates will be in a place where he is welcomed. 

    \begin{mdframed}[]
        \begin{quotation}
            Cr: \emph{What's more, Socrates, what you are doing doesn't seem right to me, giving yourself up when you could have been saved, ready to have happen to you what your enemies would urge—and did urge—in their wish to destroy you.}
        \end{quotation}
        From \emph{The Crito}, 45c
    \end{mdframed}
    Crito now presents the ethical argument that, if Socrates were to be killed, it would be a success to whoever put Socrates in prison. From the above, it is implied that Socrates was imprisoned unjustly, further implying that whoever imprisoned Socrates could be an enemy. From a moral standpoint, Crito is worried that, if Socrates refuses to be saved and is put to death by this enemy, then Crito would have aided in Socrates' death. 


    \begin{mdframed}[]
        \begin{quotation}
            Cr: \emph{In addition, I think you are betraying your sons, whom you could raise and educate, by going away and abandoning them, and, as far as you are concerned, they can experience whatever happens to come their way, when it's likely that as orphans they'll get the usual treatment of orphans. One should either not have children or endure the hardship of raising and educating them, but it looks to me as though you are taking the laziest path, whereas you must choose the path a good and brave man would choose, especially when you keep saying that you care about virtue your whole life long.}
        \end{quotation}
        From \emph{The Crito}, 45d
    \end{mdframed}
    Crito presents a second ethical argument that, if Socrates were to die now, he would leave behind his sons, which would then put his sons in a worse position than they were before this occurred. Crito furthers this particular argument -- the abandonment of his sons -- by accusing Socrates of taking the easy way out; that is, when you raise a child, you should either take the time to endure the hardship associated with raising a child, or not have children at all. 


    \begin{mdframed}[]
        \begin{quotation}
            Cr: \emph{So I am ashamed both on your behalf and on behalf of us your friends, that this whole affair surrounding you will be thought to have happened due to some cowardice on our part: the hearing of the charge in court, that it came to trial when it need not have, and the legal contest itself, how it was carried on, and, as the absurd part of the affair} \dots
        \end{quotation}
        From \emph{The Crito}, 45e
    \end{mdframed}
    Once again, Crito is saying that it is quite absurd that Socrates would think that Crito helping Socrates escape is insane, especially considering how the court proceedings went. In particular, this paragraph further implies the wrongfulness of Socrates' imprisonment. 



    \item Socrates counters Crito's arguments, in particular stating that Crito's argument is not right. 
    \begin{mdframed}[]
        \begin{quotation}
            So: \emph{My dear Crito, your eagerness would be worth a lot if it were in pursuit of something righteous, but the more it is not, the more difficult it is to deal with} \dots
        \end{quotation}
        From \emph{The Crito}, 46b
    \end{mdframed}
    Socrates begins his response by stating that what Crito is saying isn't exactly moral.

    \begin{mdframed}[]
        \begin{quotation}
            So: \dots \emph{I am the sort of person who is persuaded in my soul by nothing other than the argument which seems best to me upon reflection. At present I am not able to abandon the arguments I previously made, now that this misfortune has befallen me, but they appear about the same to me, and I defer to and honor the ones I did previously} \dots
        \end{quotation}
        From \emph{The Crito}, 46c
    \end{mdframed}
    Socrates now states that, despite the situation he is in, he will follow through with what he believes. In other words, no matter what situation he is in, he refuses to change his views. 

    \begin{mdframed}[]
        \begin{quotation}
            So: \emph{I am determined to examine this together with you, Crito, whether it appears different when I consider it in this condition, or the same, and whether we should ignore it or be persuaded by it} \dots
        \end{quotation}
        From \emph{The Crito}, 46e
    \end{mdframed}
    Here, Socrates says that he wants to explain, to Crito, why he doesn't want to escape the prison. The following dialogue are essentially just them talking to each other about all of this. 

    \begin{mdframed}[]
        \begin{quotation}
            So: \emph{Shouldn't we value the good opinions, and not the worthless ones?} 
            
            Cr: \emph{Yes}

            So: \emph{Aren't the good ones the opinions of the wise, while the worthless ones come from the ignorant?}

            Cr: \emph{Of course.}

            So: \emph{So then, what did we say, again, about cases such as this: should a man in training, who takes it seriously, pay any heed to the praise and blame and opinion of everyone, or only to one person, the one who is a doctor or a trainer?}

            Cr: \emph{Only to the one}

            So: \emph{So he should fear the criticisms and welcome the praises of that one person, and not those of the many?}

            Cr: \emph{Clearly.}

            \dots
        \end{quotation}
        From \emph{The Crito}, 47a-b
    \end{mdframed}
    Socrates once again mentions that reputation shouldn't be a factor in Crito's argument. Specifically, he's saying that the opinions of a bunch of strangers are not necessarily good opinions to follow. He then relates this to a ``man in training''; specifically, he's saying that, when a man is training and is making progress, should that man listen to the doctor or trainer -- the experienced one -- or a bunch of strangers? 

    \begin{mdframed}[]
        \begin{quotation}
            So: \emph{Therefore, based on what you've agreed, we must examine the following, whether it is just or unjust for me to try to leave here} \dots \emph{As for us, since the argument requires it, I suppose we should examine precisely what we just mentioned, whether we will act justly, we who lead as well as we who are led, by giving money and thanks to those who will get me out of here, or whether we will in fact act unjustly by doing all of this.}
        \end{quotation}
        From \emph{The Crito}, 48c-d
    \end{mdframed}
    Now that Socrates has established why reputation is not a good argument, he's now going to talk about whether it is just or unjust for him to escape, using the points that Crito made in his argument. 

    \begin{mdframed}[]
        \begin{quotation}
            So: \emph{And so one should not repay an injustice with an injustice, as the many think, since one should never act unjustly.}

            Cr: \emph{It appears not.}

            \dots

            So: \emph{And then? Is returning a harm for a harm just, as the many say,
            or not just?}

            Cr: \emph{Not at all.}

            So: \emph{Because harming a man in any way is no different from doing an injustice.}

            Cr: \emph{That's true.}

            So: \emph{One must neither repay an injustice nor cause harm to any man, no matter what one suffers because of him} \dots
        \end{quotation}
        From \emph{The Crito}, 49b-d
    \end{mdframed}
    Here, it's pretty clear that Socrates is saying that, even if one might act unfairly towards you, you should never act unfairly towards them. Put it another way, no matter how someone might treat you, you should always treat them respectfully. He then compares this injustice with an injustice with harming a man; that is, even if someone harms you, harming said person is an injustice. 

    \begin{mdframed}[]
        \begin{quotation}
            So: \dots \emph{when someone has made an agreement with someone else, and it is just, must he keep to it or betray it?}

            Cr: \emph{He must keep it.}
        \end{quotation}
        From \emph{The Crito}, 49e
    \end{mdframed}
    Once again, Socrates relates what was discussed above with the idea that, if you make an agreement with someone, you must never break it. 

    \begin{mdframed}[]
        \begin{quotation}
            So: \emph{By leaving here without persuading the city are we doing someone a harm, and those whom we should least of all harm, or not? And are we keeping to the just agreements we made, or not?} \dots \emph{If the laws and the community of the city came to us when we were about to run away from here, or whatever it should be called} \dots \emph{By attempting this deed, aren't you planning to do nothing other than destroy us, the laws, and the civic community, as much as you can?} \dots
        \end{quotation}
        From \emph{The Crito}, 50a
    \end{mdframed}
    My impression is that Socrates is saying that, in some sense, by being in this city, you're agreeing to abide by its rules or laws. Thus, by trying to help Socrates escape, Socrates is saying that Crito is essentially breaking this agreement between him and the city that he is in; put it another way, Socrates is saying that he is breaking the very laws of the civic community that he is in. My impression is that this is Socrates' response to Crito's argument that, if Crito didn't do anything to help stop the death of Socrates, Crito would be responsible. 
\end{itemize}

\end{document}