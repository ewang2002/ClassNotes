\documentclass[letterpaper]{article}
\usepackage[margin=1in]{geometry}
\usepackage[utf8]{inputenc}
\usepackage{textcomp}
\usepackage{amssymb}
\usepackage{natbib}
\usepackage{graphicx}
\usepackage{gensymb}
\usepackage{amsthm, amsmath, mathtools}
\usepackage[dvipsnames]{xcolor}
\usepackage{enumerate}
\usepackage{mdframed}
\usepackage[most]{tcolorbox}
\usepackage{csquotes}
% https://tex.stackexchange.com/questions/13506/how-to-continue-the-framed-text-box-on-multiple-pages

\tcbuselibrary{theorems}

\newcommand{\R}{\mathbb{R}}
\newcommand{\Z}{\mathbb{Z}}
\newcommand{\N}{\mathbb{N}}
\newcommand{\Q}{\mathbb{Q}}
\newcommand{\C}{\mathbb{C}}
\newcommand{\code}[1]{\texttt{#1}}
\newcommand{\mdiamond}{$\diamondsuit$}
\newcommand{\PowerSet}{\mathcal{P}}
\newcommand{\Mod}[1]{\ (\mathrm{mod}\ #1)}
\DeclareMathOperator{\lcm}{lcm}

%\newtheorem*{theorem}{Theorem}
%\newtheorem*{definition}{Definition}
%\newtheorem*{corollary}{Corollary}
%\newtheorem*{lemma}{Lemma}
\newtheorem*{proposition}{Proposition}


\newtcbtheorem[number within=section]{theorem}{Theorem}
{colback=green!5,colframe=green!35!black,fonttitle=\bfseries}{th}

\newtcbtheorem[number within=section]{definition}{Definition}
{colback=blue!5,colframe=blue!35!black,fonttitle=\bfseries}{def}

\newtcbtheorem[number within=section]{corollary}{Corollary}
{colback=yellow!5,colframe=yellow!35!black,fonttitle=\bfseries}{cor}

\newtcbtheorem[number within=section]{lemma}{Lemma}
{colback=red!5,colframe=red!35!black,fonttitle=\bfseries}{lem}

\newtcbtheorem[number within=section]{example}{Example}
{colback=white!5,colframe=white!35!black,fonttitle=\bfseries}{def}

\newtcbtheorem[number within=section]{note}{Important Note}{
        enhanced,
        sharp corners,
        attach boxed title to top left={
            xshift=-1mm,
            yshift=-5mm,
            yshifttext=-1mm
        },
        top=1.5em,
        colback=white,
        colframe=black,
        fonttitle=\bfseries,
        boxed title style={
            sharp corners,
            size=small,
            colback=red!75!black,
            colframe=red!75!black,
        } 
    }{impnote}
\usepackage[utf8]{inputenc}
\usepackage[english]{babel}
\usepackage{fancyhdr}
\usepackage[hidelinks]{hyperref}

\pagestyle{fancy}
\fancyhf{}
\rhead{POLI 28}
\chead{Wednesday, April 06, 2022}
\lhead{Reading 3}
\rfoot{\thepage}

\setlength{\parindent}{0pt}

\begin{document}

\section{Polluting the Polls -- When Citizens Should Not Vote}
\emph{The right to vote does not imply the rightness of voting.} In particular, most citizens would supposedly not vote well, so voting would be \emph{wrong} for them.

\bigskip 

\textbf{This paper argues that if a person forms her political beliefs in an unreliable or irresponsible way and lives in a society in which the majority of other citizens also form their beliefs in unreliable ways, she ought not vote.}

\bigskip 

People apparently vote for what they perceive to be the \emph{national interest} rather than their narrow self-interest. However, their perception of the national interest is often wrong, as it is grounded in ignorance and unreliable, irrational processes of belief formation. In other words, their ideological bents reflect bias.

\bigskip 

A comparison that the author makes is that
\begin{mdframed}[]
    We are not obligated to become surgeons, but if we do become surgeons, we ought to be responsible, good surgeons.

    \bigskip 

    We are not obligated to drive, but if we do drive, we ought to be responsible drivers. 

    \bigskip

    The same goes for voting. 
\end{mdframed}


\subsection{Ignorance, Bias, and Irrationality}
It's not necessarily that voters \& average citizens are ignorant about politics, but more so that citizens are irrational about their political beliefs. 

\bigskip 

The author begins by talking about confirmation bias as a starting point, stating that we are often ``locked'' into our own world in the sense that we do not readily accept news that contradict our world. This is rampant in politics. 

\bigskip 

Bryan Caplan states that most people are not just ignorant, but irrational when it comes to their beliefs about economics -- these beliefs may have big emotional payoffs (e.g. if you advocate for a welfare state, then that may make you feel compassionate.) In particular, he makes the following four points of biase; in particular: 
\begin{enumerate}
    \item Conditions are worsening even when they are getting better. 
    \item Underestimate the value of interacting with foreigners. 
    \item Underestimate the ability of the market to improve people's lives. 
    \item Underestimate the value of conserving labor. 
\end{enumerate}

It should be noted that the author believes that his vote "has little probability of causing protectionist legislation to be enacted." In particular, the reason why voters can sustain irrational beliefs is that their Irrationality has \textbf{no cost}. \emph{The individual's vote rarely carries the day, and so is irrelevant to the policy outcome.} Additionally, \emph{in politics, we tend toward beliefs that make us feel good about ourselves rather than beliefs that are well supported by the evidence}. That being said, while you may not make a significant difference when voting, we -- a collective whole -- can; therefore, if we're systematically irrational, we will systematically vote for bad policies. 


\subsection{The Disutility of Individual Bad Votes}
Voters do little harm as individuals by voting; in particular, it is hard to move from \emph{It would be better if most irrational people did not vote} to \emph{Individual irrational people should not vote}.

\subsection{The Duty to Refrain From Collective Harms}
Eliminating bad voting is a collective action problem. In other words, when I refrain from voting poorly, this doesn't fix the problem. \textbf{Bad voting is collectively, not individually, harmful.}

\subsection{Doing One's Part in Modern Democracy}
Citizens of modern democracies are not obligated to vote. However, if they do vote, they are obligated to vote well, on the basis of sound political and economic beliefs. 

\bigskip 

\begin{mdframed}[]
    To live in a well functioning democracy is a great gift and something citizens should be thankful for. Yet one reason democracy is such a great gift is that it does not require us to be political animals. It makes space for many ways of life, including avowedly nonpolitical lives.
\end{mdframed}


\begin{mdframed}[]
    A good democracy is an important public good. We should all do our part to maintain it. One way a person can do his part is by bowing out.
\end{mdframed}

\end{document}