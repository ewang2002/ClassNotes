\documentclass[letterpaper]{article}
\usepackage[margin=1in]{geometry}
\usepackage[utf8]{inputenc}
\usepackage{textcomp}
\usepackage{amssymb}
\usepackage{natbib}
\usepackage{graphicx}
\usepackage{gensymb}
\usepackage{amsthm, amsmath, mathtools}
\usepackage[dvipsnames]{xcolor}
\usepackage{enumerate}
\usepackage{mdframed}
\usepackage[most]{tcolorbox}
\usepackage{csquotes}
% https://tex.stackexchange.com/questions/13506/how-to-continue-the-framed-text-box-on-multiple-pages

\tcbuselibrary{theorems}

\newcommand{\R}{\mathbb{R}}
\newcommand{\Z}{\mathbb{Z}}
\newcommand{\N}{\mathbb{N}}
\newcommand{\Q}{\mathbb{Q}}
\newcommand{\C}{\mathbb{C}}
\newcommand{\code}[1]{\texttt{#1}}
\newcommand{\mdiamond}{$\diamondsuit$}
\newcommand{\PowerSet}{\mathcal{P}}
\newcommand{\Mod}[1]{\ (\mathrm{mod}\ #1)}
\DeclareMathOperator{\lcm}{lcm}

%\newtheorem*{theorem}{Theorem}
%\newtheorem*{definition}{Definition}
%\newtheorem*{corollary}{Corollary}
%\newtheorem*{lemma}{Lemma}
\newtheorem*{proposition}{Proposition}


\newtcbtheorem[number within=section]{theorem}{Theorem}
{colback=green!5,colframe=green!35!black,fonttitle=\bfseries}{th}

\newtcbtheorem[number within=section]{definition}{Definition}
{colback=blue!5,colframe=blue!35!black,fonttitle=\bfseries}{def}

\newtcbtheorem[number within=section]{corollary}{Corollary}
{colback=yellow!5,colframe=yellow!35!black,fonttitle=\bfseries}{cor}

\newtcbtheorem[number within=section]{lemma}{Lemma}
{colback=red!5,colframe=red!35!black,fonttitle=\bfseries}{lem}

\newtcbtheorem[number within=section]{example}{Example}
{colback=white!5,colframe=white!35!black,fonttitle=\bfseries}{def}

\newtcbtheorem[number within=section]{note}{Important Note}{
        enhanced,
        sharp corners,
        attach boxed title to top left={
            xshift=-1mm,
            yshift=-5mm,
            yshifttext=-1mm
        },
        top=1.5em,
        colback=white,
        colframe=black,
        fonttitle=\bfseries,
        boxed title style={
            sharp corners,
            size=small,
            colback=red!75!black,
            colframe=red!75!black,
        } 
    }{impnote}
\usepackage[utf8]{inputenc}
\usepackage[english]{babel}
\usepackage{fancyhdr}
\usepackage[hidelinks]{hyperref}

\pagestyle{fancy}
\fancyhf{}
\rhead{Math 103B}
\chead{December 27th, 2021}
\lhead{Course Notes}
\rfoot{\thepage}

\setlength{\parindent}{0pt}

\begin{document}

\begin{titlepage}
    \begin{center}
        \vspace*{1cm}
            
        \Huge
        \textbf{Math 103B}
            
        \vspace{0.5cm}
        \LARGE
        Modern Algebra: Rings and Fields
            
        \vspace{1.5cm}
            
        \vfill
            
        Winter 2022 \\
        Taught by Professor Brandon Alberts
    \end{center}
\end{titlepage}

\pagenumbering{gobble}

\newpage 

\pagenumbering{gobble}
\begingroup
    \renewcommand\contentsname{Table of Contents}
    \tableofcontents
\endgroup

\newpage
\pagenumbering{arabic}

\section{Rings (Chapter 12)}
Recall that a group is a set equipped with a binary operation. However, often times, a lot of sets are naturally endowed with \emph{two} binary operations: addition \emph{and} multiplication. In this case, we want to account for \emph{both} of them at the same time instead of having two groups with the same sets but different operations. To that, we introduce the \emph{ring}.

\begin{definition}{Ring}{}
    A ring $R$ is a set with two \emph{binary operations} (meaning closed operations), addition (denoted by $a + b$) and multiplication (denoted by $ab$), such that for all $a, b, c \in R$:
    \begin{enumerate}
        \item \textbf{Commutative:} $a + b = b + a$
        \item \textbf{Associative:} $(a + b) + c = a + (b + c)$
        \item \textbf{Additive Identity:} There is an \underline{additive identity} $0 \in R$ such that $a + 0 = 0 + a = a$ for all $a \in R$.
        \item \textbf{Additive Inverse:} There is an element $-a \in R$ such that $a + (-a) = (-a) + a = 0$. 
        \item \textbf{Associative:} $a(bc) = (ab)c$. 
        \item \textbf{Distributive Property:} $a(b + c) = ab + ac$ and $(b + c)a = ba + ca$.  
    \end{enumerate}
    We sometimes write this ring out as $(R, +, \cdot)$. 
\end{definition}
\textbf{Remarks:}
\begin{itemize}
    \item A ring is an \underline{abelian group} under addition, but also has an associative multiplication that is \emph{left and right distributive} over addition.
    \item If $a$ and $b$ belong to a commutative ring $R$ and $a$ is nonzero, then we say that $a$ \emph{divides} $b$ (or that $a$ is a factor of $b$) and write $a | b$ if there exists $c \in R$ such that $b = ac$. If $a$ does not divide $b$, we write $a \nmid b$.
    \item If we need to deal with something like:
    \[\underbrace{a + a + \dots + a}_{n \text{ times}}\]
    Then, we will use $n \cdot a$ to mean this. 
\end{itemize}

\subsection{Basic Applications of the Ring}
Here, we introduce several examples of rings. 

\subsubsection{Example 1: Integer Rings}
\[\Z = \{\dots, -2, -1, 0, 1, 2, \dots\}\]
The set of integers under ordinary addition and multiplication is a commutative ring with unity 1. The \emph{units} of $\Z$ are 1 and -1.

\subsubsection{Example 2: Integers Mod N}
\[\Z / n\Z = \{0, 1, \dots, n - 1\}\]
The set of integers modulo $n$ under addition and multiplication is also a commutative ring with unity 1. The set of \emph{units} is $U(n)$. Here, we define $U(n)$ to be the set of integers less than $n$ and relatively prime to $n$ under multiplication modulo $n$. 

\bigskip 

This can also be written as $\Z_n$. 

\subsubsection{Example 3: Polynomial Rings}
The set $\Z[x]$ of all polynomials in the variable $x$ with integer coefficients under ordinary addition and multiplication is a commutative ring with unity $f(x) = 1$. Here, we define: 
\[\Z[x] = \{a_0 + a_1 x + a_2 x^2 + \dots + a_n x^n \mid a_i \in \Z\}\]
So, for example, $x^2 + 4x + 5 \in \Z[x]$. 

\subsubsection{Example 4: Matrix Rings}
The set $M_{2}(\Z)$ of $2 \times 2$ matrices with integer entries is a \emph{noncommutative ring} with unity $\begin{bmatrix} 1 & 0 \\ 0 & 1 \end{bmatrix}$.

\subsubsection{Example 5: Even Integer Rings}
The set $2\Z$ of even integers under ordinary addition and multiplication is a commutative ring \underline{without} unity. 

\subsubsection{Example 6: Direct Sum}
If $R_1, R_2, \dots, R_n$ are rings, then we can create a new ring: 
\[R_1 \oplus R_2 \oplus \dots \oplus R_3 = \{(a_1, a_2, \dots, a_n) \mid a_i \in R_i\}\]
From this, we can perform componentwise addition and multiplication; that is: 
\[(a_1, a_2, \dots, a_n) + (b_1, b_2, \dots, b_n) = (a_1 + b_1, a_2 + b_2, \dots, a_n + b_n)\]
\[(a_1, a_2, \dots, a_n)(b_1, b_2, \dots, b_n) = (a_1 b_1, a_2 b_2, \dots, a_n b_n)\]

\subsection{More on Rings}
\begin{definition}{Commutative Ring}{}
    A ring $R$ is \textbf{commutative} if $ab = ba$ for all $a, b \in R$.
\end{definition}
\textbf{Remark:} In other words, multiplication in a ring does \textbf{not} have to be commutative. However, \emph{if} it is commutative, we say that the ring is commutative.

\begin{definition}{Unity}{}
    A ring $R$ has \textbf{unity} if $1 \in R$ is a multiplicative identity: 
    \[1a = a1 = a\]
\end{definition}
\textbf{Remark:} A ring \emph{does not need to have} an identity under multiplication. If a ring does have a non-zero identity under multiplication, then we say that the identity is a \emph{unity}.

\begin{definition}{Unit}{}
    An element $a \in R$ is called a \textbf{unit} if it has a multiplicative inverse. In other words, $a$ is a unit if there exists an $a^{-1} \in R$ such that: 
    \[a^{-1}a = aa^{-1} = 1\]
\end{definition}
\textbf{Remarks:}
\begin{itemize}
    \item A nonzero element of a \underline{commutative ring} with unity need not have a multiplicative inverse. When it does, we say that it is a unit of the ring. In other words, $a$ is a unit if $a^{-1}$ exists. 
    \item $U(R) = \{\text{Units in } R\}$
    \item $U(n) = \{\text{Units in } \Z / n\Z\}$
\end{itemize}

\begin{definition}{Division}{}
    For $a, b \in R$, we say that $a$ \textbf{divides} $b$ and write $a | b$ if $b = ac$ for some $c \in R$.
\end{definition}

\subsection{Properties of Rings}
We begin by talking about a few important properties. 




\subsubsection{Basic Rules of Multiplication}
\begin{theorem}{}{}
    For all $a \in R$, we have: 
    \[a0 = 0a = 0\]
\end{theorem}

\begin{mdframed}[]
    \begin{proof}
        We know that: 
        \[0a = (0 + 0)a = 0a + 0a\]
        Subtracting both sides by $0a$ gives: 
        \[0 = 0a + (0a - 0a) \implies 0 = 0a\]
        By symmetry, we can do the same for $0a$. Therefore, we are done.
    \end{proof}    
\end{mdframed}

\begin{theorem}{}{}
    For all $a, b \in R$, we have:
    \[a(-b) = (-a)b = -(ab)\]
\end{theorem}

\begin{mdframed}[]
    \begin{proof}
        First, we have: 
        \[a(-b) + ab = a(-b + b) = a0 = 0\]
        Now, if we add $-(ab)$ to both sides, we have: 
        \[a(-b) + ab + -(ab) = -(ab) \implies a(-b) = -(ab)\]
        By symmetry, $(-a)b = -(ab)$ as well. 
    \end{proof}
\end{mdframed}

\begin{theorem}{}{}
    For all $a, b \in R$, we have: 
    \[(-a)(-b) = ab\]
\end{theorem}

\begin{mdframed}[nobreak=true]
    \begin{proof}
        Note that: 
        \begin{equation*}
            \begin{aligned}
                (-a)0 &= 0 \\ 
                    &\iff (-a)(b + (-b)) = 0 \\ 
                    &\iff (-a)b + -a(-b) = 0 \\ 
                    &\iff -(ab) + -a(-b) = 0 \\ 
                    &\iff ab + -(ab) + -a(-b) = ab \\
                    &\iff -a(-b) = ab
            \end{aligned}
        \end{equation*}
        So, we are done. 
    \end{proof}
\end{mdframed}

\begin{theorem}{}{}
    For all $a, b, c \in R$, we have: 
    \[a(b - c) = ab - ac \text{ and } (b - c)a = ba - ca\]
\end{theorem}

\begin{mdframed}[]
    \begin{proof}
        \begin{equation*}
            \begin{aligned}
                a(b - c) &= ab + -(ac) \\ 
                    &= ab + (-a)c \\ 
                    &= ab - ac
            \end{aligned}
        \end{equation*}
        By symmetry, we can apply the other side as well. So, we are done. 
    \end{proof}
\end{mdframed}

\subsubsection{Rules of Multiplication with Unity Element}
\begin{theorem}{}{}
    For all $a \in R$ where $R$ has a unity element 1, we have: 
    \[(-1)a = -a\]
\end{theorem}

\begin{mdframed}[]
    \begin{proof}
        Applying the theorem that we proved:
        \[(-1)a = -(1a) = -a\]
        So, we are done. 
    \end{proof}
\end{mdframed}

Alternatively:
\begin{mdframed}[]
    \begin{proof}
        Since $(\R, +)$ is an abelian group, it suffices to prove that $(-1)a + a = 0$.
        \[(-1)a + a = (-1)a + 1a = (-1 + 1)a = 0a = 0\]
        So, we are done. 
    \end{proof}
\end{mdframed}

\begin{theorem}{}{}
    \[(-1)(-1) = 1\]
\end{theorem}

\begin{mdframed}[]
    \begin{proof}
        Applying the theorem that we proved:
        \[(-1)(-1) = 1(1) = 1\]
        So, we are done.
    \end{proof}
\end{mdframed}

\subsubsection{Uniqueness of Unity and Inverses}
\begin{theorem}{}{}
    If a ring has a unity, it is unique. If a ring element has a multiplicative inverse, it is also unique. 
\end{theorem}

\begin{mdframed}[]
    \begin{proof}
        We will prove both parts individually. Suppose $R$ is a ring.  
        \begin{enumerate}
            \item Suppose $e$ and $e'$ are unity elements in a ring $R$. Then, we know that: 
            \begin{itemize}
                \item $e = ee'$ since $e'$ is a unity. 
                \item $e' = ee'$ since $e$ is a unity. 
            \end{itemize}
            Therefore: 
            \[e = ee' = e'\]
            Which means that the unity must be unique. 

            \item Suppose $a \in R$ and further suppose that $x$ and $y$ are both multiplicative inverses of $a$. Then: 
            \[x = x1 = x(ay) = (xa)y = 1y = y\]
            Therefore, $x = y$ and the two inverses are equal. 
        \end{enumerate}
        Therefore, we are done. 
    \end{proof}
\end{mdframed}

\begin{note}{}{}
    Rings are not groups under multiplication. $R - \{0\}$ is not a group under multiplication. \underline{Rings may not have multiplicative cancellations.}
\end{note}
To show that this is the case, consider the question: Which elements $a \in R$ satisfy $a^2 = a$?
\begin{itemize}
    \item If $R$ has unity, then $a = 1$.
    \item $a = 0$ is always a solution.
\end{itemize}
Now, consider $\Z / 6 \Z = \{0, 1, 2, 3, 4, 5\}$. Then, $a^2 = a$ for $a = 0, 1, 3, 4$. The only units in this ring are 1 and 5. 



















\newpage 
\section{Subrings (Chapter 12)}
Recall that, with groups, we have objects called \emph{subgroups}. The same thing applies here: with rings, we have objects called \emph{subrings}.
\begin{definition}{Subring}{}
    A nonempty subset $S$ of a ring $R$ is a \textbf{subring} of $R$ if $S$ itself is a ring with the operations of $R$.
\end{definition}
\textbf{Remark:} If $R$ is commutative, then $S$ is commutative.

\subsection{Examples of Subrings}
Below are some examples of subrings. 

\subsubsection{Example 1: Simple Subrings}
The trivial subring $\{0\}$ is a subring of any ring $R$. This is because:
\[0(0) \in R \qquad 0 - 0 \in R\]
Any ring $R$ is a subring of itself. This is because for any $a, b \in R$, we know that $a - b = a + (-b) \in R$ and $ab \in R$. 

\subsubsection{Example 2: Integers}
For any positive integer $n$, the set below is a subring of the integers $\Z$: 
\[n\Z = \{0, \pm n, \pm 2n, \pm 3n, \dots\}\]
Take any $a, b \in \Z$. Then, suppose we have $an$ and $bn$. We know that: 
\[an - bn = (a - b)n \in \Z\]
\[an(bn) = anbn\]
Since $anb \in \Z$, it follows that $(anb)n \in n\Z$. 

\subsubsection{Example 3: Rational Numbers}
The ring $\Q$ is a subring of $\R$. 


\subsubsection{Example 4: Gaussian Integers}
Consider the Gaussian integers:
\[\Z[i] = \{a + bi \mid a, b \in \Z\}\]
This is a subring of $\C$. Note that $i^2 = -1$ so:
\[(a + bi)(c + di) = ac + adi + bci + bdi^2 = ac + adi + bci - bd = (ac - bd) + (ad + bc)i\]

\subsubsection{Example 5: Integers with Square Root 2}
Consider the following set: 
\[\Q[\sqrt{2}] = \{a + b\sqrt{2} \mid a, b \in \Q\}\]
This is a subring of $\R$. This is because: 
\[(a + b\sqrt{2})(c + d\sqrt{2}) = ac + ad\sqrt{2} + bc\sqrt{2} + 2bd\]
Note that we can apply the same work used in the previous example. 

\subsubsection{Example 6: Diagonal Matrices}
The set of diagonal matrices is a subring of $M_{2}(\Z)$. 
\[\left\{\begin{bmatrix}
    a & 0 \\ 0 & d
\end{bmatrix} \mid a, d \in \Z\right\}\]

\subsection{Subring Test}
\begin{theorem}{Subring Test}{}
    A nonempty subset $S$ of a ring $R$ is a subring if $S$ is closed under subtraction and multiplication; that is, if $a - b \in S$ and $ab \in S$ whenever $a, b \in S$. 
\end{theorem}

\begin{mdframed}[]
    \begin{proof}
        If $S$ is a subring, then it is a ring and so $S$ must be closed under subtraction and multiplication. 

        \bigskip 

        Suppose $S$ is closed under subtraction and multiplication. Then, we know the following properties (inherited from $R$): 
        \begin{itemize}
            \item $a + b = b + a$
            \item $(a + b) + c = a + (b + c)$
            \item $a(bc) = (ab)c$
            \item $a(b + c) = ab + ac$
            \item $(a + b)c = ac + bc$
        \end{itemize}
        We need to check if $S$ has 0. Since $S$ is not empty, pick some $a \in S$. Then, it follows that: 
        \[a - a = 0 \in S\]
        So, the additive identity exists. Now, if $a \in S$, then $-a = 0 - a \in S$, so additive inverses exist. 

        \bigskip 

        Finally, we need to show that addition is closed. We know that subtraction is closed, so if $a, b \in S$, then $-b \in S$ and $a + b = a - (-b) \in S$. 
    \end{proof}
\end{mdframed}



















\newpage 
\section{Integral Domains (Chapter 13)}
Recall that rings do not have multiplicative cancellation. That is, $ab = ac$ does not imply that $b = c$. However, there are exceptions to this rule.

\subsection{Zero-Divisors}
First, we briefly talk about zero-divisors. 
\begin{definition}{Zero-Divisors}{}
    A \textbf{zero-divisor} is a nonzero element $a$ of a commutative ring $R$ such that there is a nonzero element $b \in R$ with $ab = 0$. 
\end{definition}
For example, $2 \in \Z / 4 \Z$ is a zero divisor. That is: 
\[2 \cdot 2 \equiv 0 \Mod{4}\]
Another example is $M_{2}(\R)$. Take $A = \begin{bmatrix}
    0 & 0 \\ 1 & 0 
\end{bmatrix}$ and $B = \begin{bmatrix}
    0 & 0 \\ 0 & 1
\end{bmatrix}$. Then:
\[A \cdot B = \begin{bmatrix}
    0 & 0 \\ 0 & 0
\end{bmatrix}\] 

\subsection{Integral Domains}
Now, we talk about integral domains. 
\begin{definition}{Integral Domain}{}
    An \textbf{integral domain} is a commutative ring with unity and no zero-divisors.
\end{definition}
\textbf{Remarks:}
\begin{itemize}
    \item Recall that a ring $R$ has \textbf{unity} if $1 \in R$ is a multiplicative identity; that is, $1a = a1 = a$. 
    \item Essentially, in an integral domain, a product is 0 only when one of the factors is 0. That is, $ab = 0$ only when $a = 0$ or $b = 0$. 
\end{itemize}

\subsection{Examples of Integral Domains}
Here are some examples of integral domains. 

\subsubsection{Example 1: The Integers}
The ring of integers is an integral domain. 

\subsubsection{Example 2: Gaussian Integers}
The ring of Gaussian integers $Z[i] = \{a + bi \mid a, b \in \Z\}$ is an integral domain. 

\subsubsection{Example 3: Ring of Polynomials}
The ring $\Z[x]$ of polynomials with integer coefficients is an integral domain. 

\subsubsection{Example 4: Square Root 2}
The ring $\Z[\sqrt{2}] = \{a + b\sqrt{2} \mid a, b \in \Z\}$ is an integral domain. 

\subsubsection{Example 5: Modulo Prime Integers}
The ring $\Z / p\Z$ of integers modulo $a$ prime $p$ is an integral domain. This is because:
\[ab \equiv 0 \Mod{p} \iff p | ab \implies p | a \text{ or } p | b \implies a \equiv 0 \Mod{p} \text{ or } b \equiv 0 \Mod{p}\] 

\subsubsection{Non-Example 1: Modulo Integers}
The ring $\Z / n\Z$ of integers modulo $n$ is not an integral domain when $n$ is not prime. If we write $n = ab$, then $1 < a$ and $b < n$ implies that $ab \equiv 0 \Mod{b}$. We saw an example of this in the zero-divisors section; that is, $\Z / 4\Z$ is not an integral domain because we had $2 \cdot 2 \equiv 0 \Mod{n}$

\subsubsection{Non-Example 2: Matrices}
The ring $M_{2}(\Z)$ of $2 \times 2$ matrices over the integers is not an integral domain. 

\subsubsection{Non-Example 3: Direct Product}
$\Z \oplus \Z$ is not an integral domain. The reason why is because, for:
\[\Z \oplus \Z = \{(x, y) \mid x, y \in \Z\}\]
Take $(0, 1) \in \Z \oplus \Z$ and $(1, 0) \in \Z \oplus \Z$. Then:
\[(0, 1) \cdot (1, 0) = (0, 0)\]

\subsection{Properties of Integral Domains}
\begin{theorem}{Cancellation}{}
    Let $a$, $b$, and $c$ belong to an integral domain. If $ab = ac$, then: 
    \[a = 0 \text{ or } b = c\]
\end{theorem}
\begin{mdframed}[]
    \begin{proof}
        From $ab = ac$, we know that $ab - ac = 0$. Then, we know that $a(b - c) = 0$. There are two cases to consider:
        \begin{itemize}
            \item If $a \neq 0$, it follows that $b - c = 0$.
            \item Otherwise, $a = 0$ and it's trivial.
        \end{itemize}
        So, we are done. 
    \end{proof}
\end{mdframed}


\subsection{Fields}
\begin{definition}{Field}{}
    A \textbf{field} is a commutative ring with unity in which every nonzero element is a unit (i.e. every nonzero element has a multiplicative inverse).
\end{definition}
\textbf{Remarks:} 
\begin{itemize}
    \item To verify that every field is an integral domain, observe that if $a$ and $b$ belong to a field with $a \neq 0$ and $ab = 0$, we can multiply both sides of the last expression by $a^{-1}$ to obtain $b = 0$.
    \item In other words, you can never have an $x$ such that $0x = 1$. 
\end{itemize}


\subsubsection{Properties of Fields}
\begin{theorem}{}{}
    A finite integral domain is a field.
\end{theorem}

\begin{mdframed}[]
    \begin{proof}
        Let $R$ be a finite integral domain and suppose $a \in R \setminus \{0\}$. Consider the set: 
        \[\{a, a^2, a^3, a^4, \dots\} \subseteq R\]
        Because $R$ is finite, there must be some overlap, i.e. there exists two integers $j < i$ such that $a^j = a^i$. This implies that $a^j = a^{i - j} a^j$. Since we have an integral domain, we can perform multiplicative cancellation; so: 
        \[1 = a^{i - j} \text{ for } i - j \geq 1\]
        Then, $i - j - 1 \geq 0$ with $(a)(a^{i - j - 1}) = a^{i - j} = 1$. So, $a^{-1} = a^{i - j - 1}$ is a multiplicative inverse of $a$. 
    \end{proof}
\end{mdframed}


\begin{corollary}{}{}
    For every prime $p$, $\Z / p \Z$, the ring of integers modulo $p$ is a field.
\end{corollary}
\textbf{Remark:} This is often denoted $\F_p$ in this context. 

\subsubsection{Examples of Fields}
Here are some examples and non-examples of fields.

\begin{enumerate}
    \item \underline{Infinite Sets:} $\R$, $\C$, and $\Q$ are all fields. 
    \item \underline{Matrices:} $M_{2}(\R)$ is not a field because not all matrcies have an inverse. 
    \item \underline{(Non-Example) Polynomials:} $\R[x]$ is not a field. This is because not all functions have a \emph{polynomial} inverse. For example, the inverse of $x + 3$ is $\frac{1}{x + 3}$, which isn't a polynomial. However, $\R[x]$ is an integral domain. 
    \item \underline{Field w/ 9 Elements:} $\F_{3}[i] = \{0, 1, 2, i, 1 + i, 2 + i, 2i, 1 + 2i, 2 + 2i\}$. Recall that $i^2 = -1 \equiv 2 \Mod{3}$. 
    \item \underline{Field w/ 4 Elements:} $\{0, 1, a, b\}$. 
    \begin{center}
        \begin{tabular}{c|c c c c}
            $+$ & 0 & 1 & $a$ & $b$ \\  
            \hline 
            0   & 0 & 1 & $a$ & $b$ \\ 
            1   & 1 & 0 & $b$ & $a$ \\
            $a$ & $a$ & $b$ & 0 & 1 \\ 
            $b$ & $b$ & $a$ & 1 & 0
        \end{tabular}

        \begin{tabular}{c|c c c c}
            $\cdot$ & 0 & 1 & $a$ & $b$ \\ 
            \hline 
            0   & 0 & 0 & 0 & 0 \\ 
            1   & 0 & 1 & $a$ & $b$ \\
            $a$ & 0 & $a$ & $b$ & 1 \\ 
            $b$ & 0 & $b$ & 1 & $a$
        \end{tabular}
    \end{center}
    \item \underline{Rational Numbers:} $\Q[\sqrt{5}] = \{a + b\sqrt{5} \mid a, b \in \Q\} \subseteq \R$ is a field. First, to show that it's a field, we need to show that every nonzero element has multiplicative inverses. Suppose $a + b\sqrt{5} \neq 0$. Then: 
    \[a + b\sqrt{5} \neq 0 \iff b\sqrt{5} \neq a \iff (a, b) \neq (0, 0)\]
    In other words, $a, b$ are not both zero. Note that since $\sqrt{5}$ is irrational, $\sqrt{5} \neq \frac{a}{b}$. So, $\frac{1}{a + b\sqrt{5}} \in \R$ by $a + b\sqrt{5}$ is not zero and $\R$ is a field. With this in mind, we have: 
    \begin{equation*}
        \begin{aligned}
            \frac{1}{a + b\sqrt{5}} &= \frac{1}{a + b\sqrt{5}} \cdot \frac{a - b\sqrt{5}}{a - b\sqrt{5}} \\ 
                &= \frac{a - b\sqrt{5}}{a^2 - ab\sqrt{5} + ab\sqrt{5} - 5b^2} \\ 
                &= \frac{a - b\sqrt{5}}{a^2 - 5b^2} \\ 
                &= \frac{a}{a^2 - 5b^2} + \frac{-b}{a^2 - 5b^2} \sqrt{5} \in \Q[\sqrt{5}]
        \end{aligned}
    \end{equation*}
\end{enumerate}





\subsection{Characteristic of a Ring}
Consider the ring $\Z_{3}[i]$, with the elements:
\[\{0, 1, 2, i, 1 + i, 2 + i, 2i, 1 + 2i, 2 + 2i\}\]
For any element $x$ in this ring, we have: 
\[3x = x + x + x = 0\]
To better see this process, consider the following examples of elements in the ring:
\begin{itemize}
    \item $2i + 2i + 2i = 6i = 0i = 0$
    \item $(1 + 2i) + (1 + 2i) + (1 + 2i) = 3 + 6i = 0$
    \item And so on.
\end{itemize}
Similarly, in the ring $\{0, 3, 6, 9\} \subset \Z_{12}$, we have, for all $x$: 
\[4x = x + x + x + x = 0\]
We can formalize this into a definition. 
\begin{definition}{Characteristic of a Ring}{}
    The \textbf{characteristic} of a ring $R$ is the least positive integer $n$ such that $nx = 0$ for all $x \in R$. If no such integer exists, we say that $R$ has characteristic 0. The characteristic of $R$ is denoted by $\ch{R}$.    
\end{definition}
So, for example, the ring of integers $\Z$ has characteristic 0 and $\Z_n$ has characteristic $n$. Also, consider $\Z_{3} = \{0, 1, 2\}$. Then, we know that:
\[3x = x + x + x = 0 \qquad \forall x\]
So the characteristic of $\Z_{3}$ is \boxed{3}. Now, consider $\Z_{6}$. We know that:
\[6x = x + x + x + x + x + x = 0 \qquad \forall x\]
So, its characteristic is \boxed{6}. As a final example, $\{0\}$ has characteristic \boxed{1}.


\subsubsection{Characteristic of a Ring with Unity}
Occasionally, we might have more complicated rings where the above theorem may be hard to apply. 
\begin{theorem}{Characteristic of a Ring with Unity}{}
    Let $R$ be a ring with unity 1. If 1 has infinite order under addition, then the characteristic of $R$ is 0. If 1 has order $n$ under addition, then the characteristic of $R$ is $n$. 
\end{theorem}
\textbf{Remark:} Here, suppose $(\R, +)$ is a group. Then, we say that $x \in \R$ has an additive order $n$ if $nx = 0$ and $n$ is the smallest positive number with this property. 

\begin{mdframed}[]
    \begin{proof}
        Suppose 1 has infinite order. Then, there is no positive integer $n$ such that $n \cdot 1 = 0$, so $R$ must have characteristic 0. Now, let's suppose that 1 does have additive order $n$. Then, we know that $n \cdot 1 = 0$ and $n$ is the least positive integer with this property. So, for any $x \in R$, we have: 
        \begin{equation*}
            \begin{aligned}
                n \cdot x &= \overbrace{x + x + \dots + x}^{n \text{ times}} \\ 
                    &= \overbrace{1x + 1x + \dots + 1x}^{n \text{ times}} \\ 
                    &= (\overbrace{1 + 1 + \dots + 1}^{n \text{ times}})x \\ 
                    &= (n \cdot 1)x = 0x = 0
            \end{aligned}
        \end{equation*}
        So, $R$ has characteristic $n$.
    \end{proof}
\end{mdframed}
For example, take $R = \Z / 6\Z \oplus \Z / 4\Z \oplus \Z / 10\Z$.
\begin{enumerate}
    \item \underline{Does this ring have unity?} Each member of this direct product ring has 1, so the unity would be $(1, 1, 1) \in R$. 
    \item \underline{What is the characteristic of $R$?} The characteristic order of $R$ is the additive order of $(1, 1, 1) \in R$. Well, we have that: 
    \[n(1, 1, 1) = (n1, n1, n1)\]
    Consider the first element in the pair. When is $n1 \equiv 0 \Mod{6}$? This is when $6 | n$, or: 
    \[n \in \{6, 12, 18, 24, \dots\}\]
    For the third element in the pair, we need to know when $n1 \equiv 0 \Mod{10}$. This is when $10 | n$, or: 
    \[n \in \{10, 20, 30, \dots\}\]
    Here, it's clear that the answer is $\lcm(6, 4, 10) = 60$. 
\end{enumerate}

\begin{theorem}{Characteristic of an Integral Domain}{}
    The characteristic of an integral domain is 0 or prime.
\end{theorem}

\begin{mdframed}[]
    \begin{proof}
        It suffices to consider the additive order of 1. Suppose towards a contradiction that 1 has composite order $n$ and $1 < s$ and $t < n$ such that $n = st$. Then, we know that: 
        \[0 = n1 = (st)1 = s(t1) = (s1)(t1)\]
        But, $1 < s$ and $t < n$, so by minimality of $n$ being the order of 1, it must be that $s1, t1 \neq 0$ and are thus zero-divisors. But, this is a contradiction.
    \end{proof}
\end{mdframed}

\subsection{Summary of Rings}
\begin{center}
    \begin{tabular}{c|c|c}
        \textbf{Ring}   & \textbf{Characteristic} & \textbf{Integral Domain?} \\ 
        \hline 
        $\Z$            & 0                       & Yes \\ 
        $M_{2}(\Z)$     & 0                       & No \\ 
        $\Z \oplus \Z$  & 0                       & No \\ 
        $\F_p$ ($\Z / p\Z$) & $p$                 & Yes \\ 
        $\F_p \oplus \F_p$ & $p$                  & No \\ 
        $\F_{p}[x]$     & $p$                     & Yes \\ 
        $\Z / n\Z[i]$   & $n$                     & $\begin{cases}
            \text{No} & n \text{ not prime.} \\ 
            \text{Maybe} & n \text{ prime.}
        \end{cases}$
    \end{tabular}
\end{center}











\newpage 
\section{Ideals \& Quotient Rings (Chapter 14)}
Recall that if $H$ is a \emph{normal} subgroup of $G$, then there exists a quotient group $G / H$ defined by: 
\[G / H = \{gH \mid g \in G\}\]
Where the operation of the quotient group is: 
\[(g_1 H)(g_2 H) = (g_1 g_2) H\]
We can use this same notion in \emph{rings}. In our case, the normal subgroup of groups is the same as \emph{ideals} in rings. 

\subsection{Ideals}

\begin{definition}{Ideal}{}
    A subring $A$ of a ring $R$ is called a (two-sided) \textbf{ideal} of $R$ if for every element $r \in R$ and every $a \in A$ then: 
    \[ra \in A \text{ and } ar \in A\]
    That is, $rA = \{ra \mid a \in A\} \subseteq A$ and $Ar \subseteq A$. 
\end{definition}

\begin{definition}{Proper Ideal}{}
    An ideal $A$ is called \textbf{proper} if $A \subset R$. 
\end{definition}

\subsubsection{Ideal Test}
\begin{theorem}{Ideal Test}{}
    A nonempty subset $A \subseteq R$ is an ideal if and only if: 
    \begin{enumerate}
        \item $a, b \in A \implies a - b \in A$.
        \item $a \in A, r \in R \implies ra, ar \in R$.
    \end{enumerate}
\end{theorem}

\begin{mdframed}[]
    \begin{proof}
        This is similar to the subring test.
    \end{proof}
\end{mdframed}

\subsubsection{Principal Ideal}
If $R$ is a commutative ring with unity, then the \underline{principal ideal} generated by $a \in R$ is:
\[\cyclic{a} = (a) = \{ra \mid r \in R\}\]

\begin{mdframed}[]
    \begin{proof}
        Pick two elements $ra, sa \in \cyclic{a}$. Then, $ra - sa = (r - s)a \in \cyclic{a}$. Likewise, if $r \in R$, then $sa \in \cyclic{a}$ so:
        \[(sa)r = r(sa) = (rs)a \in \cyclic{a}\]
        So, we are done. 
    \end{proof}
\end{mdframed}

\subsubsection{Basic Examples of Ideals}
We now go over some basic examples of ideals. 
\begin{enumerate}
    \item \underline{Even Integers:} $2\Z \subseteq \Z$ is an ideal. Suppose that there is some integer $r \in \Z$ and $a \in 2\Z$. Then, $a = 2k$ for some $k \in \Z$ so that $ra = r \cdot 2k = 2(rk) \in 2\Z$. Note that this also extends to any $n\Z$; that is, $n\Z$ is an ideal. 
    \item \underline{Trivial Subring:} $\{0\} \subseteq R$ is a trivial ideal because $r \{0\} = \{0\} r = \{0\}$.
    \item \underline{Integers/Rationals:} $\Z \subseteq \Q$ is \emph{not} an ideal. Take $r = \frac{1}{2} \in \Q$ and $a = 1 \in \Z$. Then
    \[ra = \frac{1}{2} (1) = \frac{1}{2} \notin \Z\]
    \item \underline{Integers:} Consider $R = \Z$ with $\cyclic{n} = n\Z$. This is a principal ideal. 
    \item \underline{Polynomials:} If $R = \R[x]$, then
    \begin{equation*}
        \begin{aligned}
            \cyclic{x} &= \{f(x)x \mid f(x) \in \R[x]\} \\ 
                &= \{\text{Polynomials divisible by } x\} \\
                &= \{f(x) \in \R[x] \mid f(0) = 0\}
        \end{aligned}
    \end{equation*}
    \item \underline{Ring of Unity:} The ideal generated by $a_1, a_2, \dots, a_n \in R$, where $R$ is a commutative ring of unity, is:
    \[\cyclic{a_1, a_2, \dots, a_n} = \{r_1 a_1 + r_2 a_2 + \dots + r_n a_n \mid r_1, \dots, r_n \in R\}\]
    \item \underline{Two Elements:} Consider $\cyclic{2, x} \subseteq \Z[x]$. This is defined by
    \[\{f(x) \in \Z[x] \mid f(0) \text{ is even.}\}\]
\end{enumerate}

\subsection{Quotient Ring}
Now, we talk about quotient rings, which are very similar to quotient groups. 
\begin{definition}{Quotient Ring}{}
    Let $I \subseteq R$ be an ideal of $R$. Then, the \textbf{quotient ring} (or factor ring) is the set of \emph{cosets}
    \[R / I = \{r + I \mid r \in R\}\]
    with the operations
    \[(r + I) + (s + I) = (r + s) + I\]
    \[(r + I)(s + I) = (rs) + I\]
\end{definition}

\begin{proposition}
    $R / I$ is a ring. 
\end{proposition}

\begin{mdframed}[]
    \begin{proof}
        \begin{itemize}
            \item For addition, we know that $(R, +)$ is an abelian group. This implies that $(I, +)$ is a normal subgroup of $(R, +)$, so $(R / I, +)$ is a group.
            \item For multiplication, suppose $r + I = r' + I$ and $s + I = s' + I$, i.e.
            \[r = r' + a \text{ and } s = s' + b \text{ for some } a, b \in I\]
            Then, $(rs) = (r' + a)(s' + b) = r's' + r'b + as' + ab$. Note that $r'b, as', ab$ all belong to the ideal. So $r's' + r'b + as' + ab \in r's' + I$.
        \end{itemize}
        And, we are done. 
    \end{proof}
\end{mdframed}




\subsubsection{Examples of Quotient Rings}
Now, we go over some examples of quotient rings. 

\begin{enumerate}
    \item \underline{Integer Ring Modulo 5:} Consider $\Z / 5\Z = \{0 + 5\Z, 1 + 5\Z, 2 + 5\Z, 3 + 5\Z, 4 + 5\Z\}$. We know that $5\Z \subseteq \Z$ is an ideal. 

    \item \underline{Polynomial Ideal:} Consider $\R[x] / \cyclic{x^2 + 1}$. This ring is ``isomorphic'' to $\C$. By identifying $x + \cyclic{x^2 + 1} \in \R[x] / \cyclic{x^2 + 1}$ as $i \in \C$, then
    \[(x + \cyclic{x^2 + 1})^2 = x^2 + \cyclic{x^2 + 1} = x^2 + -(x^2 + 1) + \cyclic{x^2 + 1} = -1 + \cyclic{x^2 + 1}\]
    We can also see this through polynomial long division. There is a unique way to write $f(x) = g(x)q(x) + r(x)$ with $\text{deg } r(x) < \text{deg } g(x)$. From this, we can tell that
    \[f(x) + \cyclic{x^2 + 1} = (x^2 + 1)q(x) + (a + bx) + \cyclic{x^2 + 1} = (a + bx) + \cyclic{x^2 + 1}\]
    
    \item \underline{Gaussian Integers:} Take $\Z[i] / \cyclic{2 - i}$. We claim that this is ``isomorphic'' to $\Z / 5\Z$. It turns out:
    \[\Z[i] / \cyclic{2 - i} = \{0 + \cyclic{2 - i}, 1 + \cyclic{2 - i}, 2 + \cyclic{2 - i}, 3 + \cyclic{2 - i}, 4 + \cyclic{2 - i}\}\] 
    Consider that $2 + \cyclic{2 - i} = i + \cyclic{2 - i}$ because $2 - i \in \cyclic{2 - i}$. Then:
    \begin{equation*}
        \begin{aligned}
            2^2 &+ \cyclic{2 - i} = i^2 + \cyclic{2 - i} \\ 
                &\implies 4 + \cyclic{2 - i} = -1 + \cyclic{2 - i} \\ 
                &\implies 5 \in \cyclic{2 - i}
        \end{aligned}
    \end{equation*}
    Thus, $a + bi + \cyclic{2 - i} = a + 2b + \cyclic{2 - i} = r + \cyclic{2 - i}$ for $0 \leq r < 5$ such that $a + 2b = 5q + r$. Now, how do we know that these cosets are distinct? It suffices to show that $1 + \cyclic{2 - i}$ has additive order 5. So: 
    \[5(1 + \cyclic{2 - i}) = 5 + \cyclic{2 - i} = 0 + \cyclic{2 - i}\]
    Where the last step is due to $5 \in \cyclic{2 - i}$. This tells us that the additive order of $1 + \cyclic{2 - i}$ divides 5. This implies that the order is either 1 or 5. If the order is 5, we are done since this implies that there are 5 distinct cosets. Otherwise, suppose towards a contradiction that $1 + \cyclic{2 - i} \in \Z[i] / \cyclic{2 - i}$ has additive order 1. In this case:
    \begin{equation*}
        \begin{aligned}
            1(1 &+ \cyclic{2 - i}) = 0 + \cyclic{2 - i} \\ 
                &\implies 1 \in \cyclic{2 - i} = \{(2 - i)r \mid r \in \Z[i]\}\\ 
                &\implies 1 = (2 - i)(a + bi) \text{ for some } a, b \in \Z \\
                &\implies 1 = 2a + 2bi - ai + b \\ 
                &\implies 1 + 0i = (2a + b) + (2b - a)i \\ 
                &\implies \begin{cases}
                    1 = 2a + b \\ 
                    0 = 2b - a 
                \end{cases}  \\ 
                &\implies a = \frac{1}{5} \text{ and } \frac{2}{5}
        \end{aligned}
    \end{equation*}
    However, $a, b \in \Z$ so we have a contradiction and so $1 + \cyclic{2 - i}$ must have additive order 5. 
\end{enumerate}


\subsection{Prime and Maximal Ideals}
\begin{definition}{Prime Ideals}{}
    A \textbf{prime ideal} $A$ of a commutative ring $R$ is a proper ideal of $R$ such that $a, b \in R$ and $ab \in A$ imply $a \in A$ or $b \in A$. 
\end{definition}
Consider the following examples:
\begin{itemize}
    \item Consider $R = \Z$. The ideals of $\Z$ are $\{0\}$ and $n\Z$ for $n = 1, 2, \dots$. We know that $2\Z$ is prime. So, if $nm \in 2\Z$, then $nm = 2k$, which is even. This implies that one of $n$ or $m$ is even, so $n \in 2\Z$ or $m \in 2\Z$. 

    \bigskip 
    
    This is true in general. If $p$ is prime, then $p\Z$ is a prime ideal. Recall that if $p | ab$, then $p | a$ or $p | b$ by Euclid's Lemma. 

    \item Consider $6\Z$, which is not prime. We want to show that this is not a prime ideal. To do this, we want to find an $n, m \in \Z$ such that $nm \in 6\Z$ but $n, m \notin 6\Z$. An obvious example is $n = 2$ and $m = 3$. 
    
    \bigskip
    
    In general, if $n = st$ is composite, then $st \in n\Z$ but $s, t \notin n\Z$. 

    \item Consider $R = \{0\}$. This is a prime ideal. Suppose $n, m \in \Z$ with $nm \in R$. Then, $nm = 0$ means that one of $n$ or $m$ is 0, which implies that $n \in R$ or $m \in R$. 
\end{itemize}
\textbf{Fact:} $\{0\} \subseteq R$ is a prime ideal if and only if $R$ is an integral domain. 

\begin{definition}{Maximal Ideals}{}
    A \textbf{maximal ideal} of a commutative ring $R$ is a proper ideal of $R$ such that, when $B$ is an ideal of $R$ and $A \subseteq B \subseteq R$, then $B = A$ or $B = R$. 
    
    \bigskip

    Put it another way, a maximal ideal $I$ of a commutative ring $R$ is a proper ideal which is not contained in any other proper ideals, i.e. if $I \subseteq A \subseteq R$ for some ideal $A$, then $A = I$ or $A = R$. 
\end{definition}


\subsubsection{Properties of Prime \& Maximal Ideals}
\begin{theorem}{}{}
    Let $R$ be a commutative ring with unity and $I \subseteq R$ an ideal. Then, $R / I$ is an integral domain if and only if $I$ is prime. 
\end{theorem}

\begin{mdframed}[]
    \begin{proof}
        Supose $R / I$ is an integral domain. Suppose then that $a, b \in R$ with $ab \in I$. Then, $ab + I = 0 + I$. This further implies that $(a + I)(b + I) = 0 + I$. This implies that $a + I = 0 + I$ \emph{or} $b + I = 0 + I$ by integral domain definition. By the definition of a coset, $a \in I$ or $b \in I$. Thus, $I$ is prime. 
        
        \bigskip

        Suppose now that $I$ is prime. Suppose $a, b \in R$ with $(a + I)(b + I) = 0 + I$ with $ab + I = 0 + I$. This implies that $ab \in I$, which further means that $a \in I$ or $b \in I$ by prime. Thus, $a + I = 0 + I$ or $b + I = 0 + I$. Thus, $R / I$ is an integral domain.
    \end{proof}
\end{mdframed}

\begin{theorem}{}{}
    Let $R$ be a commutative ring with unity and $I \subseteq R$ an ideal. Then, $R / I$ is a field if and only if $I$ is maximal.
\end{theorem}

\begin{mdframed}[]
    \begin{proof}
        Suppose $R / I$ is a field. We want to show that if $I \subseteq A \subseteq R$, then $A = I$ or $A = R$.\footnote{We can prove the fact that $I \subseteq A \subseteq R$ and $A \neq I$ \emph{implies that} $A = R$.} Suppose $A \subseteq R$ is an ideal satisfying $I \subseteq A$ and $A \neq I$. The fact that $A \neq I$ implies that we can choose some $b \in A \setminus I$. This implies that $b + I \neq 0 + I$ and so $b + I \in R / I$ is a unit. This implies that there exists some $c + I \in R / I$ with $(b + I)(c + I) = 1 + I$, which further implies that $bc + I = 1 + I$. Thus, $\dots$. We know that $1 - bc \in A$, but $b \in A \setminus I \subseteq A$ so $bc \in A$ and thus $1 = (1 - bc) + bc \in A$. So, $R = R \cdot 1 \subseteq A$ so that $A = R$. Thus, $I$ is maximal. 

        \bigskip 

        Suppose that $I$ is maximal. We want to show that any $b + I \neq 0 + I$ is a unit in $R / I$. Choose some $b + I \in R / I$ with $b + I \neq 0 + I$, i.e. choose some $b \in R \setminus I$. Consider $B = \{rb + a \mid r \in R, a \in I\}$. Thus, $B = R$ by $I \subseteq B \subseteq R$ and $b \neq I$ ($b \in B, b \in I$).\footnote{Exercise: Show that $B$ is an ideal with contains $I$} From there, $1 \in B$ which means that $1 = rb + a$ for some $r \in R$ and $a \in I$, which finally implies that $1 + I = (r + I)(b + I)$.
    \end{proof}
    % TODO finish this proof
\end{mdframed}

\begin{corollary}{}{}
    All maximal ideals are prime ideals.
\end{corollary}

\begin{mdframed}[]
    \begin{proof}
        Suppose $I \subseteq R$ is maximal.
        \begin{equation*}
            \begin{aligned}
                R / I& \text{ is a field.} \\ 
                    &\implies R / I \text{ is an integral domain.} \\ 
                    &\implies R / I \text{ is prime.}
            \end{aligned}
        \end{equation*}
        So, we are done.
    \end{proof}
\end{mdframed}

\textbf{Remark:} The converse is not true. Consider $\cyclic{x} \subseteq \Z[x]$. This is not maximal by $\cyclic{x} \subset \cyclic{2, x} \subset \Z[x]$. 
\[\Z[x] / \cyclic{x} \longleftrightarrow \Z\]
\[f(x) + \cyclic{x} \longleftrightarrow f(0)\]
\[f(x) + \cyclic{x} = h(x) + \cyclic{x} \iff f(x) - h(x) = g(x)x \text{ for some } g(x) \iff f(0) - h(0) = 0\]
Thus, this ideal $\cyclic{x}$ is prime.

















\newpage
\section{Ring Homomorphisms}
Ring homomorphism is very similar in nature to group homomorphisms. Here, a ring homomorphism preserves the ring operations.

\begin{definition}{Ring Homomorphism}{}
    A \textbf{ring homomorphism} $\varphi$ from a ring $R$ to a ring $S$ is a mapping from $R$ to $S$ that preserves the ring operation. That is, for all $a, b \in R$:
    \[\varphi(a + b) = \varphi(a) + \varphi(b) \qquad \varphi(ab) = \varphi(a)\varphi(b)\]
\end{definition}
\textbf{Remark:} As is the case for groups, the operations on the left of the equal signs are those of $R$, while the operations on the right side are those of $S$. 

\bigskip

Along with ring homomorphisms, there is also ring isomorphisms.
\begin{definition}{Ring Isomorphism}{}
    A \textbf{ring isomorphism} is a ring homomorphism that is both one-to-one and onto (i.e. bijective).
\end{definition}

\subsection{Properties of Ring Homomorphisms}
\begin{theorem}{}{}
    Let $\varphi$ be a ring homomorphism from a ring $R$ to a ring $S$, and let $A$ be a subring of $R$ and let $B$ be an ideal of $S$.
    \begin{enumerate}
        \item For any $r \in R$ and any positive integer $n$, $\varphi(nr) = n\varphi(r)$ and $\varphi(r^n) = (\varphi(r))^n$.
        \item $\varphi(A) = \{\varphi(a) \mid a \in A\}$ is a subring of $S$. 
        \item If $A$ is an ideal and $\varphi$ is onto $S$, then $\varphi(A)$ is an ideal. 
        \item $\varphi^{-1}(B) = \{r \in R \mid \varphi(r) \in B\}$ is an ideal of $R$. 
        \item If $R$ is commutative, then $\varphi(R)$ is commutative.
        \item If $R$ has a unity 1, $S \neq \{0\}$, and $\varphi$ is onto, then $\varphi(1)$ is the unity of $S$. 
        \item $\varphi$ is an isomorphism if and only if $\varphi$ is onto and $\ker(\varphi) = \{r \in R \mid \varphi(r) = 0\} = \{0\}$. 
        \item If $\varphi$ is an isomorphism from $R$ onto $S$, then $\varphi^{-1}$ is an isomorphism from $S$ onto $R$. 
    \end{enumerate}
\end{theorem}

\section{Ring Homomorphisms}
\begin{theorem}{}{}
    Let $\varphi: R \mapsto S$ be a ring homomorphism. Then, $\ker \varphi = \{r \in R \mid \varphi(r) = 0\}$ is an ideal of $R$. 
\end{theorem}

\begin{mdframed}[]
    \begin{proof}
        If $a, b \in \ker \varphi$, then $\varphi(a - b) = \varphi(a) - \varphi(b) = 0 - 0 = 0$, which implies that $a - b \in \ker \varphi$. Now, if we check $a \in \ker \varphi$ and $r \in R$, then $\varphi(ra) = \varphi(r) \varphi(a) = \varphi(r) 0 = 0$. Therefore, $ra \in \ker \varphi$. Thus, $\ker \varphi$ is an ideal by the ideal test. 
    \end{proof}
\end{mdframed}

\subsection{Examples of Ring Homomorphism}
Here are some examples of ring homomorphisms.

\subsubsection{Example 1: Integers and Modulo}
Consider the mapping: 
\[k \mapsto k \Mod{n}\]
This is a ring homomorphism from $\Z$ onto $\Z_{n}$, and is called the natural homomorphism from $\Z$ to $\Z_{n}$. 

\subsubsection{Example 2: Complex Numbers}
Consider the mapping: 
\[a + bi \mapsto a - bi\]
This is a ring homomorphism from the complex numbers onto the complex numbers. 

\subsubsection{Example 3: Functions}
Consider the ring of all polynomials with real coefficients $\R[x]$. Consider the mapping:
\[f(x) \mapsto f(1)\]
This is a ring homomorphism from $\R[x]$ onto $\R$. 


\subsection{First Isomorphism Theorem}
\begin{theorem}{First Isomorphism Theorem}{}
    Let $\varphi: R \mapsto S$ be a ring homomorphism. Then, the map
    \[\overline{\varphi} = R / \ker\varphi \mapsto \varphi(R)\]
    defined by the mapping
    \[r + \ker\varphi \mapsto \varphi(r)\]
    is an isomorphism.
\end{theorem}

\begin{mdframed}[]
    \begin{proof}
        We already know that $\overline{\varphi}: R / \ker \varphi \mapsto \varphi(R)$ is an isomorphism of additive groups; in particular,
        \[(R / \ker \varphi, +) \mapsto (\varphi(R), +)\]
        by the First Isomorphism Theorem for groups. Thus, it suffices to check that:
        \[\overline{\varphi}(xy) = \overline{\varphi}(x) \overline{\varphi}(y)\]
        So, it suffices to check:
        \begin{equation*}
            \begin{aligned}
                \overline{\varphi}((r + \ker\varphi)(s + \ker\varphi)) &= \overline{\varphi}(rs + \ker\varphi) \\ 
                    &= \varphi(rs) \\ 
                    &= \varphi(r)\varphi(s) \\ 
                    &= \overline{\varphi}(r + \ker\varphi) \overline{\varphi}(s + \ker\varphi)
            \end{aligned}
        \end{equation*}
        And so we are done. 
    \end{proof}
\end{mdframed}
\textbf{Remark:} If $I \subseteq R$ is an ideal, then $I = \ker q$ where $q: R \mapsto R / I$, defined by the mapping $r \mapsto r + I$, is the quotient homomorphism.

\subsection{Examples}

\begin{enumerate}
    \item Consider the homomorphism $\varphi: \Z[x] \mapsto \Z$ defined by the mapping $f(x) \mapsto f(0)$. $\varphi$ is a surjective\footnote{If $a \in \Z$, then $(x + a) \xrightarrow{\varphi} 0 + a = a$} homomorphism. By the First Isomorphism Theorem:
    \[\Z[x] / \ker\varphi \cong \Z\]
    Here, we define $\ker\varphi = \{a_1 x + a_2 x^2 + \dots + a_n x^n \mid a_i \in \Z\}$ because $f(0)$ is a constant term. However, we can factor $x$ out to get:
    \[\ker\varphi = \{x(a_1 + a_2 x^1 + \dots + a_n x^{n - 1}) \mid a_i \in \Z\} = \cyclic{x}\]
    And so it follows that:
    \[\boxed{\Z[x] / \cyclic{x} \cong \Z}\]


    \item Consider the homomorphism $\varphi: \R[x] \mapsto \C$ defined by the mapping $f(x) \mapsto f(i)$. $\varphi$ is surjective because $f(a + bx) = a + bi$ for any $a, b \in \R$. We also know that $x^2 + 1 \in \ker \varphi$ by $i^2 + 1 = 0$. This implies that: 
    \[\cyclic{x^2 + 1} \subseteq \ker \varphi \subset \R[x]\]
    \textbf{Fact:} $\cyclic{x^2 + 1}$ is maximal, which implies that $\cyclic{x^2 + 1} = \ker \varphi$.
    \begin{mdframed}[]
        \begin{proof}
            (Of fact.) We prove that $\R[x] / I$ for $I = \cyclic{x^2 + 1}$ is a field for any $a + bx + I$ with $a, b$ not both zero, then $(a + b + I)^{-1} = \frac{a - bx}{a^2 + b^2} + I$.
        \end{proof}
    \end{mdframed}
    Therefore, $\boxed{\R[x] / \cyclic{x^2 + 1} \cong \C}$ by the First Isomorphism Theorem. 
\end{enumerate}

\subsection{Rings with Unity}
\begin{proposition}
    If $R$ has unity, then $\varphi: \Z \mapsto \R$ defined by
    \[\varphi(n) = n \cdot 1 = \begin{cases}
        \underbrace{1 + \dots + 1}_{n \text{ times}} & n > 0 \\ 
        0 & n = 0 \\ 
        \underbrace{-1 - 1 - \dots - 1}_{-n \text{ times}} & n < 0
    \end{cases}\]
    is a homomorphism.
\end{proposition}

\begin{mdframed}[]
    \begin{proof}
        Left as an exercise.
    \end{proof}
\end{mdframed}


\begin{proposition}
    If $R$ is a ring with unity, then:
    \begin{enumerate}[(a)]
        \item If $\ch R = n > 0$, then $R$ contains a subring isomorphic to $\Z / n\Z$.
        \item If $\ch R = 0$, then $R$ contains a subring isomorphic to $\Z$.
    \end{enumerate}
\end{proposition}

\begin{mdframed}[]
    \begin{proof}
        Let $\varphi: \Z \mapsto \R$ with $\varphi(n) = n \cdot 1$. Then, $\ch R$ is the additive order of 1. This implies that if $\ch R = n > 0$, then $\ker \varphi = n\Z$ so $\Z / n\Z \cong \varphi(\Z) \subseteq R$ by the First Isomorphism Theorem. Likewise, if $\ch R = 0$, then $\ker \varphi = \{0\}$ and it follows that $\Z \cong \varphi(\Z) \subseteq R$ by the First Isomorphism Theorem. 
    \end{proof}
\end{mdframed}

\subsection{Fields}

\begin{definition}{Prime Subfield}{}
    The subfield of a field isomorphic to $\F_p$ or $\Q$ is called the \textbf{prime subfield}.
\end{definition}

\begin{theorem}{}{}
    \begin{itemize}
        \item If $F$ is a field of characteristic $p$, then $F$ contains a subfield isomorphic to $\F_p$.
        \item If $F$ is a field of characteristic 0, then $F$ contains a subfield isomorphic to $\Q$. 
    \end{itemize} 
\end{theorem}

\begin{mdframed}[]
    \begin{proof}
        We prove both parts.
        \begin{itemize}
            \item By the previous proposition, $\ch F = p$ implies that the subring is isomorphic to $\Z / p\Z = \F_p$.
            \item $\ch F = 0$ implies that the subring is isomorphic to $\Z$, given by 
            \[\varphi: \Z \mapsto F\]
            which sends $n \mapsto n \cdot 1$. Consider the set 
            \[T = \{ab^{-1} \mid a, b \in \varphi(\Z), b \neq 0\} \subseteq F\]
            We claim that $T$ is a subring isomorphic to $\Q$. 
            \begin{proof}
                Define $\overline{\varphi}: \Q \mapsto F$ by $\overline{\varphi}(a / b) = \varphi(a) \varphi(b)^{-1}$. 

                \begin{itemize}
                    \item \underline{Well-Defined:} This is well-defined since $\frac{a}{b} = \frac{c}{d}$ if and only if $ad = bc$, which then implies that $\varphi(a) \varphi(d) = \varphi(b) \varphi(c)$ for $\varphi: \Z \mapsto F$. This implies that $\varphi(a) \varphi(b)^{-1} = \varphi(c) \varphi(d)^{-1}$, which again implies that $\overline{\varphi}(a / b) = \overline{\varphi}(c / d)$.
                    \item \underline{Homomorphism:} Addition is left as an exercise. For multiplication, see lecture. 
                \end{itemize}
                So, we are done. 
            \end{proof}
        \end{itemize}
        And so on (need to come back).
    \end{proof}
\end{mdframed}

\textbf{Remark:} If $F$ is a field and $I \subseteq F$ is an ideal, then $I = \{0\}$ or $I = F$.

\begin{mdframed}[]
    \begin{proof}
        $F / \{0\} \cong F$ is a field. Thus, $\{0\}$ is a maximal ideal of $F$. This implies that, for all ideals $I$ with $\{0\} \subseteq I \subseteq F$, $I = \{0\}$ or $I = F$. The fact that all ideals satisfy $\{0\} \subseteq I \subseteq F$ concludes the proof.
    \end{proof}
\end{mdframed}


\end{document}