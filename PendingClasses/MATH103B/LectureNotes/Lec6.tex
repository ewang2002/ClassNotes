\documentclass[letterpaper]{article}
\usepackage[margin=1in]{geometry}
\usepackage[utf8]{inputenc}
\usepackage{textcomp}
\usepackage{amssymb}
\usepackage{natbib}
\usepackage{graphicx}
\usepackage{gensymb}
\usepackage{amsthm, amsmath, mathtools}
\usepackage{xcolor}
\usepackage{enumerate}
\usepackage{framed}
\usepackage{tcolorbox}
\tcbuselibrary{theorems}

\newcommand{\R}{\mathbb{R}}
\newcommand{\Z}{\mathbb{Z}}
\newcommand{\N}{\mathbb{N}}
\newcommand{\Q}{\mathbb{Q}}
\newcommand{\code}[1]{\texttt{#1}}
\newcommand{\mdiamond}{$\diamondsuit$}

%\newtheorem*{theorem}{Theorem}
%\newtheorem*{definition}{Definition}
\newtheorem*{proposition}{Proposition}
%\newtheorem*{corollary}{Corollary}
%\newtheorem*{lemma}{Lemma}

\newtcbtheorem[number within=section]{theorem}{Theorem}
{colback=green!5,colframe=green!35!black,fonttitle=\bfseries}{def}

\newtcbtheorem[number within=section]{definition}{Definition}
{colback=blue!5,colframe=blue!35!black,fonttitle=\bfseries}{def}

\newtcbtheorem[number within=section]{corollary}{Corollary}
{colback=yellow!5,colframe=yellow!35!black,fonttitle=\bfseries}{def}

\newtcbtheorem[number within=section]{lemma}{Lemma}
{colback=red!5,colframe=red!35!black,fonttitle=\bfseries}{def}
\usepackage[utf8]{inputenc}
\usepackage[english]{babel}
\usepackage{fancyhdr}
\usepackage[hidelinks]{hyperref}

\pagestyle{fancy}
\fancyhf{}
\rhead{Math 103B}
\chead{Friday, January 14, 2022}
\lhead{Lecture 6}
\rfoot{\thepage}

\setlength{\parindent}{0pt}

\begin{document}

\section{Prime Ideals and Maximal Ideals}

\begin{definition}{Prime Ideals}{}
    A \textbf{prime ideal} $A$ of a commutative ring $R$ is a proper ideal of $R$ such that $a, b \in R$ and $ab \in A$ imply $a \in A$ or $b \in A$. 
\end{definition}
Consider the following examples:
\begin{itemize}
    \item Consider $R = \Z$. The ideals of $\Z$ are $\{0\}$ and $n\Z$ for $n = 1, 2, \dots$. We know that $2\Z$ is prime. So, if $nm \in 2\Z$, then $nm = 2k$, which is even. This implies that one of $n$ or $m$ is even, so $n \in 2\Z$ or $m \in 2\Z$. 

    \bigskip 
    
    This is true in general. If $p$ is prime, then $p\Z$ is a prime ideal. Recall that if $p | ab$, then $p | a$ or $p | b$ by Euclid's Lemma. 

    \item Consider $6\Z$, which is not prime. We want to show that this is not a prime ideal. To do this, we want to find an $n, m \in \Z$ such that $nm \in 6\Z$ but $n, m \notin 6\Z$. An obvious example is $n = 2$ and $m = 3$. 
    
    \bigskip
    
    In general, if $n = st$ is composite, then $st \in n\Z$ but $s, t \notin n\Z$. 

    \item Consider $R = \{0\}$. This is a prime ideal. Suppose $n, m \in \Z$ with $nm \in R$. Then, $nm = 0$ means that one of $n$ or $m$ is 0, which implies that $n \in R$ or $m \in R$. 
\end{itemize}
\textbf{Fact:} $\{0\} \subseteq R$ is a prime ideal if and only if $R$ is an integral domain. 

\begin{definition}{Maximal Ideals}{}
    A \textbf{maximal ideal} of a commutative ring $R$ is a proper ideal of $R$ such that, when $B$ is an ideal of $R$ and $A \subseteq B \subseteq R$, then $B = A$ or $B = R$. 
    
    \bigskip

    Put it another way, a maximal ideal $I$ of a commutative ring $R$ is a proper ideal which is not contained in any other proper ideals, i.e. if $I \subseteq A \subseteq R$ for some ideal $A$, then $A = I$ or $A = R$. 
\end{definition}

\begin{theorem}{}{}
    Let $R$ be a commutative ring with unity and $I \subseteq R$ an ideal. Then, $R / I$ is an integral domain if and only if $I$ is prime. 
\end{theorem}

\begin{mdframed}[]
    \begin{proof}
        Supose $R / I$ is an integral domain. Suppose then that $a, b \in R$ with $ab \in I$. Then, $ab + I = 0 + I$. This further implies that $(a + I)(b + I) = 0 + I$. This implies that $a + I = 0 + I$ \emph{or} $b + I = 0 + I$ by integral domain definition. By the definition of a coset, $a \in I$ or $b \in I$. Thus, $I$ is prime. 
        
        \bigskip

        Suppose now that $I$ is prime. Suppose $a, b \in R$ with $(a + I)(b + I) = 0 + I$ with $ab + I = 0 + I$. This implies that $ab \in I$, which further means that $a \in I$ or $b \in I$ by prime. Thus, $a + I = 0 + I$ or $b + I = 0 + I$. Thus, $R / I$ is an integral domain.
    \end{proof}
\end{mdframed}

\begin{theorem}{}{}
    Let $R$ be a commutative ring with unity and $I \subseteq R$ an ideal. Then, $R / I$ is a field if and only if $I$ is maximal.
\end{theorem}

\begin{mdframed}[]
    \begin{proof}
        Suppose $R / I$ is a field. We want to show that if $I \subseteq A \subseteq R$, then $A = I$ or $A = R$.\footnote{We can prove the fact that $I \subseteq A \subseteq R$ and $A \neq I$ \emph{implies that} $A = R$.} Suppose $A \subseteq R$ is an ideal satisfying $I \subseteq A$ and $A \neq I$. The fact that $A \neq I$ implies that we can choose some $b \in A \setminus I$. This implies that $b + I \neq 0 + I$ and so $b + I \in R / I$ is a unit. This implies that there exists some $c + I \in R / I$ with $(b + I)(c + I) = 1 + I$, which further implies that $bc + I = 1 + I$. Thus, $\dots$. We know that $1 - bc \in A$, but $b \in A \setminus I \subseteq A$ so $bc \in A$ and thus $1 = (1 - bc) + bc \in A$. So, $R = R \cdot 1 \subseteq A$ so that $A = R$. Thus, $I$ is maximal. 

        \bigskip 

        Suppose that $I$ is maximal. We want to show that any $b + I \neq 0 + I$ is a unit in $R / I$. Choose some $b + I \in R / I$ with $b + I \neq 0 + I$, i.e. choose some $b \in R \setminus I$. Consider $B = \{rb + a \mid r \in R, a \in I\}$. Thus, $B = R$ by $I \subseteq B \subseteq R$ and $b \neq I$ ($b \in B, b \in I$).\footnote{Exercise: Show that $B$ is an ideal with contains $I$} From there, $1 \in B$ which means that $1 = rb + a$ for some $r \in R$ and $a \in I$, which finally implies that $1 + I = (r + I)(b + I)$.
    \end{proof}
    % TODO finish this proof
\end{mdframed}

\begin{corollary}{}{}
    All maximal ideals are prime ideals.
\end{corollary}

\begin{mdframed}[]
    \begin{proof}
        Suppose $I \subseteq R$ is maximal.
        \begin{equation*}
            \begin{aligned}
                R / I& \text{ is a field.} \\ 
                    &\implies R / I \text{ is an integral domain.} \\ 
                    &\implies R / I \text{ is prime.}
            \end{aligned}
        \end{equation*}
        So, we are done.
    \end{proof}
\end{mdframed}

\textbf{Remark:} The converse is not true. Consider $\cyclic{x} \subseteq \Z[x]$. This is not maximal by $\cyclic{x} \subset \cyclic{2, x} \subset \Z[x]$. 
\[\Z[x] / \cyclic{x} \longleftrightarrow \Z\]
\[f(x) + \cyclic{x} \longleftrightarrow f(0)\]
\[f(x) + \cyclic{x} = h(x) + \cyclic{x} \iff f(x) - h(x) = g(x)x \text{ for some } g(x) \iff f(0) - h(0) = 0\]
Thus, this ideal $\cyclic{x}$ is prime.

\end{document}