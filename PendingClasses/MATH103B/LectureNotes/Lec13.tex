\documentclass[letterpaper]{article}
\usepackage[margin=1in]{geometry}
\usepackage[utf8]{inputenc}
\usepackage{textcomp}
\usepackage{amssymb}
\usepackage{natbib}
\usepackage{graphicx}
\usepackage{gensymb}
\usepackage{amsthm, amsmath, mathtools}
\usepackage[dvipsnames]{xcolor}
\usepackage{enumerate}
\usepackage{mdframed}
\usepackage[most]{tcolorbox}
\usepackage{csquotes}
% https://tex.stackexchange.com/questions/13506/how-to-continue-the-framed-text-box-on-multiple-pages

\tcbuselibrary{theorems}

\newcommand{\R}{\mathbb{R}}
\newcommand{\Z}{\mathbb{Z}}
\newcommand{\N}{\mathbb{N}}
\newcommand{\Q}{\mathbb{Q}}
\newcommand{\C}{\mathbb{C}}
\newcommand{\code}[1]{\texttt{#1}}
\newcommand{\mdiamond}{$\diamondsuit$}
\newcommand{\PowerSet}{\mathcal{P}}
\newcommand{\Mod}[1]{\ (\mathrm{mod}\ #1)}
\DeclareMathOperator{\lcm}{lcm}

%\newtheorem*{theorem}{Theorem}
%\newtheorem*{definition}{Definition}
%\newtheorem*{corollary}{Corollary}
%\newtheorem*{lemma}{Lemma}
\newtheorem*{proposition}{Proposition}


\newtcbtheorem[number within=section]{theorem}{Theorem}
{colback=green!5,colframe=green!35!black,fonttitle=\bfseries}{th}

\newtcbtheorem[number within=section]{definition}{Definition}
{colback=blue!5,colframe=blue!35!black,fonttitle=\bfseries}{def}

\newtcbtheorem[number within=section]{corollary}{Corollary}
{colback=yellow!5,colframe=yellow!35!black,fonttitle=\bfseries}{cor}

\newtcbtheorem[number within=section]{lemma}{Lemma}
{colback=red!5,colframe=red!35!black,fonttitle=\bfseries}{lem}

\newtcbtheorem[number within=section]{example}{Example}
{colback=white!5,colframe=white!35!black,fonttitle=\bfseries}{def}

\newtcbtheorem[number within=section]{note}{Important Note}{
        enhanced,
        sharp corners,
        attach boxed title to top left={
            xshift=-1mm,
            yshift=-5mm,
            yshifttext=-1mm
        },
        top=1.5em,
        colback=white,
        colframe=black,
        fonttitle=\bfseries,
        boxed title style={
            sharp corners,
            size=small,
            colback=red!75!black,
            colframe=red!75!black,
        } 
    }{impnote}
\usepackage[utf8]{inputenc}
\usepackage[english]{babel}
\usepackage{fancyhdr}
\usepackage[hidelinks]{hyperref}

\pagestyle{fancy}
\fancyhf{}
\rhead{Math 103B}
\chead{Wednesday, February 2, 2022}
\lhead{Lecture 13}
\rfoot{\thepage}

\setlength{\parindent}{0pt}

\begin{document}

\section{Midterm Review}
Today is simply midterm review. 

\subsection{Page 257, Problem 29}
\begin{mdframed}[]
    List the distinct elements of $\Z[x] / \cyclic{3, x^2 + 1}$.
\end{mdframed}

\begin{proof}
    First, note that 
    \[\cyclic{3, x^2 + 1} = \{3 f_{1}(x) + (x^2 + 1)f_{2}(x) \mid f_{i}(x) \in \Z[x]\}\]
    Now, the elements of this ring is in the form 
    \[f(x) + \cyclic{3, x^2 + 1}\]
    An observation to see is that $f(x)$ must have degree less than 2. This is because 
    \[x^2 + 1 \in \cyclic{x^2 + 1} \implies x^2 + \cyclic{x^2 + 1} = -1 + \cyclic{x^2 + 1}\]
    In other words, the coset corresponding to $x^2$ is the same as the coset corresponding to $-1$. For example, if we had $x^3$, this would correspond to $x^2 x = (-1)x$ and $x^4$ would correspond to $x^2 x^2 = (-1)(-1) = 1$ and so on. So, we have 
    \[\Z[x] / \cyclic{3, x^2 + 1} = \left\{a + bx + \cyclic{3, x^2 + 1}\right\}\]
    Of course, we can't forget the 3 is a member of our ideal. So, 
    \[a = 3q_1 + r_1\]
    \[b = 3q_2 + r_2\]
    where $r_1, r_2 \in \{0, 1, 2\}$ so that 
    \[a + bx = 3q_1 + r_1 + (3q_2 + r_2)x = 3(q_1 + q_2 x) + r_1 + r_2 x\]
    and so it follows that 
    \[a + bx + \cyclic{3, x^2 + 1} = r_1 + r_2 x + \cyclic{3, x^2 + 1}\]
    Therefore, our answer is 
    \[\Z[x] / \cyclic{3, x^2 + 1} = \{r_1 + r_2 + \cyclic{3, x^2 + 1} \mid r_1, r_2 \in \Z / 3\Z\}\]
\end{proof}

\end{document}