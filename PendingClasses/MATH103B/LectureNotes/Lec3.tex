\documentclass[letterpaper]{article}
\usepackage[margin=1in]{geometry}
\usepackage[utf8]{inputenc}
\usepackage{textcomp}
\usepackage{amssymb}
\usepackage{natbib}
\usepackage{graphicx}
\usepackage{gensymb}
\usepackage{amsthm, amsmath, mathtools}
\usepackage{xcolor}
\usepackage{enumerate}
\usepackage{framed}
\usepackage{tcolorbox}
\tcbuselibrary{theorems}

\newcommand{\R}{\mathbb{R}}
\newcommand{\Z}{\mathbb{Z}}
\newcommand{\N}{\mathbb{N}}
\newcommand{\Q}{\mathbb{Q}}
\newcommand{\code}[1]{\texttt{#1}}
\newcommand{\mdiamond}{$\diamondsuit$}

%\newtheorem*{theorem}{Theorem}
%\newtheorem*{definition}{Definition}
\newtheorem*{proposition}{Proposition}
%\newtheorem*{corollary}{Corollary}
%\newtheorem*{lemma}{Lemma}

\newtcbtheorem[number within=section]{theorem}{Theorem}
{colback=green!5,colframe=green!35!black,fonttitle=\bfseries}{def}

\newtcbtheorem[number within=section]{definition}{Definition}
{colback=blue!5,colframe=blue!35!black,fonttitle=\bfseries}{def}

\newtcbtheorem[number within=section]{corollary}{Corollary}
{colback=yellow!5,colframe=yellow!35!black,fonttitle=\bfseries}{def}

\newtcbtheorem[number within=section]{lemma}{Lemma}
{colback=red!5,colframe=red!35!black,fonttitle=\bfseries}{def}
\usepackage[utf8]{inputenc}
\usepackage[english]{babel}
\usepackage{fancyhdr}
\usepackage[hidelinks]{hyperref}

\pagestyle{fancy}
\fancyhf{}
\rhead{Math 103B}
\chead{Friday, January 07, 2022}
\lhead{Lecture 3}
\rfoot{\thepage}

\setlength{\parindent}{0pt}

\begin{document}

\section{Integral Domains}
\begin{definition}{Zero-Divisors}{}
    A \textbf{zero-divisor} is a nonzero element $a$ of a commutative ring $R$ such that there is a nonzero element $b \in R$ with $ab = 0$. 
\end{definition}

\begin{definition}{Integral Domain}{}
    An \textbf{integral domain} is a commutative ring with unity and no zero-divisors.
\end{definition}
\textbf{Remarks:}
\begin{itemize}
    \item Recall that a ring $R$ has \textbf{unity} if $1 \in R$ is a multiplicative identity; that is, $1a = a1 = a$. 
    \item Essentially, in an integral domain, a product is 0 only when one of the facts is 0. That is, $ab = 0$ only when $a = 0$ or $b = 0$. 
\end{itemize}

\subsection{Examples}
Here are some examples of integral domains. 

\subsubsection{Example 1: The Integers}
The ring of integers is an integral domain. 

\subsubsection{Example 2: Gaussian Integers}
The ring of Gaussian integers $Z[i] = \{a + bi \mid a, b \in \Z\}$ is an integral domain. 

\subsubsection{Example 3: Ring of Polynomials}
The ring $\Z[x]$ of polynomials with integer coefficients is an integral domain. 

\subsubsection{Example 4: Square Root 2}
The ring $\Z[\sqrt{2}] = \{a + b\sqrt{2} \mid a, b \in \Z\}$ is an integral domain. 

\subsubsection{Example 5: Modulo Prime Integers}
The ring $\Z / p\Z$ of integers modulo $a$ prime $p$ is an integral domain. 

\subsubsection{Non-Example 1: Modulo Integers}
The ring $\Z / n\Z$ of integers modulo $n$ is not an integral domain when $n$ is not prime.

\subsubsection{Non-Example 2: Matrices}
The ring $M_{2}(\Z)$ of $2 \times 2$ matrices over the integers is not an integral domain. 

\subsubsection{Non-Example 3: Direct Product}
$\Z \oplus \Z$ is not an integral domain. 


\subsection{Properties of Integral Domains}
\begin{theorem}{Cancellation}{}
    Let $a$, $b$, and $c$ belong to an integral domain. If $a \neq 0$ and $ab = ac$, then: 
    \[b = c\]
\end{theorem}
\begin{mdframed}[]
    \begin{proof}
        From $ab = ac$, we know that $a(b - c) = 0$. Since $a \neq 0$, it follows that $b - c = 0$.
    \end{proof}
\end{mdframed}

\section{Fields}

\begin{definition}{Field}{}
    A \textbf{field} is a commutative ring with unity in which every nonzero element is a unit.
\end{definition}
\textbf{Remark:} To verify that every field is an integral domain, observe that if $a$ and $b$ belong to a field with $a \neq 0$ and $ab = 0$, we can multiply both sides of the last expression by $a^{-1}$ to obtain $b = 0$.

\begin{theorem}{}{}
    A finite integral domain is a field.
\end{theorem}

\begin{corollary}{}{}
    For every prime $p$, $\Z / p \Z$, the ring of integers modulo $p$ is a field.
\end{corollary}



\end{document}