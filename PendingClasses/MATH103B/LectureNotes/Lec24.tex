\documentclass[letterpaper]{article}
\usepackage[margin=1in]{geometry}
\usepackage[utf8]{inputenc}
\usepackage{textcomp}
\usepackage{amssymb}
\usepackage{natbib}
\usepackage{graphicx}
\usepackage{gensymb}
\usepackage{amsthm, amsmath, mathtools}
\usepackage[dvipsnames]{xcolor}
\usepackage{enumerate}
\usepackage{mdframed}
\usepackage[most]{tcolorbox}
\usepackage{csquotes}
% https://tex.stackexchange.com/questions/13506/how-to-continue-the-framed-text-box-on-multiple-pages

\tcbuselibrary{theorems}

\newcommand{\R}{\mathbb{R}}
\newcommand{\Z}{\mathbb{Z}}
\newcommand{\N}{\mathbb{N}}
\newcommand{\Q}{\mathbb{Q}}
\newcommand{\C}{\mathbb{C}}
\newcommand{\code}[1]{\texttt{#1}}
\newcommand{\mdiamond}{$\diamondsuit$}
\newcommand{\PowerSet}{\mathcal{P}}
\newcommand{\Mod}[1]{\ (\mathrm{mod}\ #1)}
\DeclareMathOperator{\lcm}{lcm}

%\newtheorem*{theorem}{Theorem}
%\newtheorem*{definition}{Definition}
%\newtheorem*{corollary}{Corollary}
%\newtheorem*{lemma}{Lemma}
\newtheorem*{proposition}{Proposition}


\newtcbtheorem[number within=section]{theorem}{Theorem}
{colback=green!5,colframe=green!35!black,fonttitle=\bfseries}{th}

\newtcbtheorem[number within=section]{definition}{Definition}
{colback=blue!5,colframe=blue!35!black,fonttitle=\bfseries}{def}

\newtcbtheorem[number within=section]{corollary}{Corollary}
{colback=yellow!5,colframe=yellow!35!black,fonttitle=\bfseries}{cor}

\newtcbtheorem[number within=section]{lemma}{Lemma}
{colback=red!5,colframe=red!35!black,fonttitle=\bfseries}{lem}

\newtcbtheorem[number within=section]{example}{Example}
{colback=white!5,colframe=white!35!black,fonttitle=\bfseries}{def}

\newtcbtheorem[number within=section]{note}{Important Note}{
        enhanced,
        sharp corners,
        attach boxed title to top left={
            xshift=-1mm,
            yshift=-5mm,
            yshifttext=-1mm
        },
        top=1.5em,
        colback=white,
        colframe=black,
        fonttitle=\bfseries,
        boxed title style={
            sharp corners,
            size=small,
            colback=red!75!black,
            colframe=red!75!black,
        } 
    }{impnote}
\usepackage[utf8]{inputenc}
\usepackage[english]{babel}
\usepackage{fancyhdr}
\usepackage[hidelinks]{hyperref}

\pagestyle{fancy}
\fancyhf{}
\rhead{Math 103B}
\chead{Monday, March 7, 2022}
\lhead{Lecture 24}
\rfoot{\thepage}

\setlength{\parindent}{0pt}

\begin{document}

\section{Extension Fields}
We continue our discussion on extension fields. 

\subsection{Formal Derivative}
\begin{definition}{Formal Derivative}{}
    The \textbf{formal derivative} of $f(x) = a_n x^n + \dots + a_1 x + a_0 \in F[x]$ is $f'(x) = a_n n x^{n - 1} + \dots + a_1 \in F[x]$. 
\end{definition}

\begin{lemma}{Properties of the Derivative}{}
    Let $f(x), g(x) \in F[x]$ and $a \in F$, where $F$ is a field. Then: 
    \begin{enumerate}
        \item $(f(x) + g(x))' = f'(x) + g'(x)$.
        \item $(af(x))' = af'(x)$.
        \item $(f(x)g(x))' = f(x)g'(x) + g(x)f'(x)$. 
    \end{enumerate}
\end{lemma}

\subsubsection{Example 1: Derivative}
The derivative of $x^3 + x^2 + 1$ is $3x^2 + 2x$. 

\subsection{Criterion for Multiple Zeros}
\begin{theorem}{}{}
    $f(x) \in F[x]$ has a multiple zero in an extension $E / F$ if and only if $f(x)$ and $f'(x)$ have a common factor in $F[x]$. 
\end{theorem}

\subsubsection{Example 1: Multiple Roots}
Consider $f(x) = x^2 + 2x + 1 = (x + 1)^2$. Then, $f'(x) = 2x + 2 = 2(x + 1)$. Here, $x + 1$ is a common factor. 

\subsubsection{Example 2: Multiple Roots}
Consider $g(x) = x^4 + 2x^2 + 1 = (x^2 + 1)^2$. Then, $g'(x) = 4x^3 + 4x = 4x(x^2 + 1)$. Here, $x^2 + 1$ is a common factor. 

\subsubsection{Example 3: Multiple Roots}
Consider $f(x) = x^5 + x^3 + x^2 + 1 \in \F_{3}[x]$. We note that 
\[f'(x) = 5x^4 + 3x^2 + 2x \equiv 2x^4 + 2x \in \F_{3}[x]\]
We now want to see if both polynomials have a common factor. We start with $f'(x)$. 
\[f'(x) = 2x^4 + 2x = 2x(x^3 + 1)\]
We see that $x = 2$ is a root. So: 
\[2x(x^3 + 1) = 2x(x + 1)(x^2 + 2x + 1)\]
We note that $x^2 + 2x + 1$ is reducible, so: 
\[2x(x + 1)(x^2 + 2x + 1) = 2x(x + 1)^3\]
Now, if $f(x)$ and $f'(x)$ have a common factor $p(x)$, then either $x | p(x)$ or $x + 1 | p(x)$. Because $p(x)$ is a factor of $f(x)$, this implies that $x | f(x)$ or $x + 1 | f(x)$. Note that: 
\begin{itemize}
    \item $x | f(x) \iff f(0) = 0$. 
    \item $x + 1 | f(x) \iff f(2) = 0$ since $x + 1$ is the same thing as $x - 2$. Here, we see that $f(2) = 0$, so $f(x)$ have multiple zeros.
\end{itemize}

\subsection{Zeros of an Irreducible}
\begin{theorem}{}{}
    Let $f(x) \in F[x]$ be irreducible. If $F$ has characteristic 0, then $f(x)$ has no multiple roots. If $F$ has characteristic $p$, then $f(x)$ has multiple roots if and only if $f(x) = g(x^p)$ for some $g(x)$ in $F[x]$. 
\end{theorem}

\subsubsection{Example 1: Irreducible Polynomials}
Consider $x^6 + x^2 + 1 \in \F_{2}[x]$. Knowing that $\text{char } \F_{2} = 2$, we have  
\[(x^2)^3 + (x^2)^1 + 1 = (x^3)^2 + (x^1)^2 + 1^2 = (x^3 + x + 1)^2\]

\end{document}