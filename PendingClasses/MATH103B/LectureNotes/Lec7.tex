\documentclass[letterpaper]{article}
\usepackage[margin=1in]{geometry}
\usepackage[utf8]{inputenc}
\usepackage{textcomp}
\usepackage{amssymb}
\usepackage{natbib}
\usepackage{graphicx}
\usepackage{gensymb}
\usepackage{amsthm, amsmath, mathtools}
\usepackage[dvipsnames]{xcolor}
\usepackage{enumerate}
\usepackage{mdframed}
\usepackage[most]{tcolorbox}
\usepackage{csquotes}
% https://tex.stackexchange.com/questions/13506/how-to-continue-the-framed-text-box-on-multiple-pages

\tcbuselibrary{theorems}

\newcommand{\R}{\mathbb{R}}
\newcommand{\Z}{\mathbb{Z}}
\newcommand{\N}{\mathbb{N}}
\newcommand{\Q}{\mathbb{Q}}
\newcommand{\C}{\mathbb{C}}
\newcommand{\code}[1]{\texttt{#1}}
\newcommand{\mdiamond}{$\diamondsuit$}
\newcommand{\PowerSet}{\mathcal{P}}
\newcommand{\Mod}[1]{\ (\mathrm{mod}\ #1)}
\DeclareMathOperator{\lcm}{lcm}

%\newtheorem*{theorem}{Theorem}
%\newtheorem*{definition}{Definition}
%\newtheorem*{corollary}{Corollary}
%\newtheorem*{lemma}{Lemma}
\newtheorem*{proposition}{Proposition}


\newtcbtheorem[number within=section]{theorem}{Theorem}
{colback=green!5,colframe=green!35!black,fonttitle=\bfseries}{th}

\newtcbtheorem[number within=section]{definition}{Definition}
{colback=blue!5,colframe=blue!35!black,fonttitle=\bfseries}{def}

\newtcbtheorem[number within=section]{corollary}{Corollary}
{colback=yellow!5,colframe=yellow!35!black,fonttitle=\bfseries}{cor}

\newtcbtheorem[number within=section]{lemma}{Lemma}
{colback=red!5,colframe=red!35!black,fonttitle=\bfseries}{lem}

\newtcbtheorem[number within=section]{example}{Example}
{colback=white!5,colframe=white!35!black,fonttitle=\bfseries}{def}

\newtcbtheorem[number within=section]{note}{Important Note}{
        enhanced,
        sharp corners,
        attach boxed title to top left={
            xshift=-1mm,
            yshift=-5mm,
            yshifttext=-1mm
        },
        top=1.5em,
        colback=white,
        colframe=black,
        fonttitle=\bfseries,
        boxed title style={
            sharp corners,
            size=small,
            colback=red!75!black,
            colframe=red!75!black,
        } 
    }{impnote}
\usepackage[utf8]{inputenc}
\usepackage[english]{babel}
\usepackage{fancyhdr}
\usepackage[hidelinks]{hyperref}

\pagestyle{fancy}
\fancyhf{}
\rhead{Math 103B}
\chead{Wednesday, January 19, 2022}
\lhead{Lecture 7}
\rfoot{\thepage}

\setlength{\parindent}{0pt}

\begin{document}

\section{Ring Homomorphism}
Ring homomorphism is very similar in nature to group homomorphisms. Here, a ring homomorphism preserves the ring operations.

\begin{definition}{Ring Homomorphism}{}
    A \textbf{ring homomorphism} $\varphi$ from a ring $R$ to a ring $S$ is a mapping from $R$ to $S$ that preserves the ring operation. That is, for all $a, b \in R$:
    \[\varphi(a + b) = \varphi(a) + \varphi(b) \qquad \varphi(ab) = \varphi(a)\varphi(b)\]
\end{definition}
\textbf{Remark:} As is the case for groups, the operations on the left of the equal signs are those of $R$, while the operations on the right side are those of $S$. 

\bigskip

Along with ring homomorphisms, there is also ring isomorphisms.
\begin{definition}{Ring Isomorphism}{}
    A \textbf{ring isomorphism} is a ring homomorphism that is both one-to-one and onto (i.e. bijective).
\end{definition}

\subsection{Properties of Ring Homomorphisms}
\begin{theorem}{}{}
    Let $\varphi$ be a ring homomorphism from a ring $R$ to a ring $S$, and let $A$ be a subring of $R$ and let $B$ be an ideal of $S$.
    \begin{enumerate}
        \item For any $r \in R$ and any positive integer $n$, $\varphi(nr) = n\varphi(r)$ and $\varphi(r^n) = (\varphi(r))^n$.
        \item $\varphi(A) = \{\varphi(a) \mid a \in A\}$ is a subring of $S$. 
        \item If $A$ is an ideal and $\varphi$ is onto $S$, then $\varphi(A)$ is an ideal. 
        \item $\varphi^{-1}(B) = \{r \in R \mid \varphi(r) \in B\}$ is an ideal of $R$. 
        \item If $R$ is commutative, then $\varphi(R)$ is commutative.
        \item If $R$ has a unity 1, $S \neq \{0\}$, and $\varphi$ is onto, then $\varphi(1)$ is the unity of $S$. 
        \item $\varphi$ is an isomorphism if and only if $\varphi$ is onto and $\ker(\varphi) = \{r \in R \mid \varphi(r) = 0\} = \{0\}$. 
        \item If $\varphi$ is an isomorphism from $R$ onto $S$, then $\varphi^{-1}$ is an isomorphism from $S$ onto $R$. 
    \end{enumerate}
\end{theorem}

\subsection{Examples of Ring Homomorphism}
Here are some examples of ring homomorphisms.

\subsubsection{Example 1: Integers and Modulo}
Consider the mapping: 
\[k \mapsto k \Mod{n}\]
This is a ring homomorphism from $\Z$ onto $\Z_{n}$, and is called the natural homomorphism from $\Z$ to $\Z_{n}$. 

\subsubsection{Example 2: Complex Numbers}
Consider the mapping: 
\[a + bi \mapsto a - bi\]
This is a ring homomorphism from the complex numbers onto the complex numbers. 

\subsubsection{Example 3: Functions}
Consider the ring of all polynomials with real coefficients $\R[x]$. Consider the mapping:
\[f(x) \mapsto f(1)\]
This is a ring homomorphism from $\R[x]$ onto $\R$. 

\end{document}