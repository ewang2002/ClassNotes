\documentclass[letterpaper]{article}
\usepackage[margin=1in]{geometry}
\usepackage[utf8]{inputenc}
\usepackage{textcomp}
\usepackage{amssymb}
\usepackage{natbib}
\usepackage{graphicx}
\usepackage{gensymb}
\usepackage{amsthm, amsmath, mathtools}
\usepackage{xcolor}
\usepackage{enumerate}
\usepackage{framed}
\usepackage{tcolorbox}
\tcbuselibrary{theorems}

\newcommand{\R}{\mathbb{R}}
\newcommand{\Z}{\mathbb{Z}}
\newcommand{\N}{\mathbb{N}}
\newcommand{\Q}{\mathbb{Q}}
\newcommand{\code}[1]{\texttt{#1}}
\newcommand{\mdiamond}{$\diamondsuit$}

%\newtheorem*{theorem}{Theorem}
%\newtheorem*{definition}{Definition}
\newtheorem*{proposition}{Proposition}
%\newtheorem*{corollary}{Corollary}
%\newtheorem*{lemma}{Lemma}

\newtcbtheorem[number within=section]{theorem}{Theorem}
{colback=green!5,colframe=green!35!black,fonttitle=\bfseries}{def}

\newtcbtheorem[number within=section]{definition}{Definition}
{colback=blue!5,colframe=blue!35!black,fonttitle=\bfseries}{def}

\newtcbtheorem[number within=section]{corollary}{Corollary}
{colback=yellow!5,colframe=yellow!35!black,fonttitle=\bfseries}{def}

\newtcbtheorem[number within=section]{lemma}{Lemma}
{colback=red!5,colframe=red!35!black,fonttitle=\bfseries}{def}
\usepackage[utf8]{inputenc}
\usepackage[english]{babel}
\usepackage{fancyhdr}
\usepackage[hidelinks]{hyperref}

\pagestyle{fancy}
\fancyhf{}
\rhead{Math 103B}
\chead{Wednesday, January 05, 2022}
\lhead{Lecture 2}
\rfoot{\thepage}

\setlength{\parindent}{0pt}

\begin{document}

\section{Properties of Rings}
We begin by talking about a few important properties. 

\subsection{Basic Rules of Multiplication}
\begin{theorem}{}{}
    For all $a \in R$, we have: 
    \[a0 = 0a = 0\]
\end{theorem}

\begin{mdframed}[]
    \begin{proof}
        We know that: 
        \[0a = (0 + 0)a = 0a + 0a\]
        Subtracting both sides by $0a$ gives: 
        \[0 = 0a + (0a - 0a) \implies 0 = 0a\]
        By symmetry, we can do the same for $0a$. Therefore, we are done.
    \end{proof}    
\end{mdframed}

\begin{theorem}{}{}
    For all $a, b \in R$, we have:
    \[a(-b) = (-a)b = -(ab)\]
\end{theorem}

\begin{mdframed}[]
    \begin{proof}
        First, we have: 
        \[a(-b) + ab = a(-b + b) = a0 = 0\]
        Now, if we add $-(ab)$ to both sides, we have: 
        \[a(-b) + ab + -(ab) = -(ab) \implies a(-b) = -(ab)\]
        By symmetry, $(-a)b = -(ab)$ as well. 
    \end{proof}
\end{mdframed}

\begin{theorem}{}{}
    For all $a, b \in R$, we have: 
    \[(-a)(-b) = ab\]
\end{theorem}

\begin{mdframed}[]
    \begin{proof}
        \begin{equation*}
            \begin{aligned}
                (-a)0 &= 0 \\ 
                    &\iff (-a)(b + (-b)) = 0 \\ 
                    &\iff (-a)b + -a(-b) = 0 \\ 
                    &\iff -(ab) + -a(-b) = 0 \\ 
                    &\iff ab + -(ab) + -a(-b) = ab \\
                    &\iff -a(-b) = ab
            \end{aligned}
        \end{equation*}
        So, we are done. 
    \end{proof}
\end{mdframed}

\begin{theorem}{}{}
    For all $a, b, c \in R$, we have: 
    \[a(b - c) = ab - ac \text{ and } (b - c)a = ba - ca\]
\end{theorem}

\begin{mdframed}[]
    \begin{proof}
        \begin{equation*}
            \begin{aligned}
                a(b - c) &= ab + -(ac) \\ 
                    &= ab + (-a)c \\ 
                    &= ab - ac
            \end{aligned}
        \end{equation*}
        By symmetry, we can apply the other side as well. So, we are done. 
    \end{proof}
\end{mdframed}

\subsection{Rules of Multiplication with Unity Element}
\begin{theorem}{}{}
    For all $a \in R$ where $R$ has a unity element 1, we have: 
    \[(-1)a = -a\]
\end{theorem}

\begin{mdframed}[]
    \begin{proof}
        Applying the theorem that we proved:
        \[(-1)a = -(1a) = -a\]
        So, we are done. 
    \end{proof}
\end{mdframed}

\begin{theorem}{}{}
    \[(-1)(-1) = 1\]
\end{theorem}

\begin{mdframed}[]
    \begin{proof}
        Applying the theorem that we proved:
        \[(-1)(-1) = 1(1) = 1\]
        So, we are done.
    \end{proof}
\end{mdframed}

\subsection{Uniqueness of Unity and Inverses}
\begin{theorem}{}{}
    If a ring has a unity, it is unique. If a ring element has a multiplicative inverse, it is also unique. 
\end{theorem}

\begin{mdframed}[]
    \begin{proof}
        We will prove both parts individually. Suppose $R$ is a ring.  
        \begin{enumerate}
            \item Suppose $e$ and $e'$ are unity elements in a ring $R$. Then, we know that: 
            \begin{itemize}
                \item $e = ee'$ since $e'$ is a unity. 
                \item $e' = ee'$ since $e$ is a unity. 
            \end{itemize}
            Therefore: 
            \[e = ee' = e'\]
            Which means that the unity must be unique. 

            \item Suppose $a \in R$ and further suppose that $x$ and $y$ are both multiplicative inverses of $a$. Then: 
            \[x = x1 = x(ay) = (xa)y = 1y = y\]
            Therefore, $x = y$ and the two inverses are equal. 
        \end{enumerate}
        Therefore, we are done. 
    \end{proof}
\end{mdframed}


\newpage 
\section{Subring}
Recall that, with groups, we have objects called \emph{subgroups}. The same thing applies here: with rings, we have objects called \emph{subrings}.
\begin{definition}{Subring}{}
    A nonempty subset $S$ of a ring $R$ is a \textbf{subring} of $R$ if $S$ itself is a ring with the operations of $R$.
\end{definition}

\subsection{Subring Test}
\begin{theorem}{Subring Test}{}
    A nonempty subset $S$ of a ring $R$ is a subring if $S$ is closed under subtraction and multiplication; that is, if $a - b \in S$ and $ab \in S$ whenever $a, b \in S$. 
\end{theorem}

\subsection{Examples of Subrings}
Below are some examples of subrings. 

\subsubsection{Example 1: Trivial Subring}
The trivial subring $\{0\}$ is a subring of any ring $R$. This is because:
\[0(0) \in R \qquad 0 - 0 \in R\]

\subsubsection{Example 2: Ring}
Any ring $R$ is a subring of itself. This is because for any $a, b \in R$, we know that $a - b = a + (-b) \in R$ and $ab \in R$. 

\subsubsection{Example 3: Integers}
For any positive integer $n$, the set below is a subring of the integers $\Z$: 
\[n\Z = \{0, \pm n, \pm 2n, \pm 3n, \dots\}\]
Take any $a, b \in \Z$. Then, suppose we have $an$ and $bn$. We know that: 
\[an - bn = (a - b)n \in \Z\]
\[an(bn) = anbn\]
Since $anb \in \Z$, it follows that $(anb)n \in n\Z$. 

\end{document}