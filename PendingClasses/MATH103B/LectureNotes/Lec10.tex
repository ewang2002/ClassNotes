\documentclass[letterpaper]{article}
\usepackage[margin=1in]{geometry}
\usepackage[utf8]{inputenc}
\usepackage{textcomp}
\usepackage{amssymb}
\usepackage{natbib}
\usepackage{graphicx}
\usepackage{gensymb}
\usepackage{amsthm, amsmath, mathtools}
\usepackage[dvipsnames]{xcolor}
\usepackage{enumerate}
\usepackage{mdframed}
\usepackage[most]{tcolorbox}
\usepackage{csquotes}
% https://tex.stackexchange.com/questions/13506/how-to-continue-the-framed-text-box-on-multiple-pages

\tcbuselibrary{theorems}

\newcommand{\R}{\mathbb{R}}
\newcommand{\Z}{\mathbb{Z}}
\newcommand{\N}{\mathbb{N}}
\newcommand{\Q}{\mathbb{Q}}
\newcommand{\C}{\mathbb{C}}
\newcommand{\code}[1]{\texttt{#1}}
\newcommand{\mdiamond}{$\diamondsuit$}
\newcommand{\PowerSet}{\mathcal{P}}
\newcommand{\Mod}[1]{\ (\mathrm{mod}\ #1)}
\DeclareMathOperator{\lcm}{lcm}

%\newtheorem*{theorem}{Theorem}
%\newtheorem*{definition}{Definition}
%\newtheorem*{corollary}{Corollary}
%\newtheorem*{lemma}{Lemma}
\newtheorem*{proposition}{Proposition}


\newtcbtheorem[number within=section]{theorem}{Theorem}
{colback=green!5,colframe=green!35!black,fonttitle=\bfseries}{th}

\newtcbtheorem[number within=section]{definition}{Definition}
{colback=blue!5,colframe=blue!35!black,fonttitle=\bfseries}{def}

\newtcbtheorem[number within=section]{corollary}{Corollary}
{colback=yellow!5,colframe=yellow!35!black,fonttitle=\bfseries}{cor}

\newtcbtheorem[number within=section]{lemma}{Lemma}
{colback=red!5,colframe=red!35!black,fonttitle=\bfseries}{lem}

\newtcbtheorem[number within=section]{example}{Example}
{colback=white!5,colframe=white!35!black,fonttitle=\bfseries}{def}

\newtcbtheorem[number within=section]{note}{Important Note}{
        enhanced,
        sharp corners,
        attach boxed title to top left={
            xshift=-1mm,
            yshift=-5mm,
            yshifttext=-1mm
        },
        top=1.5em,
        colback=white,
        colframe=black,
        fonttitle=\bfseries,
        boxed title style={
            sharp corners,
            size=small,
            colback=red!75!black,
            colframe=red!75!black,
        } 
    }{impnote}
\usepackage[utf8]{inputenc}
\usepackage[english]{babel}
\usepackage{fancyhdr}
\usepackage[hidelinks]{hyperref}

\pagestyle{fancy}
\fancyhf{}
\rhead{Math 103B}
\chead{Wednesday, January 26, 2022}
\lhead{Lecture 10}
\rfoot{\thepage}

\setlength{\parindent}{0pt}

\begin{document}

\section{Polynomial Rings}

\begin{definition}{Polynomial Ring}{}
    Let $R$ be a commutative ring. The \textbf{polynomial ring} over $R$ in the indeterminate $x$ is defined by
    \[R[x] = \{a_0 + a_1 x + a_2 x^2 + \dots + a_n x^n \mid a_0, a_1, \dots, a_n \in R, n \in \Z_{\geq 0}\}\]
\end{definition}
\textbf{Remark:} We say that the set represented by $R[x]$ is a set of ``formal symbols.'' In other words, these are things we can write down, not functions. 

\begin{definition}{}{}
    We say that
    \[a_0 + a_1 x + a_2 x^2 + \dots + a_n x^n = b_0 + b_1 x + b_2 x^2 + \dots + b_m x^m\]
    are equal if $a_i = b_i$ for all $i = 0, 1, 2, \dots$. Here, we define $a_i = 0$ if $i > n$, and $b_i = 0$ if $i > m$. 
\end{definition}
Consider $\F_{2}[x]$. Here, polynomials determine functions. Consider $f(x) = x$ and $g(x) = x^2$. Then, this determines a function: 
\begin{center}
    \begin{tabular}{p{3in}|p{3in}}
        $f(x)$ & $g(x)$ \\ 
        \hline 
        $\varphi_{f}: \F_2 \mapsto \F_2$ defined by: 
        \begin{itemize}
            \item $\varphi_{f}(0) = 0$
            \item $\varphi_{f}(1) = 1$
        \end{itemize} & 
        $\varphi_{g}: \F_2 \mapsto \F_{2}$ defined by: 
        \begin{itemize}
            \item $\varphi_{g}(0) = 0^2 = 0$
            \item $\varphi_{g}(1) = 1^2 = 1$
        \end{itemize}
    \end{tabular}
\end{center}
Here, $f(x) \neq g(x)$ but they determine the same function $\F_{2} \mapsto \F_{2}$



\begin{definition}{}{}
    In $R[x]$, if
    \[f(x) = a_0 + a_1 x + \dots + a_n x^n\]
    \[g(x) = b_0 + b_1 x + \dots + b_m x^m\]
    then
    \[f(x) + g(x) = (a_0 + b_0) + (a_1 + b_1)x + \dots + (a_s + b_s) x^s\]
    for $s = \max\{n, m\}$. Additionally,
    \[f(x)g(x) = c_0 + c_1 x + \dots + c_{n + m} x^{n + m}\]
    where 
    \[c_0 = a_0 b_0\]
    \[c_1 = a_1 b_0 + a_0 b_1\]
    \[c_2 = a_2 b_0 + a_1 b_1 + a_0 b_2\]
    \[c_k = a_k b_0 + a_{k - 1}b_1 + \dots + a_1 b_{k - 1} + a_0 b_k\]
\end{definition}
\textbf{Remark:} In practice, we use distributive law for multiplication.

\begin{definition}{}{}
    When $f(x) = a_0 + a_1 x + \dots + a_n x^n$ with $a_n \neq 0$, we say that: 
    \begin{itemize}
        \item $f(x)$ has \textbf{degree} $n$, written $\deg f(x) = n$. However\footnote{If $f(x)$ is degree 0, then $f(x) = a_0 x^0 = a_0$ for $a_0 \neq 0$.}, if $f(x) = 0$, then we say that $f(x)$ has \emph{no} degree \emph{or} $f(x)$ has degree $-\infty$. 
        \item $a_n$ is the \textbf{leading coefficient} of $f(x)$. 
        \item $f(x) = a_0$ is a \textbf{constant} polynomial.
        \item If $a_n = 1$, we say that $f(x)$ is a \textbf{monic} polynomial.
    \end{itemize}
\end{definition}
\textbf{Remarks:}
\begin{itemize}
    \item We omit terms like $0x^k$. For example, if our polynomial is $1 + 0x + 1x^2$, then we write $1 + 1x^2$.
    \item We write $1x^k$ as just $x^k$. So, for example, we write $1 + 1x^2$ as $1 + x^2$. 
    \item We write $\dots + (-a_k) x^k + \dots$ as $\dots - a_k x^k + \dots$. For example, $1 + (-1)x^2 = 1 - x^2$.
\end{itemize}

\subsection{Properties of Polynomial Rings}

\begin{proposition}
    Let $R$ be a commutative ring and $r \in R$. Then, the evaluation map 
    \[\varphi_{r}: R[X] \mapsto R\]
    \[f(x) \mapsto f(r) = a_0 + a_1 r + a_2 r^2 + \dots + a_n r^n\]
    is a homomorphism.
\end{proposition}

\begin{mdframed}[]
    \begin{proof}
        The proof is straightforward. 
        \begin{itemize}
            \item \underline{Addition:}
            \begin{equation*}
                \begin{aligned}
                    \varphi_{r}(f(x) + g(x)) &= \varphi_{r}((a_0 + b_0) + (a_1 + b_1)x + \dots + (a_s + b_s)x^s) \\ 
                        &= (a_0 + b_0) + (a_1 + b_1) r + \dots + (a_s + b_s) r^s \\ 
                        &= a_0 + a_1 r + \dots + a_n r^n + b_0 + b_1 r + \dots + b_m r^m \\ 
                        &= \varphi_{r}(f(x)) + \varphi_{r}(g(x))
                \end{aligned}
            \end{equation*}

            \item \underline{Multiplication:}
            \begin{equation*}
                \begin{aligned}
                    \varphi_{r}(f(x) g(x)) &= \varphi_{r}(c_0 + c_1 x + \dots + c_{n + m} x^{n + m}) \\ 
                        &= c_0 + c_1 r + \dots + c_{n + m} r^{n + m} \\ 
                        &= (a_0 + a_1 r + \dots + a_n r^n)(b_0 + b_1 r + \dots + b_m r^m) \\ 
                        &= \varphi_{r}(f(x)) \varphi_{r}(g(x))
                \end{aligned}
            \end{equation*}
        \end{itemize}
        This proves that this is a homomorphism. 
    \end{proof}
\end{mdframed}

\textbf{Remark:} This is not an injective homomorphism, but it \textbf{is} a surjective homomorphism.

\begin{theorem}{}{}
    If $D$ is an integral domain, then $D[x]$ is an integral domain. 
\end{theorem}

\begin{mdframed}[]
    \begin{proof}
        $D[x]$ is commutative by definition. We know that $1 \in D$ so $f(x) = 1$ is the unity of $D[x]$. Suppose $f(x), g(x) \in D[x] \setminus \{0\}$ so that 
        \[f(x) = a_0 + a_1 x + \dots + a_n x^n\]
        \[g(x) = b_0 + b_1 x + \dots + b_m x^m\]
        for $a_n \neq 0$ and $b_m \neq 0$. Then,
        \[f(x) g(x) = a_0 b_0 + (a_1 b_0 + a_1 b_0) x + \dots + a_n b_m x^{n + m}\]
        but, $a_n \neq 0$ and $b_m \neq 0$ which implies that $a_n b_m \neq 0$ by $D$ being an integral domain. Thus, $f(x) g(x) \neq 0$, so there are no zero-divisors and $D[x]$ is an integral domain. 
    \end{proof}
\end{mdframed}

\end{document}