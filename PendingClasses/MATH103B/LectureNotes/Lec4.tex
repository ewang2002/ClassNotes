\documentclass[letterpaper]{article}
\usepackage[margin=1in]{geometry}
\usepackage[utf8]{inputenc}
\usepackage{textcomp}
\usepackage{amssymb}
\usepackage{natbib}
\usepackage{graphicx}
\usepackage{gensymb}
\usepackage{amsthm, amsmath, mathtools}
\usepackage[dvipsnames]{xcolor}
\usepackage{enumerate}
\usepackage{mdframed}
\usepackage[most]{tcolorbox}
\usepackage{csquotes}
% https://tex.stackexchange.com/questions/13506/how-to-continue-the-framed-text-box-on-multiple-pages

\tcbuselibrary{theorems}

\newcommand{\R}{\mathbb{R}}
\newcommand{\Z}{\mathbb{Z}}
\newcommand{\N}{\mathbb{N}}
\newcommand{\Q}{\mathbb{Q}}
\newcommand{\C}{\mathbb{C}}
\newcommand{\code}[1]{\texttt{#1}}
\newcommand{\mdiamond}{$\diamondsuit$}
\newcommand{\PowerSet}{\mathcal{P}}
\newcommand{\Mod}[1]{\ (\mathrm{mod}\ #1)}
\DeclareMathOperator{\lcm}{lcm}

%\newtheorem*{theorem}{Theorem}
%\newtheorem*{definition}{Definition}
%\newtheorem*{corollary}{Corollary}
%\newtheorem*{lemma}{Lemma}
\newtheorem*{proposition}{Proposition}


\newtcbtheorem[number within=section]{theorem}{Theorem}
{colback=green!5,colframe=green!35!black,fonttitle=\bfseries}{th}

\newtcbtheorem[number within=section]{definition}{Definition}
{colback=blue!5,colframe=blue!35!black,fonttitle=\bfseries}{def}

\newtcbtheorem[number within=section]{corollary}{Corollary}
{colback=yellow!5,colframe=yellow!35!black,fonttitle=\bfseries}{cor}

\newtcbtheorem[number within=section]{lemma}{Lemma}
{colback=red!5,colframe=red!35!black,fonttitle=\bfseries}{lem}

\newtcbtheorem[number within=section]{example}{Example}
{colback=white!5,colframe=white!35!black,fonttitle=\bfseries}{def}

\newtcbtheorem[number within=section]{note}{Important Note}{
        enhanced,
        sharp corners,
        attach boxed title to top left={
            xshift=-1mm,
            yshift=-5mm,
            yshifttext=-1mm
        },
        top=1.5em,
        colback=white,
        colframe=black,
        fonttitle=\bfseries,
        boxed title style={
            sharp corners,
            size=small,
            colback=red!75!black,
            colframe=red!75!black,
        } 
    }{impnote}
\usepackage[utf8]{inputenc}
\usepackage[english]{babel}
\usepackage{fancyhdr}
\usepackage[hidelinks]{hyperref}

\pagestyle{fancy}
\fancyhf{}
\rhead{Math 103B}
\chead{Monday, January 10, 2022}
\lhead{Lecture 4}
\rfoot{\thepage}

\setlength{\parindent}{0pt}

\begin{document}

\section{Characteristic of a Ring}
Consider the ring $\Z_{3}[i]$, with the elements:
\[\{0, 1, 2, i, 1 + i, 2 + i, 2i, 1 + 2i, 2 + 2i\}\]
For any element $x$ in this ring, we have: 
\[3x = x + x + x = 0\]
For example:
\begin{itemize}
    \item $2i + 2i + 2i = 6i = 0i = 0$
    \item $(1 + 2i) + (1 + 2i) + (1 + 2i) = 3 + 6i = 0$
    \item And so on.
\end{itemize}
Similarly, in the ring $\{0, 3, 6, 9\} \subset \Z_{12}$, we have, for all $x$: 
\[4x = x + x + x + x = 0\]

\subsection{Characteristic of a Ring}
\begin{definition}{Characteristic of a Ring}{}
    The \textbf{characteristic} of a ring $R$ is the least positive integer $n$ such that $nx = 0$ for all $x \in R$. If no such integer exists, we say that $R$ has characteristic 0. The characteristic of $R$ is denoted by $\ch{R}$.    
\end{definition}
So, for example, the ring of integers $\Z$ has characteristic 0 and $\Z_n$ has characteristic $n$. For example, consider $\Z_{3} = \{0, 1, 2\}$. Then, we know that:
\[3x = x + x + x = 0 \qquad \forall x\]
So the characteristic of $\Z_{3}$ is \boxed{3}. Now, consider $\Z_{6}$. We know that:
\[6x = x + x + x + x + x + x = 0 \qquad \forall x\]
So, its characteristic is \boxed{6}.

\subsection{Characteristic of a Ring with Unity}
\begin{theorem}{Characteristic of a Ring with Unity}{}
    Let $R$ be a ring with unity 1. If 1 has infinite order under addition, then the characteristic of $R$ is 0. If 1 has order $n$ under addition, then the characteristic of $R$ is $n$. 
\end{theorem}

\begin{mdframed}[]
    \begin{proof}
        Suppose 1 has infinite order. Then, there is no positive integer $n$ such that $n \cdot 1 = 0$, so $R$ must have characteristic 0. Now, let's suppose that 1 does have additive order $n$. Then, we know that $n \cdot 1 = 0$ and $n$ is the least positive integer with this property. So, for any $x \in R$, we have: 
        \begin{equation*}
            \begin{aligned}
                n \cdot x &= \overbrace{x + x + \dots + x}^{n \text{ times}} \\ 
                    &= \overbrace{1x + 1x + \dots + 1x}^{n \text{ times}} \\ 
                    &= (\overbrace{1 + 1 + \dots + 1}^{n \text{ times}})x \\ 
                    &= (n \cdot 1)x = 0x = 0
            \end{aligned}
        \end{equation*}
        So, $R$ has characteristic $n$.
    \end{proof}
\end{mdframed}

\begin{theorem}{Characteristic of an Integral Domain}{}
    The characteristic of an integral domain is 0 or prime.
\end{theorem}

\end{document}