\documentclass[letterpaper]{article}
\usepackage[margin=1in]{geometry}
\usepackage[utf8]{inputenc}
\usepackage{textcomp}
\usepackage{amssymb}
\usepackage{natbib}
\usepackage{graphicx}
\usepackage{gensymb}
\usepackage{amsthm, amsmath, mathtools}
\usepackage[dvipsnames]{xcolor}
\usepackage{enumerate}
\usepackage{mdframed}
\usepackage[most]{tcolorbox}
\usepackage{csquotes}
% https://tex.stackexchange.com/questions/13506/how-to-continue-the-framed-text-box-on-multiple-pages

\tcbuselibrary{theorems}

\newcommand{\R}{\mathbb{R}}
\newcommand{\Z}{\mathbb{Z}}
\newcommand{\N}{\mathbb{N}}
\newcommand{\Q}{\mathbb{Q}}
\newcommand{\C}{\mathbb{C}}
\newcommand{\code}[1]{\texttt{#1}}
\newcommand{\mdiamond}{$\diamondsuit$}
\newcommand{\PowerSet}{\mathcal{P}}
\newcommand{\Mod}[1]{\ (\mathrm{mod}\ #1)}
\DeclareMathOperator{\lcm}{lcm}

%\newtheorem*{theorem}{Theorem}
%\newtheorem*{definition}{Definition}
%\newtheorem*{corollary}{Corollary}
%\newtheorem*{lemma}{Lemma}
\newtheorem*{proposition}{Proposition}


\newtcbtheorem[number within=section]{theorem}{Theorem}
{colback=green!5,colframe=green!35!black,fonttitle=\bfseries}{th}

\newtcbtheorem[number within=section]{definition}{Definition}
{colback=blue!5,colframe=blue!35!black,fonttitle=\bfseries}{def}

\newtcbtheorem[number within=section]{corollary}{Corollary}
{colback=yellow!5,colframe=yellow!35!black,fonttitle=\bfseries}{cor}

\newtcbtheorem[number within=section]{lemma}{Lemma}
{colback=red!5,colframe=red!35!black,fonttitle=\bfseries}{lem}

\newtcbtheorem[number within=section]{example}{Example}
{colback=white!5,colframe=white!35!black,fonttitle=\bfseries}{def}

\newtcbtheorem[number within=section]{note}{Important Note}{
        enhanced,
        sharp corners,
        attach boxed title to top left={
            xshift=-1mm,
            yshift=-5mm,
            yshifttext=-1mm
        },
        top=1.5em,
        colback=white,
        colframe=black,
        fonttitle=\bfseries,
        boxed title style={
            sharp corners,
            size=small,
            colback=red!75!black,
            colframe=red!75!black,
        } 
    }{impnote}
\usepackage[utf8]{inputenc}
\usepackage[english]{babel}
\usepackage{fancyhdr}
\usepackage[hidelinks]{hyperref}

\pagestyle{fancy}
\fancyhf{}
\rhead{Math 103B}
\chead{Monday, February 7, 2022}
\lhead{Lecture 15}
\rfoot{\thepage}

\setlength{\parindent}{0pt}

\begin{document}

\section{More on Irreducible Polynomials}
We will continue our discussion on irreducible polynomials.

\subsection{Primitive Polynomials}
\begin{lemma}{Gauss's Lemma}{}
    The product of two primitive polynomials is primitive.
\end{lemma}

\begin{mdframed}[]
    \begin{proof}
        Let $f(x), g(x) \in \Z[x]$ be primitive, and suppose $f(x)g(x)$ is not primitive. Choose a prime $p$ which divides the content of $f(x)g(x)$. Then 
        \[f(x)g(x) \equiv 0 \Mod{p}\]
        Then, $f(x)g(x) \equiv 0 \Mod{p}$. However, $f(x), g(x) \not\equiv 0 \Mod{p}$. Let $\overline{f}(x), \overline{g}(x) \in \F_{p}[x]$ be the polynomials with $f(x) \equiv \overline{f}(x) \Mod{p}$ and $g(x) \equiv \overline{g}(x) \Mod{p}$. Then
        \[f(x)g(x) \equiv \overline{f}(x)\overline{g}(x) \Mod{p}\]
        which implies that $\overline{f}(x), \overline{g}(x)$ are zero-divisors in $\F_{p}[x]$. But, this is a contradiction as $\F_{p}[x]$ is a field and thus an integral domain. 
    \end{proof}
\end{mdframed}

\subsection{Reducibility over Rational Numbers Implies Reducibility Over Integers}
Recall, from earlier, the following theorem. 
\begin{theorem}{}{}
    Let $f(x) \in \Z[x]$. $f(x)$ is reducible over $\Q \implies f(x)$ is reducible over $\Z$.
\end{theorem}

\begin{mdframed}[]
    \begin{proof}
        Write $f(x) = g(x)h(x)$ for $g(x), h(x) \in \Q[x]$. Without loss of generality, suppose $f(x)$ is primitive; otherwise, let $c$ be the content of $f(x)$ and then let $\frac{1}{c}f(x) = \left(\frac{1}{c}g(x)\right)h(x)$. Let $a$ be the least common multiple of the denominator of coefficients of $g(x)$, and let $b$ be the least common multiple of the denominator of coefficients of $h(x)$. We can now clear denominators like so
        \[abf(x) = (ag(x))(bh(x))\]
        with $ag(x), bh(x) \in \Z[x]$. Let $c_1$ be the content of $ag(x)$ and $c_2$ be the content of $bh(x)$ so that 
        \[ag(x) = c_1 g_1 (x)\]
        \[bh(x) = c_2 h_1 (x)\]
        for primitives $g_1, h_1$. Then, $abf(x) = c_1 c_2 g_1 (x) h_1 (x)$ (here, we replaced $ag(x)$ and $bg(x)$ and reorganized them). Since $f(x)$ is primitive, the content of $abf(x)$ is $ab$. Since $g_1 (x)$ and $h_1 (x)$ is primitive, by Gauss's Lemma, we know that $g_1 (x) h_1 (x)$ is primitive; this implies that the content of $c_1 c_2 g_1 (x) h_1 (x)$ is $c_1 c_2$. This means that 
        \[ab = c_1 c_2\]
        and thus $ab = c_1 c_2$, $\Z[x]$ is an integral domain, and so by multiplicative cancellation, $f(x) = g_1 (x) h_1 (x)$. By construction, we know that $g_1 (x), h_1 (x) \in \Z[x]$. 
    \end{proof}
\end{mdframed}

\subsubsection{Example: Concrete Polynomial}
Suppose we are given the polynomial
\[f(x) = 6x^2 + x - 2 = \left(3x - \frac{3}{2}\right) \left(2x + \frac{4}{3}\right)\]
and we want to find a factorization of said polynomial over the integers. 

\begin{mdframed}[]
    Since $f(x)$ is reducible over $\Q$, it must be reducible over $\Z$. We can use the proof above as a guide for finding a factorization. Note that the coefficients of $f(x)$, on the left-hand side, are 6, 1, and -2. Therefore, $\gcd(6, 1, -2) = 1$ so $f(x)$ is a primitive polynomial. 

    \bigskip 

    We now need to clear the denominators of the two factors of $6x^2 + x - 2$. The left-hand factor $\left(3x - \frac{3}{2}\right)$ has common denominator 2 and the right-hand factor $\left(2x + \frac{4}{3}\right)$ has common denominator 3. So, we multiply both sides of the equation by $2 \cdot 3$, like so 
    \begin{equation*}
        \begin{aligned}
            6x^2 + x - 2 &= \left(3x - \frac{3}{2}\right) \left(2x + \frac{4}{3}\right) \\ 
                &\implies 2 \cdot 3 (6x^2 + x - 2) = 2\left(3x - \frac{3}{2}\right) 3\left(2x + \frac{4}{3}\right) \\ 
                &\implies 2 \cdot 3 (6x^2 + x - 2) = \boxed{(6x - 3)(6x + 4)}
        \end{aligned}
    \end{equation*}
    We now look for the content of the two integer factors (boxed above). The left factor $(6x - 3)$ has content 3, while the right factor $(6x + 4)$ has content 2. We can now factor these constants out, giving us  
    \[2 \cdot 3 (6x^2 + x - 2) = 3(2x - 1) \cdot 2(3x + 2)\]
    We see that the constants on both sides can be canceled out. Doing so, we have 
    \[6x^2 + x - 2 = (2x - 1) (3x + 2)\] 
    so we are done. 
\end{mdframed}

\subsection{Mod p Irreducibility Test}
\begin{theorem}{}{}
    Let $p$ be a prime and $f(x) \in \Z[x]$ with $\deg f(x) \geq 1$. Let $\overline{f}(x) \in \F_{p}[x]$ be such that
    \[f(x) \equiv \overline{f}(x) \Mod{p}\]
    If $\overline{f}(x)$ is irreducible over $\F_p$ and $\deg \overline{f}(x) = \deg f(x)$, then $f(x)$ is irreducible over $\Q$. 
\end{theorem}

\begin{mdframed}[]
    \begin{proof}
        We prove the contrapositive. Suppose $f(x)$ is reducible over $\Q$. Then, if $\deg \overline{f}(x) \neq \deg f(x)$, then we are done. Otherwise, we have that $f(x) = g(x) h(x)$ over $\Q$, which implies that 
        \[0 < \deg g(x), \deg h(x) < \deg f(x)\]
        This implies that $f(x) = g_1 (x) h_1 (x)$ over $\Z$ by the theorem we proved above and with
        \[0 \leq \deg g_1 (x), \deg h_1 (x) < \deg f(x)\]
        This implies that 
        \[\overline{f}(x) = \overline{g_1}(x) \overline{h_1}(x) \text{ over } \F_p\]
        We know that $\deg f(x) = \deg \overline{f}(x)$ and $\deg g(x) \geq \deg \overline{g_1}(x)$ and $\deg h_1 (x) \geq \deg \overline{h_1}(x)$. We then have 
        \begin{equation*}
            \begin{aligned}
                \deg \overline{f}(x) &= \deg \overline{g_1}(x) + \deg \overline{h_1}(x) \\ 
                    &\leq \deg g_1 (x) + \deg h_1 (x) \\ 
                    &= \deg f(x) = \deg \overline{f}(x)
            \end{aligned}
        \end{equation*}
        So, thus, $\deg g_1 (x) = \deg \overline{g_1}(x)$ and $\deg h_1 (x) = \deg \overline{h_1}(x)$ and so we have that 
        \[0 < \deg \overline{g}(x), \deg \overline{h}(x) < \deg f(x) = \deg \overline{f}(x)\]
        and thus we are done. 
    \end{proof}
\end{mdframed}

\subsubsection{Example 1: Degree 3 Polynomial}
Is the polynomial
\[f(x) = 21x^3 - 3x^2 + 2x + 9\]
irreducible?

\begin{mdframed}[]
    We pick $p = 2$. This gives us 
    \[f(x) \equiv x^3 + x^2 + 1 = \overline{f}(x) \Mod{p}\]
    Since $\deg f(x) = \deg \overline{f}(x) = 3$, this condition is satisfied. We now need to check if $\overline{f}(x)$ is irreducible over $\F_2$. To do so, we can just brute-force it:
    \[\overline{f}(0) = 0 + 0 + 1 = 1\]
    \[\overline{f}(1) = 1 + 1 + 1 = 3 \equiv 1\]
    As this polynomial has no roots and $\deg \overline{f}(x) = 3$, the reducibility test for degrees 2 and 3 states that $\overline{f}(x)$ is irreducible. Therefore, $f(x)$ is irreducible over $\Z$. 
\end{mdframed}

\subsubsection{Example 2: Degree 4 Polynomial}
Is the polynomial
\[f(x) = \frac{3}{7}x^4 - \frac{2}{7}x^2 + \frac{9}{35}x + \frac{3}{5}\]
irreducible?

\begin{mdframed}[]
    Let $h(x) = 35f(x) = 15x^4 - 10x^2 + 9x + 21$. We let $p = 2$ so 
    \[h(x) \equiv x^4 + x + 1 \Mod{p}\]
    By brute-force, we know that $h(x)$ has no zeros in $\F_2$. It also doesn't have any quadratic or linear factors, and so $\overline{h}(x)$ is irreducible over $\F_2[x]$ and thus $h(x)$ is irreducible over $\Q$. 
\end{mdframed}

\end{document}