\documentclass[letterpaper]{article}
\usepackage[margin=1in]{geometry}
\usepackage[utf8]{inputenc}
\usepackage{textcomp}
\usepackage{amssymb}
\usepackage{natbib}
\usepackage{graphicx}
\usepackage{gensymb}
\usepackage{amsthm, amsmath, mathtools}
\usepackage[dvipsnames]{xcolor}
\usepackage{enumerate}
\usepackage{mdframed}
\usepackage[most]{tcolorbox}
\usepackage{csquotes}
% https://tex.stackexchange.com/questions/13506/how-to-continue-the-framed-text-box-on-multiple-pages

\tcbuselibrary{theorems}

\newcommand{\R}{\mathbb{R}}
\newcommand{\Z}{\mathbb{Z}}
\newcommand{\N}{\mathbb{N}}
\newcommand{\Q}{\mathbb{Q}}
\newcommand{\C}{\mathbb{C}}
\newcommand{\code}[1]{\texttt{#1}}
\newcommand{\mdiamond}{$\diamondsuit$}
\newcommand{\PowerSet}{\mathcal{P}}
\newcommand{\Mod}[1]{\ (\mathrm{mod}\ #1)}
\DeclareMathOperator{\lcm}{lcm}

%\newtheorem*{theorem}{Theorem}
%\newtheorem*{definition}{Definition}
%\newtheorem*{corollary}{Corollary}
%\newtheorem*{lemma}{Lemma}
\newtheorem*{proposition}{Proposition}


\newtcbtheorem[number within=section]{theorem}{Theorem}
{colback=green!5,colframe=green!35!black,fonttitle=\bfseries}{th}

\newtcbtheorem[number within=section]{definition}{Definition}
{colback=blue!5,colframe=blue!35!black,fonttitle=\bfseries}{def}

\newtcbtheorem[number within=section]{corollary}{Corollary}
{colback=yellow!5,colframe=yellow!35!black,fonttitle=\bfseries}{cor}

\newtcbtheorem[number within=section]{lemma}{Lemma}
{colback=red!5,colframe=red!35!black,fonttitle=\bfseries}{lem}

\newtcbtheorem[number within=section]{example}{Example}
{colback=white!5,colframe=white!35!black,fonttitle=\bfseries}{def}

\newtcbtheorem[number within=section]{note}{Important Note}{
        enhanced,
        sharp corners,
        attach boxed title to top left={
            xshift=-1mm,
            yshift=-5mm,
            yshifttext=-1mm
        },
        top=1.5em,
        colback=white,
        colframe=black,
        fonttitle=\bfseries,
        boxed title style={
            sharp corners,
            size=small,
            colback=red!75!black,
            colframe=red!75!black,
        } 
    }{impnote}
\usepackage[utf8]{inputenc}
\usepackage[english]{babel}
\usepackage{fancyhdr}
\usepackage[hidelinks]{hyperref}

\pagestyle{fancy}
\fancyhf{}
\rhead{Math 103B}
\chead{Monday, February 14, 2022}
\lhead{Lecture 18}
\rfoot{\thepage}

\setlength{\parindent}{0pt}

\begin{document}

\section{Divisibility in Integral Domains}
We will continue our discussion on divisibility in integral domains.

\subsection{Unique Factorization Domain}
\begin{definition}{}{}
    An integral domain $D$ is a \textbf{unique factorization domain} (UFD) if it satisfies two properties: 
    \begin{enumerate}
        \item Every non-zero, non-unit element of $D$ can be written as a product of irreducibles. 
        \item Up to reordering and up to associates, this factorization is unique. 
    \end{enumerate}
\end{definition}

\subsubsection{Example 1: The Integers}
Show that $\Z$ is a UFD. 

\begin{mdframed}[]
    \begin{proof}
        (Sketch.) We show existence and uniqueness. 
        \begin{itemize}
            \item \textbf{Existence:} We induct on the integer $N > 1$. 
            \begin{itemize}
                \item \underline{Base Case:} $N = 2$ is irreducible since it is prime. 
                \item \underline{Inductive Step:} If $N$ is prime, it's already irreducible. Otherwise, $N = ab$ for $a, b < N$. But, by the inductive hypothesis, $a, b$ are products of irreducible, so $N$ is irreducible. 
            \end{itemize}

            \item \textbf{Uniqueness:} Suppose $p_1 p_2 \ldots p_n = q_1 q_2 \ldots q_m$. WLOG $n \leq m$, $p_1 | q_1 q_2 \ldots q_m$. By Euclid's lemma, we know that 
            \[p_1|q_i\]
            for some $i$. WLOG, $i = 1$. But, $q_i = \pm p_1$. We repeat this process until 
            \[\pm 1 = q_{n + 1} \ldots q_m\]
            but this isn't possible unless $n = m$, in which case you get $\pm 1 = 1$. 
        \end{itemize}
        This concludes this proof. 
    \end{proof}
\end{mdframed}

\subsubsection{Example 2: Another Ring}
Show that $\Z[\sqrt{-3}]$ is not a UFD. 

\begin{mdframed}[]
    \begin{proof}
        This is not a UFD because 
        \begin{itemize}
            \item There are non-unique factorizations 
            \[4 = 2 \cdot 2 = (1 + \sqrt{-3})(1 - \sqrt{-3})\]
            \item And 2, $1 \pm \sqrt{-3}$ are irreducibles but not primes.
        \end{itemize}
        Which means we are done. 
    \end{proof}
\end{mdframed}

\subsection{PIDs and UFDs}
\begin{theorem}{}{}
    Every PID is a UFD. 
\end{theorem}

\begin{lemma}{}{}
    In a PID, any strictly ascending chain of ideals 
    \[I_1 \subset I_2 \subset I_3 \subset \dots\]
    must have finite length. 
\end{lemma}

\begin{mdframed}[]
    \begin{proof}
        We prove existence and uniqueness.
        \begin{itemize}
            \item \textbf{Existence:} Let $a_0 \in D$ be non-zero, non-unit. If $a_0$ is irreducible, we're done. Otherwise, write $a_0 = b_1 a_1$ for $b_1, a_1 \in D$ non-units. This implies that $\cyclic{a_0} \subset \cyclic{a_1}$. We can repeat this process over and over again (this part omitted), but note that this chain is finite, so it terminates at some $\cyclic{a_n}$ for an irreducible number, or that 
            \[a_0 = (b_1 b_2 \dots b_r) a_r\]
            i.e. $a_r$ is irreducible and $a_r | a_0$. Write $a_0 = c_1 p_1$ for $p_1$ irreducible. We can recursively define this process like so 
            \begin{mdframed}[]
                if $c_i$ is irreducible, stop. Otherwise, $c_i = c_{i + 1} p_{i + 1}$, where $p_{i + 1}$ is irreducible with $c_{i + 1}$ being a non-unit. This gives us
                \[\cyclic{c_i} \subset \cyclic{c_{i + 1}}\]
                Thus, $\cyclic{c_1} \subset \cyclic{c_2} \subset \cyclic{c_3} \subset \dots$. This chain has finite length, so it terminates at $\cyclic{c_s}$ for some integer $s$. This implies that $c_s$ is irreducible. Therefore, 
                \[a_0 = \underbrace{p_1 p_2 p_3 p_4 \dots p_s}_{\text{Irreducible by construction}} c_s\]
                and $c_s$ is irreducible. So, we wrote a product of irreducibles. 
            \end{mdframed}

            \item \textbf{Uniqueness:} Same idea as above. 
        \end{itemize}
        So, we are done. 
    \end{proof}
\end{mdframed}

\begin{mdframed}[]
    \begin{proof}
        Let $I_1 \subset I_2 \subset \dots$ be a strictly ascending chain of ideals. Let
        \[I = \bigcup_{k} I_k \subseteq D\]
        where $I$ is itself an ideal. Since $D$ is a PID, there exists a $d \in D$ such that 
        \[I = \cyclic{d}\]
        but $d \in I = \bigcup_{k} I_k$. Thus, $d \in I_j$ for some $j$. This implies that $\cyclic{d} \subseteq I_j \subseteq I = \cyclic{d}$, so these are all equalities. But, $I_j$ must be the last element; otherwise, $I_j \subseteq I_{j + 1}$ since it is a strictly ascending chain, but this would imply that $I_j \subset I$. 
    \end{proof}
\end{mdframed}

\subsubsection{Example 1: Polynomial Rings}
If $\F$ is a field, then $\F[x]$ is a PID. This implies that $\F[x]$ is a UFD. 

\subsubsection{Example 2: Chains}
In $\Z$, consider the following ideals in our chain:
\[\{0\} \subset \cyclic{2} \subset \Z\]
since $\cyclic{2}$ is maximal. If we wanted a longer chain, we could have 
\[\{0\} \subset \cyclic{100} \subset \cyclic{50} \subset \ldots \subset \Z\]
Here, there are only a finite number of choices we can pick. 

\end{document}