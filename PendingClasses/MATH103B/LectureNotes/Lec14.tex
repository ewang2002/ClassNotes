\documentclass[letterpaper]{article}
\usepackage[margin=1in]{geometry}
\usepackage[utf8]{inputenc}
\usepackage{textcomp}
\usepackage{amssymb}
\usepackage{natbib}
\usepackage{graphicx}
\usepackage{gensymb}
\usepackage{amsthm, amsmath, mathtools}
\usepackage[dvipsnames]{xcolor}
\usepackage{enumerate}
\usepackage{mdframed}
\usepackage[most]{tcolorbox}
\usepackage{csquotes}
% https://tex.stackexchange.com/questions/13506/how-to-continue-the-framed-text-box-on-multiple-pages

\tcbuselibrary{theorems}

\newcommand{\R}{\mathbb{R}}
\newcommand{\Z}{\mathbb{Z}}
\newcommand{\N}{\mathbb{N}}
\newcommand{\Q}{\mathbb{Q}}
\newcommand{\C}{\mathbb{C}}
\newcommand{\code}[1]{\texttt{#1}}
\newcommand{\mdiamond}{$\diamondsuit$}
\newcommand{\PowerSet}{\mathcal{P}}
\newcommand{\Mod}[1]{\ (\mathrm{mod}\ #1)}
\DeclareMathOperator{\lcm}{lcm}

%\newtheorem*{theorem}{Theorem}
%\newtheorem*{definition}{Definition}
%\newtheorem*{corollary}{Corollary}
%\newtheorem*{lemma}{Lemma}
\newtheorem*{proposition}{Proposition}


\newtcbtheorem[number within=section]{theorem}{Theorem}
{colback=green!5,colframe=green!35!black,fonttitle=\bfseries}{th}

\newtcbtheorem[number within=section]{definition}{Definition}
{colback=blue!5,colframe=blue!35!black,fonttitle=\bfseries}{def}

\newtcbtheorem[number within=section]{corollary}{Corollary}
{colback=yellow!5,colframe=yellow!35!black,fonttitle=\bfseries}{cor}

\newtcbtheorem[number within=section]{lemma}{Lemma}
{colback=red!5,colframe=red!35!black,fonttitle=\bfseries}{lem}

\newtcbtheorem[number within=section]{example}{Example}
{colback=white!5,colframe=white!35!black,fonttitle=\bfseries}{def}

\newtcbtheorem[number within=section]{note}{Important Note}{
        enhanced,
        sharp corners,
        attach boxed title to top left={
            xshift=-1mm,
            yshift=-5mm,
            yshifttext=-1mm
        },
        top=1.5em,
        colback=white,
        colframe=black,
        fonttitle=\bfseries,
        boxed title style={
            sharp corners,
            size=small,
            colback=red!75!black,
            colframe=red!75!black,
        } 
    }{impnote}
\usepackage[utf8]{inputenc}
\usepackage[english]{babel}
\usepackage{fancyhdr}
\usepackage[hidelinks]{hyperref}

\pagestyle{fancy}
\fancyhf{}
\rhead{Math 103B}
\chead{Friday, February 4, 2022}
\lhead{Lecture 14}
\rfoot{\thepage}

\setlength{\parindent}{0pt}

\begin{document}

\section{Reducible and Irreducible Polynomials}
The idea behind a reducible or irreducible polynomial is very similar in nature to factoring and finding zeros of a polynomial. 

\subsection{Definition}
\begin{definition}{}{}
    Let $D$ be an integral domain. A polynomial $f(x)$ from $D[x]$ that is neither the zero polynomial nor a unit in $D[x]$ is said to be \textbf{irreducible} over $D$ if, whenever $f(x)$ is expressed as a product 
    \[f(x) = g(x)h(x)\]
    with $g(x), h(x) \in D[x]$, then $g(x)$ \emph{or} $h(x)$ is a unit in $D[x]$. A non-zero, non-unit element of $D[x]$ that is not irreducible over $D$ is called \textbf{reducible} over $D$.  
\end{definition}
\textbf{Fact:} If $F$ is a field, $f(x) \in F[x]$ is irreducible if and only if $f(x) = g(x) h(x)$ implies that one of $g(x)$ or $h(x)$ have degree 0. 

\bigskip 

We can try to make a similar definition for the integers to get a better idea of what this means. We can define an ``irreducible'' integer $n \in \Z$ is one such that 
\[n = ab \implies a \in \{\pm 1\} \text{ or } b \in \{\pm 1\}\]
So, in the integers, the only set of ``irreducible'' integers are $\pm p$ for primes $p$. 

\subsubsection{Example 1: Polynomial}
Consider the polynomial $f(x) = 2x^2 + 4$. 
\begin{itemize}
    \item This is \textbf{reducible} over $\Z$ since $2x^2 + 4 = 2(x^2 + 2)$ and neither 2 nor $x^2 + 2$ is a unit in $\Z[x]$. 
    \item This is \textbf{irreducible} over $\Q$. If we use the same factorization described above, then note that $2$ has a unit in $Q[x]$. 
    \item This is \textbf{reducible} over $\C$ since $2x^2 + 4 = 2(x - i\sqrt{2})(x + i\sqrt{2})$. Here, if $g(x) = 2(x - i\sqrt{2})$ and $h(x) = x + i\sqrt{2}$, then none of $g$ or $h$ are units. 
\end{itemize}

\subsection{Reducibility Test for Degrees 2 and 3}
\begin{theorem}{}{}
    Let $F$ be a field. If $f(x) \in F[x]$ and $\deg f(x)$ is 2 or 3, then $f(x)$ is reducible over $F$ if and only if $f(x)$ has a zero in $F$. 
\end{theorem}

\begin{mdframed}[]
    \begin{proof}
        We will prove the contrapositive; that is, $f(x)$ is reducible if and only if $f(x)$ has a root in $F$. 
        \begin{itemize}
            \item \underline{Backwards Direction:} Suppose $a \in F$ with $f(a) = 0$. This implies that $(x - a) | f(a)$ which implies that $f(x) = (x - a)g(x)$. Thus, $\deg g(x) = \deg f(x) - 1 \geq 1$. But, we found a factorization, so $f(x)$ is reducible. 
            \item \underline{Forward Direction:} If $f(x)$ is reducible, then $f(x) = g(x) h(x)$ with $\deg g(x), \deg h(x) \neq 0$. The only options are 
            \[\deg f(x) = \deg g(x) + \deg h(x)\]
            So, we can brute-force the possible degrees: 
            \begin{itemize}
                \item 2 = 1 + 1
                \item 3 = 1 + 2 or 3 = 2 + 1
            \end{itemize}
            Thus, there exists $ax + b \in F[x]$, $a \neq 0$, with $(ax + b) | f(x)$ which implies that $f(x) = (ax + b)q(x)$. This further implies that $f\left(-\frac{b}{a}\right) = 0 \cdot q\left(-\frac{b}{a}\right) = 0$. So, $f(x)$ has a root $-\frac{b}{a} \in F$. 
        \end{itemize}
        This concludes the proof. 
    \end{proof}
\end{mdframed}

\subsubsection{Example 2: Polynomial}
Consider the polynomial $f(x) = 2x^3 + 4$. 
\begin{itemize}
    \item Is $f(x)$ irreducible over $\Q$? Using the theorem above, we have 
    \[2x^3 + 4 = 0 \implies 2x^3 = -4 \implies x^3 = -sqrt{2} \implies x = - \sqrt[3]{2}\]
    But, $-\sqrt[3]{2} \notin \Q$ so this is \textbf{irreducible}.

    \item This is \textbf{reducible} over $\R$. 
\end{itemize}

\subsubsection{Example 3: Polynomial}
Consider the field $\F_{2}[x]$. Are the polynomials with coefficients in this field reducible?
\begin{itemize}
    \item \underline{Degree 0:} 
    \begin{itemize}
        \item $0$: Reducible.
        \item $1$: Irreducible\footnote{This can be generalized to any non-zero constant polynomial.}.
    \end{itemize}

    \item \underline{Degree 1:}
    \begin{itemize}
        \item $x$: Irreducible\footnote{Cannot be factored since it is linear.}. 
        \item $x + 1$: Irreducible\footnote{Cannot be factored since it is linear. In general, a degree 1 polynomial with coefficients in a field are always irreducible.}.
    \end{itemize}

    \item \underline{Degree 2:}
    \begin{itemize}
        \item $x^2 = xx$: Reducible. 
        \item $x^2 + 1$: Reducible\footnote{Using the theorem, note that $1 \in F_3$ and $1^2 + 1 = 2 \equiv 0$.}.
        \item $x^2 + x = x(x + 1)$: Reducible.
        \item $x^2 + x + 1$: Irreducible.
    \end{itemize}

    \item \underline{Degree 3:}
    \begin{itemize}
        \item Left as an exercise. 
    \end{itemize}
\end{itemize}

\subsection{Relation Between Integer Coefficient and Rational Coefficient Polynomials}
\begin{theorem}{}{}
    Let $f(x) \in \Z[x]$. $f(x)$ is reducible over $\Q \implies f(x)$ is reducible over $\Z$.
\end{theorem}
\textbf{Remark:} The contrapositive of this theorem is important. In particular, $f(x)$ is irreducible over $\Z \implies f(x)$ is irreducible over $\Q$. 

\textbf{Warning:} The \emph{converse} of this theorem is not true. For an example, see $f(x) = 2x^2 + 4$. 

\begin{definition}{Content}{}
    The \textbf{content} of a non-zero polynomial $a_0 + a_1 x + \dots + a_n x^n \in \Z[x]$ is $\gcd(a_0, a_1, \dots, a_n)$. 
\end{definition}

\begin{definition}{Primitive Polynomial}
    A \textbf{primitive polynomial} is an element of $\Z[x]$ with content 1. 
\end{definition}

\end{document}