\documentclass[letterpaper]{article}
\usepackage[margin=1in]{geometry}
\usepackage[utf8]{inputenc}
\usepackage{textcomp}
\usepackage{amssymb}
\usepackage{natbib}
\usepackage{graphicx}
\usepackage{gensymb}
\usepackage{amsthm, amsmath, mathtools}
\usepackage[dvipsnames]{xcolor}
\usepackage{enumerate}
\usepackage{mdframed}
\usepackage[most]{tcolorbox}
\usepackage{csquotes}
% https://tex.stackexchange.com/questions/13506/how-to-continue-the-framed-text-box-on-multiple-pages

\tcbuselibrary{theorems}

\newcommand{\R}{\mathbb{R}}
\newcommand{\Z}{\mathbb{Z}}
\newcommand{\N}{\mathbb{N}}
\newcommand{\Q}{\mathbb{Q}}
\newcommand{\C}{\mathbb{C}}
\newcommand{\code}[1]{\texttt{#1}}
\newcommand{\mdiamond}{$\diamondsuit$}
\newcommand{\PowerSet}{\mathcal{P}}
\newcommand{\Mod}[1]{\ (\mathrm{mod}\ #1)}
\DeclareMathOperator{\lcm}{lcm}

%\newtheorem*{theorem}{Theorem}
%\newtheorem*{definition}{Definition}
%\newtheorem*{corollary}{Corollary}
%\newtheorem*{lemma}{Lemma}
\newtheorem*{proposition}{Proposition}


\newtcbtheorem[number within=section]{theorem}{Theorem}
{colback=green!5,colframe=green!35!black,fonttitle=\bfseries}{th}

\newtcbtheorem[number within=section]{definition}{Definition}
{colback=blue!5,colframe=blue!35!black,fonttitle=\bfseries}{def}

\newtcbtheorem[number within=section]{corollary}{Corollary}
{colback=yellow!5,colframe=yellow!35!black,fonttitle=\bfseries}{cor}

\newtcbtheorem[number within=section]{lemma}{Lemma}
{colback=red!5,colframe=red!35!black,fonttitle=\bfseries}{lem}

\newtcbtheorem[number within=section]{example}{Example}
{colback=white!5,colframe=white!35!black,fonttitle=\bfseries}{def}

\newtcbtheorem[number within=section]{note}{Important Note}{
        enhanced,
        sharp corners,
        attach boxed title to top left={
            xshift=-1mm,
            yshift=-5mm,
            yshifttext=-1mm
        },
        top=1.5em,
        colback=white,
        colframe=black,
        fonttitle=\bfseries,
        boxed title style={
            sharp corners,
            size=small,
            colback=red!75!black,
            colframe=red!75!black,
        } 
    }{impnote}
\usepackage[utf8]{inputenc}
\usepackage[english]{babel}
\usepackage{fancyhdr}
\usepackage[hidelinks]{hyperref}

\pagestyle{fancy}
\fancyhf{}
\rhead{Math 103B}
\chead{Monday, January 3rd, 2022}
\lhead{Lecture 1}
\rfoot{\thepage}

\setlength{\parindent}{0pt}

\begin{document}

\section{Ring}
Recall that a group is a set equipped with a binary operation. However, often times, a lot of sets are naturally endowed with \emph{two} binary operations: addition \emph{and} multiplication. In this case, we want to account for \emph{both} of them at the same time instead of having two groups with the same sets but different operations. To that, we introduce the \emph{ring}.

\subsection{The Ring: Definition}

\begin{definition}{Ring}{}
    A ring $R$ is a set with two binary operations, addition (denoted by $a + b$) and multiplication (denoted by $ab$), such that for all $a, b, c \in R$:
    \begin{enumerate}
        \item \textbf{Commutative:} $a + b = b + a$
        \item \textbf{Associative:} $(a + b) + c = a + (b + c)$
        \item \textbf{Additive Identity:} There is an \underline{additive identity} $0 \in R$ such that $a + 0 = 0 + a = a$ for all $a \in R$.
        \item \textbf{Additive Inverse:} There is an element $-a \in R$ such that $a + (-a) = (-a) + a = 0$. 
        \item \textbf{Associative:} $a(bc) = (ab)c$. 
        \item \textbf{Distributive Property:} $a(b + c) = ab + ac$ and $(b + c)a = ba + ca$.  
    \end{enumerate}
    We sometimes write this ring out as $(R, +, \cdot)$. 
\end{definition}
\textbf{Remarks:}
\begin{itemize}
    \item A ring is an \underline{abelian group} under addition, but also has an associative multiplication that is \emph{left and right distributive} over addition.
    \item Multiplication does \textbf{not} have to be commutative. If it is commutative, we say that the ring is commutative.
    \item A ring \emph{does not need to have} an identity under multiplication. A \textbf{unity} (or identity) in a ring is a \emph{nonzero element} that is an identity under multiplication.
    \item A nonzero element of a \underline{commutative ring} with unity need not have a multiplicative inverse. When it does, we say that it is a \textbf{unit} of the ring. In other words, $a$ is a unit if $a^{-1}$ exists. 
    \item If $a$ and $b$ belong to a commutative ring $R$ and $a$ is nonzero, then we say that $a$ \emph{divides} $b$ (or that $a$ is a factor of $b$) and write $a | b$ if there exists $c \in R$ such that $b = ac$. If $a$ does not divide $b$, we write $a \nmid b$.
    \item If we need to deal with something like:
    \[\underbrace{a + a + \dots + a}_{n \text{ times}}\]
    Then, we will use $n \cdot a$ to mean this. 
\end{itemize}


\subsection{Basic Applications of the Ring}
Here, we introduce several examples of rings. 

\subsubsection{Example 1: Integer Rings}
\[\Z = \{\dots, -2, -1, 0, 1, 2, \dots\}\]
The set of integers under ordinary addition and multiplication is a commutative ring with unity 1. The \emph{units} of $\Z$ are 1 and -1.

\subsubsection{Example 2: Integers Mod N}
\[\Z / n\Z = \{0, 1, \dots, n - 1\}\]
The set of integers modulo $n$ under addition and multiplication is also a commutative ring with unity 1. The set of \emph{units} is $U(n)$. Here, we define $U(n)$ to be the set of integers less than $n$ and relatively prime to $n$ under multiplication modulo $n$. 

\bigskip 

This can also be written as $\Z_n$. 

\subsubsection{Example 3: Polynomial Rings}
The set $\Z[x]$ of all polynomials in the variable $x$ with integer coefficients under ordinary addition and multiplication is a commutative ring with unity $f(x) = 1$. Here, we define: 
\[\Z[x] = \{a_0 + a_1 x + a_2 x^2 + \dots + a_n x^n \mid a_i \in \Z\}\]
So, for example, $x^2 + 4x + 5 \in \Z[x]$. 

\subsubsection{Example 4: Matrix Rings}
The set $M_{2}(\Z)$ of $2 \times 2$ matrices with integer entries is a \emph{noncommutative ring} with unity $\begin{bmatrix} 1 & 0 \\ 0 & 1 \end{bmatrix}$.

\subsubsection{Example 5: Even Integer Rings}
The set $2\Z$ of even integers under ordinary addition and multiplication is a commutative ring \underline{without} unity. 

\subsubsection{Example 6: Direct Sum}
If $R_1, R_2, \dots, R_n$ are rings, then we can create a new ring: 
\[R_1 \oplus R_2 \oplus \dots \oplus R_3 = \{(a_1, a_2, \dots, a_n) \mid a_i \in R_i\}\]
From this, we can perform componentwise addition and multiplication; that is: 
\[(a_1, a_2, \dots, a_n) + (b_1, b_2, \dots, b_n) = (a_1 + b_1, a_2 + b_2, \dots, a_n + b_n)\]
\[(a_1, a_2, \dots, a_n)(b_1, b_2, \dots, b_n) = (a_1 b_1, a_2 b_2, \dots, a_n b_n)\]

\subsection{More on Rings}
\begin{definition}{Commutative Ring}{}
    A ring $R$ is \textbf{commutative} if $ab = ba$ for all $a, b \in R$.
\end{definition}

\begin{definition}{Unity}{}
    A ring $R$ has \textbf{unity} if $1 \in R$ is a multiplicative identity: 
    \[1a = a1 = a\]
\end{definition}

\begin{definition}{Unit}{}
    An element $a \in R$ is called a \textbf{unit} if it has a multiplicative inverse. In other words, $a$ is a unit if there exists an $a^{-1} \in R$ such that: 
    \[a^{-1}a = aa^{-1} = 1\]
\end{definition}
\textbf{Remarks:}
\begin{itemize}
    \item $U(R) = \{\text{Units in } R\}$
    \item $U(n) = \{\text{Units in } \Z / n\Z\}$
\end{itemize}

\begin{definition}{Division}{}
    For $a, b \in R$, we say that $a$ \textbf{divides} $b$ and write $a | b$ if $b = ac$ for some $c \in R$.
\end{definition}


\end{document}
