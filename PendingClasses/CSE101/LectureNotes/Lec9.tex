\documentclass[letterpaper]{article}
\usepackage[margin=1in]{geometry}
\usepackage[utf8]{inputenc}
\usepackage{textcomp}
\usepackage{amssymb}
\usepackage{natbib}
\usepackage{graphicx}
\usepackage{gensymb}
\usepackage{amsthm, amsmath, mathtools}
\usepackage[dvipsnames]{xcolor}
\usepackage{enumerate}
\usepackage{mdframed}
\usepackage[most]{tcolorbox}
\usepackage{csquotes}
% https://tex.stackexchange.com/questions/13506/how-to-continue-the-framed-text-box-on-multiple-pages

\tcbuselibrary{theorems}

\newcommand{\R}{\mathbb{R}}
\newcommand{\Z}{\mathbb{Z}}
\newcommand{\N}{\mathbb{N}}
\newcommand{\Q}{\mathbb{Q}}
\newcommand{\C}{\mathbb{C}}
\newcommand{\code}[1]{\texttt{#1}}
\newcommand{\mdiamond}{$\diamondsuit$}
\newcommand{\PowerSet}{\mathcal{P}}
\newcommand{\Mod}[1]{\ (\mathrm{mod}\ #1)}
\DeclareMathOperator{\lcm}{lcm}

%\newtheorem*{theorem}{Theorem}
%\newtheorem*{definition}{Definition}
%\newtheorem*{corollary}{Corollary}
%\newtheorem*{lemma}{Lemma}
\newtheorem*{proposition}{Proposition}


\newtcbtheorem[number within=section]{theorem}{Theorem}
{colback=green!5,colframe=green!35!black,fonttitle=\bfseries}{th}

\newtcbtheorem[number within=section]{definition}{Definition}
{colback=blue!5,colframe=blue!35!black,fonttitle=\bfseries}{def}

\newtcbtheorem[number within=section]{corollary}{Corollary}
{colback=yellow!5,colframe=yellow!35!black,fonttitle=\bfseries}{cor}

\newtcbtheorem[number within=section]{lemma}{Lemma}
{colback=red!5,colframe=red!35!black,fonttitle=\bfseries}{lem}

\newtcbtheorem[number within=section]{example}{Example}
{colback=white!5,colframe=white!35!black,fonttitle=\bfseries}{def}

\newtcbtheorem[number within=section]{note}{Important Note}{
        enhanced,
        sharp corners,
        attach boxed title to top left={
            xshift=-1mm,
            yshift=-5mm,
            yshifttext=-1mm
        },
        top=1.5em,
        colback=white,
        colframe=black,
        fonttitle=\bfseries,
        boxed title style={
            sharp corners,
            size=small,
            colback=red!75!black,
            colframe=red!75!black,
        } 
    }{impnote}
\usepackage[utf8]{inputenc}
\usepackage[english]{babel}
\usepackage{fancyhdr}
\usepackage[hidelinks]{hyperref}

\pagestyle{fancy}
\fancyhf{}
\rhead{CSE 101}
\chead{Wednesday, January 26, 2022}
\lhead{Lecture 9}
\rfoot{\thepage}

\setlength{\parindent}{0pt}

\begin{document}

\section{Fundamental Shortest Paths Formula}
For any vertex $w$ that isn't the source, $w \neq s$,
\[\text{dist}(w) = \min_{(v, w) \in E} \text{dist}(v) + \ell(v, w)\]
We can use a system of equations to solve for the distances. When $\ell \geq 0$, Dijsktra gives an order to solve in.

\subsection{Algorithm Idea}
Instead of finding the shortest paths, which may not exist, we instead find the shortest paths of length at most $k$. So, for $w \neq s$, we have: 
\[\text{dist}_{k}(w) = \min_{(v, w) \in E} \text{dist}_{k - 1}(v) + \ell(v, w)\]

\subsection{Algorithm}
\begin{verbatim}
    Bellman-Ford(G, s, l)
        dist_{0}(v) = infinity for all v 
        dist_{0}(s) = 0
        For k = 1 to n
            For w in V
                dist_{k}(w) = min(dist_{k - 1}(v) + l(v, w))
            dist_{k}(s) = min(dist_{k}(s), 0)
\end{verbatim}

\subsection{Analysis}
\begin{proposition}
    If $n \geq |V| - 1$ and if $G$ has no negative weight cycles, then for all $v$,
    \[\text{dist}(v) = \text{dist}_{n}(v)\]
\end{proposition}
In particular:
\begin{itemize}
    \item If there is a negative weight cycle, there is probably no shortest path. 
    \item If not, we only need to run our algorithm for $|V|$ rounds, for a final runtime of $\BigO(|V||E|)$. 
\end{itemize}

\subsection{Detecting Negative Cycles}
\emph{If} there are no negative weight cycles, Bellman-Ford computes shortest paths (and they might not exist otherwise). Howe do we know whether or not there are any? 

\subsubsection{Cycle Detection}

\begin{proposition}
    For any $n \geq |V| - 1$, there are no negative weight cycles reachable from $s$ if and only if, for every $v \in V$:
    \[\text{dist}_{n}(v) = \text{dist}_{n + 1}(v)\]
\end{proposition}

\subsection{Potential Function}
Let 
\[\ell'(v, w) = \ell(v, w) - d(v) + d(w) \geq 0\]
Imagine someone lending you $d(w)$ when you arrive at $w$, but then you have to pay it back when you leave.

\end{document}