\documentclass[letterpaper]{article}
\usepackage[margin=1in]{geometry}
\usepackage[utf8]{inputenc}
\usepackage{textcomp}
\usepackage{amssymb}
\usepackage{natbib}
\usepackage{graphicx}
\usepackage{gensymb}
\usepackage{amsthm, amsmath, mathtools}
\usepackage{xcolor}
\usepackage{enumerate}
\usepackage{framed}
\usepackage{tcolorbox}
\tcbuselibrary{theorems}

\newcommand{\R}{\mathbb{R}}
\newcommand{\Z}{\mathbb{Z}}
\newcommand{\N}{\mathbb{N}}
\newcommand{\Q}{\mathbb{Q}}
\newcommand{\code}[1]{\texttt{#1}}
\newcommand{\mdiamond}{$\diamondsuit$}

%\newtheorem*{theorem}{Theorem}
%\newtheorem*{definition}{Definition}
\newtheorem*{proposition}{Proposition}
%\newtheorem*{corollary}{Corollary}
%\newtheorem*{lemma}{Lemma}

\newtcbtheorem[number within=section]{theorem}{Theorem}
{colback=green!5,colframe=green!35!black,fonttitle=\bfseries}{def}

\newtcbtheorem[number within=section]{definition}{Definition}
{colback=blue!5,colframe=blue!35!black,fonttitle=\bfseries}{def}

\newtcbtheorem[number within=section]{corollary}{Corollary}
{colback=yellow!5,colframe=yellow!35!black,fonttitle=\bfseries}{def}

\newtcbtheorem[number within=section]{lemma}{Lemma}
{colback=red!5,colframe=red!35!black,fonttitle=\bfseries}{def}
\usepackage[utf8]{inputenc}
\usepackage[english]{babel}
\usepackage{fancyhdr}
\usepackage[hidelinks]{hyperref}

\pagestyle{fancy}
\fancyhf{}
\rhead{CSE 101}
\chead{Friday, January 14, 2022}
\lhead{Lecture 6}
\rfoot{\thepage}

\setlength{\parindent}{0pt}

\begin{document}

\section{Strongly Connected Components}
The issue with our algorithm is that we recompute the postorder for every SCC we need to find. However, we don't need to do this; rather, after removing some strongly connected components to get $G'$, the largest postorder number of vertices in $G'$ is still in a sink component of $G'$. 

\subsection{Better Algorithm}
\begin{verbatim}
    SCCs(G)
        Run DFS(G^R), record postorders
        Mark all vertices as unvisited 
        For v in V in reverse postorder 
            If v not in a component yet 
                explore(v) on G-components found,
                marking new component 
\end{verbatim}

So, really, this is just 2 DFSs, so the runtime is $O(|V| + |E|)$. 

\section{Paths in Graphs}
DFS and \code{explore} allow us to determine \emph{if} it is possible to get from one vertex to another, and using the DFS tree, you can also find a path. However, this is often not an efficient path. 

\subsection{Goal}
Given a graph $G$ with two vertices $s$ and $t$ in the same connected component, find the \emph{best} path from $s$ to $t$. What do we mean by the best? 
\begin{itemize}
    \item Least expensive. 
    \item Best scenery.
    \item Shortest.
\end{itemize}
For now, we want the fewest edges. 

\subsection{Observation}
If there is a path from $s$ to $v$ with length at most $d$, then there is some $w$ adjacent to $v$ with a length at most $\leq (d - 1)$ for a path from $s$ to $w$. 

\subsection{Algorithm Idea}
For each $d$, create a list of all vertices at distance $d$ from $s$.
\begin{itemize}
    \item For $d = 0$, this is just $\{s\}$.
    \item For larger $d$, we want all new vertices adjacent to vertices at distance $d - 1$.
\end{itemize}

\begin{verbatim}
1    ShortestPaths(G, s)
2        Initialize Array A
3        A[0] = {s}
4        dist(s) = 0
5        For d = 0 to n
6            For u in A[d]
7                For (u, v) in E
8                    if dist(v) undefined
9                        dist(v) = d + 1
10                       add v to A[d + 1]
\end{verbatim}

How can we improve this?
\begin{itemize}
    \item What if \code{dist(v)} undefined at end? We can set the distances of all vertices to undefined. 
    \item The algorithm goes through \code{A[0]}, \code{A[1]}, in order. We can just use a queue.
    \begin{verbatim}
        1    ShortestPaths(G, s)
        2        Initialize Queue Q
        +        Q.enqueue(s)
        4        dist(s) = 0
        +        While Q not empty
        +        u = front(Q) 
        7            For (u, v) in E
        8                if dist(v) = infinity
        9                    dist(v) = dist(u) + 1
        10                   Q.enqueue(v)
        \end{verbatim}

    \item What if we want to keep track of the paths? 
    \begin{verbatim}
        1    ShortestPaths(G, s)
        2        Initialize Queue Q
        3        Q.enqueue(s)
        4        dist(s) = 0
        5        While Q not empty
        6        u = front(Q) 
        7            For (u, v) in E
        8                if dist(v) = infinity
        9                    dist(v) = dist(u) + 1
        10                   Q.enqueue(v)
        +                    v.prev = u
        \end{verbatim}
\end{itemize}

\subsection{Breadth First Search}
In our last change above, we note that we simply have BFS.
\begin{verbatim}
    BFS(G, s)
        For v in V, dist(v) = infinity 
        Initialize Queue Q
        Q.enqueue(s)
        dist(s) = 0
        While Q is not empty
            u = front(Q)
            For (u, v) in E
                If dist(v) = infinity
                    dist(v) = dist(u) + 1
                    Q.enqueue(v)
                    v.prev = u
\end{verbatim}
The total runtime is $O(|V| + |E|)$.

\subsection{DFS vs. BFS}
\begin{itemize}
    \item Similarities:
    \begin{itemize}
        \item The way both algorithms process vertices is the same (\code{visited} vs. \code{dist < infinity}).
        \item For each vertex, process all unprocessed neighbors.
    \end{itemize}
    \item Differences:
    \begin{itemize}
        \item DFS uses a stack to store vertices waiting to be processed. 
        \item BFS uses a queue.
    \end{itemize}
    \item Big Effect:
    \begin{itemize}
        \item DFS goes depth-first: very long path. Get a very ``skinny'' tree. 
        \item BFS is breadth first: visits all side paths. Get a very shallow tree since we process all of the neighbors.
    \end{itemize}
\end{itemize}

\subsection{Edge Length}
The number of edges in a path is not always the right measure of distance. Sometimes, taking several shorter steps is preferably to taking a few longer ones. 

\bigskip 

We can assign each edge $(u, v)$ a non-negative \underline{length} $\ell(u, v)$. The length of a path is the sum of the lengths of its edges.

\subsection{Problem: Shortest Path}
Coming soon!


\end{document}