\documentclass[letterpaper]{article}
\usepackage[margin=1in]{geometry}
\usepackage[utf8]{inputenc}
\usepackage{textcomp}
\usepackage{amssymb}
\usepackage{natbib}
\usepackage{graphicx}
\usepackage{gensymb}
\usepackage{amsthm, amsmath, mathtools}
\usepackage{xcolor}
\usepackage{enumerate}
\usepackage{framed}
\usepackage{tcolorbox}
\tcbuselibrary{theorems}

\newcommand{\R}{\mathbb{R}}
\newcommand{\Z}{\mathbb{Z}}
\newcommand{\N}{\mathbb{N}}
\newcommand{\Q}{\mathbb{Q}}
\newcommand{\code}[1]{\texttt{#1}}
\newcommand{\mdiamond}{$\diamondsuit$}

%\newtheorem*{theorem}{Theorem}
%\newtheorem*{definition}{Definition}
\newtheorem*{proposition}{Proposition}
%\newtheorem*{corollary}{Corollary}
%\newtheorem*{lemma}{Lemma}

\newtcbtheorem[number within=section]{theorem}{Theorem}
{colback=green!5,colframe=green!35!black,fonttitle=\bfseries}{def}

\newtcbtheorem[number within=section]{definition}{Definition}
{colback=blue!5,colframe=blue!35!black,fonttitle=\bfseries}{def}

\newtcbtheorem[number within=section]{corollary}{Corollary}
{colback=yellow!5,colframe=yellow!35!black,fonttitle=\bfseries}{def}

\newtcbtheorem[number within=section]{lemma}{Lemma}
{colback=red!5,colframe=red!35!black,fonttitle=\bfseries}{def}
\usepackage[utf8]{inputenc}
\usepackage[english]{babel}
\usepackage{fancyhdr}
\usepackage[hidelinks]{hyperref}

\pagestyle{fancy}
\fancyhf{}
\rhead{CSE 101}
\chead{Monday, January 10, 2022}
\lhead{Lecture 4}
\rfoot{\thepage}

\setlength{\parindent}{0pt}

\begin{document}

\section{Directed Graphs}
A directed graph an be thought of as a graph of dependencies. 

\begin{definition}{Topological Ordering}{}
    A \textbf{topological ordering} of a directed graph is an ordering of the vertices so that for each edge $(v, w)$, $v$ comes before $w$ in the ordering.
\end{definition}

\subsection{Cycles}
\begin{definition}{Cycle}{}
    A cycle in a directed graph is a sequence of vertices $v_1, v_2, \dots, v_n$ so that there are edges: 
    \[(v_1, v_2), (v_2, v_3), \dots, (v_n, v_1)\]
\end{definition}

\subsubsection{Obstacle}
\begin{proposition}
    If $G$ is a directed graph with a cycle, then $G$ has no topological ordering.
\end{proposition}

\begin{mdframed}[]
    \begin{proof}
        Suppose we have a cycle $v_1, \dots, v_n$. Assume for the sake of contradiction taht we have an ordering. Then, we can find the earliest $v_i$ in the ordering. But, the $v_i$ comes before $v_{i - 1}$, in contradiction to the order property. 
    \end{proof}
\end{mdframed}

\subsection{Directed Acyclic Graph (DAG)}
\begin{definition}{}{}
    A \textbf{directed acylic graph} (DAG) is a directed graph which contains no cycles. 
\end{definition}

The previous result said that only DAGs can be topologically ordered. However, is the reverse true? \textbf{Yes}. 

\subsection{Existence of Orderings}
\begin{theorem}{}{}
    Let $G$ be a finite DAG. Then, $G$ has a topological ordering.
\end{theorem}

\begin{mdframed}[]
    \begin{proof}
        We consider the last vertex in the ordering. This must be a sink, or a vertex with no outgoing edges. So, once we find the sink, we can put the graph at the end of the topological graph, and then order the remaining vertices. 
    \end{proof}
\end{mdframed}

\subsubsection{Sinks}
\begin{lemma}{}{}
    Every finite DAG contains at least one sink. 
\end{lemma}

\begin{mdframed}[]
    \begin{proof}
        Start at a vertex $v = v_1$. Then, we can ``follow the trail,'' or in other words follow the edges. Eventually, we will either find: 
        \begin{itemize}
            \item Some vertices repeat (which creates a cycle).
            \item Gets stuck (found a sink).
        \end{itemize}
        So, we are done. 
    \end{proof}
\end{mdframed}

\subsection{Algorithm}
Suppose we want to design an algorithm that, given a DAG $G$, computes a topological ordering on G. We can use the proof to create a naive algorithm. 
\begin{verbatim}
    TopologicalOrdering(G)
        If |G| = 0
            Return {}
        Let v in G 
        While there is an edge (v, w)
            v = w
        Return (Ordering(G - v), v)
\end{verbatim}
The runtime is $O(|V|^2)$. This is because we need $|V|$ time to find each sink and have $|V|$ sinks. This is suboptimal, however.
\begin{verbatim}
    TopologicalOrdering(G)
        Run DFS(G) w/ Pre/Post Numbers 
        Return Vertices in Reverse Postorder
\end{verbatim}
This runs in $O(|V| + |E|)$.

\subsection{Topological Sort}
This is a particularly useful algorithm. 
\begin{itemize}
    \item Many graph algorithms are relatively easy to find the answer for $v$ if you've already found the answer for everything downstream of $v$. 
    \begin{itemize}
        \item We can topologically sort $G$. 
        \item Then, solve for $v$ in reverse topological order.
    \end{itemize}
\end{itemize}

\subsection{Connectivity in Digraphs}
In undirected graphs, we had a very clean description of reachability: $v$ was reachable from $w$ if and only if they were in the same connected component. Well, this no longer works for digraphs. 

\end{document}