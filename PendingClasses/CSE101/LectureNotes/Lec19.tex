\documentclass[letterpaper]{article}
\usepackage[margin=1in]{geometry}
\usepackage[utf8]{inputenc}
\usepackage{textcomp}
\usepackage{amssymb}
\usepackage{natbib}
\usepackage{graphicx}
\usepackage{gensymb}
\usepackage{amsthm, amsmath, mathtools}
\usepackage[dvipsnames]{xcolor}
\usepackage{enumerate}
\usepackage{mdframed}
\usepackage[most]{tcolorbox}
\usepackage{csquotes}
% https://tex.stackexchange.com/questions/13506/how-to-continue-the-framed-text-box-on-multiple-pages

\tcbuselibrary{theorems}

\newcommand{\R}{\mathbb{R}}
\newcommand{\Z}{\mathbb{Z}}
\newcommand{\N}{\mathbb{N}}
\newcommand{\Q}{\mathbb{Q}}
\newcommand{\C}{\mathbb{C}}
\newcommand{\code}[1]{\texttt{#1}}
\newcommand{\mdiamond}{$\diamondsuit$}
\newcommand{\PowerSet}{\mathcal{P}}
\newcommand{\Mod}[1]{\ (\mathrm{mod}\ #1)}
\DeclareMathOperator{\lcm}{lcm}

%\newtheorem*{theorem}{Theorem}
%\newtheorem*{definition}{Definition}
%\newtheorem*{corollary}{Corollary}
%\newtheorem*{lemma}{Lemma}
\newtheorem*{proposition}{Proposition}


\newtcbtheorem[number within=section]{theorem}{Theorem}
{colback=green!5,colframe=green!35!black,fonttitle=\bfseries}{th}

\newtcbtheorem[number within=section]{definition}{Definition}
{colback=blue!5,colframe=blue!35!black,fonttitle=\bfseries}{def}

\newtcbtheorem[number within=section]{corollary}{Corollary}
{colback=yellow!5,colframe=yellow!35!black,fonttitle=\bfseries}{cor}

\newtcbtheorem[number within=section]{lemma}{Lemma}
{colback=red!5,colframe=red!35!black,fonttitle=\bfseries}{lem}

\newtcbtheorem[number within=section]{example}{Example}
{colback=white!5,colframe=white!35!black,fonttitle=\bfseries}{def}

\newtcbtheorem[number within=section]{note}{Important Note}{
        enhanced,
        sharp corners,
        attach boxed title to top left={
            xshift=-1mm,
            yshift=-5mm,
            yshifttext=-1mm
        },
        top=1.5em,
        colback=white,
        colframe=black,
        fonttitle=\bfseries,
        boxed title style={
            sharp corners,
            size=small,
            colback=red!75!black,
            colframe=red!75!black,
        } 
    }{impnote}
\usepackage[utf8]{inputenc}
\usepackage[english]{babel}
\usepackage{fancyhdr}
\usepackage[hidelinks]{hyperref}

\pagestyle{fancy}
\fancyhf{}
\rhead{CSE 101}
\chead{Wednesday, February 18, 2022}
\lhead{Lecture 18}
\rfoot{\thepage}

\setlength{\parindent}{0pt}

\begin{document}

\section{Dynamic Programming}
We continue our discussion on dynamic programming. 

\subsection{Problem: Knapsack}
You are a burglar and are in the process of robbing a home. You have found several valuable items, but the sack you brought can only hold so much weight. What is the best combination of items to steal? 

\bigskip 

Some alternative formulations are: 
\begin{itemize}
    \item You are packing for a trip and your suitcase can only store so much stuff. You want to pack the most useful items for this trip. 
    \item You're building a spacecraft, but launching a spacecraft is incredibly expensive based on its weight. So, you want to decide what are the best modules to put on a spacecraft. 
\end{itemize}

\textbf{Problem Statement:} You have an available list of items. Each item has a non-negative integer weight. Your sack also has a capacity. The goal is to find the collection of items so that:
\begin{enumerate}
    \item The total weight of all of the items is less than or equal to the capacity.
    \item Subject to condition 1, the total value is as large as possible. 
\end{enumerate}
There are two slight variations of this problem:
\begin{itemize}
    \item Each item can be taken as many times as you want. 
    \item Each item can be taken at most once. 
\end{itemize}
In our case, we'll assume that you can take each item as many times as you want. 

\subsubsection{Example: Knapsack}
Suppose you have the following items: 
\begin{center}
    \begin{tabular}{c|c|c}
        \textbf{Item} & \textbf{Weight} & \textbf{Value} \\ 
        \hline 
        A & 1 & \$1 \\ 
        B & 2 & \$4 \\ 
        C & 3 & \$3 \\ 
        D & 4 & \$5
    \end{tabular}
\end{center}
Further suppose that the maximum capacity is 6. What is the best set of items to take, assuming you can only take one copy of each item? 

\begin{mdframed}[]
    We can brute-force this. 
    \begin{itemize}
        \item With weights $1 + 2 + 3 = 6$ (A, B, C), you can get $1 + 4 + 3 = \$8$. 
        \item With weights $4 + 2 = 6$ (B, D), you can get $5 + 4 = \$9$. 
    \end{itemize}
    The other weights smaller than 6 do not give us the best choice. Thus, the best set of items to take is $\{B, D\}$ with a value of \$9. 
\end{mdframed}


\subsubsection{Greedy Algorithms Don't Work}
Problems like this one usually suggest greedy algorithms as a solution. However, greedy algorithms don't work. To see that this is the case, consider the following two counterexamples.
\begin{enumerate}
    \item \underline{Most Valuable Item:} Suppose the greedy algorithm defines ``best'' to be the most valuable item. Suppose you have a knapsack of capacity 6 and the following set of items that you can steal, where you can only take one of each item. 
    \begin{center}
        \begin{tabular}{c|c|c}
            \textbf{Item} & \textbf{Weight} & \textbf{Value} \\ 
            \hline 
            A & 6 & \$10 \\ 
            B & 3 & \$9 \\ 
            C & 3 & \$9 
        \end{tabular}
    \end{center}
    Then: 
    \begin{itemize}
        \item The greedy solution would immediately go for $A$, since it is the most valuable item, which has value \textbf{\$10} and weight 6. 
        \item The optimal solution is $B$ and $C$, which has value \textbf{\$18} and weight 6.
    \end{itemize}

    \item \underline{Biggest Value/Weight:} Suppose the greedy algorithm defines ``best'' to be the item with the highest value to weight ratio. Again, suppose you have a knapsack of capacity 6 and the following set of items that you can steal, where you can only take one of each item. 
    \begin{center}
        \begin{tabular}{c|c|c}
            \textbf{Item} & \textbf{Weight} & \textbf{Value} \\ 
            \hline 
            A & 4 & \$5 \\ 
            B & 3 & \$3 \\ 
            C & 3 & \$3 
        \end{tabular}
    \end{center}
    Then:
    \begin{itemize}
        \item The greedy solution would immediately go for $A$, since $A$ has the highest value/weight ratio, which has value \textbf{\$5} and weight 4. Since the other items have weight 3 and $4 + 3 > 6$, we can't pick any of those items.  
        \item The optimal solution is $B$ and $C$, which has value \textbf{\$6} and weight 6.
    \end{itemize}
\end{enumerate}

\subsubsection{Subproblems}
Suppose we make \emph{one} choice of an item to go into the bag. Then, what is left? 
\begin{itemize}
    \item The remaining items must have a total weight at most $\text{Capacity} - \text{Weight of Item}$. 
    \item The total value of the items is 
    \[\text{Value of Item} + \text{Value of Other Items}\]
    \item Therefore, we want to maximize the value of the other items subject to their weight not exceeding the $\text{Capacity} - \text{Weight of Chosen Item}$.
\end{itemize}
Therefore, the subproblem is 
\[\text{BestValue}(\text{c'})\]
where $c'$ is the new capacity. 

\subsubsection{Recursion}
What is $\text{BestValue}(C)$, where $C$ is the capacity?
\begin{itemize}
    \item If there are \emph{no items} in the bag, then
    \[\text{Value} = 0\]
    \item If item $i$ is in the bag, then the value is given by 
    \[\text{Value} = \text{BestValue}(C - \text{Weight}(i)) + \text{Value}(i)\]
\end{itemize}
So, the best attainable value for a sack with capacity $C$ is the maximum of either: 
\begin{itemize}
    \item 0, \emph{or}
    \item For the item $i$ with weight such that $\text{Weight}(i) \leq C$ and $\text{Weight}(i)$ is maximal,
    \[\text{Value}(i) + \text{BestValue}(C - \text{Weight}(i))\]
\end{itemize}
Therefore, the recursion is given by 
\[\text{BestValue}(C) = \max(0, \max_{\text{Weight}(i) \leq C} \left(\text{Value}(i) + \text{BestValue}(C - \text{Weight}(i))\right))\]
Here, the 0 takes into account the possibility that the sack might be too small to fit any items. 

\subsubsection{Algorithm}
The algorithm, which makes use of the recurrence above, can be described like so: 
\begin{verbatim}
    Knapsack(Wt, Val, Cap):
        Create Array T[0..Cap]
        // Consider all capacities from 0..Cap 
        For C = 0 to Cap: 
            // Accounts for the fact that the sack might be too small to 
            // hold any items
            T[C] = 0
            // This is computing the maximum of all of the possible items
            For items i with Wt(i) <= C:
                If T[C] < Val(i) + T[C - Wt(i)]:
                    T[C] = Val(i) + T[C - Wt(i)]
        Return T[Cap]
\end{verbatim}
To see the runtime, we note that: 
\begin{itemize}
    \item There are $\BigO(\text{Cap})$ subproblems (the outer \code{for}-loop).
    \item For each subproblem, we need to (possibly) consider all of the possible items (the inner \code{for}-loop).
\end{itemize}
Therefore, the runtime is given by 
\[\BigO(\text{Cap} \cdot \text{Num. Items})\]

\subsubsection{Example: Knapsack Redux}
Suppose you have the following items: 
\begin{center}
    \begin{tabular}{c|c|c}
        \textbf{Item} & \textbf{Weight} & \textbf{Value} \\ 
        \hline 
        A & 1 & \$1 \\ 
        B & 2 & \$4 \\ 
        C & 3 & \$3 \\ 
        D & 4 & \$5
    \end{tabular}
\end{center}
Further suppose that the maximum capacity is 6. 

\begin{enumerate}
    \item What is the highest possible value you can get, (assuming you can only take as many of the same item as you want)? 

    \begin{mdframed}[]
        We can make use of the algorithm above. We define the table $T$ like so: 
        \begin{center}
            \begin{tabular}{|c|c|c|c|c|c|c|c|}
                \hline 
                \code{C}         & 0 & 1 & 2 & 3 & 4 & 5 & 6 \\ 
                \hline 
                \code{BestValue} &   &   &   &   &   &   &   \\ 
                \hline 
            \end{tabular}
        \end{center}
        \begin{itemize}
            \item Start at $c = 0$. Then, it's trivial to see that you can't get anything of value since, well, your capacity is 0. So, $T[0] = 0$ and 
            \begin{center}
                \begin{tabular}{|c|c|c|c|c|c|c|c|}
                    \hline 
                    \code{C}         & 0 & 1 & 2 & 3 & 4 & 5 & 6 \\ 
                    \hline 
                    \code{BestValue} & 0 &   &   &   &   &   &   \\ 
                    \hline 
                \end{tabular}
            \end{center}
    
            \item Now, we're at $c = 1$. By the algorithm we can initially set $T[1] = 0$. There are two choices that we can consider:
            \begin{itemize}
                \item We can fit item A in and get $\$1$.
                \item Or, we can fit nothing and get $\$0$.  
            \end{itemize}
            Clearly, the better choice is $\$1$, so we put that into $T[1]$. 
            \begin{center}
                \begin{tabular}{|c|c|c|c|c|c|c|c|}
                    \hline 
                    \code{C}         & 0 & 1 & 2 & 3 & 4 & 5 & 6 \\ 
                    \hline 
                    \code{BestValue} & 0 & 1 &   &   &   &   &   \\ 
                    \hline 
                \end{tabular}
            \end{center}
    
            \item Next, we're at $c = 2$. By the algorithm we can initially set $T[2] = 0$. There are several choices we can consider. 
            \begin{itemize}
                \item We can fit item B in and get $\$4$. 
                \item Or, we can take A, along with $T[1]$, and get $\$1 + \$1 = \$2$. 
            \end{itemize}
            Clearly, \$4 is the better choice, so we put that into $T[2]$. 
            \begin{center}
                \begin{tabular}{|c|c|c|c|c|c|c|c|}
                    \hline 
                    \code{C}         & 0 & 1 & 2 & 3 & 4 & 5 & 6 \\ 
                    \hline 
                    \code{BestValue} & 0 & 1 & 4 &   &   &   &   \\ 
                    \hline 
                \end{tabular}
            \end{center}
    
            \item Next, we're at $c = 3$. There are several choices to consider.
            \begin{itemize}
                \item We can fit item C in and get $\$3$.
                \item We can fit item A, along with what we already have in $T[2]$, and get $\$1 + \$4 = \$5$.
                \item We can fit item B, along with what we already have in $T[1]$, and get $\$2 + \$1 = \$3$. 
            \end{itemize}
            Clearly, \$5 is the better choice, so we put that into $T[3]$. 
            \begin{center}
                \begin{tabular}{|c|c|c|c|c|c|c|c|}
                    \hline 
                    \code{C}         & 0 & 1 & 2 & 3 & 4 & 5 & 6 \\ 
                    \hline 
                    \code{BestValue} & 0 & 1 & 4 & 5 &   &   &   \\ 
                    \hline 
                \end{tabular}
            \end{center}
    
            \item Next, we're at $c = 4$. There are, again, several cases to consider. 
            \begin{itemize}
                \item We can fit item D in and get $\$4$. 
                \item We can fit item C, along with what we have in $T[1]$, and get $\$3 + \$1 = \$4$.
                \item We can fit item B, along with what we have in $T[2]$, and get $\$4 + \$4 = \$8$. 
                \item We can fit item A, along with what we have in $T[3]$, and get $\$1 + \$5 = \$6$.
            \end{itemize}
            Here, we see that \$8 is the best choice, so we put that into $T[4]$. 
            \begin{center}
                \begin{tabular}{|c|c|c|c|c|c|c|c|}
                    \hline 
                    \code{C}         & 0 & 1 & 2 & 3 & 4 & 5 & 6 \\ 
                    \hline 
                    \code{BestValue} & 0 & 1 & 4 & 5 & 8 &   &   \\ 
                    \hline 
                \end{tabular}
            \end{center}
    
            \item Next, we're at $c = 5$. There are several choices to consider. 
            \begin{itemize}
                \item We can fit item D, along with $T[1]$, and get $\$5 + \$1 = \$6$.
                \item We can fit item C, along with $T[2]$, and get $\$3 + \$4 = \$7$.
                \item We can fit item B, along with $T[3]$, and get $\$4 + \$5 = \$9$.
                \item We can fit item A, along with $T[4]$, and get $\$1 + \$8 = \$9$.
            \end{itemize}
            Here, we see that \$9 is the best choice. So, we put that into $T[5]$. 
            \begin{center}
                \begin{tabular}{|c|c|c|c|c|c|c|c|}
                    \hline 
                    \code{C}         & 0 & 1 & 2 & 3 & 4 & 5 & 6 \\ 
                    \hline 
                    \code{BestValue} & 0 & 1 & 4 & 5 & 8 & 9 &   \\ 
                    \hline 
                \end{tabular}
            \end{center}
    
            \item Next, we're at $c = 6$. There are several choices to consider.
            \begin{itemize}
                \item We can fit item D, along with $T[2]$, and get $\$5 + \$4 = \$9$.
                \item We can fit item C, along with $T[3]$, and get $\$3 + \$5 = \$8$.
                \item We can fit item B, along with $T[4]$, and get $\$4 + \$8 = \$12$.
                \item We can fit item A, along with $T[5]$, and get $\$1 + \$9 = \$10$. 
            \end{itemize}
            Here, we see that \$12 is clearly the better choice. So, we put that in $T[6]$.
            \begin{center}
                \begin{tabular}{|c|c|c|c|c|c|c|c|}
                    \hline 
                    \code{C}         & 0 & 1 & 2 & 3 & 4 & 5 & 6 \\ 
                    \hline 
                    \code{BestValue} & 0 & 1 & 4 & 5 & 8 & 9 & 12 \\ 
                    \hline 
                \end{tabular}
            \end{center}
        \end{itemize}
        So, the answer is given by $T[6] = 12$. 
    \end{mdframed}

    \item What was the best possible set of items to collect? 
    \begin{mdframed}[]
        Remember that our table $T$ was populated to have the following values: 
        \begin{center}
            \begin{tabular}{|c|c|c|c|c|c|c|c|}
                \hline 
                \code{C}         & 0 & 1 & 2 & 3 & 4 & 5 & 6 \\ 
                \hline 
                \code{BestValue} & 0 & 1 & 4 & 5 & 8 & 9 & 12 \\ 
                \hline 
            \end{tabular}
        \end{center}
        So, we can backtrack to find out what items we collected. 
        \begin{itemize}
            \item First, we note that, to get \$12, we had to pick item B along with whatever was in $T[4]$. So, we know that \textbf{B} was an item that we collected. But, since we had to use whatever was in $T[4]$, we consider that entry next. 
            \item At $T[4] = 8$, we note that, to get $\$8$, we had to pick item B along with whatever was in $T[2]$. So, we know that \textbf{B} was an item that was collected. Additionally, since we had to use whatever was in $T[2]$, we consider that entry next. 
            \item At $T[2] = 4$, we note that, to get $\$4$, we had to pick item B. So, we know that \textbf{B} was an item that was collected. Since we didn't have to depend on the maximum value at some other entry, we're done. 
        \end{itemize}

        Therefore, we collected three \textbf{B}'s for a total of \textbf{\$12}. 
    \end{mdframed}
\end{enumerate}

\subsubsection{Non-Repeating Items: Subproblems}
Suppose we tried to do this process with non-repeating items. We can try to repeat what we did above (with the subproblems) for this case. That is, suppose we put one item into the bag. 
\begin{itemize}
    \item Then, the remaining items must have total weight at most $\text{Capacity} - \text{Weight of Item}$.
    \item The total value of the items is 
    \[\text{Value of Item} + \text{Value of Other Items}\]
    \item Therefore, we want to maximize the value of the other items subject to their weight not exceeding the $\text{Capacity} - \text{Weight of Chosen Item}$ \textbf{and} such that the chosen item(s) cannot be picked again. 
\end{itemize}
The recursion needs to now keep track of the remaining capacity, since the item that was picked cannotbe used again. This makes the problem a little harder. So, we can try to make several subproblems (where $c'$ is the new capacity).
\begin{enumerate}
    \item Suppose the subproblem we use is $\text{BestValue}_{\neq i}(c')$. In other words, the best value we can achieve without using item $i$ that doesn't go over the capacity. However, we \textbf{cannot} make a recursion out of this. This is because after using item $j$, the remaining items cannot include $i$ and $j$. However, we need to create a recursion such that a recursive subproblem can be solved in terms of other recursive subproblems.
    
    \item Suppose the subproblem we include is $\text{BestValue}_{\neq i, \neq j}(c')$. Again, we cannot make a recursion for this because this would imply the need to exclude a third item. 
    
    \item Suppose the subproblem we use is
    \[\text{BestValue}_{S}(c') = \max\left(0, \max_{i \in S}(\text{Value}(i) - \text{BestValue}_{S \setminus \{i\}}(\text{Cap} - \text{Weight}(i)))\right)\]
    In other words, the best value achievable using only items from $S$ with total weight at most $\text{Cap}$. This recursion \emph{works}, but this doesn't give us a very good algorithm. This is because the number of subproblems is more than $2^{\text{Number of Items}}$ since we're effectively considering every possible subset ($S \setminus \{i\}$). 
    
    \item Instead of trying the above approaches, let's think of something else. Suppose we have items coming along a conveyor belt. You decide, one at a time, whether to add the item to your sack. Then, the question we can think about is -- do we take the last item or leave it alone? 
    \begin{itemize}
        \item If we take the item, the recursion becomes
        \[\text{BestValue}_{\leq n - 1}(\text{Cap} - \text{Weight}(n)) + \text{Value}(n)\]
        \item If we don't take the item, then the recursion becomes 
        \[\text{BestValue}_{\leq n - 1}(\text{Cap})\]
    \end{itemize}
    If we tried to convert this into a recursion, then we only need subproblems of the form 
    \[\text{BestValue}_{\leq k}(\text{Cap})\]
    In other words, what happens to the first $k$ items? Somehow, by imposing this order, we don't need to deal with every possible subset of items.
\end{enumerate}

\subsubsection{Non-Repeating Items: Recursion}
So, our recursion $\text{BestValue}_{\leq k}(\text{Cap})$ is defined by the highest total value of items with the total weight at most \text{Cap} using only items from the first $k$. Thus:
\begin{itemize}
    \item \underline{Base Case:} If $k = 0$, then we can't take any items. So: 
    \[\text{BestValue}_{\leq 0}(\text{Cap}) = 0\]

    \item \underline{Recursion:} $\text{BestValue}_{\leq k}(\text{Cap})$ is the maximum of either: 
    \begin{enumerate}
        \item $\text{BestValue}_{\leq k - 1}(\text{Cap})$: You don't take the item, so we're finding the best value of the $k - 1$ items. 
        \item If $\text{Weight}(k) \leq \text{Cap}$, then $\text{BestValue}_{\leq k - 1}(\text{Cap} - \text{Weight}(k)) + \text{Value}(k)$: You take the item, add on the value, and then add on the best value of the $k - 1$ items. 
    \end{enumerate}
\end{itemize}

\subsubsection{Non-Repeating Items: Example}
Suppose you have the following items: 
\begin{center}
    \begin{tabular}{c|c|c}
        \textbf{Item} & \textbf{Weight} & \textbf{Value} \\ 
        \hline 
        A & 1 & \$1 \\ 
        B & 2 & \$4 \\ 
        C & 3 & \$3 \\ 
        D & 4 & \$5
    \end{tabular}
\end{center}
Further suppose that the maximum capacity is 6. 

\begin{enumerate}
    \item Find the highest possible value you can get, assuming you can only take one copy of each item. 

    \begin{mdframed}[]
        Instead of a one-dimensional table, we have a two-dimensional table $T$. 
        \begin{center}
            \begin{tabular}{|c|c|c|c|c|c|c|c|}
                \hline 
                \code{Cap}  & 0 & 1 & 2 & 3 & 4 & 5 & 6 \\ 
                \hline 
                $\emptyset$ &   &   &   &   &   &   &    \\ 
                \hline 
                \code{A}    &   &   &   &   &   &   &    \\ 
                \hline 
                \code{AB}   &   &   &   &   &   &   &    \\ 
                \hline 
                \code{ABC}  &   &   &   &   &   &   &    \\
                \hline 
                \code{ABCD} &   &   &   &   &   &   &    \\ 
                \hline 
            \end{tabular}
        \end{center}
        Here, we say that $T[\code{x}, n]$ refers to the cell with the row being \code{x} (the items that we can pick) and the column being $n$ (the capacity).
        \begin{itemize}
            \item At the row with the $\emptyset$, we essentially have $k = 0$. In other words, we're not able to pick anything. So, it follows that, regardless of capacity, if there's nothing to pick, we'll end up with $\$0$.
            \begin{center}
                \begin{tabular}{|c|c|c|c|c|c|c|c|}
                    \hline 
                    \code{Cap}  & 0 & 1 & 2 & 3 & 4 & 5 & 6 \\ 
                    \hline 
                    $\emptyset$ & 0 & 0 & 0 & 0 & 0 & 0 & 0  \\ 
                    \hline 
                    \code{A}    &   &   &   &   &   &   &    \\ 
                    \hline 
                    \code{AB}   &   &   &   &   &   &   &    \\ 
                    \hline 
                    \code{ABC}  &   &   &   &   &   &   &    \\
                    \hline 
                    \code{ABCD} &   &   &   &   &   &   &    \\ 
                    \hline 
                \end{tabular}
            \end{center}
    
            \item Suppose we're at the row with \code{A}; that is, we have $k = 1$ so we can pick one item, which in our case is $A$. Now, when the capacity is 0, clearly we cannot pick anything. But, as long as the capacity is greater than or equal to 1, we can pick A (value of 1). So: 
            \begin{center}
                \begin{tabular}{|c|c|c|c|c|c|c|c|}
                    \hline 
                    \code{Cap}  & 0 & 1 & 2 & 3 & 4 & 5 & 6 \\ 
                    \hline 
                    $\emptyset$ & 0 & 0 & 0 & 0 & 0 & 0 & 0  \\ 
                    \hline 
                    \code{A}    & 0 & 1 & 1 & 1 & 1 & 1 & 1  \\ 
                    \hline 
                    \code{AB}   &   &   &   &   &   &   &    \\ 
                    \hline 
                    \code{ABC}  &   &   &   &   &   &   &    \\
                    \hline 
                    \code{ABCD} &   &   &   &   &   &   &    \\ 
                    \hline 
                \end{tabular}
            \end{center}
    
            \item Suppose we're at the row with \code{AB}; that is, we're at $k = 2$ so we can pick two items, which in our case is $A$ or $B$. Now, when the capacity is 0, clearly we cannot pick anything. When the capacity is 1, clearly we can only pick $A$, so we can take $T[\code{A}, 1]$. When the capacity is 2, clearly the most optimal choice is $B$ (value of 4). When the capacity is greater than 2, we can pick $B$ and $T[\code{A}, c]$, where $c$ is the capacity such that $c > 2$ (value of 5).
            \begin{center}
                \begin{tabular}{|c|c|c|c|c|c|c|c|}
                    \hline 
                    \code{Cap}  & 0 & 1 & 2 & 3 & 4 & 5 & 6 \\ 
                    \hline 
                    $\emptyset$ & 0 & 0 & 0 & 0 & 0 & 0 & 0  \\ 
                    \hline 
                    \code{A}    & 0 & 1 & 1 & 1 & 1 & 1 & 1  \\ 
                    \hline 
                    \code{AB}   & 0 & 1 & 4 & 5 & 5 & 5 & 5  \\ 
                    \hline 
                    \code{ABC}  &   &   &   &   &   &   &    \\
                    \hline 
                    \code{ABCD} &   &   &   &   &   &   &    \\ 
                    \hline 
                \end{tabular}
            \end{center}
    
            \item Suppose we're at the row with \code{ABC}; that is, we're at $k = 3$ so we can pick three items, which in our case is $A$ or $B$ or $C$. 
            \begin{itemize}
                \item Now, when the capacity is 0, clearly we cannot pick anything. 
                
                \item When the capacity is 1, clearly we can pick $T[\code{AB}, 1]$ (value of 1). 
                
                \item When the capacity is 2, clearly the most optimal choice is $T[\code{AB}, 2]$ (value of 4). This is because there isn't any space for item $C$. 
                
                \item When the capacity is 3, clearly the most optimal choice is just $T[\code{AB}, 3]$. Note that if we picked $C$, then we would be worse-off. 
                
                \item When the capacity is 4, clearly the most optimal choice is just $T[\code{AB}, 4]$. Note that if we picked $C$, then we would be worse-off. 
                
                \item When the capacity is 5, we can pick $C$ and also $T[\code{AB}, 2]$ (value of 7). 
                
                \item When the capacity is 6, we can pick $C$ and also $T[\code{AB}, 3]$ (value of 8). 
            \end{itemize}
            Thus: 
            \begin{center}
                \begin{tabular}{|c|c|c|c|c|c|c|c|}
                    \hline 
                    \code{Cap}  & 0 & 1 & 2 & 3 & 4 & 5 & 6 \\ 
                    \hline 
                    $\emptyset$ & 0 & 0 & 0 & 0 & 0 & 0 & 0  \\ 
                    \hline 
                    \code{A}    & 0 & 1 & 1 & 1 & 1 & 1 & 1  \\ 
                    \hline 
                    \code{AB}   & 0 & 1 & 4 & 5 & 5 & 5 & 5  \\ 
                    \hline 
                    \code{ABC}  & 0 & 1 & 4 & 5 & 5 & 7 & 8  \\
                    \hline 
                    \code{ABCD} &   &   &   &   &   &   &    \\ 
                    \hline 
                \end{tabular}
            \end{center}
    
            \item Suppose we're at the row with \code{ABCD}; that is, we're at $k = 4$ so we can pick all four items if we can, which in our case is $A$ or $B$ or $C$ or $D$. 
            \begin{itemize}
                \item Now, when the capacity is 0, clearly we cannot pick anything. 
                
                \item When the capacity is 1, clearly we can pick $T[\code{ABC}, 1]$ (value of 1). 
                
                \item When the capacity is 2, clearly the most optimal choice is $T[\code{ABC}, 2]$ (value of 4). 
                
                \item When the capacity is 3, clearly the most optimal choice is just $T[\code{ABC}, 3]$. 
                
                \item When the capacity is 4, clearly the most optimal choice is just $T[\code{ABC}, 4]$. Here, we could also just pick $D$.
                
                \item When the capacity is 5, the optimal choice is just $T[\code{ABC}, 3]$ (value of 7). Note that the other option is $D$ and $A$, which is worse. 
                
                \item When the capacity is 6, we can pick $D$ and also $T[\code{ABC}, 2]$ (value of 9). 
            \end{itemize}
            Thus: 
            \begin{center}
                \begin{tabular}{|c|c|c|c|c|c|c|c|}
                    \hline 
                    \code{Cap}  & 0 & 1 & 2 & 3 & 4 & 5 & 6 \\ 
                    \hline 
                    $\emptyset$ & 0 & 0 & 0 & 0 & 0 & 0 & 0  \\ 
                    \hline 
                    \code{A}    & 0 & 1 & 1 & 1 & 1 & 1 & 1  \\ 
                    \hline 
                    \code{AB}   & 0 & 1 & 4 & 5 & 5 & 5 & 5  \\ 
                    \hline 
                    \code{ABC}  & 0 & 1 & 4 & 5 & 5 & 7 & 8  \\
                    \hline 
                    \code{ABCD} & 0 & 1 & 4 & 5 & 5 & 7 & 9  \\ 
                    \hline 
                \end{tabular}
            \end{center}
        \end{itemize}
        Thus, the final answer is given by $T[\code{ABCD}, 6] = 9$.
    \end{mdframed}

    \item Find the items that you took to get the maximum value found in the previous part. 
    
    \begin{mdframed}[]
        Using the table $T$ from the previous part: 
        \begin{center}
            \begin{tabular}{|c|c|c|c|c|c|c|c|}
                \hline 
                \code{Cap}  & 0 & 1 & 2 & 3 & 4 & 5 & 6 \\ 
                \hline 
                $\emptyset$ & 0 & 0 & 0 & 0 & 0 & 0 & 0  \\ 
                \hline 
                \code{A}    & 0 & 1 & 1 & 1 & 1 & 1 & 1  \\ 
                \hline 
                \code{AB}   & 0 & 1 & 4 & 5 & 5 & 5 & 5  \\ 
                \hline 
                \code{ABC}  & 0 & 1 & 4 & 5 & 5 & 7 & 8  \\
                \hline 
                \code{ABCD} & 0 & 1 & 4 & 5 & 5 & 7 & 9  \\ 
                \hline 
            \end{tabular}
        \end{center}
        We can backtrack to find out what items we collected. 
        \begin{itemize}
            \item Starting at $T[\code{ABCD}, 6]$, note that, in order to get to this position, we picked $T[\code{ABC}, 2]$ and $D$. So, we know that \textbf{D} was an item that we collected, and we additionally consider $T[\code{ABC}, 4]$. 
            \item Now that we're at $T[\code{ABC}, 2]$, note that, in order to get to this position, we picked $T[\code{AB}, 2]$. As we didn't pick up any other items along th way, we can just consider $T[\code{AB}, 2]$. 
            \item Now that we're at $T[\code{AB}, 2]$, note that, in order to get to this position, we picked \textbf{B}. Since we didn't consider any other entries in the table, we're done. 
        \end{itemize}
        Thus, the solution is \textbf{B} and \textbf{D}. 
    \end{mdframed}
\end{enumerate}

\end{document}