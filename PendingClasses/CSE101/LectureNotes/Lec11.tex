\documentclass[letterpaper]{article}
\usepackage[margin=1in]{geometry}
\usepackage[utf8]{inputenc}
\usepackage{textcomp}
\usepackage{amssymb}
\usepackage{natbib}
\usepackage{graphicx}
\usepackage{gensymb}
\usepackage{amsthm, amsmath, mathtools}
\usepackage[dvipsnames]{xcolor}
\usepackage{enumerate}
\usepackage{mdframed}
\usepackage[most]{tcolorbox}
\usepackage{csquotes}
% https://tex.stackexchange.com/questions/13506/how-to-continue-the-framed-text-box-on-multiple-pages

\tcbuselibrary{theorems}

\newcommand{\R}{\mathbb{R}}
\newcommand{\Z}{\mathbb{Z}}
\newcommand{\N}{\mathbb{N}}
\newcommand{\Q}{\mathbb{Q}}
\newcommand{\C}{\mathbb{C}}
\newcommand{\code}[1]{\texttt{#1}}
\newcommand{\mdiamond}{$\diamondsuit$}
\newcommand{\PowerSet}{\mathcal{P}}
\newcommand{\Mod}[1]{\ (\mathrm{mod}\ #1)}
\DeclareMathOperator{\lcm}{lcm}

%\newtheorem*{theorem}{Theorem}
%\newtheorem*{definition}{Definition}
%\newtheorem*{corollary}{Corollary}
%\newtheorem*{lemma}{Lemma}
\newtheorem*{proposition}{Proposition}


\newtcbtheorem[number within=section]{theorem}{Theorem}
{colback=green!5,colframe=green!35!black,fonttitle=\bfseries}{th}

\newtcbtheorem[number within=section]{definition}{Definition}
{colback=blue!5,colframe=blue!35!black,fonttitle=\bfseries}{def}

\newtcbtheorem[number within=section]{corollary}{Corollary}
{colback=yellow!5,colframe=yellow!35!black,fonttitle=\bfseries}{cor}

\newtcbtheorem[number within=section]{lemma}{Lemma}
{colback=red!5,colframe=red!35!black,fonttitle=\bfseries}{lem}

\newtcbtheorem[number within=section]{example}{Example}
{colback=white!5,colframe=white!35!black,fonttitle=\bfseries}{def}

\newtcbtheorem[number within=section]{note}{Important Note}{
        enhanced,
        sharp corners,
        attach boxed title to top left={
            xshift=-1mm,
            yshift=-5mm,
            yshifttext=-1mm
        },
        top=1.5em,
        colback=white,
        colframe=black,
        fonttitle=\bfseries,
        boxed title style={
            sharp corners,
            size=small,
            colback=red!75!black,
            colframe=red!75!black,
        } 
    }{impnote}
\usepackage[utf8]{inputenc}
\usepackage[english]{babel}
\usepackage{fancyhdr}
\usepackage[hidelinks]{hyperref}

\pagestyle{fancy}
\fancyhf{}
\rhead{CSE 101}
\chead{Friday, January 28, 2022}
\lhead{Lecture 11}
\rfoot{\thepage}

\setlength{\parindent}{0pt}

\begin{document}

\section{Divide and Conquer}
The idea behind divide and conquer is as follows: 
\begin{enumerate}
    \item Break problems into similar pieces. 
    \item Solve pirces recursively. 
    \item Recombine the pieces to get an answer.
\end{enumerate}

\subsection{Integer Multiplication}
Suppose we're given two $n$-bit numbers and are asked to find their product. 

\subsubsection{Naive Algorithm}
The naive algorithm is to do multiplication like we would from elementary school. This runs in $\BigO(n^2)$ time because we need to write down $\BigO(n^2)$ bits of numbers to add (addition is done in linear time and is omitted).

\subsubsection{Improving the Algorithm: Two-Digit Multiplication}
If we multiplied $ab \times cd$ digit-wise, we would get: 
\[ab \times cd = (ac)(bc + ad)(bd)\]
This requires 4 one-digit multiplications and one addition. The \textbf{trick} is to compute $ac$, $bd$, $(a + b)(c + d)$. We note that: 
\[bc + ad = (a + b)(c + d) - ac - bd\]
This requires \emph{3} one-digit multiplications and 4 addition/subtractions.

% TODO 

\subsection{Generalization}
We will often get runtime recurrences with divide and conquer looking something like: 
\[T(n) = \begin{cases}
    O(1) & n = O(1) \\ 
    aT\left(\frac{n}{b} + O(1)\right) + O(n^d) & \text{Otherwise}
\end{cases}\]
Here, the second line is saying $a$ subproblems of size $\frac{n}{b}$. 

\subsubsection{Tracking Recursive Calls}
We have: 
\begin{itemize}
    \item 1 recurisve calls of size $n$
    \item $a$ recursive calls of size $n / b + O(1)$
    \item $a^2$ recursive calls of size $n / b^2 + O(1)$
    \item \dots
    \item $a^k$ recursive calls of size $n / b^k + O(1)$
\end{itemize}
So, the total runtime is: 
\begin{equation*}
    \begin{aligned}
        \text{Total Runtime } &= \sum_{k = 0}^{\log_{b}(n)} a^k O\left(\left(\frac{n}{b^k}\right)^d\right) \\ 
            &= O(n^d) \sum_{k = 0}^{\log){b}(n)} \left(\frac{a}{b^d}\right)^k
    \end{aligned}
\end{equation*}

There are several cases to consider. 
\begin{enumerate}
    \item $a > b^d$: The runtime would be $O(n^{\log){b}(a)})$.
    \item $a < b^d$: The runtime is $O(n^d)$. 
    \item $a = b^d$: The runtime is $O(n^d \log(n))$. 
\end{enumerate}

\subsubsection{Master Theorem}
\begin{theorem}{Master Theorem}{}
    Let $T(n)$ be given by the recurrence:
    \[T(n) = \begin{cases}
        O(1) & n = O(1) \\ 
        aT\left(\frac{n}{b} + O(1)\right) + O(n^d) & \text{Otherwise}
    \end{cases}\]
    Then we have: 
    \[T(n) = \begin{cases}
        O(n^{\log_{b}(a)}) & a > b^d \\ 
        O(n^d \log(n)) & a = b^d \\ 
        O(n^d) & a < b^d
    \end{cases}\]
\end{theorem}

\end{document}