\documentclass[letterpaper]{article}
\usepackage[margin=1in]{geometry}
\usepackage[utf8]{inputenc}
\usepackage{textcomp}
\usepackage{amssymb}
\usepackage{natbib}
\usepackage{graphicx}
\usepackage{gensymb}
\usepackage{amsthm, amsmath, mathtools}
\usepackage[dvipsnames]{xcolor}
\usepackage{enumerate}
\usepackage{mdframed}
\usepackage[most]{tcolorbox}
\usepackage{csquotes}
% https://tex.stackexchange.com/questions/13506/how-to-continue-the-framed-text-box-on-multiple-pages

\tcbuselibrary{theorems}

\newcommand{\R}{\mathbb{R}}
\newcommand{\Z}{\mathbb{Z}}
\newcommand{\N}{\mathbb{N}}
\newcommand{\Q}{\mathbb{Q}}
\newcommand{\C}{\mathbb{C}}
\newcommand{\code}[1]{\texttt{#1}}
\newcommand{\mdiamond}{$\diamondsuit$}
\newcommand{\PowerSet}{\mathcal{P}}
\newcommand{\Mod}[1]{\ (\mathrm{mod}\ #1)}
\DeclareMathOperator{\lcm}{lcm}

%\newtheorem*{theorem}{Theorem}
%\newtheorem*{definition}{Definition}
%\newtheorem*{corollary}{Corollary}
%\newtheorem*{lemma}{Lemma}
\newtheorem*{proposition}{Proposition}


\newtcbtheorem[number within=section]{theorem}{Theorem}
{colback=green!5,colframe=green!35!black,fonttitle=\bfseries}{th}

\newtcbtheorem[number within=section]{definition}{Definition}
{colback=blue!5,colframe=blue!35!black,fonttitle=\bfseries}{def}

\newtcbtheorem[number within=section]{corollary}{Corollary}
{colback=yellow!5,colframe=yellow!35!black,fonttitle=\bfseries}{cor}

\newtcbtheorem[number within=section]{lemma}{Lemma}
{colback=red!5,colframe=red!35!black,fonttitle=\bfseries}{lem}

\newtcbtheorem[number within=section]{example}{Example}
{colback=white!5,colframe=white!35!black,fonttitle=\bfseries}{def}

\newtcbtheorem[number within=section]{note}{Important Note}{
        enhanced,
        sharp corners,
        attach boxed title to top left={
            xshift=-1mm,
            yshift=-5mm,
            yshifttext=-1mm
        },
        top=1.5em,
        colback=white,
        colframe=black,
        fonttitle=\bfseries,
        boxed title style={
            sharp corners,
            size=small,
            colback=red!75!black,
            colframe=red!75!black,
        } 
    }{impnote}
\usepackage[utf8]{inputenc}
\usepackage[english]{babel}
\usepackage{fancyhdr}
\usepackage[hidelinks]{hyperref}

\pagestyle{fancy}
\fancyhf{}
\rhead{CSE 101}
\chead{Wednesday, January 12, 2022}
\lhead{Lecture 5}
\rfoot{\thepage}

\setlength{\parindent}{0pt}

\begin{document}

\section{Connectivity in Digraphs}
How do we achieve a clean description of reachability in a directed graph? 
\begin{itemize}
    \item Note that reachability is no longer symmetric. That is, we can reach $w$ from $v$ but not the other way around. 
    \item Can we maybe allow reachability in either direction?
    \item Maybe we can allow the ability to follow edges in either direction? However, this treats a digraph as an undirected graph. 
\end{itemize}

\subsection{Strongly Connected Components}
\begin{definition}{Strongly Connected Components}{}
    In a directed graph $G$, two vertices $v$ and $w$ are in the same \textbf{strongly connected component} if $v$ is reachable from $w$ and $w$ is reachable from $v$.
\end{definition}

\subsection{Equivalence Relation}
Let $v \sim w$ if $v$ is reachable from $w$ and vice versa.
\begin{proposition}
    This is an equivalence relation. Namely:
    \begin{itemize}
        \item $v \sim v$ ($v$ is reachable from itself).
        \item $v \sim w \implies w \sim v$ (relation is symmetric).
        \item $u \sim v \text{ and } v \sim w \implies u \sim w$.
    \end{itemize}
\end{proposition}
If we take any $v$, the set of all $w$ so that $v \sim w$ is the component of $v$. Everything connects to everything else in this equivalence class and does not connect (both ways) to anything outside. 

\subsection{Connectivity}
Do strongly connected components completely describe connectivity in $G$?
\begin{itemize}
    \item \textbf{No}. In directed cases, we can have an edge between strongly connected components.
\end{itemize}

\subsection{Metagraph}
\begin{definition}{Metagraph}{}
    The \textbf{metagraph} of a directed graph $G$ is a graph whose vertices are the strongly connected components of $G$.
\end{definition}


\subsubsection{Result}
\begin{theorem}{}{}
    The metagraph of any directed graph is a DAG. 
\end{theorem}

\subsection{Computing SCCs}
Given a directed graph $G$, compute the SCCs of $G$ and its metagraph. 

\subsubsection{Easy Algorithm}
\begin{itemize}
    \item For each $v$, compute vertices reachable from $v$.
    \item Find pairs $v$, $w$ so that $v$ reachable from $w$ and vice versa.
    \item For each $v$, the corresponding $w$'s are in the SCC of $v$.
\end{itemize}
The runtime is $O(|V|(|V| + |E|))$.

\subsubsection{Better Algorithm}
Suppose that $SCC(v)$ is a sink in the metagraph.
\begin{itemize}
    \item $G$ has no edges from $SCC(v)$ to another $SCC$. 
    \item We can run \code{explore(v)} to find all vertices reachable from $v$. This will contain all vertices in $SCC(v)$. But, it contains no other vertices. 
    \item If $v$ is in the SCC, then \code{explore(v)} finds exactly $v$'s component. 
\end{itemize}
With this observation, we consider the following strategy:
\begin{itemize}
    \item Find $v$ in a sink SCC of $G$. 
    \item Run \code{explore(v)} to find the component $C_1$.
    \item Repeat process on $G \setminus C_1$.
\end{itemize}
The problem is, how do we find $v$ that is in a sink?
\begin{proposition}
    Let $C_1$ and $C_2$ be SCCs of $G$ with an edge from $C_1$ to $C_2$. If we run \code{DFS} on $G$, the largest postorder number of any vertex in $C_1$ will be larger than the largest postorder number in $C_2$. 
\end{proposition}
The reason why we care is because if $v$ is the vertex with the largest postorder number, then: 
\begin{itemize}
    \item There is no edge to $SCC(V)$ from any other SCC. 
    \item SCC is a \underline{source} SCC.
\end{itemize}
However, we wanted a \emph{sink} SCC. So, how do we relate these two?
\begin{itemize}
    \item A sink is like a source, only with edges going in the opposite direction.
\end{itemize}
So, we define a reverse graph like so: 
\begin{definition}{}{}
    Given a directed graph $G$, the \textbf{reverse graph} of $G$ (denoted $G^R$) is obtained by reversing the directions of all the edges of $G$.
\end{definition}
Some properties of reverse graphs are: 
\begin{itemize}
    \item $G$ and $G^R$ have the same SCCs. 
    \item The sink SCCs of $G$ are the source SCCs of $G^R$. 
    \item The source SCCs of $G$ are the sink SCCs of $G^R$. 
\end{itemize}

\end{document}