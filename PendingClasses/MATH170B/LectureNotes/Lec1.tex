\documentclass[letterpaper]{article}
\usepackage[margin=1in]{geometry}
\usepackage[utf8]{inputenc}
\usepackage{textcomp}
\usepackage{amssymb}
\usepackage{natbib}
\usepackage{graphicx}
\usepackage{gensymb}
\usepackage{amsthm, amsmath, mathtools}
\usepackage[dvipsnames]{xcolor}
\usepackage{enumerate}
\usepackage{mdframed}
\usepackage[most]{tcolorbox}
\usepackage{csquotes}
% https://tex.stackexchange.com/questions/13506/how-to-continue-the-framed-text-box-on-multiple-pages

\tcbuselibrary{theorems}

\newcommand{\R}{\mathbb{R}}
\newcommand{\Z}{\mathbb{Z}}
\newcommand{\N}{\mathbb{N}}
\newcommand{\Q}{\mathbb{Q}}
\newcommand{\C}{\mathbb{C}}
\newcommand{\code}[1]{\texttt{#1}}
\newcommand{\mdiamond}{$\diamondsuit$}
\newcommand{\PowerSet}{\mathcal{P}}
\newcommand{\Mod}[1]{\ (\mathrm{mod}\ #1)}
\DeclareMathOperator{\lcm}{lcm}

%\newtheorem*{theorem}{Theorem}
%\newtheorem*{definition}{Definition}
%\newtheorem*{corollary}{Corollary}
%\newtheorem*{lemma}{Lemma}
\newtheorem*{proposition}{Proposition}


\newtcbtheorem[number within=section]{theorem}{Theorem}
{colback=green!5,colframe=green!35!black,fonttitle=\bfseries}{th}

\newtcbtheorem[number within=section]{definition}{Definition}
{colback=blue!5,colframe=blue!35!black,fonttitle=\bfseries}{def}

\newtcbtheorem[number within=section]{corollary}{Corollary}
{colback=yellow!5,colframe=yellow!35!black,fonttitle=\bfseries}{cor}

\newtcbtheorem[number within=section]{lemma}{Lemma}
{colback=red!5,colframe=red!35!black,fonttitle=\bfseries}{lem}

\newtcbtheorem[number within=section]{example}{Example}
{colback=white!5,colframe=white!35!black,fonttitle=\bfseries}{def}

\newtcbtheorem[number within=section]{note}{Important Note}{
        enhanced,
        sharp corners,
        attach boxed title to top left={
            xshift=-1mm,
            yshift=-5mm,
            yshifttext=-1mm
        },
        top=1.5em,
        colback=white,
        colframe=black,
        fonttitle=\bfseries,
        boxed title style={
            sharp corners,
            size=small,
            colback=red!75!black,
            colframe=red!75!black,
        } 
    }{impnote}
\usepackage[utf8]{inputenc}
\usepackage[english]{babel}
\usepackage{fancyhdr}
\usepackage[hidelinks]{hyperref}

\pagestyle{fancy}
\fancyhf{}
\rhead{Math 170B}
\chead{Monday, April 03, 2023}
\lhead{Lecture 1}
\rfoot{\thepage}

\setlength{\parindent}{0pt}

\begin{document}

\section{Basic Concepts \& Taylor's Theorem (Section 1.1)}
Let $f(x): \R \mapsto \R$ be a general function (typically nonlinear). We may also write $f([a, b]): [a, b] \mapsto \R$ to denote a general function over an interval $[a, b]$. We also write $C^n (\R)$ or $C^n ([a, b])$ to denote the \emph{classes} of $n$-times continuously differentiable functions. We write $C^0 (\R) = C(\R)$ to mean the class of only continuous functions. 

\begin{mdframed}
    (Example.) $f(x) = |x|$ is continuous but is not differentiable at $x = 0$. Thus, $f(x) = |x|$ is in $C^0 (\R)$. 

    \bigskip 

    $f(x) = e^x$ is in $C^\infty (\R)$.
\end{mdframed}

\subsection{Taylor Series}
\begin{theorem}{Taylor Series with Lagrange Remainder}{}
    Let $f \in C^m ([a, b])$, with the derivative $f^{(m + 1)}$ exists on the open interval $(a, b)$ and with values $x, c \in [a, b]$. Then, 
    \[f(x) = \sum_{k = 0}^{m} \frac{f^{(k)} (c)}{k!} (x - c)^k + E_{m}(\psi),\] 
    where $E_{m}(\psi)$ is the remainder (or error) term. We define 
    \[E_{m}(\psi) = \frac{f^{(m + 1)}(\psi)}{(m + 1)!} (x - c)^{(m + 1)},\] 
    where $c < \psi < x$ or $x < \psi < c$ depending on the values of $x$ and $c$. 
\end{theorem}

\begin{mdframed}
    (Example.) Suppose $f(x) = \ln(x)$ with interval $[a, b] = [1, 10]$ and $c = e^1.$ Let $|x - c| < 1$ (i.e., $x$ is relatively close to $c$). Then, 
    \[f^{(1)}(x) = f'(x) = \frac{1}{x}.\]
    \[f^{(2)}(x) = f''(x) = -\frac{1}{x^2}.\]
    \[f^{(3)}(x) = f'''(x) = \frac{2}{x^3}.\]
    \[f^{(4)}(x) = -2 \cdot 3 \frac{1}{x^4}.\]
    \[f^{(5)}(x) = 2 \cdot 3 \cdot 4 \frac{1}{x^5}.\]
    \[\vdots\]
    \[f^{(k)}(x) = (-1)^{k - 1} (k - 1)! \frac{1}{x^k}\]
    for $k = 1, 2, \hdots$. Then, 
    \[E_m (\psi) = \frac{1}{(m + 1)!} f^{(m + 1)}(\psi) (x - c)^{m + 1}.\] Using the value of $c = e^1$, \[f^{(k)}(c) = (-1)^{k - 1}(k - 1)! \frac{1}{e^k}.\] Combining everything, we end up with 
    \begin{equation*}
        \begin{aligned}
            f(x) &= \sum_{k = 0}^{m} \frac{f^{(k)} (c)}{k!} (x - c)^k + E_{m}(\psi) \\ 
                &= 1 + \sum_{k = 1}^{m} (-1)^{k - 1} \frac{(k - 1)!}{k!} \frac{1}{e^k} + E_{m}(\psi) \\ 
                &= 1 + \sum_{k = 1}^{m} (-1)^{k - 1} \frac{1}{k} \frac{1}{e^x}(x - e)^k + \frac{1}{(m + 1)} f^{(m + 1)} (\psi) (x - e)^{m + 1}.
        \end{aligned}
    \end{equation*}
    How many terms in this approximation do we need in order for the error to be below a certain amount? In other words, what is the minimum $m$ so that a Taylor expansion is accurate up to $\frac{1}{\alpha} \cdot 10^{-9}$? We have 
    \[|E_m (\psi)| \leq \frac{1}{\alpha} \cdot 10^{-9}.\]
    We already computed the remainder, so 
    \[\left| \frac{1}{(m + 1)} f^{(m + 1)} (\psi) (x - e)^{m + 1}\right| \leq \frac{1}{\alpha} \cdot 10^{-9}.\] Using $|x - e| < 1$, we want to find $m$. Thus, $|\psi| < 1$. 
\end{mdframed}


\end{document}