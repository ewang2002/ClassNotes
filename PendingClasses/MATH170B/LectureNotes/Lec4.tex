\documentclass[letterpaper]{article}
\usepackage[margin=1in]{geometry}
\usepackage[utf8]{inputenc}
\usepackage{textcomp}
\usepackage{amssymb}
\usepackage{natbib}
\usepackage{graphicx}
\usepackage{gensymb}
\usepackage{amsthm, amsmath, mathtools}
\usepackage[dvipsnames]{xcolor}
\usepackage{enumerate}
\usepackage{mdframed}
\usepackage[most]{tcolorbox}
\usepackage{csquotes}
% https://tex.stackexchange.com/questions/13506/how-to-continue-the-framed-text-box-on-multiple-pages

\tcbuselibrary{theorems}

\newcommand{\R}{\mathbb{R}}
\newcommand{\Z}{\mathbb{Z}}
\newcommand{\N}{\mathbb{N}}
\newcommand{\Q}{\mathbb{Q}}
\newcommand{\C}{\mathbb{C}}
\newcommand{\code}[1]{\texttt{#1}}
\newcommand{\mdiamond}{$\diamondsuit$}
\newcommand{\PowerSet}{\mathcal{P}}
\newcommand{\Mod}[1]{\ (\mathrm{mod}\ #1)}
\DeclareMathOperator{\lcm}{lcm}

%\newtheorem*{theorem}{Theorem}
%\newtheorem*{definition}{Definition}
%\newtheorem*{corollary}{Corollary}
%\newtheorem*{lemma}{Lemma}
\newtheorem*{proposition}{Proposition}


\newtcbtheorem[number within=section]{theorem}{Theorem}
{colback=green!5,colframe=green!35!black,fonttitle=\bfseries}{th}

\newtcbtheorem[number within=section]{definition}{Definition}
{colback=blue!5,colframe=blue!35!black,fonttitle=\bfseries}{def}

\newtcbtheorem[number within=section]{corollary}{Corollary}
{colback=yellow!5,colframe=yellow!35!black,fonttitle=\bfseries}{cor}

\newtcbtheorem[number within=section]{lemma}{Lemma}
{colback=red!5,colframe=red!35!black,fonttitle=\bfseries}{lem}

\newtcbtheorem[number within=section]{example}{Example}
{colback=white!5,colframe=white!35!black,fonttitle=\bfseries}{def}

\newtcbtheorem[number within=section]{note}{Important Note}{
        enhanced,
        sharp corners,
        attach boxed title to top left={
            xshift=-1mm,
            yshift=-5mm,
            yshifttext=-1mm
        },
        top=1.5em,
        colback=white,
        colframe=black,
        fonttitle=\bfseries,
        boxed title style={
            sharp corners,
            size=small,
            colback=red!75!black,
            colframe=red!75!black,
        } 
    }{impnote}
\usepackage[utf8]{inputenc}
\usepackage[english]{babel}
\usepackage{fancyhdr}
\usepackage[hidelinks]{hyperref}

\pagestyle{fancy}
\fancyhf{}
\rhead{Math 170B}
\chead{Monday, April 10, 2023}
\lhead{Lecture 4}
\rfoot{\thepage}

\setlength{\parindent}{0pt}

\begin{document}

\section{Machine (Floating-Point) Numbers (Section 1.3, Continued)}
\begin{mdframed}
    (Example.) Let $x = \frac{2}{3}$. 
    
    \begin{enumerate}[(1)]
        \item What is the binary form of $x$?
        \begin{mdframed}
            The algorithm for finding the binary form of the decimal is as follows: 
            \begin{itemize}
                \item Given $x$, multiply it by 2. If the integer part of the result is 1, set the $i$th bit to 1. Otherwise, set it to 0. 
                \item If the $i$th bit is 1, subtract $x \times 2$ by 1. 
                \item Repeat the above until one of the following occurs: 
                \begin{itemize}
                    \item You hit exactly 1, or 
                    \item You hit 23 bits after the binary point (most likely, you'll see that the bits repeat in some way). 
                \end{itemize}
            \end{itemize}
            \[\begin{aligned}
                \frac{2}{3} \cdot 2 &= \frac{4}{3} \geq 1 &\implies 1 \\ 
                \frac{1}{3} \cdot 2 &= \frac{2}{3} < 0 &\implies 0 \\
                \frac{2}{3} \cdot 2 &= \frac{4}{3} \geq 1 &\implies 1 \\ 
                \frac{1}{3} \cdot 2 &= \frac{2}{3} < 0 &\implies 0 \\
                \frac{2}{3} \cdot 2 &= \frac{4}{3} \geq 1 &\implies 1 \\ 
                \frac{1}{3} \cdot 2 &= \frac{2}{3} < 0 &\implies 0 \\
                \frac{2}{3} \cdot 2 &= \frac{4}{3} \geq 1 &\implies 1 \\ 
                \frac{1}{3} \cdot 2 &= \frac{2}{3} < 0 &\implies 0 \\
                &\vdots
            \end{aligned}\]
            This gives us the binary representation \code{0.1010101010...}. Normalizing this gives us 
            \[(1.01010101010101010101010101010101010\hdots)_2 \times 2^{-1}.\]
        \end{mdframed}
        \item Find $x_\_$ and $x_+$.
        \begin{mdframed}
            Note that $x_\_$ is just what we found in the previous step, but with 23 bits to the right of the binary point,
            \[(1.01010101010101010101010)_2 \times 2^{-1}.\]
            Then, $x_+$ is 
            \[(1.01010101010101010101011)_2 \times 2^{-1}.\] 
        \end{mdframed}
        \item What is $\text{fl}(x)$?
        \begin{mdframed}
            We now consider $x - x_\_$ and $x - x_+$. Here, 
            \begin{itemize}
                \item For $x - x_\_$, 
                \begin{verbatim}
                   1111111111222222222233333333...
          1234567890123456789012345678901234567...

x       1.0101010101010101010101010101010101010 x 2^-1
- x_    1.01010101010101010101010               x 2^-1
        ===============================================
        0.0000000000000000000000010101010101010 x 2^-1\end{verbatim}
                This gives us \[0.101\hdots \times 2^{-24} = \frac{2}{3} \times 2^{-24}.\]
                    
                \item For $x - x_+$, we know this will be negative since $x_+ > x$ (since we're rounding \emph{up}). So, generally, we'll consider $x_+ - x$: 
                \[x_+ - x = (x_+ - x_\_) - (x - x_\_) = \left(2^{-23} \times 2^{-1}\right) - \left(\frac{2}{3} \times 2^{-24}\right) = \frac{1}{3} \times 2^{-24}.\]
                Note that $(x_+ - x_\_) = \left(2^{-23} \times 2^{-1}\right)$ came from 
                \begin{verbatim}
x+      1.01010101010101010101011               x 2^-1
- x_    1.01010101010101010101010               x 2^-1
        ===============================================
        0.00000000000000000000001               x 2^-1\end{verbatim}
                so, we have $2^{-23} \times 2^{-1}$. 
            \end{itemize}
            Notice that $x_{+}$ has the smaller error, so $\text{fl}(x) = x^* = x_{+}$. 
        \end{mdframed}
        \item What is the relative round-off error?  
        \begin{mdframed}
            This is 
            \[\frac{|\text{fl}(x) - x|}{|x|} = \frac{|x_+ - x|}{|x|} = \frac{\frac{1}{3} \times 2^{-24}}{\frac{2}{3}} = 2^{-25}.\]

            Notice that \[\frac{|x^* - x|}{|x|} \leq 2^{-24}.\]
        \end{mdframed}
    \end{enumerate}
    
\end{mdframed}


\end{document}