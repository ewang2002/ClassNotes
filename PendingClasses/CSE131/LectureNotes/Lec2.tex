\documentclass[letterpaper]{article}
\usepackage[margin=1in]{geometry}
\usepackage[utf8]{inputenc}
\usepackage{textcomp}
\usepackage{amssymb}
\usepackage{natbib}
\usepackage{graphicx}
\usepackage{gensymb}
\usepackage{amsthm, amsmath, mathtools}
\usepackage{xcolor}
\usepackage{enumerate}
\usepackage{framed}
\usepackage{tcolorbox}
\tcbuselibrary{theorems}

\newcommand{\R}{\mathbb{R}}
\newcommand{\Z}{\mathbb{Z}}
\newcommand{\N}{\mathbb{N}}
\newcommand{\Q}{\mathbb{Q}}
\newcommand{\code}[1]{\texttt{#1}}
\newcommand{\mdiamond}{$\diamondsuit$}

%\newtheorem*{theorem}{Theorem}
%\newtheorem*{definition}{Definition}
\newtheorem*{proposition}{Proposition}
%\newtheorem*{corollary}{Corollary}
%\newtheorem*{lemma}{Lemma}

\newtcbtheorem[number within=section]{theorem}{Theorem}
{colback=green!5,colframe=green!35!black,fonttitle=\bfseries}{def}

\newtcbtheorem[number within=section]{definition}{Definition}
{colback=blue!5,colframe=blue!35!black,fonttitle=\bfseries}{def}

\newtcbtheorem[number within=section]{corollary}{Corollary}
{colback=yellow!5,colframe=yellow!35!black,fonttitle=\bfseries}{def}

\newtcbtheorem[number within=section]{lemma}{Lemma}
{colback=red!5,colframe=red!35!black,fonttitle=\bfseries}{def}
\usepackage[utf8]{inputenc}
\usepackage[english]{babel}
\usepackage{fancyhdr}
\usepackage[hidelinks]{hyperref}

\pagestyle{fancy}
\fancyhf{}
\rhead{CSE 131}
\chead{Wednesday, April 05, 2023}
\lhead{Lecture 2}
\rfoot{\thepage}

\setlength{\parindent}{0pt}

\begin{document}

\section{Introduction to Compilers (Continued)}
We'll be using s-expressions to represent our program's source code. That is, each \code{snek} file will contain s-expressions. S-expressions are defined as either 
\begin{itemize}
    \item An \textbf{atom} of the form $x$, or 
    \item An \textbf{expression} of the form $(x \quad y)$, where $x$ and $y$ are s-expressions. 
\end{itemize}
For example, \code{(sub1 2)} is an expression with two atoms, \code{sub1} and \code{2}. 

\subsection{The \code{sexp} Crate}
Most programming language will have a parser for s-expressions. In Rust, we have the \code{sexp} crate. This crate has the following \code{enum}s\footnote{In Rust, \code{enum}s are algebraic data types.}:
\begin{itemize}
    \item A \code{Sexp} enum, representing either an \code{Atom} or a vector of s-expressions. Vectors of s-expressions will contain two elements (since an expression has the form $(x \quad y)$, which has two expressions). 
    \begin{verbatim}
pub enum Sexp {
    Atom(Atom),
    List(Vec<Sexp>),
}
    \end{verbatim}

    \item An \code{Atom} enum, representing one of three different types of atoms: a \code{String}, \code{i64} (64-bit signed integer), or \code{f64} (double-precision float). 
    \begin{verbatim}
pub enum Atom {
    S(String),
    I(i64),
    F(f64)
}\end{verbatim}
\end{itemize}

\begin{mdframed}
    (Exercise.) Why is \code{Vec<Box<Sexp>>} or \code{Box<Vec<Sexp>>} not used in the \code{enum} definitions above? 

    \begin{mdframed}
        Remember that the reason why something like 
        \begin{verbatim}
enum Expr {
    Num(i32),
    Add1(Expr),
    Sub1(Expr)
}\end{verbatim}
        wouldn't work is because \code{Expr}, as a recursive type, could have infinite size. That is, we could nest \emph{many} \code{Expr}s, and since this value could theoretically continue infinitely, so we don't know how much space this recursive type needs. However, \code{Box<T>} is a pointer type that allocates memory on the heap. \code{Box<T>} has a \emph{fixed} size, so Rust is fine if we have \code{Box<Expr>}. 

        \bigskip

        The reason why we \emph{don't} need \code{Vec<Box<Sexpr>>} or \code{Box<Vec<Sexpr>>} in the above \code{enum}s is because \code{Vec<T>} allocates to the heap (when you add any elements; a vector created initially with no elements does not allocate). In other words, think of \code{Vec<T>} as being a resizable \code{Box<T>}. More accurately, a \code{Vec<T>} is a fixed-size type with a reference to variable-sized heap data.
    \end{mdframed}
\end{mdframed}


\begin{mdframed}
    (Exercise.) Represent the following s-expression in Rust: 
    \begin{verbatim}
        (sub1 (sub1 (add1 73)))\end{verbatim}
    
    \begin{mdframed}
        First, the s-expression itself is roughly similar to a tree, where each \emph{atom} is a leaf node and each \emph{list} is another s-expression. In any case, we can roughly structure the above s-expression like so: 
        \begin{verbatim}
(               // List
    sub1            // Atom 
    (               // List 
        sub1            // Atom 
        (               // List 
            add1            // Atom 
            73              // Atom 
        )
    )
) \end{verbatim}
        In other words, we have a list whose elements are an \emph{atom} and another \emph{list}. With this in mind, the Rust form is
        \begin{verbatim}
List(vec![
    Atom(S("sub1")),
    List(vec![
        Atom(S("sub1")),
        List(vec![
            Atom(S("add1")),
            Atom(I(73))
        ])
    ])
])\end{verbatim}
    \end{mdframed}
\end{mdframed}

\end{document}