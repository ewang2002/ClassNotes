\section{Introduction to Recursion}
In this section, we'll talk about recursion. Note that, in our examples, we'll assume that the callee manages (i.e., moves) the stack pointer. In particular, this means everything will have a positive offset from \code{rsp}.

\subsection{Recursive Sum Example}
Let's consider the following code:
\begin{verbatim}
(fun (sumrec num)
    (if (= num 0)
        0
        (+ num (sumrec (+ num -1)))
    )
)\end{verbatim}
This program simply performs $1 + 2 + 3 + \hdots + \code{num}$. The generated assembly would look something like what is shown below. 
\begin{verbatim}
sumrec:
    sub rsp, 16
    mov rax, [rsp + 24]
    ... if (= num 0)
    cmp rax, 1
    je ifelse_1
         mov rax, 0
         jmp ifend_0
    ifelse_1:
         ... put temp num on stack for LHS
         mov [rsp + 0], rax
         mov rax, [rsp + 24]
         ... (+ num -1) stored in rax
         ... now do 1-arg calling conv
         sub rsp, 16
         mov [rsp], rax
         mov [rsp+8], rdi
         call sumrec
         mov rdi, [rsp+8]
         add rsp, 16
         ... do addition on the waiting num ...
         add rax, [rsp + 0]
    ifend_0:
    
    add rsp, 16
    ret\end{verbatim}
Note that only relevant assembly is shown. Some things to point out: 
\begin{itemize}
    \item In the second assembly line, \code{sub rsp, 16}, the \code{16} is the \emph{depth} that we calculated. 
    \item In the lines before the recursive call, i.e., 
    \begin{verbatim}
        sub rsp, 16
        mov [rsp], rax
        mov [rsp+8], rdi
        call sumrec\end{verbatim}
    we're moving the arguments into the correct position in memory so the recursive call can make use of them. 
    \item When we run a \code{call} instruction, \code{rsp} is moved up one word and the return pointer to the next line of instruction (program counter) is put in that location in memory (where \code{rsp} is pointing to). 
\end{itemize}
To see how the memory looks when each line of assembly is executed, see \code{Lec12Trace.pdf}.




\subsection{Second Recursive Sum Example}
Let's rewrite the recursive sum example a bit.  
\begin{verbatim}
(fun (sumrec num sofar)
    (if (= num 0)
        sofar 
        (sumrec (+ num -1) (+ sofar num))
    )
)\end{verbatim}
The generated assembly might look like 
\begin{verbatim}
sumrec:
    sub rsp, 16
    
    mov rax, [rsp + 24]
    mov [rsp + 0], rax
    ... if (= num 0)  
    cmp rax, 1
    je ifelse_1
        mov rax, [rsp + 32]
        jmp ifend_0
    ifelse_1:
        mov rax, [rsp + 24]
        ... add -1 to num, store on stack as tmp ...
        mov [rsp + 0], rax

        mov rax, [rsp + 32]
        ... add sofar to num, store in rax ...
        add rax, [rsp + 8]
  
        ... 2-arg calling convention from class ...
        sub rsp, 24
        mov rbx, [rsp+24]
        mov [rsp], rbx
        mov [rsp+8], rax
        mov [rsp+16], rdi
        call sumrec             ; (A)
        mov rdi, [rsp+16]       ; (B)
        add rsp, 24             ; (C)
    
    ifend_0:
    add rsp, 16                 ; (C)
    ret                         ; (D)\end{verbatim}
An interesting thing to note is that, after reaching the base case, there's no additional calculation that needs to be made. In particular, the steps after returning is 
\begin{enumerate}[(a)]
    \item Move \code{rsp} back. Remember that, after \code{call} is done (i.e., when \code{ret} is executed), \code{rsp} is moved back one word.
    \item Restore \code{rdi}.
    \item Move \code{rsp} back more. 
    \item Return!
\end{enumerate}
No local variables or arguments were accessed.