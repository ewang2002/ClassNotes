\documentclass[letterpaper]{article}
\usepackage[margin=1in]{geometry}
\usepackage[utf8]{inputenc}
\usepackage{textcomp}
\usepackage{amssymb}
\usepackage{natbib}
\usepackage{graphicx}
\usepackage{gensymb}
\usepackage{amsthm, amsmath, mathtools}
\usepackage{xcolor}
\usepackage{enumerate}
\usepackage{framed}
\usepackage{tcolorbox}
\tcbuselibrary{theorems}

\newcommand{\R}{\mathbb{R}}
\newcommand{\Z}{\mathbb{Z}}
\newcommand{\N}{\mathbb{N}}
\newcommand{\Q}{\mathbb{Q}}
\newcommand{\code}[1]{\texttt{#1}}
\newcommand{\mdiamond}{$\diamondsuit$}

%\newtheorem*{theorem}{Theorem}
%\newtheorem*{definition}{Definition}
\newtheorem*{proposition}{Proposition}
%\newtheorem*{corollary}{Corollary}
%\newtheorem*{lemma}{Lemma}

\newtcbtheorem[number within=section]{theorem}{Theorem}
{colback=green!5,colframe=green!35!black,fonttitle=\bfseries}{def}

\newtcbtheorem[number within=section]{definition}{Definition}
{colback=blue!5,colframe=blue!35!black,fonttitle=\bfseries}{def}

\newtcbtheorem[number within=section]{corollary}{Corollary}
{colback=yellow!5,colframe=yellow!35!black,fonttitle=\bfseries}{def}

\newtcbtheorem[number within=section]{lemma}{Lemma}
{colback=red!5,colframe=red!35!black,fonttitle=\bfseries}{def}
\usepackage[utf8]{inputenc}
\usepackage[english]{babel}
\usepackage{fancyhdr}
\usepackage[hidelinks]{hyperref}

\pagestyle{fancy}
\fancyhf{}
\rhead{Math 20D}
\chead{September 14th, 2021}
\lhead{Course Notes}
\rfoot{\thepage}

\setlength{\parindent}{0pt}

\begin{document}

\begin{titlepage}
    \begin{center}
        \vspace*{1cm}
            
        \Huge
        \textbf{Math 20D Notes}
            
        \vspace{0.5cm}
        \LARGE
        Introduction to Differential Equations
            
        \vspace{1.5cm}
            
        \vfill
            
        Fall 2021\\
        Taught by Professor Tu
    \end{center}
\end{titlepage}

\pagenumbering{gobble}

\newpage 

\pagenumbering{gobble}
\begingroup
    \renewcommand\contentsname{Table of Contents}
    \tableofcontents
\endgroup

\newpage
\pagenumbering{arabic}

\section{Introduction to Differential Equations}
To motivate our definition of a differential equation, we first begin by talking about what an algebraic equation is. When we are given an algebraic equation like: 
\[x^2 + 5x - 6 = 0\]
Our goal is to find the value of $x$. Here, $x$ is the unknown. We need to find a solution (i.e. a number) for $x$ such that the equation is satisfied. 

\bigskip 

In a differential equation, we are essentially doing the same thing. Given a differential equation, we need to find the unknown function that satisfies it. For instance, if we are given: 
\[f'(x) - x = 0\]
Then $f'(x)$ is the unknown; that is, we need to find the function that satisfies this.

\bigskip 

Of course, we aren't just limited to $f'(x)$. In fact, we may see the second, third, fourth derivative, and so on; essentially, given a differential equation, we need to find the unknown function involving derivatives of any order. 

\subsection{Independent vs. Dependent Variables}
If a differential equation involvs the derivative of one variable with respect to another, then the former variable is called a \textbf{dependent variable} and the latter is called an \textbf{independent variable}. 

Let's suppose we have the following differential equation: 
\[\frac{dy}{dx} - x = 0\]
\begin{itemize}
    \item $y$ is the dependent variable. 
    \item $x$ is the independent variable.
\end{itemize}
If we wrote the above differential equation like so: 
\[y' - x = 0\]
Then again: 
\begin{itemize} 
    \item $y$ is the dependent variable.
    \item $x$ is the independent variable. 
\end{itemize}

\subsection{Classifying Differential Equations}
There are three different criterias for classifying differential equations.

\subsubsection{Linear vs. Non-Linear Differential Equation}
A linear equation means that there is \textbf{no} power (or, more specifically, a power of 1) on any of the dependent variables. For example, the following differential equation is linear: 
\[y' - x = 0\]
It should be noted that having $y$, $y'$, $y''$, etc. (essentially, any derivative) does not automatically make a differential equation non-linear. The following differential equation is also linear: 
\[2y + 3y' - xy'' = 0\]
It doesn't matter what the power of any dependent terms is; so, you can have terms like $x^{15}y''$ in a differential equation and that (alone) would make it linear.

\bigskip 

A differential equation is non-linear if:
\begin{itemize}
    \item Any of the dependent variables have a power that isn't 1. This makes the following differential equations non-linear: 
    \[2\boxed{y^2} + 3y' - xy'' = 0\]
    \[\boxed{\sqrt{y - 3}} + y' = x\]

    \item The dependent variables are being multiplied with each other. For example, $y''$ by itself is fine but $y'y''$ is not. This makes the following differential equation non-linear: 
    \[2y + 3\boxed{y'y''} - xy'' = 0\]
\end{itemize}

Keep in mind that we can multiply the derivatives with some power of the independent variables 


\subsubsection{Homogeneous/Non-Homogeneous}
A homogeneous differential equation is one where there are no \textbf{lone} independent terms. 

\subsubsection{Order of Equation}
The order of a differential equation depends on the highest derivative \textbf{alone}. For example, the following differential equation has an order of 2:
\[y'' - x = 0\]
The following differential equations have an order of 3:
\[y''' - y'  = 1\]
\[(y''')^3 - y' = 10\]
Here, it doesn't matter what the power of any $y$ term is. 

\end{document}