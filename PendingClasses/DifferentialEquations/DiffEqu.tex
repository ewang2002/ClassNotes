\documentclass[letterpaper]{article}
\usepackage[margin=1in]{geometry}
\usepackage[utf8]{inputenc}
\usepackage{textcomp}
\usepackage{amssymb}
\usepackage{natbib}
\usepackage{graphicx}
\usepackage{gensymb}
\usepackage{amsthm, amsmath, mathtools}
\usepackage[dvipsnames]{xcolor}
\usepackage{enumerate}
\usepackage{mdframed}
\usepackage[most]{tcolorbox}
\usepackage{csquotes}
% https://tex.stackexchange.com/questions/13506/how-to-continue-the-framed-text-box-on-multiple-pages

\tcbuselibrary{theorems}

\newcommand{\R}{\mathbb{R}}
\newcommand{\Z}{\mathbb{Z}}
\newcommand{\N}{\mathbb{N}}
\newcommand{\Q}{\mathbb{Q}}
\newcommand{\C}{\mathbb{C}}
\newcommand{\code}[1]{\texttt{#1}}
\newcommand{\mdiamond}{$\diamondsuit$}
\newcommand{\PowerSet}{\mathcal{P}}
\newcommand{\Mod}[1]{\ (\mathrm{mod}\ #1)}
\DeclareMathOperator{\lcm}{lcm}

%\newtheorem*{theorem}{Theorem}
%\newtheorem*{definition}{Definition}
%\newtheorem*{corollary}{Corollary}
%\newtheorem*{lemma}{Lemma}
\newtheorem*{proposition}{Proposition}


\newtcbtheorem[number within=section]{theorem}{Theorem}
{colback=green!5,colframe=green!35!black,fonttitle=\bfseries}{th}

\newtcbtheorem[number within=section]{definition}{Definition}
{colback=blue!5,colframe=blue!35!black,fonttitle=\bfseries}{def}

\newtcbtheorem[number within=section]{corollary}{Corollary}
{colback=yellow!5,colframe=yellow!35!black,fonttitle=\bfseries}{cor}

\newtcbtheorem[number within=section]{lemma}{Lemma}
{colback=red!5,colframe=red!35!black,fonttitle=\bfseries}{lem}

\newtcbtheorem[number within=section]{example}{Example}
{colback=white!5,colframe=white!35!black,fonttitle=\bfseries}{def}

\newtcbtheorem[number within=section]{note}{Important Note}{
        enhanced,
        sharp corners,
        attach boxed title to top left={
            xshift=-1mm,
            yshift=-5mm,
            yshifttext=-1mm
        },
        top=1.5em,
        colback=white,
        colframe=black,
        fonttitle=\bfseries,
        boxed title style={
            sharp corners,
            size=small,
            colback=red!75!black,
            colframe=red!75!black,
        } 
    }{impnote}
\usepackage[utf8]{inputenc}
\usepackage[english]{babel}
\usepackage{fancyhdr}
\usepackage[hidelinks]{hyperref}

\pagestyle{fancy}
\fancyhf{}
\rhead{Math 20D}
\chead{September 14th, 2021}
\lhead{Course Notes}
\rfoot{\thepage}

\setlength{\parindent}{0pt}

\begin{document}

\begin{titlepage}
    \begin{center}
        \vspace*{1cm}
            
        \Huge
        \textbf{Math 20D Notes}
            
        \vspace{0.5cm}
        \LARGE
        Introduction to Differential Equations
            
        \vspace{1.5cm}
            
        \vfill
            
        Fall 2021\\
        Taught by Professor Tu
    \end{center}
\end{titlepage}

\pagenumbering{gobble}

\newpage 

\pagenumbering{gobble}
\begingroup
    \renewcommand\contentsname{Table of Contents}
    \tableofcontents
\endgroup

\newpage
\pagenumbering{arabic}

\section{Introduction to Differential Equations}
To motivate our definition of a differential equation, we first begin by talking about what an algebraic equation is. When we are given an algebraic equation like: 
\[x^2 + 5x - 6 = 0\]
Our goal is to find the value of $x$. Here, $x$ is the unknown. We need to find a solution (i.e. a number) for $x$ such that the equation is satisfied. 

\bigskip 

In a differential equation, we are essentially doing the same thing. Given a differential equation, we need to find the unknown function that satisfies it. For instance, if we are given: 
\[f'(x) - x = 0\]
It follows that $f'(x)$ is the unknown; that is, we need to find the function that satisfies this.

\subsection{Why Study Differential Equations?}
To answer this question, suppose we have a radioactive substance. Now, suppose the decay rate is 0.2 (in other words, the probability of decay at every instance is 0.2, or 20\%). It follows that every atom could decay together or by themselves (depends on the chance). 

\bigskip 

Let $A$ be the number of original atoms left. Let $A(t)$ be a function of $t$ and let $A(0) = 10^{10}$ atoms. Then, we let: 
\[\frac{dA}{dt} = -20\% \cdot A = \boxed{-0.2A}\]
Which is the same as saying: 
\[A'(t) = -0.2A(t)\]
Which our goal is to find $A$. 


\end{document}