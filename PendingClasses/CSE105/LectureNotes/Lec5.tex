\documentclass[letterpaper]{article}
\usepackage[margin=1in]{geometry}
\usepackage[utf8]{inputenc}
\usepackage{textcomp}
\usepackage{amssymb}
\usepackage{natbib}
\usepackage{graphicx}
\usepackage{gensymb}
\usepackage{amsthm, amsmath, mathtools}
\usepackage{xcolor}
\usepackage{enumerate}
\usepackage{framed}
\usepackage{tcolorbox}
\tcbuselibrary{theorems}

\newcommand{\R}{\mathbb{R}}
\newcommand{\Z}{\mathbb{Z}}
\newcommand{\N}{\mathbb{N}}
\newcommand{\Q}{\mathbb{Q}}
\newcommand{\code}[1]{\texttt{#1}}
\newcommand{\mdiamond}{$\diamondsuit$}

%\newtheorem*{theorem}{Theorem}
%\newtheorem*{definition}{Definition}
\newtheorem*{proposition}{Proposition}
%\newtheorem*{corollary}{Corollary}
%\newtheorem*{lemma}{Lemma}

\newtcbtheorem[number within=section]{theorem}{Theorem}
{colback=green!5,colframe=green!35!black,fonttitle=\bfseries}{def}

\newtcbtheorem[number within=section]{definition}{Definition}
{colback=blue!5,colframe=blue!35!black,fonttitle=\bfseries}{def}

\newtcbtheorem[number within=section]{corollary}{Corollary}
{colback=yellow!5,colframe=yellow!35!black,fonttitle=\bfseries}{def}

\newtcbtheorem[number within=section]{lemma}{Lemma}
{colback=red!5,colframe=red!35!black,fonttitle=\bfseries}{def}
\usepackage[utf8]{inputenc}
\usepackage[english]{babel}
\usepackage{fancyhdr}
\usepackage[hidelinks]{hyperref}

\pagestyle{fancy}
\fancyhf{}
\rhead{CSE 105}
\chead{Wednesday, January 19, 2022}
\lhead{Lecture 5}
\rfoot{\thepage}

\setlength{\parindent}{0pt}

\begin{document}

\section{Nondeterministic Finite Automata (1.2, Continued)}
This continues from the notes from Monday, January 12.

\subsection{Equivalence of NFAs and DFAs}
Deterministic and nondeterministic finite automata both recognize the same class of languages.

\begin{theorem}{}{}
    Every nondeterministic finitne automaton has an equivalent deterministic finite automaton.
\end{theorem}
\textbf{Remark:} Here, we say that two machines are equivalent if they recognize the same language. 

\begin{mdframed}[]
    \begin{proof}
        The proof is omitted due to potential academic integrity issues.
    \end{proof}
\end{mdframed}

\subsection{Applications of Theorem}
There are several applications of this theorem. 

\begin{corollary}{}{}
    A language is regular if and only if some nondeterministic finite automaton recognizes it.
\end{corollary}

\subsubsection{Example: NFA to DFA}
Consider the following NFA:
\begin{center}
    \includegraphics[scale=0.5]{../assets/nfa_to_dfa_1.png}
\end{center}


\end{document}