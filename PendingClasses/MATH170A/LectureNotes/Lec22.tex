\documentclass[letterpaper]{article}
\usepackage[margin=1in]{geometry}
\usepackage[utf8]{inputenc}
\usepackage{textcomp}
\usepackage{amssymb}
\usepackage{natbib}
\usepackage{graphicx}
\usepackage{gensymb}
\usepackage{amsthm, amsmath, mathtools}
\usepackage[dvipsnames]{xcolor}
\usepackage{enumerate}
\usepackage{mdframed}
\usepackage[most]{tcolorbox}
\usepackage{csquotes}
% https://tex.stackexchange.com/questions/13506/how-to-continue-the-framed-text-box-on-multiple-pages

\tcbuselibrary{theorems}

\newcommand{\R}{\mathbb{R}}
\newcommand{\Z}{\mathbb{Z}}
\newcommand{\N}{\mathbb{N}}
\newcommand{\Q}{\mathbb{Q}}
\newcommand{\C}{\mathbb{C}}
\newcommand{\code}[1]{\texttt{#1}}
\newcommand{\mdiamond}{$\diamondsuit$}
\newcommand{\PowerSet}{\mathcal{P}}
\newcommand{\Mod}[1]{\ (\mathrm{mod}\ #1)}
\DeclareMathOperator{\lcm}{lcm}

%\newtheorem*{theorem}{Theorem}
%\newtheorem*{definition}{Definition}
%\newtheorem*{corollary}{Corollary}
%\newtheorem*{lemma}{Lemma}
\newtheorem*{proposition}{Proposition}


\newtcbtheorem[number within=section]{theorem}{Theorem}
{colback=green!5,colframe=green!35!black,fonttitle=\bfseries}{th}

\newtcbtheorem[number within=section]{definition}{Definition}
{colback=blue!5,colframe=blue!35!black,fonttitle=\bfseries}{def}

\newtcbtheorem[number within=section]{corollary}{Corollary}
{colback=yellow!5,colframe=yellow!35!black,fonttitle=\bfseries}{cor}

\newtcbtheorem[number within=section]{lemma}{Lemma}
{colback=red!5,colframe=red!35!black,fonttitle=\bfseries}{lem}

\newtcbtheorem[number within=section]{example}{Example}
{colback=white!5,colframe=white!35!black,fonttitle=\bfseries}{def}

\newtcbtheorem[number within=section]{note}{Important Note}{
        enhanced,
        sharp corners,
        attach boxed title to top left={
            xshift=-1mm,
            yshift=-5mm,
            yshifttext=-1mm
        },
        top=1.5em,
        colback=white,
        colframe=black,
        fonttitle=\bfseries,
        boxed title style={
            sharp corners,
            size=small,
            colback=red!75!black,
            colframe=red!75!black,
        } 
    }{impnote}
\usepackage[utf8]{inputenc}
\usepackage[english]{babel}
\usepackage{fancyhdr}
\usepackage[hidelinks]{hyperref}
\usepackage{csquotes}

\pagestyle{fancy}
\fancyhf{}
\rhead{Math 170A}
\chead{March 08, 2023}
\lhead{Lecture 23}
\rfoot{\thepage}

\setlength{\parindent}{0pt}
\newcommand{\0}{\mathbf{0}}
\newcommand{\y}{\mathbf{y}}
\renewcommand{\b}{\mathbf{b}}
\newcommand{\x}{\mathbf{x}}
\newcommand{\e}{\mathbf{e}}
\newcommand{\rr}{\mathbf{r}}
\newcommand{\vv}{\mathbf{v}}
\renewcommand{\u}{\mathbf{u}}

\begin{document}

\section{Reduction to Hessenberg and Triangular Form (5.5)}
One of the issues with the QR iteration is its computational cost; it's not very efficient. In this section, we'll learn about how we can convert $A$ into a matrix that is faster to compute with. 

\subsection{Upper Hessenberg Matrix}
\begin{definition}{Upper Hessenberg Matrix}{}
    An $n \times n$ matrix $H$ is called \textbf{upper Hessenberg} if $h_{ij} = 0$ for $i > j + 1$. 
\end{definition}
\textbf{Remarks:} 
\begin{itemize}
    \item An upper Hessenberg matrix might look like 
    \[\begin{bmatrix}
        * & * & * & * & * & * & * \\ 
        * & * & * & * & * & * & * \\
        0 & * & * & * & * & * & * \\
        0 & 0 & * & * & * & * & * \\ 
        0 & 0 & 0 & * & * & * & * \\ 
        0 & 0 & 0 & 0 & * & * & * \\ 
        0 & 0 & 0 & 0 & 0 & * & *
    \end{bmatrix}\]
    \item If $H$ is also symmetric ($A = A^T$), then we get a tridiagonal matrix. This matrix might look like 
    \[\begin{bmatrix}
        * & * & 0 & 0 & 0 & 0 & 0 \\ 
        * & * & * & 0 & 0 & 0 & 0 \\
        0 & * & * & * & 0 & 0 & 0 \\
        0 & 0 & * & * & * & 0 & 0 \\ 
        0 & 0 & 0 & * & * & * & 0 \\ 
        0 & 0 & 0 & 0 & * & * & * \\ 
        0 & 0 & 0 & 0 & 0 & * & *
    \end{bmatrix}\]
\end{itemize}
% TODO get matrix representation.

\begin{mdframed}
    (Example.) Is the following matrix 
    \[H = \begin{bmatrix}
        4 & 0 & -1 & 2 \\ 
        -8 & -3 & 5 & 6 \\ 
        0 & -5 & 2 & 7 \\ 
        0 & 0 & 4 & 1
    \end{bmatrix}\]
    an upper Hessenberg matrix? 

    \begin{mdframed}
        Yes. 
    \end{mdframed}
\end{mdframed}

\subsection{Revised QR Iteration: Preconditioning}
We want to transform $A$ to an upper Hessenberg matrix. This process is known as \textbf{preconditioning}. The process is as follows: 
\begin{enumerate}
    \item The idea is that we want to find an $H$ such that \[A = QHQ^T \text{ or } A = QHQ^{*}\]
    for some unitary $Q$ matrix. This can be found in finite steps, unlike Schur. Remember that $A$ and $H$ should have the same eigenvalues, since they are similar matrices. 
    \item Apply QR iteration to $H$. The QR iteration of $H$ only needs $\BigO(n^2)$ flops. In addition, the QR iteration on a Hessenberg matrix stays within the set of Hessenberg matrices. That is, for $H_0 = H$ where $H$ is Hessenberg, then all $H_k$ obtained by QR iteration are also Hessenberg matrices. 
\end{enumerate}

\[\begin{bmatrix}
    * & \hdots & * \\ 
    * & \hdots & * \\ 
    \vdots & & \vdots \\ 
    * & * \hdots & *
\end{bmatrix} \xrightarrow[\text{Step 1}]{\text{Preconditioning}} \underbrace{\begin{bmatrix}
    \underline{*} & * & * & \hdots & * \\
    \underline{*} & \underline{*} & * & \hdots & * \\ 
    0 & \underline{*} & \underline{*} & \hdots & * \\ 
    \vdots & \vdots & \vdots & \ddots & \vdots \\ 
    0 & 0 & 0 & \hdots & \underline{*} 
    \end{bmatrix}}_{\text{H: Hessenberg}} \xrightarrow[\text{Step 2}]{\text{QR Iteration}} \underbrace{\dots}_{H_k} \xrightarrow{\text{QR Iteration}} \underbrace{\begin{bmatrix}
        * & * & * & \hdots & * \\
        0 & * & * & \hdots & * \\ 
        0 & 0 & * & \hdots & * \\ 
        \vdots & \vdots & \vdots & \ddots & \vdots \\ 
        0 & 0 & 0 & \hdots & *
\end{bmatrix}}_{T: \text{Upper Triangular}}\]
The idea is, as we apply the iteration more times, we eventually end up with $T$; that is, 
\[\lim_{k \mapsto \infty} H_k = T.\]

\subsubsection{Why Does It Work?}
\begin{theorem}{}{}
    An $A \in \R^{n \times n}$ matrix can be decomposed as $A = QHQ^T$  with $H$ upper Hessenberg and $Q$ orthogonal. For $B \in \C^{n \times n}$, the decomposition becomes $B = UHU^*$ with $U$ being unitary. 
\end{theorem}

\begin{proof}
    The idea is very similar to QR decomposition in the sense that we need reflectors. Note that the QR process looks like \[Q_n \hdots Q_1 Q_1 A \mapsto R,\] but with this process the eigenvalues are changed. So, we need \[Q_n \hdots Q_2 Q_1 A Q_1^T Q_2^T \hdots Q_n^T \mapsto H.\]
    Here, $H$ is similar to $A$. 
\end{proof}

\begin{mdframed}
    (Example.) Consider 
    \[A = \begin{bmatrix}
        a_{11} & a_{12} & a_{13} & a_{14} \\ 
        a_{21} & a_{22} & a_{23} & a_{24} \\ 
        a_{31} & a_{32} & a_{33} & a_{34} \\ 
        a_{41} & a_{42} & a_{43} & a_{44} \\ 
    \end{bmatrix}.\]
    Consider \[\x = \begin{bmatrix}
        a_{21} \\ a_{31} \\ a_{41}
    \end{bmatrix} \quad \y = \begin{bmatrix}
        a_{12} \\ a_{13} \\ a_{14}
    \end{bmatrix}.\]
    This gives us 
    \[A = \begin{bmatrix}
        a_{11} & \y^T \\ 
        \x & B
    \end{bmatrix}.\]
    So, we can map $\x$ to $\begin{bmatrix}
        ||x||_2 \\ 0
    \end{bmatrix}$ using a reflector $\hat{Q}_1 \in \R^{3 \times 3}$. Then, 
    \[Q_1 = \begin{bmatrix}
        1 & 0 \\ 0 & \hat{Q}_1
    \end{bmatrix} \in \R^{4 \times 4}\] is still orthogonal. From there, \[Q_1 A = \begin{bmatrix}
        1 & 0 \\ 0 & \hat{Q}_1
    \end{bmatrix} \begin{bmatrix}
        a_{11} & \y^T \\ 
        \x & B 
    \end{bmatrix} = \begin{bmatrix}
        a_{11} & y^T \\ 
        \hat{Q}_1 \x & \hat{Q}_1 B
    \end{bmatrix}.\]
    From there, 
    \[Q_1 A Q_1^T = \begin{bmatrix}
        a_{11} & y^T \\ 
        \hat{Q}_1 \x & \hat{Q}_1 B
    \end{bmatrix} \begin{bmatrix}
        1 & 0 \\ 
        0 & \hat{Q}_1^T
    \end{bmatrix} = \begin{bmatrix}
        a_{11} & \y^T \hat{Q}_1^T \\ 
        \hat{Q}_1 \x & \hat{Q}_1 B \hat{Q}_1^T 
    \end{bmatrix} = \begin{bmatrix}
        a_{11} & * & * & * \\ 
        ||x||_2 & c_{11} & c_{12} & c_{13} \\ 
        0 & c_{21} & c_{22} & c_{23} \\ 
        0 & c_{31} & c_{32} & c_{33}
    \end{bmatrix}.\]
    At this point, we now only look at the matrix formed by the $c_{ij}$ elements. We need to do $Q_2(Q_1 A Q_1^T)Q_2^T$ to make $c_{31}$ into 0 using a reflector. Then, we'll get 
    \[H = \begin{bmatrix}
        * & * & * & * \\ 
        * & * & * & * \\ 
        0 & * & * & * \\ 
        0 & 0 & * & * 
    \end{bmatrix}.\]
\end{mdframed}
\textbf{Remark:} If $A$ is $n \times n$, then we need more reflectors, i.e., 
\[Q_{n - 1} \hdots Q_2 Q_1 A Q_1^T Q_2^T \hdots Q_{n - 1}^T.\]


\subsubsection{QR Iteration on Hessenberg}
If $H$ is an Hessenberg matrix, then all $H_k$ obtained via QR iteration are Hessenberg matrices as well. With that said, the QR iteration algorithm is as follows: 

\begin{enumerate}
    \item Find $H$; $A = QHQ^T$ with $H$ being Hessenberg. We can use the MATLAB command \code{H = hess(A)} to get the Hessenberg matrix. 
    \item We can perform QR iteration on $H$ to get $H_0 = H$. Using a for- or while- loop, we can run \[[Q_k, R_k] = \code{qr}(H_k) \qquad H_{k + 1} = R_k Q_k.\]
    We stop iteration when the subdiagonal of $H_{k + 1}$ is smaller than a threshold $\epsilon > 0$. That is, when the underlined portion is less than $\epsilon$. 
    \[\begin{bmatrix}
        * & * & * & * & * & * & * \\ 
        \underline{*} & * & * & * & * & * & * \\
        0 & \underline{*} & * & * & * & * & * \\
        0 & 0 & \underline{*} & * & * & * & * \\ 
        0 & 0 & 0 & \underline{*} & * & * & * \\ 
        0 & 0 & 0 & 0 & \underline{*} & * & * \\ 
        0 & 0 & 0 & 0 & 0 & \underline{*} & *
    \end{bmatrix}\]
    Put it a different way, when 
    \[\sqrt{\sum_{i = 1}^{n - 1} H_{k + 1} (i + 1, i)^2} < \epsilon.\]

    \item The eigenvalues are on the diagonal of $H_{k + 1}$. 
\end{enumerate}


\subsubsection{Flop Count}
In summary, the flop count is \[\BigO\left(\frac{10}{3}n^3\right) + \BigO\left(n^2 N\right),\]
where $N$ is the number of iterations and $n$ is the size of $H$. Then, 
\begin{itemize}
    \item $\BigO\left(\frac{10}{3}n^3\right)$ represents the operation for getting the Hessenberg matrix, and 
    \item $\BigO\left(n^2 N\right)$ represents the operation of finding the QR decomposition of $H$.
\end{itemize}
\textbf{Note:} If we didn't use the Hessenberg matrix, the flop count of QR iteration is $\BigO(n^3 N)$. 

\textbf{Conclusion:} When $N$ is large, Hessenberg will be very helpful in reducing computational costs.

\begin{mdframed}
    (Example: Hermition Matrices.) The \textbf{Hermition Matrix} is a matrix $A$ such that $A = A^*$. If we do an Hessenberg matrix, then\footnote{$(CD)^* = D^* C^*$.} \[A^* = UH^* U^*.\] In particular, 
    \[A = A^* \implies UHU^* = UH^* U^* \implies H = H^*.\] 
    The QR decomposition of a tridiagonal matrix is only $\BigO(n)$, so the iteration part in the QR iteration reduces to $\BigO(nN)$. So, less work for % TODO 
\end{mdframed}
\textbf{Remark:} The Hermition matrix mimics the symmetric matrix in the reals.

\section{Note on SVD Computation (5.8)}
For $A \in \R^{n \times m}, n \geq m$, the eigenvalues of $AA^T$ and $A^T A$ are $\sigma_1^2, \hdots, \sigma_r^2, 0, \hdots, 0$ (with $\sigma_i$ singular values of $A$, $r = \text{rank}(A)$). Remember that 
\begin{itemize}
    \item $AA^T$ is an $n \times n$ matrix, and 
    \item $A^T A$ is an $m \times m$ matrix.
\end{itemize}
So, to find the SVD of $A$, we can find the eigenvalues and eigenvectors of $AA^T$ and $A^T A$. Note that both $AA^T$ and $A^T A$ are symmetric. 

\bigskip 

Then, it follows that \[\kappa_{2}(AA^T) = \kappa_{2}(A)^2.\] If $\kappa_{2}(A)$ is large, then for $AA^T$, the operation is \emph{squared}, thus affecting accuracy. So, we generally want to avoid finding eigenvalues of $AA^T$ and $A^T A$ directly.


\end{document}