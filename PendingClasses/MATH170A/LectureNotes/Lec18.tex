\documentclass[letterpaper]{article}
\usepackage[margin=1in]{geometry}
\usepackage[utf8]{inputenc}
\usepackage{textcomp}
\usepackage{amssymb}
\usepackage{natbib}
\usepackage{graphicx}
\usepackage{gensymb}
\usepackage{amsthm, amsmath, mathtools}
\usepackage{xcolor}
\usepackage{enumerate}
\usepackage{framed}
\usepackage{tcolorbox}
\tcbuselibrary{theorems}

\newcommand{\R}{\mathbb{R}}
\newcommand{\Z}{\mathbb{Z}}
\newcommand{\N}{\mathbb{N}}
\newcommand{\Q}{\mathbb{Q}}
\newcommand{\code}[1]{\texttt{#1}}
\newcommand{\mdiamond}{$\diamondsuit$}

%\newtheorem*{theorem}{Theorem}
%\newtheorem*{definition}{Definition}
\newtheorem*{proposition}{Proposition}
%\newtheorem*{corollary}{Corollary}
%\newtheorem*{lemma}{Lemma}

\newtcbtheorem[number within=section]{theorem}{Theorem}
{colback=green!5,colframe=green!35!black,fonttitle=\bfseries}{def}

\newtcbtheorem[number within=section]{definition}{Definition}
{colback=blue!5,colframe=blue!35!black,fonttitle=\bfseries}{def}

\newtcbtheorem[number within=section]{corollary}{Corollary}
{colback=yellow!5,colframe=yellow!35!black,fonttitle=\bfseries}{def}

\newtcbtheorem[number within=section]{lemma}{Lemma}
{colback=red!5,colframe=red!35!black,fonttitle=\bfseries}{def}
\usepackage[utf8]{inputenc}
\usepackage[english]{babel}
\usepackage{fancyhdr}
\usepackage[hidelinks]{hyperref}
\usepackage{csquotes}

\pagestyle{fancy}
\fancyhf{}
\rhead{Math 170A}
\chead{Monday, February 27, 2023}
\lhead{Lecture 19}
\rfoot{\thepage}

\setlength{\parindent}{0pt}
\newcommand{\0}{\mathbf{0}}
\newcommand{\y}{\mathbf{y}}
\renewcommand{\b}{\mathbf{b}}
\newcommand{\x}{\mathbf{x}}
\newcommand{\e}{\mathbf{e}}
\newcommand{\rr}{\mathbf{r}}
\newcommand{\vv}{\mathbf{v}}
\renewcommand{\u}{\mathbf{u}}

\begin{document}

\section{Eigenvector and Eigenvalues (5.1, 5.2)}
Fundamentally, we want to solve \begin{equation}{\label{eq:2-27:1}}
    Av = \lambda v,
\end{equation} where
\begin{itemize}
    \item $A$ is a matrix either in $\R^{n \times n}$ or $\C^{n \times n}$, and 
    \item $\lambda \in \C$ (the \emph{eigenvalue}), and 
    \item $v \in \C^n$ (the \emph{eigenvector}).
\end{itemize}
The pair $(\lambda, v)$ is called an \textbf{eigenpair} of $A$. Whereas each eigenvector has a unique eigenvalue associated with it, each eigenvalue is associated with many eigenvectors. For example, if $v$ is an eigenvector of $A$ associated with the eigenvalue $\lambda$, then every nonzero multiple of $v$ is also an eigenvector of $A$ associated with $\lambda$. Now, note that 
\[Av = \lambda v \implies Av = \lambda I v \implies \0 = Av - \lambda I v \implies  \0 = (A - \lambda I)v,\]
where $I$ is the $n \times n$ identity matrix. In any case, if $v$ is an eigenvector with eigenvalue $\lambda$, then $v$ is a nonzero solution of the homogeneous matrix equation \[(A - \lambda I)v = \0.\] Therefore, the matrix is singular\footnote{Not invertible.} and $\det(A - \lambda I) = 0$. This establishes the following theorem. 

\begin{theorem}{}{2-27:2}
    $\lambda$ is an eigenvalue of $A$ if and only if 
    \[\det(A - \lambda I) = 0.\]
    This equation is known as the \textbf{characteristic equation} of $A$.
\end{theorem}
\textbf{Remark:} Although this is useful theoretically, this equation has little value for actual computations, as we will see later. 

We note that $\det(A - \lambda I)$ is a polynomial in $\lambda$ of degree $n$; it is called the \textbf{characteristic polynomial} of $A$. Therefore, the eigenvalues, $\lambda$, are the roots of this polynomial. There are exactly $n$ roots, not all real (i.e., some complex), and thus $n$ eigenvalues.

\begin{mdframed}
    (Exercise.) Find the characteristic polynomial of \[B = \begin{bmatrix}
        1 & 2 \\ 3 & 4
    \end{bmatrix}.\]

    \begin{mdframed}
        We have 
        \begin{equation*}
            \begin{aligned}
                \det(B - \lambda I) &= \det\left(\begin{bmatrix}
                    1 & 2 \\ 3 & 4
                \end{bmatrix} - \lambda \begin{bmatrix}
                    1 & 0 \\ 0 & 1
                \end{bmatrix}\right) \\ 
                    &= \det\left(\begin{bmatrix}
                        1 - \lambda & 2 \\ 
                        3 & 4 - \lambda
                    \end{bmatrix}\right) \\ 
                    &= (1 - \lambda)(4 - \lambda) - 3(2) \\ 
                    &= 4 - \lambda - 4\lambda + \lambda^2 - 6 \\ 
                    &= \lambda^2 - 5\lambda - 2.
            \end{aligned}
        \end{equation*}
    \end{mdframed}
\end{mdframed}

\subsection{Getting Eigenvectors}
We can get the eigenvectors by solving \begin{equation}{\label{eq:2-27:3}}
    (A - \lambda I)v = \0,
\end{equation}
where $\lambda$ are the eigenvalues that we found by solving (\ref{th:2-27:2}). Note that 
\begin{itemize}
    \item Eigenvalues are the roots of the polynomials of (\ref{th:2-27:2}).
    \item Roots can be complex, thus eigenvalues can also be complex \emph{even} when $A$ is real. 
\end{itemize}

\begin{mdframed}
    (Example.) Consider \[A = \begin{bmatrix}
        0 & -1 \\ 1 & 0
    \end{bmatrix} \in \R^{2 \times 2}.\] 
    \begin{itemize}
        \item To find the eigenvalues, we have 
        \[\det(A - \lambda I) = \det\left(\begin{bmatrix}
            -\lambda & -1 \\ 1 & -\lambda
        \end{bmatrix}\right) = \lambda^2 + 1.\]
        Then, we can solve \[\lambda^2 + 1 = 0,\] giving us $\lambda = \pm i$, which is \emph{complex}.  

        \item To find the eigenvectors, we use the formula (\ref{eq:2-27:3}) for each eigenvalue.
        \begin{itemize}
            \item For $\lambda = i$, we have 
            \[\left(\begin{bmatrix}
                0 & - 1 \\ 1 & 0 
            \end{bmatrix} - \begin{bmatrix}
                i & 0 \\ 
                0 & i
            \end{bmatrix}\right)v = \0 \implies \begin{bmatrix}
                -i & -1 \\ 
                1 & -i
            \end{bmatrix}v = \0 \implies v = \begin{bmatrix}
                i \\ 1
            \end{bmatrix}.\]
            
            \item For $\lambda = -i$, we have 
            \[\left(\begin{bmatrix}
                0 & -1 \\ 1 & 0
            \end{bmatrix} - \begin{bmatrix}
                -i & 0 \\ 
                0 & -i
            \end{bmatrix}\right)v = \0 \implies \begin{bmatrix}
                i & -1 \\ 1 & i
            \end{bmatrix}v = \0 \implies v = \begin{bmatrix}
                -i \\ 1
            \end{bmatrix}.\]
        \end{itemize}
    \end{itemize}
\end{mdframed}

\subsection{Eigenvectors \& Eigenvalues for Large Matrices}
Suppose we have a matrix of size $200 \times 200$. The characteristic equation will be a polynomial of degree 200. This means that there are 200 roots and thus 200 eigenvalues. Now, we should note that there is no formula to find the roots when $r > 4$ (we have the quadratic equation when $r = 2$, for example), so this process is tedious. Additionally, numerically finding the roots is ill-conditioned\footnote{Small perturbations to the coefficients lead to large changes in the roots.}.

\bigskip 

So, how do we find the eigenvalues numerically? As implied earlier, the characteristic equation is not useful for finding eigenvalues numerically, so we will perform iterative methods to find eigenvalues. 

\subsubsection{QR Iteration}
Before we begin this section, let's note a theorem. 
\begin{theorem}{}{}
    Let $T \in \C^{n \times n}$ be a (lower- or upper-) triangular matrix. Then, the eigenvalues of $T$ are the main-diagonal entries $t_{11}, \hdots, t_{nn}$. 
\end{theorem}
We will construct a sequence of matrices with the same eigenvalues that converge to a triangular matrix. The reason why we choose a triangular matrix is because its eigenvalues are on the diagonal. 



\end{document}