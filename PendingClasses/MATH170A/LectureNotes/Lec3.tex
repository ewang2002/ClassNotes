\documentclass[letterpaper]{article}
\usepackage[margin=1in]{geometry}
\usepackage[utf8]{inputenc}
\usepackage{textcomp}
\usepackage{amssymb}
\usepackage{natbib}
\usepackage{graphicx}
\usepackage{gensymb}
\usepackage{amsthm, amsmath, mathtools}
\usepackage[dvipsnames]{xcolor}
\usepackage{enumerate}
\usepackage{mdframed}
\usepackage[most]{tcolorbox}
\usepackage{csquotes}
% https://tex.stackexchange.com/questions/13506/how-to-continue-the-framed-text-box-on-multiple-pages

\tcbuselibrary{theorems}

\newcommand{\R}{\mathbb{R}}
\newcommand{\Z}{\mathbb{Z}}
\newcommand{\N}{\mathbb{N}}
\newcommand{\Q}{\mathbb{Q}}
\newcommand{\C}{\mathbb{C}}
\newcommand{\code}[1]{\texttt{#1}}
\newcommand{\mdiamond}{$\diamondsuit$}
\newcommand{\PowerSet}{\mathcal{P}}
\newcommand{\Mod}[1]{\ (\mathrm{mod}\ #1)}
\DeclareMathOperator{\lcm}{lcm}

%\newtheorem*{theorem}{Theorem}
%\newtheorem*{definition}{Definition}
%\newtheorem*{corollary}{Corollary}
%\newtheorem*{lemma}{Lemma}
\newtheorem*{proposition}{Proposition}


\newtcbtheorem[number within=section]{theorem}{Theorem}
{colback=green!5,colframe=green!35!black,fonttitle=\bfseries}{th}

\newtcbtheorem[number within=section]{definition}{Definition}
{colback=blue!5,colframe=blue!35!black,fonttitle=\bfseries}{def}

\newtcbtheorem[number within=section]{corollary}{Corollary}
{colback=yellow!5,colframe=yellow!35!black,fonttitle=\bfseries}{cor}

\newtcbtheorem[number within=section]{lemma}{Lemma}
{colback=red!5,colframe=red!35!black,fonttitle=\bfseries}{lem}

\newtcbtheorem[number within=section]{example}{Example}
{colback=white!5,colframe=white!35!black,fonttitle=\bfseries}{def}

\newtcbtheorem[number within=section]{note}{Important Note}{
        enhanced,
        sharp corners,
        attach boxed title to top left={
            xshift=-1mm,
            yshift=-5mm,
            yshifttext=-1mm
        },
        top=1.5em,
        colback=white,
        colframe=black,
        fonttitle=\bfseries,
        boxed title style={
            sharp corners,
            size=small,
            colback=red!75!black,
            colframe=red!75!black,
        } 
    }{impnote}
\usepackage[utf8]{inputenc}
\usepackage[english]{babel}
\usepackage{fancyhdr}
\usepackage[hidelinks]{hyperref}

\pagestyle{fancy}
\fancyhf{}
\rhead{Math 170A}
\chead{Friday, January 13, 2023}
\lhead{Lecture 3}
\rfoot{\thepage}

\setlength{\parindent}{0pt}

\newcommand{\0}{\mathbf{0}}
\newcommand{\y}{\mathbf{y}}
\renewcommand{\b}{\mathbf{b}}
\newcommand{\x}{\mathbf{x}}
\newcommand{\e}{\mathbf{e}}

\begin{document}

\section{Triangular Systems (1.3)}
Generally, it's common practice to reduce general systems down to a triangular form, generally through a process known as Gaussian elimination. They are also easy to solve, and can be solved inexpensively. 

\subsection{Lower and Upper Triangular Matrices}
We say that $L \in \R^{n \times n}$ is a \textbf{lower triangular} matrix if $\ell_{ij} = 0$ whenever $i < j$. Thus, a lower triangular matrix has the form 
\[A = \begin{bmatrix}
    \ell_{11} & 0 & 0 & 0 & 0 & \hdots & 0 \\ 
    \ell_{21} & \ell_{22} & 0 & 0 & 0 & \hdots & 0 \\ 
    \ell_{31} & \ell_{32} & \ell_{33} & 0 & 0 & \hdots & 0 \\ 
    \ell_{41} & \ell_{42} & \ell_{43} & \ell_{44} & 0 & \hdots & 0 \\ 
    \ell_{51} & \ell_{52} & \ell_{53} & \ell_{54} & \ell_{55} & \hdots & 0 \\
    \vdots & \vdots & \vdots & \vdots & \vdots & \ddots & \vdots \\  
    \ell_{n1} & \ell_{n2} & \ell_{n3} & \ell_{n4} & \ell_{n5} & \hdots & \ell_{nn}
\end{bmatrix}.\]
Similarly, we say that $U \in \R^{n \times n}$ is an \textbf{upper triangular} matrix if $u_{ij} = 0$ whenever $i > j$; they have the form 
\[A = \begin{bmatrix}
    u_{11} & u_{12} & u_{13} & u_{14} & u_{15} & \hdots & u_{1n} \\ 
    0 & u_{22} & u_{23} & u_{24} & u_{25} & \hdots & u_{2n} \\ 
    0 & 0 & u_{33} & u_{34} & u_{35} & \hdots & u_{3n} \\ 
    0 & 0 & 0 & u_{44} & u_{45} & \hdots & u_{4n} \\ 
    0 & 0 & 0 & 0 & u_{55} & \hdots & u_{5n} \\
    \vdots & \vdots & \vdots & \vdots & \vdots & \ddots & \vdots \\  
    0 & 0 & 0 & 0 & 0 & \hdots & u_{nn}
\end{bmatrix}.\]
A \textbf{triangular matrix} is one that is either upper or lower triangular.

\begin{mdframed}
    (Example.) Consider the following matrix \[\begin{bmatrix}
        2 & 0 & 0 \\ 
        4 & -1 & 0 \\ 
        -2 & 0 & 3
    \end{bmatrix}.\]
    This is a \textbf{lower triangular matrix}. 
\end{mdframed}
\textbf{Remarks:} 
\begin{itemize}
    \item We can have 0's in the places where there are normally nonzero numbers. This does not violate the definition of a lower- or upper-triangular matrix. 
    \item Because of this, we can say that a diagonal matrix is both a lower- and upper-triangular matrix.
\end{itemize} 

\subsection{Uniqueness of Solution}
When is there a unique system to $L\x = \b$ or $U\x = \b$? 

\begin{mdframed}
    A triangular system (lower or upper) has a unique solution if and only if \emph{all \underline{diagonal} entries} are non-zero.
\end{mdframed}

In fact, as long as the \emph{determinant} of the triangular matrix $A$ is not 0, there will be a unique solution. With triangular matrices, computing the determinant is easy: we just need to multiply all the elements on the \emph{diagonal} together. More technically, as long as 
\[\det(A) = a_{11} \cdot a_{22} \cdot a_{33} \cdot \hdots \cdot a_{nn} \neq 0,\]
then $A\x = \b$ will be a unique solution.

\subsection{Solving Lower Triangular Systems: Forward Substitution}
Suppose we want to solve the following system 
\[\begin{bmatrix}
    \ell_{11} & 0 & 0 & \hdots & 0  \\ 
    \ell_{21} & \ell_{22} & 0 & \hdots & 0  \\ 
    \ell_{31} & \ell_{32} & \ell_{33} & \hdots & 0  \\ 
    \vdots & \vdots & \vdots & \ddots & \vdots \\ 
    \ell_{n1} & \ell_{n2} & \ell_{n3} & \hdots & \ell_{nn}
\end{bmatrix} \begin{bmatrix}
    x_1 \\ x_2 \\ x_3 \\ \vdots \\ x_n
\end{bmatrix} = \begin{bmatrix}
    b_1 \\ b_2 \\ b_3 \\ \vdots \\ b_n
\end{bmatrix}.\]
Notice that \[\ell_{11} x_1 = b_1 \implies x_1 = \frac{b_1}{\ell_{11}}.\]
Also notice that \[\ell_{21} x_1 + \ell_{22}x_2 = b_2 \implies \ell_{22}x_2 = b_2 - \ell_{21}x_1 \implies x_2 = \frac{b_2 - \ell_{21} x_1}{\ell_{22}}.\]
And then notice that \[\ell_{31} x_1 + \ell_{32} x_2 + \ell_{33} x_3 = b_3 \implies \ell_{33}x_3 = b_3 - \ell_{31}x_1 - \ell_{32}x_2 \implies x_3 = \frac{b_3 - \ell_{31}x_1 - \ell_{32}x_2}{\ell_{33}}.\]
Notice how we started off with a simple linear equation, which gave us the answer for $x_1$, and then we can use $x_1$ to find $x_2$ easily in the next equation, and so on. This algorithm is known as \textbf{forward substitution}. For $i = 1, \hdots, n$, it makes use of the formula 
\[\boxed{x_i = \frac{b_i - \ell_{i1} x_1 - \ell_{i2} x_2 - \hdots - \ell_{i,i - 1} x_{i - 1}}{\ell_{ii}} = \frac{b_i - \sum_{j = 1}^{i - 1} \ell_{ij} x_j}{\ell_{ii}}.}\]
Roughly speaking, the algorithm looks like the following:
\begin{algorithmic}
    \For{$i = 1, \hdots, n$}
        \For{$j = 1, \hdots, i - 1$}
            \State $x_i \gets x_i - \ell_{ij} x_j$
        \EndFor

        \If{$g_{ii} = 0$}
            \State Set Error Flag, Exit 
        \EndIf

        \State $x_i \gets x_i / \ell_{ii}$
    \EndFor
\end{algorithmic}

\end{document}