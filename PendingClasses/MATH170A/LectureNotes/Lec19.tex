\documentclass[letterpaper]{article}
\usepackage[margin=1in]{geometry}
\usepackage[utf8]{inputenc}
\usepackage{textcomp}
\usepackage{amssymb}
\usepackage{natbib}
\usepackage{graphicx}
\usepackage{gensymb}
\usepackage{amsthm, amsmath, mathtools}
\usepackage[dvipsnames]{xcolor}
\usepackage{enumerate}
\usepackage{mdframed}
\usepackage[most]{tcolorbox}
\usepackage{csquotes}
% https://tex.stackexchange.com/questions/13506/how-to-continue-the-framed-text-box-on-multiple-pages

\tcbuselibrary{theorems}

\newcommand{\R}{\mathbb{R}}
\newcommand{\Z}{\mathbb{Z}}
\newcommand{\N}{\mathbb{N}}
\newcommand{\Q}{\mathbb{Q}}
\newcommand{\C}{\mathbb{C}}
\newcommand{\code}[1]{\texttt{#1}}
\newcommand{\mdiamond}{$\diamondsuit$}
\newcommand{\PowerSet}{\mathcal{P}}
\newcommand{\Mod}[1]{\ (\mathrm{mod}\ #1)}
\DeclareMathOperator{\lcm}{lcm}

%\newtheorem*{theorem}{Theorem}
%\newtheorem*{definition}{Definition}
%\newtheorem*{corollary}{Corollary}
%\newtheorem*{lemma}{Lemma}
\newtheorem*{proposition}{Proposition}


\newtcbtheorem[number within=section]{theorem}{Theorem}
{colback=green!5,colframe=green!35!black,fonttitle=\bfseries}{th}

\newtcbtheorem[number within=section]{definition}{Definition}
{colback=blue!5,colframe=blue!35!black,fonttitle=\bfseries}{def}

\newtcbtheorem[number within=section]{corollary}{Corollary}
{colback=yellow!5,colframe=yellow!35!black,fonttitle=\bfseries}{cor}

\newtcbtheorem[number within=section]{lemma}{Lemma}
{colback=red!5,colframe=red!35!black,fonttitle=\bfseries}{lem}

\newtcbtheorem[number within=section]{example}{Example}
{colback=white!5,colframe=white!35!black,fonttitle=\bfseries}{def}

\newtcbtheorem[number within=section]{note}{Important Note}{
        enhanced,
        sharp corners,
        attach boxed title to top left={
            xshift=-1mm,
            yshift=-5mm,
            yshifttext=-1mm
        },
        top=1.5em,
        colback=white,
        colframe=black,
        fonttitle=\bfseries,
        boxed title style={
            sharp corners,
            size=small,
            colback=red!75!black,
            colframe=red!75!black,
        } 
    }{impnote}
\usepackage[utf8]{inputenc}
\usepackage[english]{babel}
\usepackage{fancyhdr}
\usepackage[hidelinks]{hyperref}
\usepackage{csquotes}

\pagestyle{fancy}
\fancyhf{}
\rhead{Math 170A}
\chead{Wednesday, March 01, 2023}
\lhead{Lecture 20}
\rfoot{\thepage}

\setlength{\parindent}{0pt}
\newcommand{\0}{\mathbf{0}}
\newcommand{\y}{\mathbf{y}}
\renewcommand{\b}{\mathbf{b}}
\newcommand{\x}{\mathbf{x}}
\newcommand{\e}{\mathbf{e}}
\newcommand{\rr}{\mathbf{r}}
\newcommand{\vv}{\mathbf{v}}
\renewcommand{\u}{\mathbf{u}}

\begin{document}

\section{The Power Method (5.3)}
Let $A \in \C^{n \times n}$, and assume that $A$ is semisimple. Let $\lambda_1, \lambda_2, \hdots, \lambda_n$ denote the eigenvalues associated with $v_1, \hdots, v_n$, respectively. Assume that the vectors are ordered so that $|\lambda_1| \geq |\lambda_2| \geq \hdots \geq |\lambda_n|$. If $|\lambda_1| > |\lambda_2|$, then $\lambda_1$ is called the \textbf{dominant eigenvalue}\footnote{Basically, the largest absolute eigenvalue.} and $v_1$ is called the \textbf{dominant eigenvector} of $A$.  

\subsection{The Iterative Power Method}
Assuming we have $|\lambda_1| > |\lambda_2|$ as described above (otherwise, this method may not work), the general idea behind the iterative power method is that we can pick $q \in \R^n$ randomly. Then, we can form the sequence of vectors
\[q, Aq, A^2 q, A^3 q, \hdots.\]
To calculate this sequence, we don't necessarily need to form the powers of $A$ explicitly. Each vector in the sequence can be obtained by multiplying the previous vector by $A$, e.g., $A^{j + 1}q = A(A^j q)$. It's easy to show that the sequence converge, in a sense, to a dominant eigenvector, for almost all choices of $q$. Since $v_1, \hdots, v_n$ form a basis for $\C^n$, there exists constants $c_1, \hdots, c_n$ such that  
\[q = c_1 v_1 + c_2 v_2 + \hdots + c_n v_n.\]
We don't know what $v_1, \hdots, v_n$ are, so we don't know what $c_1, \hdots, c_n$ are, either. However, it's clear that, for any choice of $q$, $c_1$ will be nonzero. The argument that follows is valid for every $q$ for which $c_1 \neq 0$; multiplying by $A$, we have 
\begin{equation*}
    \begin{aligned}
        Aq &= c_1 Av_1 + c_2 Ac_2 + \hdots + c_n Av_n \\ 
            &= c_1 \lambda_1 v_1 + c_2 \lambda_2 v_2 + \hdots + c_n  \lambda_n v_n.
    \end{aligned}
\end{equation*}
Similarly, 
\begin{equation*}
    \begin{aligned}
        A^2 q &= A(c_1 \lambda_1 v_1 + c_2 \lambda_2 v_2 + \hdots + c_n  \lambda_n v_n) \\
            &= c_1 \lambda_1 (Av_1) + c_2 \lambda_2 (Av_2) + \hdots + c_n \lambda_n (Av_n) \\ 
            &= c_1 \lambda_1 (\lambda_1 v_1) + c_2 \lambda_2 (\lambda_2 v_2) + \hdots + c_n \lambda_n (\lambda_n v_n) \\ 
            &= c_1 \lambda_1^2 v_1 + c_2 \lambda_2^2 v_2 + \hdots + c_n \lambda_n^2 v_n.
    \end{aligned}
\end{equation*}
In general, we have 
\begin{equation*}
    \begin{aligned}
        A^j q &= c_1 \lambda_1^j v_1 + c_2 \lambda_2^j v_2 + \hdots + c_n \lambda_n^j v_n \\ 
            &= \lambda_1^j \left(c_1 v_1 + c_2 \left(\frac{\lambda_2}{\lambda_1}\right)^j v_2 + \hdots + c_n \left(\frac{\lambda_n}{\lambda_1}\right)^j v_n\right).
    \end{aligned}
\end{equation*}
In particular, we have 
\[\frac{1}{\lambda_1^j} A^i q = c_1 v_1 + c_2 \left(\frac{\lambda_2}{\lambda_1}\right)^j v_2 + \hdots + c_n \left(\frac{\lambda_n}{\lambda_1}\right)^j v_n.\]
Notice that $\lim_{j \mapsto \infty} \left(\frac{\lambda_i}{\lambda_1}\right)^j = 0$, so \[\lim_{j \mapsto \infty} \frac{1}{\lambda_1^j} A^j q = c_1 v_1,\]
the dominant eigenvector.

\bigskip 

\textbf{Remark:} This only works if $\lambda_1$ is known. 


\end{document}