\documentclass[letterpaper]{article}
\usepackage[margin=1in]{geometry}
\usepackage[utf8]{inputenc}
\usepackage{textcomp}
\usepackage{amssymb}
\usepackage{natbib}
\usepackage{graphicx}
\usepackage{gensymb}
\usepackage{amsthm, amsmath, mathtools}
\usepackage[dvipsnames]{xcolor}
\usepackage{enumerate}
\usepackage{mdframed}
\usepackage[most]{tcolorbox}
\usepackage{csquotes}
% https://tex.stackexchange.com/questions/13506/how-to-continue-the-framed-text-box-on-multiple-pages

\tcbuselibrary{theorems}

\newcommand{\R}{\mathbb{R}}
\newcommand{\Z}{\mathbb{Z}}
\newcommand{\N}{\mathbb{N}}
\newcommand{\Q}{\mathbb{Q}}
\newcommand{\C}{\mathbb{C}}
\newcommand{\code}[1]{\texttt{#1}}
\newcommand{\mdiamond}{$\diamondsuit$}
\newcommand{\PowerSet}{\mathcal{P}}
\newcommand{\Mod}[1]{\ (\mathrm{mod}\ #1)}
\DeclareMathOperator{\lcm}{lcm}

%\newtheorem*{theorem}{Theorem}
%\newtheorem*{definition}{Definition}
%\newtheorem*{corollary}{Corollary}
%\newtheorem*{lemma}{Lemma}
\newtheorem*{proposition}{Proposition}


\newtcbtheorem[number within=section]{theorem}{Theorem}
{colback=green!5,colframe=green!35!black,fonttitle=\bfseries}{th}

\newtcbtheorem[number within=section]{definition}{Definition}
{colback=blue!5,colframe=blue!35!black,fonttitle=\bfseries}{def}

\newtcbtheorem[number within=section]{corollary}{Corollary}
{colback=yellow!5,colframe=yellow!35!black,fonttitle=\bfseries}{cor}

\newtcbtheorem[number within=section]{lemma}{Lemma}
{colback=red!5,colframe=red!35!black,fonttitle=\bfseries}{lem}

\newtcbtheorem[number within=section]{example}{Example}
{colback=white!5,colframe=white!35!black,fonttitle=\bfseries}{def}

\newtcbtheorem[number within=section]{note}{Important Note}{
        enhanced,
        sharp corners,
        attach boxed title to top left={
            xshift=-1mm,
            yshift=-5mm,
            yshifttext=-1mm
        },
        top=1.5em,
        colback=white,
        colframe=black,
        fonttitle=\bfseries,
        boxed title style={
            sharp corners,
            size=small,
            colback=red!75!black,
            colframe=red!75!black,
        } 
    }{impnote}
\usepackage[utf8]{inputenc}
\usepackage[english]{babel}
\usepackage{fancyhdr}
\usepackage[hidelinks]{hyperref}
\usepackage{csquotes}

\pagestyle{fancy}
\fancyhf{}
\rhead{Math 170A}
\chead{Friday, February 17, 2023}
\lhead{Lecture 16}
\rfoot{\thepage}

\setlength{\parindent}{0pt}
\newcommand{\0}{\mathbf{0}}
\newcommand{\y}{\mathbf{y}}
\renewcommand{\b}{\mathbf{b}}
\newcommand{\x}{\mathbf{x}}
\newcommand{\e}{\mathbf{e}}
\newcommand{\rr}{\mathbf{r}}
\newcommand{\vv}{\mathbf{v}}
\renewcommand{\u}{\mathbf{u}}

\begin{document}

\section{Singular Value Decomposition \& Basic Applications (4.1, 4.2)}
\begin{theorem}{Geometric Singular Value Decomposition Theorem}{}
    Let $A \in \R^{n \times m}$ be a nonzero matrix with rank $r$. Then, $\R^m$ has an orthogonal basis $v_1, v_2, \hdots, v_m$, $\R^n$ has an orthonormal basis $u_1, u_2, \hdots, u_n$, and there exists $\sigma_1 \geq \sigma_2 \geq \hdots \geq \sigma_r > 0$ such that 
    \[Av_i = \begin{cases}
        \sigma_i u_i & i = 1, \hdots, r \\ 
        \0 & i = r + 1, \hdots, m
    \end{cases} \qquad A^T u_i = \begin{cases}
        \sigma_i v_i & i = 1, \hdots, r \\ 
        \0 & i = r + 1, \hdots, n
    \end{cases}.\]
    For SVD, there exists orthogonal matrices $U \in \R^{n \times n}$ and $V \in \R^{m \times m}$ such that \[A = U \Sigma V^T\] with \[\Sigma = \begin{bmatrix}
        \sigma_1 & 0 & 0 & 0 & 0 & \hdots & 0 \\ 
        0 & \sigma_2 & 0 & 0 & 0 & \hdots & 0 \\ 
        0 & 0 & \ddots & 0 & 0 & \hdots & 0 \\ 
        0 & 0 & 0 & \sigma_r & 0 & \hdots & 0 \\ 
        0 & 0 & 0 & 0 & 0 & \hdots & 0 \\ 
        \vdots & \vdots & \vdots & \vdots & \vdots & \ddots & \vdots \\ 
        0 & 0 & 0 & 0 & 0 & \hdots & 0
    \end{bmatrix} \in \R^{n \times m}.\]
\end{theorem}
\textbf{Remarks:}
\begin{itemize}
    \item This is called a full SVD\footnote{Later, we will introduced a reduced SVD.} Here, $\sigma_1, \sigma_2, \hdots, \sigma_r$ are called the \emph{singular values}.
    \item Note that we can write $U$ and $V$ as a vector of vectors, 
    \[U = \begin{bmatrix}
        u_1, u_2, \hdots, u_n
    \end{bmatrix}, u_i \in \R^{n}\]
    and \[V = \begin{bmatrix}
        v_1, v_2, \hdots, v_m
    \end{bmatrix}, v_m \in \R^m.\]
\end{itemize}

\subsection{Intuition}
To get some intuition, let's suppose we start with $A = U\Sigma V^T$. Then, we know that \[AV = U\Sigma V^T V \implies AV = U\Sigma.\] Rewriting $V$ and $U$ as columns, we have 
\begin{equation}{\label{lec15:1}}
    A\begin{bmatrix}
        v_1 & v_2 & \hdots & v_m
    \end{bmatrix} = \begin{bmatrix}
        u_1 & u_2 & \hdots & u_n
    \end{bmatrix} \begin{bmatrix}
        \sigma_1 & 0 & 0 & 0 & 0 & \hdots & 0 \\ 
        0 & \sigma_2 & 0 & 0 & 0 & \hdots & 0 \\ 
        0 & 0 & \ddots & 0 & 0 & \hdots & 0 \\ 
        0 & 0 & 0 & \sigma_r & 0 & \hdots & 0 \\ 
        0 & 0 & 0 & 0 & 0 & \hdots & 0 \\ 
        \vdots & \vdots & \vdots & \vdots & \vdots & \ddots & \vdots \\ 
        0 & 0 & 0 & 0 & 0 & \hdots & 0
    \end{bmatrix}.
\end{equation}
For some matrix 
\[B = \begin{bmatrix}
    b_1 & b_2 & \hdots & b_m
\end{bmatrix},\] then \[B \begin{bmatrix}
    \sigma_1 & 0 & 0 & \hdots & 0 \\ 
    0 & 0 & 0 & \hdots & 0 \\ 
    0 & 0 & 0 & \hdots & 0 \\ 
    \vdots & \vdots & \vdots & \ddots & \vdots \\
    0 & 0 & 0 & \hdots & 0
\end{bmatrix} = \begin{bmatrix}
    b_1 & b_2 & \hdots & b_m
\end{bmatrix}\begin{bmatrix}
    \sigma_1 & 0 & 0 & \hdots & 0 \\ 
    0 & 0 & 0 & \hdots & 0 \\ 
    0 & 0 & 0 & \hdots & 0 \\ 
    \vdots & \vdots & \vdots & \ddots & \vdots \\
    0 & 0 & 0 & \hdots & 0
\end{bmatrix} = \begin{bmatrix}
    \sigma_1 b_1 & \0 & \hdots & \0
\end{bmatrix}.\]
Likewise, \[B \begin{bmatrix}
    0 & 0 & 0 & \hdots & 0 \\ 
    0 & \sigma_2 & 0 & \hdots & 0 \\ 
    0 & 0 & 0 & \hdots & 0 \\ 
    \vdots & \vdots & \vdots & \ddots & \vdots \\
    0 & 0 & 0 & \hdots & 0
\end{bmatrix} = \begin{bmatrix}
    b_1 & b_2 & \hdots & b_m
\end{bmatrix}\begin{bmatrix}
    0 & 0 & 0 & \hdots & 0 \\ 
    0 & \sigma_2 & 0 & \hdots & 0 \\ 
    0 & 0 & 0 & \hdots & 0 \\ 
    \vdots & \vdots & \vdots & \ddots & \vdots \\
    0 & 0 & 0 & \hdots & 0
\end{bmatrix} = \begin{bmatrix}
    \0 & \sigma_2 b_2 & \hdots & \0
\end{bmatrix}.\] 
Notice how $\sigma_i$ scales the $i$th column of $B$. Now, let's suppose we combine the operations above into one matrix: 
\[B \begin{bmatrix}
    \sigma_1 & 0 & 0 & \hdots & 0 \\ 
    0 & \sigma_2 & 0 & \hdots & 0 \\ 
    0 & 0 & 0 & \hdots & 0 \\ 
    \vdots & \vdots & \vdots & \ddots & \vdots \\
    0 & 0 & 0 & \hdots & 0
\end{bmatrix} = \begin{bmatrix}
    0 & b_2 & \hdots & 0
\end{bmatrix}\begin{bmatrix}
    \sigma_1 & 0 & 0 & \hdots & 0 \\ 
    0 & \sigma_2 & 0 & \hdots & 0 \\ 
    0 & 0 & 0 & \hdots & 0 \\ 
    \vdots & \vdots & \vdots & \ddots & \vdots \\
    0 & 0 & 0 & \hdots & 0
\end{bmatrix} = \begin{bmatrix}
    \sigma_1 b_1 & \sigma_2 b_2 & \hdots & \0
\end{bmatrix}.\] 
Here, both column 1 and 2 are scaled. So, going back to equation (\ref{lec15:1}), we have 
\[\begin{bmatrix}
    \sigma_1 u_1 & \sigma_2 u_2 & \hdots & \sigma_r u_r & \0 & \hdots & \0
\end{bmatrix}.\]
This gives us the formula,
\[Av_i = \begin{cases}
    \sigma_i u_i & i = 1, \hdots, r \\ 
    \0 & i = r + 1, \hdots, m
\end{cases}.\]

Likewise, let's consider $A^T$; 
\begin{equation*}
    \begin{aligned}
        A^T &= (U\Sigma V^T)^T \\ 
            &= (V^T) \Sigma^T U^T \\ 
            &= V\Sigma U^T.
    \end{aligned}
\end{equation*}
From there, we have $A^T U = V\Sigma^T U^T U = V\Sigma^T.$ Expanding out the matrices, we have 
\[A^T \begin{bmatrix}
    u_1 & u_2 & \hdots & u_n
\end{bmatrix} = \begin{bmatrix}
    v_1 & v_2 & \hdots & v_m
\end{bmatrix} \begin{bmatrix}
    \sigma_1 & 0 & 0 & \hdots & 0 \\ 
    0 & \sigma_2 & 0 & \hdots & 0 \\ 
    0 & 0 & \sigma_r & \hdots & 0 \\ 
    \vdots & \vdots & \vdots & \ddots & \vdots \\
    0 & 0 & 0 & \hdots & 0
\end{bmatrix} = \begin{bmatrix}
    \sigma_1 v_1 & \sigma_2 v_2 & \hdots & \sigma_r v_r & \0 & \hdots
\end{bmatrix}.\]
Note that $\Sigma^T$ is not a tall matrix, but a wide one; instead of being $n \times m$, it's $m \times n$. This gives us the formula
\[A^T u_i = \begin{cases}
    \sigma_i v_i & i = 1, \hdots, r \\ 
    \0 & i = r + 1, \hdots, m
\end{cases}.\] In particular, we can see that 
\begin{center}
    \begin{tabular}{p{2in} p{2in}}
        \[v_1 \xrightarrow[\sigma_1]{A} u_1\]
        \[v_2 \xrightarrow[\sigma_2]{} u_2\]
        \[v_3 \xrightarrow[\sigma_3]{} u_3\]
        \[\vdots\]
        \[v_r \xrightarrow[\sigma_r]{} u_r\]
        \[v_{r + 1} \xrightarrow{} \0\]
        \[\vdots\]
        \[v_{m} \xrightarrow{} \0\] &  
        \[u_1 \xrightarrow[\sigma_1]{A^T} v_1\]
        \[u_2 \xrightarrow[\sigma_2]{} v_2\]
        \[u_3 \xrightarrow[\sigma_3]{} v_3\]
        \[\vdots\]
        \[u_r \xrightarrow[\sigma_r]{} v_r\]
        \[u_{r + 1} \xrightarrow{} \0\]
        \[\vdots\]
        \[u_{n} \xrightarrow{} \0\]
    \end{tabular}
\end{center}
So, in particular, the range of matrix A, $\mathcal{R}(A)$, is spanned by 
\[v_1 \xrightarrow[\sigma_1]{A} u_1\]
\[v_2 \xrightarrow[\sigma_2]{} u_2\]
\[v_3 \xrightarrow[\sigma_3]{} u_3\]
\[\vdots\]
\[v_r \xrightarrow[\sigma_r]{} u_r.\]
The null space of $A$, $\mathcal{N}(A)$, is spanned by 
\[v_{r + 1} \xrightarrow{} \0\]
\[\vdots\]
\[v_{m} \xrightarrow{} \0.\]
For $A^T$, this is analogous.

\subsection{Fundamental Subspaces}
The SVD displays orthogonal bases for the four Fundamental subspaces, $\mathcal{R}(A)$, $\mathcal{N}(A)$, $\mathcal{R}(A^T)$, and $\mathcal{N}(A^T)$, where $\mathcal{R}$ is the \emph{range} and $\mathcal{N}$ is the null space. 
\[\mathcal{R}(A) = \text{span}\{u_1, u_2, \hdots, u_r\}.\]
\[\mathcal{N}(A) = \text{span}\{v_{r + 1}, v_{r + 2}, \hdots, v_m\}.\]
\[\mathcal{R}(A^T) = \text{span}\{v_1, v_2, \hdots, v_r\}.\]
\[\mathcal{N}(A^T) = \text{span}\{u_{r + 1}, u_{r + 2}, \hdots, u_n\}.\]
In particular, 
\[\mathcal{R}(A) + \mathcal{N}(A^T) = \text{span}\{u_1, u_2, \hdots, u_n\}.\]
\[\mathcal{R}(A^T) + \mathcal{N}(A) = \text{span}\{v_1, v_2, \hdots, v_m\}.\]
We can see that $\mathcal{R}(A^T) = \mathcal{N}(A)^{\perp}$ and $\mathcal{R}(A) = \mathcal{N}(A^T)^{\perp}.$

\begin{corollary}{}{}
    Let $A \in \R^{n \times m}$. Then, $\dim(\mathcal{R}(A)) + \dim(\mathcal{N}(A)) = m$ and $\dim(\mathcal{R}(A^T)) + \dim(\mathcal{N}(A^T)) = n$.
\end{corollary}

\subsection{Reduced SVD}
We'll introduce this section with an example. 

\begin{mdframed}
    (Example.) Suppose we have a $3 \times 3$ matrix of rank\footnote{It has 3 rows but only has 2 non-zero singular values, $\sigma_1$ and $\sigma_2$} 2. Then, 
    \begin{equation*}
        \begin{aligned}
            A &= \underbrace{\begin{bmatrix}
                u_{11} & u_{12} & u_{13} \\ 
                u_{21} & u_{22} & u_{23} \\ 
                u_{31} & u_{32} & u_{33} 
            \end{bmatrix}}_{U} \underbrace{\begin{bmatrix}
                \sigma_1 & 0 & 0 \\ 
                0 & \sigma_2 & 0 \\ 
                0 & 0 & 0 
            \end{bmatrix}}_{\Sigma} \underbrace{\begin{bmatrix}
                v_{11} & v_{21} & v_{31} \\ 
                v_{12} & v_{22} & v_{32} \\ 
                v_{13} & v_{23} & v_{33} 
            \end{bmatrix}}_{V^T} \\ 
                &= \begin{bmatrix}
                    \sigma_1 u_{11} & \sigma_2 u_{12} & 0 \\ 
                    \sigma_1 u_{21} & \sigma_2 u_{22} & 0 \\ 
                    \sigma_1 u_{31} & \sigma_2 u_{32} & 0 
                \end{bmatrix} \begin{bmatrix}
                    v_{11} & v_{21} & v_{31} \\ 
                    v_{12} & v_{22} & v_{32} \\ 
                    v_{13} & v_{23} & v_{33} 
                \end{bmatrix}.
        \end{aligned}
    \end{equation*} 
    As a result, we'll end up multiplying $v_{13}$, $v_{23}$, $v_{33}$ by 0. So, instead, what if we have: 
    \[A = \underbrace{\begin{bmatrix}
        u_{11} & u_{12} \\ 
        u_{21} & u_{22} \\ 
        u_{31} & u_{32} 
    \end{bmatrix}}_{\hat{U}} \underbrace{\begin{bmatrix}
        \sigma_1 & 0 \\ 
        0 & \sigma_2
    \end{bmatrix}}_{\hat{\Sigma}} \underbrace{\begin{bmatrix}
        v_{11} & v_{21} & v_{31} \\ 
        v_{12} & v_{22} & v_{32}
    \end{bmatrix}}_{\hat{V^T}}.\]
    This is known as the reduced SVD. 
\end{mdframed}

\begin{theorem}{Condensed SVD Theorem}{}
    Let $A \in \R^{n \times m}$ be a nonzero matrix of rank $r$. Then, there exists $\hat{U} \in \R^{n \times r}$, $\hat{\Sigma} \in \R^{r \times r}$, and $\hat{V} \in \R^{m \times r}$, such that $\hat{U}$ and $\hat{V}$ are isometries, and $\hat{\Sigma}$ is a diagonal matrix with main-diagonal entries $\sigma_1 \geq \sigma_2 \geq \hdots \geq \sigma_r > 0$ and \[A = \hat{U}\hat{\Sigma}\hat{V^T}.\]
\end{theorem}

\end{document}