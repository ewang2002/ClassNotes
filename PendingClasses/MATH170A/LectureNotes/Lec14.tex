\documentclass[letterpaper]{article}
\usepackage[margin=1in]{geometry}
\usepackage[utf8]{inputenc}
\usepackage{textcomp}
\usepackage{amssymb}
\usepackage{natbib}
\usepackage{graphicx}
\usepackage{gensymb}
\usepackage{amsthm, amsmath, mathtools}
\usepackage[dvipsnames]{xcolor}
\usepackage{enumerate}
\usepackage{mdframed}
\usepackage[most]{tcolorbox}
\usepackage{csquotes}
% https://tex.stackexchange.com/questions/13506/how-to-continue-the-framed-text-box-on-multiple-pages

\tcbuselibrary{theorems}

\newcommand{\R}{\mathbb{R}}
\newcommand{\Z}{\mathbb{Z}}
\newcommand{\N}{\mathbb{N}}
\newcommand{\Q}{\mathbb{Q}}
\newcommand{\C}{\mathbb{C}}
\newcommand{\code}[1]{\texttt{#1}}
\newcommand{\mdiamond}{$\diamondsuit$}
\newcommand{\PowerSet}{\mathcal{P}}
\newcommand{\Mod}[1]{\ (\mathrm{mod}\ #1)}
\DeclareMathOperator{\lcm}{lcm}

%\newtheorem*{theorem}{Theorem}
%\newtheorem*{definition}{Definition}
%\newtheorem*{corollary}{Corollary}
%\newtheorem*{lemma}{Lemma}
\newtheorem*{proposition}{Proposition}


\newtcbtheorem[number within=section]{theorem}{Theorem}
{colback=green!5,colframe=green!35!black,fonttitle=\bfseries}{th}

\newtcbtheorem[number within=section]{definition}{Definition}
{colback=blue!5,colframe=blue!35!black,fonttitle=\bfseries}{def}

\newtcbtheorem[number within=section]{corollary}{Corollary}
{colback=yellow!5,colframe=yellow!35!black,fonttitle=\bfseries}{cor}

\newtcbtheorem[number within=section]{lemma}{Lemma}
{colback=red!5,colframe=red!35!black,fonttitle=\bfseries}{lem}

\newtcbtheorem[number within=section]{example}{Example}
{colback=white!5,colframe=white!35!black,fonttitle=\bfseries}{def}

\newtcbtheorem[number within=section]{note}{Important Note}{
        enhanced,
        sharp corners,
        attach boxed title to top left={
            xshift=-1mm,
            yshift=-5mm,
            yshifttext=-1mm
        },
        top=1.5em,
        colback=white,
        colframe=black,
        fonttitle=\bfseries,
        boxed title style={
            sharp corners,
            size=small,
            colback=red!75!black,
            colframe=red!75!black,
        } 
    }{impnote}
\usepackage[utf8]{inputenc}
\usepackage[english]{babel}
\usepackage{fancyhdr}
\usepackage[hidelinks]{hyperref}
\usepackage{csquotes}

\pagestyle{fancy}
\fancyhf{}
\rhead{Math 170A}
\chead{Wednesday, February 15, 2023}
\lhead{Lecture 15}
\rfoot{\thepage}

\setlength{\parindent}{0pt}
\newcommand{\0}{\mathbf{0}}
\newcommand{\y}{\mathbf{y}}
\renewcommand{\b}{\mathbf{b}}
\newcommand{\x}{\mathbf{x}}
\newcommand{\e}{\mathbf{e}}
\newcommand{\rr}{\mathbf{r}}
\newcommand{\vv}{\mathbf{v}}
\renewcommand{\u}{\mathbf{u}}

\begin{document}

\textbf{Note:} A lot of lecture 15 is actually in Lecture 14's notes since most of the lecture continues on Lecture 14. 

\section{Singular Value Decomposition (4.1)}
The singular value decomposition, known as \textbf{SVD}, is a matrix decomposition (similar to eigenvector, eigenvalues, but less restrictive). SVD is used for 
\begin{itemize}
    \item low rank approximation (imaging).
    \item least squares when rank is not full. 
\end{itemize}

\begin{theorem}{SVD Theorem}{}
    Let $A \in \R^{n \times m}$, with $A \neq 0$ and assume $n \geq m$ with $\text{rank}(A) = r \leq m$. Then, there exists orthogonal matrices $U \in \R^{n \times n}$ and $V \in \R^{m \times m}$ and positive numbers $\sigma_1 \geq \sigma_2 \geq \hdots \geq \sigma_r > 0$ such that \[A = U \Sigma V^T\] with \[\Sigma = \begin{bmatrix}
        \sigma_1 & 0 & 0 & 0 & 0 & \hdots & 0 \\ 
        0 & \sigma_2 & 0 & 0 & 0 & \hdots & 0 \\ 
        0 & 0 & \ddots & 0 & 0 & \hdots & 0 \\ 
        0 & 0 & 0 & \sigma_r & 0 & \hdots & 0 \\ 
        0 & 0 & 0 & 0 & 0 & \hdots & 0 \\ 
        \vdots & \vdots & \vdots & \vdots & \vdots & \ddots & \vdots \\ 
        0 & 0 & 0 & 0 & 0 & \hdots & 0
    \end{bmatrix} \in \R^{n \times m}.\]
\end{theorem}
This is called a full SVD\footnote{Later, we will introduced a reduced SVD.} Here, $\sigma_1, \sigma_2, \hdots, \sigma_r$ are called the \emph{singular values}.

\textbf{Remarks:}
\begin{itemize}
    \item Notice that $A = U\Sigma V^T \implies AV = U\Sigma V^T V = U\Sigma$. If you compare this to eigenvectors and eigenvalues, you will notice that $AV = V\Lambda$. 
    
    \item The SVD is not unique. Instead of $U$, we can try $-U$; likewise, instead of $V$, we can use $-V$. 
\end{itemize}

\end{document}