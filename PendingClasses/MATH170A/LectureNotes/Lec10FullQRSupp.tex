\documentclass[letterpaper]{article}
\usepackage[margin=1in]{geometry}
\usepackage[utf8]{inputenc}
\usepackage{textcomp}
\usepackage{amssymb}
\usepackage{natbib}
\usepackage{graphicx}
\usepackage{gensymb}
\usepackage{amsthm, amsmath, mathtools}
\usepackage[dvipsnames]{xcolor}
\usepackage{enumerate}
\usepackage{mdframed}
\usepackage[most]{tcolorbox}
\usepackage{csquotes}
% https://tex.stackexchange.com/questions/13506/how-to-continue-the-framed-text-box-on-multiple-pages

\tcbuselibrary{theorems}

\newcommand{\R}{\mathbb{R}}
\newcommand{\Z}{\mathbb{Z}}
\newcommand{\N}{\mathbb{N}}
\newcommand{\Q}{\mathbb{Q}}
\newcommand{\C}{\mathbb{C}}
\newcommand{\code}[1]{\texttt{#1}}
\newcommand{\mdiamond}{$\diamondsuit$}
\newcommand{\PowerSet}{\mathcal{P}}
\newcommand{\Mod}[1]{\ (\mathrm{mod}\ #1)}
\DeclareMathOperator{\lcm}{lcm}

%\newtheorem*{theorem}{Theorem}
%\newtheorem*{definition}{Definition}
%\newtheorem*{corollary}{Corollary}
%\newtheorem*{lemma}{Lemma}
\newtheorem*{proposition}{Proposition}


\newtcbtheorem[number within=section]{theorem}{Theorem}
{colback=green!5,colframe=green!35!black,fonttitle=\bfseries}{th}

\newtcbtheorem[number within=section]{definition}{Definition}
{colback=blue!5,colframe=blue!35!black,fonttitle=\bfseries}{def}

\newtcbtheorem[number within=section]{corollary}{Corollary}
{colback=yellow!5,colframe=yellow!35!black,fonttitle=\bfseries}{cor}

\newtcbtheorem[number within=section]{lemma}{Lemma}
{colback=red!5,colframe=red!35!black,fonttitle=\bfseries}{lem}

\newtcbtheorem[number within=section]{example}{Example}
{colback=white!5,colframe=white!35!black,fonttitle=\bfseries}{def}

\newtcbtheorem[number within=section]{note}{Important Note}{
        enhanced,
        sharp corners,
        attach boxed title to top left={
            xshift=-1mm,
            yshift=-5mm,
            yshifttext=-1mm
        },
        top=1.5em,
        colback=white,
        colframe=black,
        fonttitle=\bfseries,
        boxed title style={
            sharp corners,
            size=small,
            colback=red!75!black,
            colframe=red!75!black,
        } 
    }{impnote}
\usepackage[utf8]{inputenc}
\usepackage[english]{babel}
\usepackage{fancyhdr}
\usepackage[hidelinks]{hyperref}

\pagestyle{fancy}
\fancyhf{}
\rhead{Math 170A}
\lhead{Lecture 10 Supplemental Notes}
\rfoot{\thepage}

\setlength{\parindent}{0pt}

\newcommand{\0}{\mathbf{0}}
\newcommand{\y}{\mathbf{y}}
\renewcommand{\b}{\mathbf{b}}
\newcommand{\x}{\mathbf{x}}
\newcommand{\e}{\mathbf{e}}
\newcommand{\rr}{\mathbf{r}}
\newcommand{\vv}{\mathbf{v}}
\renewcommand{\u}{\mathbf{u}}


\begin{document}


\section{QR Decomposition of a Tall Matrix}
\begin{mdframed}
    Find the full QR decomposition of \[A = \begin{bmatrix}
        1 & -1 & 4 \\
        1 & 4 & -2 \\ 
        1 & 4 & 2 \\ 
        1 & -1 & 0
    \end{bmatrix}.\]
\end{mdframed}

\begin{itemize}
    \item \underline{Step 1:} First, we start with $\vec{a_1} = \begin{bmatrix}
        1 \\ 1 \\ 1 \\ 1
    \end{bmatrix}$, the first column of $A$. We want to map $\vec{a_1} \mapsto ||\vec{a_1}||_2 \e_1$, so we have $||\vec{a_1}||_2 = \sqrt{4} = 2$ and \[\vec{a_1} \mapsto 2\begin{bmatrix}
        1 \\ 0 \\ 0 \\ 0
    \end{bmatrix} = \begin{bmatrix}
        2 \\ 0 \\ 0 \\ 0
    \end{bmatrix}.\] Using the lemma discussed in lecture 10, we can define \[\vec{v_1} = \vec{a_1} - ||\vec{a_1}||_2 \e_1 = \begin{bmatrix}
        1\\1\\1\\1
    \end{bmatrix} - \begin{bmatrix}
        2\\0\\0\\0
    \end{bmatrix} = \begin{bmatrix}
        -1 \\ 1 \\ 1 \\ 1
    \end{bmatrix}\] \[||\vec{v_1}||_2 = 2\] and so \[\vec{u_1} = \frac{\vec{v_1}}{||\vec{v_1}||_2} = \frac{1}{2} \begin{bmatrix}
        -1 \\ 1 \\ 1 \\ 1
    \end{bmatrix} = \begin{bmatrix}
        -\frac{1}{2} \\ \frac{1}{2} \\ \frac{1}{2} \\ \frac{1}{2}
    \end{bmatrix}.\] Then, we have \[Q_1 = I - 2\vec{u_1}\vec{u_1}^T = \begin{bmatrix}
        1 & 0 & 0 & 0 \\ 
        0 & 1 & 0 & 0 \\ 
        0 & 0 & 1 & 0 \\ 
        0 & 0 & 0 & 1
    \end{bmatrix} - 2 \begin{bmatrix}
        -\frac{1}{2} \\ \frac{1}{2} \\ \frac{1}{2} \\ \frac{1}{2}
    \end{bmatrix} \begin{bmatrix}
        -\frac{1}{2} & \frac{1}{2} & \frac{1}{2} & \frac{1}{2}
    \end{bmatrix} = \begin{bmatrix}
        \frac{1}{2} & \frac{1}{2} & \frac{1}{2} & \frac{1}{2} \\ 
        \frac{1}{2} & \frac{1}{2} & -\frac{1}{2} & -\frac{1}{2} \\ 
        \frac{1}{2} & -\frac{1}{2} & \frac{1}{2} & -\frac{1}{2} \\ 
        \frac{1}{2} & -\frac{1}{2} & -\frac{1}{2} & \frac{1}{2}
    \end{bmatrix}.\]
    From this, it follows that 
    \[Q_1 A = \begin{bmatrix}
        2 & 3 & 2 \\ 
        0 & 0 & 0 \\ 
        0 & 0 & 4 \\ 
        0 & -5 & 2
    \end{bmatrix}.\]



    \item \underline{Step 2:} We now look at the second column of $Q_1 A$ (\emph{not} $A$). Note that this is\footnote{As from lecture, we set the value in the first row and at that column to 0.} $\vec{\tilde{a_2}} = \begin{bmatrix}
        0 \\ 0 \\ 0 \\ -5
    \end{bmatrix}$, and $||\vec{\tilde{a_2}}||_2 = \sqrt{25} = 5$. So. mapping $\vec{\tilde{a_2}} \mapsto ||\vec{\tilde{a_2}}||_2 \e_2$, we have 
    \[\vec{\tilde{a_2}} \mapsto 5 \begin{bmatrix}
        0 \\ 1 \\ 0 \\ 0
    \end{bmatrix} = \begin{bmatrix}
        0 \\ 5 \\ 0 \\ 0
    \end{bmatrix}.\] So, using the lemma again, we define 
    \[\vec{v_2} = \vec{\tilde{a_2}} - ||\vec{\tilde{a_2}}||_2 \e_2 = \begin{bmatrix}
        0 \\ 0 \\ 0 \\ -5
    \end{bmatrix} - \begin{bmatrix}
        0 \\ 5 \\ 0 \\ 0
    \end{bmatrix} = \begin{bmatrix}
        0 \\ -5 \\ 0 \\ -5
    \end{bmatrix}\] \[||\vec{v_2}||_2 = \sqrt{(-5)^2 + 5^2} = \sqrt{50}\] and so \[\vec{u_2} = \frac{1}{\sqrt{50}} \begin{bmatrix}
        0 \\ -5 \\ 0 \\ -5
    \end{bmatrix} = \begin{bmatrix}
        0 \\ \frac{-5}{\sqrt{50}} \\ 0 \\ \frac{-5}{\sqrt{50}}
    \end{bmatrix}.\]
    Then, 
    \[Q_2 = I - 2\vec{u_2}\vec{u_2}^T = \begin{bmatrix}
        1 & 0 & 0 & 0 \\ 
        0 & 1 & 0 & 0 \\ 
        0 & 0 & 1 & 0 \\ 
        0 & 0 & 0 & 1
    \end{bmatrix} - 2 \begin{bmatrix}
        0 \\ \frac{-5}{\sqrt{50}} \\ 0 \\ \frac{-5}{\sqrt{50}}
    \end{bmatrix} \begin{bmatrix}
        0 & \frac{-5}{\sqrt{50}} & 0 & \frac{-5}{\sqrt{50}}
    \end{bmatrix} = \begin{bmatrix}
        1 & 0 & 0 &0\\ 
        0 & 0 & 0 &-1\\ 
        0 & 0 & 1 &0\\ 
        0 & -1 & 0 & 0
    \end{bmatrix}.\]
    From this, it follows that 
    \[Q_2 (Q_1 A) = \begin{bmatrix}
        2 & 3 & 2 \\ 
        0 & 5 & -2 \\ 
        0 & 0 & 4 \\ 
        0 & 0 & 0
    \end{bmatrix}.\]
\end{itemize}
Notice how, in step 2, we found an upper-triangular matrix. Therefore, we have that 
\[R = \begin{bmatrix}
    2 & 3 & 2 \\ 
    0 & 5 & -2 \\ 
    0 & 0 & 4 \\ 
    0 & 0 & 0
\end{bmatrix}\] and 
\begin{equation*}
    \begin{aligned}
        Q_2 (Q_1 A) &= R \\ 
            &\implies Q_1 A = Q_2^{-1} R \\ 
            &\implies A = Q_1^{-1} Q_2^{-1} R \\ 
            &\implies A =  \begin{bmatrix}
                \frac{1}{2} & \frac{1}{2} & \frac{1}{2} & \frac{1}{2} \\ 
                \frac{1}{2} & \frac{1}{2} & -\frac{1}{2} & -\frac{1}{2} \\ 
                \frac{1}{2} & -\frac{1}{2} & \frac{1}{2} & -\frac{1}{2} \\ 
                \frac{1}{2} & -\frac{1}{2} & -\frac{1}{2} & \frac{1}{2}
            \end{bmatrix}^{-1} \begin{bmatrix}
                1 & 0 & 0 &0\\ 
                0 & 0 & 0 &-1\\ 
                0 & 0 & 1 &0\\ 
                0 & -1 & 0 & 0
            \end{bmatrix}^{-1} R \\ 
            &\implies A = \begin{bmatrix}
                \frac{1}{2} & \frac{1}{2} & \frac{1}{2} & \frac{1}{2} \\ 
                \frac{1}{2} & \frac{1}{2} & -\frac{1}{2} & -\frac{1}{2} \\ 
                \frac{1}{2} & -\frac{1}{2} & \frac{1}{2} & -\frac{1}{2} \\ 
                \frac{1}{2} & -\frac{1}{2} & -\frac{1}{2} & \frac{1}{2}
            \end{bmatrix} \begin{bmatrix}
                1 & 0 & 0 &0\\ 
                0 & 0 & 0 &-1\\ 
                0 & 0 & 1 &0\\ 
                0 & -1 & 0 & 0
            \end{bmatrix} R \\
            &\implies A = \begin{bmatrix}
                \frac{1}{2}&-\frac{1}{2}&\frac{1}{2}&-\frac{1}{2} \\ 
                \frac{1}{2}&\frac{1}{2}&-\frac{1}{2}&-\frac{1}{2} \\ 
                \frac{1}{2}&\frac{1}{2}&\frac{1}{2}&\frac{1}{2} \\ 
                \frac{1}{2}&-\frac{1}{2}&-\frac{1}{2}&\frac{1}{2}
            \end{bmatrix} R.
    \end{aligned}
\end{equation*} 
So, \[Q = \begin{bmatrix}
    \frac{1}{2}&-\frac{1}{2}&\frac{1}{2}&-\frac{1}{2} \\ 
    \frac{1}{2}&\frac{1}{2}&-\frac{1}{2}&-\frac{1}{2} \\ 
    \frac{1}{2}&\frac{1}{2}&\frac{1}{2}&\frac{1}{2} \\ 
    \frac{1}{2}&-\frac{1}{2}&-\frac{1}{2}&\frac{1}{2}
\end{bmatrix}.\]










\newpage 
\section{Reflector}
\begin{mdframed}
    Find a reflector $Q$ that maps the vector $\vec{x} = \begin{bmatrix}
        3 \\ 4 \\ 1 \\ 3 \\ 1
    \end{bmatrix}$ to a multiple of the first column of the $5 \times 5$ identity, $\e_1$. Compute $Q$ by writing it as \[Q = I - 2\frac{\vec{u}\vec{u}^T}{||\vec{u}||_2^2}\] for some appropriate $\vec{u}$, and write it as a completely assembled matrix.
\end{mdframed}

We want to map $\vec{x} \mapsto ||\vec{x}||_2 \e_1$. Notice how \[||\vec{x}||_2 = \sqrt{3^2 + 4^2 + 1^2 + 3^2 + 1^2} = \sqrt{9 + 16 + 1 + 9 + 1} = \sqrt{36} = 6.\] So, we're mapping \[\vec{x} \mapsto 6\begin{bmatrix}
    1 \\ 0 \\ 0 \\ 0 \\ 0
\end{bmatrix} = \begin{bmatrix}
    6 \\ 0 \\ 0 \\ 0 \\ 0
\end{bmatrix}.\] Now, we can define \[\vec{v} = \vec{x} - ||\vec{x}||_2 \e_1 = \begin{bmatrix}
    3 \\ 4 \\ 1 \\ 3 \\ 1
\end{bmatrix} - \begin{bmatrix}
    6 \\ 0 \\ 0 \\ 0 \\ 0
\end{bmatrix} = \begin{bmatrix}
    -3 \\ 4 \\ 1 \\ 3 \\ 1
\end{bmatrix}\] \[||\vec{v}||_2 = 6.\] So, 
\[\vec{u} = \frac{1}{6}\begin{bmatrix}
    -3 \\ 4 \\ 1 \\ 3 \\ 1
\end{bmatrix} = \begin{bmatrix}
    -\frac{1}{2} \\ \frac{2}{3} \\ \frac{1}{6} \\ \frac{1}{2} \\ \frac{1}{6}
\end{bmatrix}.\] Thus, 
\[Q = I - 2\vec{u}\vec{u}^T = \begin{bmatrix}
    1 & 0 & 0 & 0 & 0 \\ 
    0 & 1 & 0 & 0 & 0 \\ 
    0 & 0 & 1 & 0 & 0 \\ 
    0 & 0 & 0 & 1 & 0 \\ 
    0 & 0 & 0 & 0 & 1 
\end{bmatrix} - 2\begin{bmatrix}
    -\frac{1}{2} \\ \frac{2}{3} \\ \frac{1}{6} \\ \frac{1}{2} \\ \frac{1}{6}
\end{bmatrix}\begin{bmatrix}
    -\frac{1}{2} & \frac{2}{3} & \frac{1}{6} & \frac{1}{2} & \frac{1}{6}
\end{bmatrix}\]
Therefore, the answer is 
\[\begin{bmatrix}
    \frac{1}{2}&\frac{2}{3}&\frac{1}{6}&\frac{1}{2}&\frac{1}{6}\\ 
    \frac{2}{3}&\frac{1}{9}&-\frac{2}{9}&-\frac{2}{3}&-\frac{2}{9}\\ 
    \frac{1}{6}&-\frac{2}{9}&\frac{17}{18}&-\frac{1}{6}&-\frac{1}{18}\\ 
    \frac{1}{2}&-\frac{2}{3}&-\frac{1}{6}&\frac{1}{2}&-\frac{1}{6}\\ 
    \frac{1}{6}&-\frac{2}{9}&-\frac{1}{18}&-\frac{1}{6}&\frac{17}{18}
\end{bmatrix}.\]




\newpage 
\section{Least Squares}
\begin{mdframed}
    Consider the overdetermined system 
    \[\begin{bmatrix}
        1 \\ 1
    \end{bmatrix} \begin{bmatrix}
        x
    \end{bmatrix} = \begin{bmatrix}
        9 \\ 5
    \end{bmatrix}.\] 
    \begin{enumerate}[(a)]
        \item Calculate a full QR decomposition of the coefficient matrix with the help of Householder reflectors.
        \item Using the QR decomposition from part (a), calculate the least squares solution (the minimizer). 
        \item Calculate the norm of the residual with the help of $Q$ (the minimum).
    \end{enumerate}
\end{mdframed}
Here, $A$ is a $2 \times 1$ matrix and so $n = 2$ and $m = 1$. 

\subsection*{Part (A)}
We want to find the QR decomposition of $\begin{bmatrix}
    1 \\ 1
\end{bmatrix}$. There's only one column, which we'll call $\vec{a} = \begin{bmatrix}
    1 \\ 1
\end{bmatrix}$. Note that $||\vec{a}||_2 = \sqrt{1^2 + 1^2} = \sqrt{2}$ and so we can map $\vec{a} \mapsto ||\vec{a}||_2 \e_1$ by 
\[\vec{a} \mapsto \sqrt{2} \begin{bmatrix}
    1 \\ 0
\end{bmatrix} = \begin{bmatrix}
    \sqrt{2} \\ 0
\end{bmatrix}.\]
Using the lemma discussed in class, we define 
\[\vec{v} = \vec{a} - ||\vec{a}||_2 \e_1 = \begin{bmatrix}
    1 \\ 1
\end{bmatrix} - \begin{bmatrix}
    \sqrt{2} \\ 0
\end{bmatrix} = \begin{bmatrix}
    1 - \sqrt{2} \\ 1
\end{bmatrix}\] \[||\vec{v}|| = \sqrt{(1 - \sqrt{2})^2 + 1^2} = \sqrt{4 - 2\sqrt{2}}\] and so we have 
\[\vec{u} = \frac{\vec{v}}{||\vec{v}||_2} = \frac{1}{\sqrt{4 - 2\sqrt{2}}} \begin{bmatrix}
    1 - \sqrt{2} \\ 1
\end{bmatrix} = \begin{bmatrix}
    \frac{1-\sqrt{2}}{\sqrt{4-2\sqrt{2}}}\\ 
    \frac{1}{\sqrt{4-2\sqrt{2}}}
\end{bmatrix}.\] From there, we have 
\[Q = I - 2\vec{u}\vec{u}^T = \begin{bmatrix}
    1 & 0 \\ 0 & 1
\end{bmatrix} - 2 \begin{bmatrix}
    \frac{1-\sqrt{2}}{\sqrt{4-2\sqrt{2}}}\\ 
    \frac{1}{\sqrt{4-2\sqrt{2}}}
\end{bmatrix} \begin{bmatrix}
    \frac{1-\sqrt{2}}{\sqrt{4-2\sqrt{2}}} & 
    \frac{1}{\sqrt{4-2\sqrt{2}}}
\end{bmatrix} = \begin{bmatrix}
    \frac{1}{\sqrt{2}} & \frac{1}{\sqrt{2}} \\ 
    \frac{1}{\sqrt{2}} & -\frac{1}{\sqrt{2}}
\end{bmatrix}.\] Then, 
\[QA = R = \begin{bmatrix}
    \sqrt{2} \\ 0
\end{bmatrix}.\] Therefore\footnote{Note that $Q$ is orthogonal and also our householder reflector, so $Q^T = Q^{-1} = Q$.}, 
\[A = \underbrace{\begin{bmatrix}
    \frac{1}{\sqrt{2}} & \frac{1}{\sqrt{2}} \\ 
    \frac{1}{\sqrt{2}} & -\frac{1}{\sqrt{2}}
\end{bmatrix}}_{Q} \underbrace{\begin{bmatrix}
    \sqrt{2} \\ 0
\end{bmatrix}}_{R}.\]


\subsection*{Part (B)}
If $\vec{y} = \begin{bmatrix}
    9 \\ 5
\end{bmatrix}$, then \[Q^T \vec{y} = Q \vec{y} = \begin{bmatrix}
    \frac{1}{\sqrt{2}} & \frac{1}{\sqrt{2}} \\ 
    \frac{1}{\sqrt{2}} & -\frac{1}{\sqrt{2}}
\end{bmatrix} \begin{bmatrix}
    9 \\ 5
\end{bmatrix} = \begin{bmatrix}
    7\sqrt{2} \\ 2\sqrt{2}
\end{bmatrix}.\] Since $m = 1$, we have 
\[\hat{c} = \begin{bmatrix}
    7 \sqrt{2}
\end{bmatrix} \quad \hat{R} = \begin{bmatrix}
    \sqrt{2}
\end{bmatrix}.\] The idea is to solve $\hat{R}\vec{x} = \hat{c}$, so \[\begin{bmatrix}
    \sqrt{2}
\end{bmatrix}\begin{bmatrix}
    x
\end{bmatrix} = \begin{bmatrix}
    7\sqrt{2}
\end{bmatrix}.\]
This gives us $\vec{x} = \begin{bmatrix}
    7
\end{bmatrix}$, the minimizer.

\subsection*{Part (C)}
From part (b), we know that $Q^T \vec{y} = \begin{bmatrix}
    7\sqrt{2} \\ 2\sqrt{2}
\end{bmatrix}$. Recall that if $\hat{c}$ consisted of the first $m$ elements, then $\hat{d}$ will consist of the remaining elements. So, \[\hat{d} = \begin{bmatrix}
    2\sqrt{2}
\end{bmatrix}\] and so \[||\hat{d}||_2 = \sqrt{(2\sqrt{2})^2} = \sqrt{8} = 2\sqrt{2},\] the minimum.


\subsection*{Remarks}
In lecture 9, we wrote 
\[Q^T\vec{y} = \begin{bmatrix}
    \hat{c} \\ \hat{d}
\end{bmatrix},\] where 
\[\hat{c} = Q^T\vec{y}(1:m) \quad \hat{d} = Q^T\vec{y}(m + 1:).\] Likewise, $\hat{R} = R(1:m, 1:m)$.








\newpage 
\section{Another QR Decomposition \& Least Squares}
\begin{mdframed}
    Find the full QR decomposition for $A$ and use it to find the minimizer $\vec{x}$ and the minimum value that solves the least squares problem $\min_{x \in \R} ||\vec{b} - A\vec{x}||_2$ with \[A = \begin{bmatrix}
        1 \\ 0 \\ 1
    \end{bmatrix} \quad \vec{b} = \begin{bmatrix}
        2 \\ 2 \\ 0
    \end{bmatrix}.\]
\end{mdframed}
Note that $n = 3$ and $m = 1$. 

\subsection*{QR Decomposition}
Let $\vec{a}$ be the first (and only) column of $A$. We know that \[||\vec{a}||_2 = \sqrt{2},\] so we're mapping $\vec{a} \mapsto \sqrt{2}\e_1$; that is, 
\[\vec{a} \mapsto \begin{bmatrix}
    \sqrt{2} \\ 0 \\ 0
\end{bmatrix}.\] Then, 
\[\vec{v} = \begin{bmatrix}
    1 \\ 0 \\ 1
\end{bmatrix} - \begin{bmatrix}
    \sqrt{2} \\ 0 \\ 0
\end{bmatrix} = \begin{bmatrix}
    1 - \sqrt{2} \\ 0 \\ 1
\end{bmatrix}\] \[||\vec{v}||_2 = \sqrt{(1 - \sqrt{2})^2 + 1^2} = \sqrt{4 - 2\sqrt{2}}.\] Then, 
\[\vec{u} = \frac{1}{\sqrt{4 - 2\sqrt{2}}} \begin{bmatrix}
    1 - \sqrt{2} \\ 0 \\ 1
\end{bmatrix} = \begin{bmatrix}
    \frac{1-\sqrt{2}}{\sqrt{4-2\sqrt{2}}}\\ 
    0
    \\ \frac{1}{\sqrt{4-2\sqrt{2}}}
\end{bmatrix}.\] From there, 
\[Q = \begin{bmatrix}
    1 & 0 & 0 \\
    0 & 1 & 0 \\ 
    0 & 0 & 1
\end{bmatrix} - 2\begin{bmatrix}
    \frac{1-\sqrt{2}}{\sqrt{4-2\sqrt{2}}}\\ 
    0
    \\ \frac{1}{\sqrt{4-2\sqrt{2}}}
\end{bmatrix}\begin{bmatrix}
    \frac{1-\sqrt{2}}{\sqrt{4-2\sqrt{2}}}& 
    0
    & \frac{1}{\sqrt{4-2\sqrt{2}}}
\end{bmatrix} = \begin{bmatrix}
    \frac{1}{\sqrt{2}} & 0 & \frac{1}{\sqrt{2}}\\ 
    0 & 1 & 0 \\ 
    \frac{1}{\sqrt{2}} & 0 & -\frac{1}{\sqrt{2}}
\end{bmatrix}\]
Then, 
\[QA = R = \begin{bmatrix}
    \sqrt{2} \\ 0 \\ 0
\end{bmatrix}.\]

\subsection*{Minimizer}
Note that 
\[Q^T\vec{b} = Q\vec{b} = \begin{bmatrix}
    \sqrt{2} \\ 2 \\ \sqrt{2}
\end{bmatrix}.\] Since $m = 1$, we have 
\[\hat{c} = Q^T\vec{b}(1 : m) = \begin{bmatrix}
    \sqrt{2}
\end{bmatrix} \quad \hat{R} = \begin{bmatrix}
    \sqrt{2}
\end{bmatrix}\] so 
\[\hat{R}\vec{x} = \hat{c} \implies \begin{bmatrix}
    \sqrt{2}
\end{bmatrix} \vec{x} = \begin{bmatrix}
    \sqrt{2}
\end{bmatrix}\] and thus $\hat{x} = \begin{bmatrix}
    1
\end{bmatrix}$. 

\subsection*{Minimum}
Likewise, 
\[\hat{d} = Q^T\hat{b}(m+1:) = \begin{bmatrix}
    2 \\ \sqrt{2}
\end{bmatrix}.\] So, \[||\hat{d}||_2 = \sqrt{2^2 + (\sqrt{2})^2} = \sqrt{4 + 2} = \sqrt{6}.\]

\end{document}