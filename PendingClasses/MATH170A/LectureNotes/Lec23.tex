\documentclass[letterpaper]{article}
\usepackage[margin=1in]{geometry}
\usepackage[utf8]{inputenc}
\usepackage{textcomp}
\usepackage{amssymb}
\usepackage{natbib}
\usepackage{graphicx}
\usepackage{gensymb}
\usepackage{amsthm, amsmath, mathtools}
\usepackage{xcolor}
\usepackage{enumerate}
\usepackage{framed}
\usepackage{tcolorbox}
\tcbuselibrary{theorems}

\newcommand{\R}{\mathbb{R}}
\newcommand{\Z}{\mathbb{Z}}
\newcommand{\N}{\mathbb{N}}
\newcommand{\Q}{\mathbb{Q}}
\newcommand{\code}[1]{\texttt{#1}}
\newcommand{\mdiamond}{$\diamondsuit$}

%\newtheorem*{theorem}{Theorem}
%\newtheorem*{definition}{Definition}
\newtheorem*{proposition}{Proposition}
%\newtheorem*{corollary}{Corollary}
%\newtheorem*{lemma}{Lemma}

\newtcbtheorem[number within=section]{theorem}{Theorem}
{colback=green!5,colframe=green!35!black,fonttitle=\bfseries}{def}

\newtcbtheorem[number within=section]{definition}{Definition}
{colback=blue!5,colframe=blue!35!black,fonttitle=\bfseries}{def}

\newtcbtheorem[number within=section]{corollary}{Corollary}
{colback=yellow!5,colframe=yellow!35!black,fonttitle=\bfseries}{def}

\newtcbtheorem[number within=section]{lemma}{Lemma}
{colback=red!5,colframe=red!35!black,fonttitle=\bfseries}{def}
\usepackage[utf8]{inputenc}
\usepackage[english]{babel}
\usepackage{fancyhdr}
\usepackage[hidelinks]{hyperref}
\usepackage{csquotes}

\pagestyle{fancy}
\fancyhf{}
\rhead{Math 170A}
\chead{March 10, 2023}
\lhead{Lecture 24}
\rfoot{\thepage}

\setlength{\parindent}{0pt}
\newcommand{\0}{\mathbf{0}}
\newcommand{\y}{\mathbf{y}}
\renewcommand{\b}{\mathbf{b}}
\newcommand{\x}{\mathbf{x}}
\newcommand{\e}{\mathbf{e}}
\newcommand{\rr}{\mathbf{r}}
\newcommand{\vv}{\mathbf{v}}
\renewcommand{\u}{\mathbf{u}}

\begin{document}

\section{Iterative Methods: Jacobi Method (8.1)}
We now return to the problem from the beginning of the class: solving $A\x = \b$, for $A \in \R^{n \times n}$ and $\b \in \R^n$. Recall that we performed Gauss elimination directly, resulting in a flop count of $\BigO(n^3)$. The problem with this is that, if $n$ is large (hundreds of thousands of equations), then Gauss elimination may not be feasible. 

\bigskip 

Aside from Gaussian Elimination, there are iterative methods that we can run to solve $A\x = \b$. The idea is that we need an initial guess $\x^{(0)}$. Then, from $\x^{(k)}$, we hope to somehow iterate to $\x^{(k + 1)}$. The idea is that, as $k \mapsto \infty$, $\x^{(k)} \mapsto \x^*$, where $\x^*$ is the solution to $A\x = \b$ (i.e., $A\x^* = \b$). Some advantages of this process include 
\begin{itemize}
    \item If $\x^{(0)}$ is already close to $\x^*$, then the iterative method converges \emph{fast}.
    \item We can stop at any point of the iteration, depending on how accurate the approximation of $\x^*$ (by $\x^{(k)}$) should be. 
\end{itemize}

\subsection{The Jacobi Method}
The idea is that $i$th equation in the system $A\x = \b$ is 
\[\sum_{j = 1}^{n} a_{ij}x_j = b_i.\]
This can be rewritten by solving for $x_i$ and assuming that $a_{ii} \neq 0$ as 
\[x_i = \frac{1}{a_{ii}}\left(b_{i} - \sum_{\substack{j = 1 \\ j \neq i}}^n a_{ij}x_{j}\right) \qquad i = 1, 2, \hdots, n.\]
The iteration process, then, is 
\[\boxed{x_{i}^{(k + 1)} = \frac{1}{a_{ii}} \left(b_{i} - \sum_{\substack{j = 1 \\ j \neq i}}^n a_{ij}x_{j}^{(k)}\right)} \qquad i = 1, 2, \hdots, n.\]
As mentioned, this is an iterative process, so we \textbf{stop} when $||\x^{(k + 1)} - \x^{(k)}|| < \epsilon$, where we can choose any vector norm. 

\subsection{Matrix Representation}
We can also represent this process using matrices. In equation form, we have 
\[x_{i}^{(k + 1)} = \frac{1}{a_{ii}} \left(b_{i} - \sum_{\substack{j = 1 \\ j \neq i}}^n a_{ij}x_{j}^{(k)}\right) \qquad i = 1, 2, \hdots, n.\]
Notice that, for $i = 1, 2, \hdots, n$, $\sum_{\substack{j = 1 \\ j \neq i}}^n a_{ij}x_{j}^{(k)}$ is the same thing as saying 
\[\sum_{i = 1}^{n} a_{ij}x_{j}^{(k)} - a_{ii}x_i^{(k)} \implies A\x^{(k)} - D\x^{(k)},\]
where 
\[D = \begin{bmatrix}
    a_{11} &        &        &        \\
           & a_{22} &        &        \\  
           &        & \ddots &        \\  
           &        &        & a_{nn}   
\end{bmatrix}.\]
Therefore, translating our equation into matrix form means translating each component into matrix form. This gives us  
\[\x^{(k + 1)} = \begin{bmatrix}
    x_1^{(k + 1)} \\ 
    x_2^{(k + 1)} \\ 
    x_3^{(k + 1)} \\
    \vdots \\  
    x_n^{(k + 1)} \\ 
\end{bmatrix} = \underbrace{\begin{bmatrix}
    \frac{1}{a_{11}} & 0 & 0 & \hdots & 0 \\ 
    0 & \frac{1}{a_{22}} & 0 & \hdots & 0 \\ 
    0 & 0 & \frac{1}{a_{33}} & \hdots & 0 \\
    \vdots & \vdots & \vdots & \ddots & \vdots \\  
    0 & 0 & 0 & \hdots & \frac{1}{a_{nn}}
\end{bmatrix}}_{D^{-1}} \left(\b - (A - D)\x^{(k)}\right).\]
This further gives us, for $k \geq 0$ and $a_{ii} \neq 0$ for all $i$, 
\[\boxed{x^{(k + 1)} = D^{-1}(\b - (A - D)\x^{(k)})}.\]

\subsection{Convergence}
Let $\x^*$ be the true solution, i.e., $A\x^* = \b$. Does $\x^{(k)} \mapsto \x^*$ as $k \mapsto \infty$? In other words, does $\x^{(k)} - \x^* \mapsto 0$ as $k \mapsto \infty$? 

\bigskip 

Let $e^{(k)} = \x^* - \x^{(k)}$. We want to show that $e^{(k)} \mapsto \infty$. So, 
\begin{equation*}
    \begin{aligned}
        e^{(k + 1)} &= \x^* - \x^{(k + 1)} \\ 
            &= \x^* - (D^{-1} (\b - (A - D)\x^{(k)})) \\ 
            &= \x^* - (D^{-1} (\b - A\x^{(k)} + D\x^{(k)})) \\ 
            &= \x^* - D^{-1} (\b - A\x^{(k)}) - (D^{-1}D)(\x^{(k)}) \\ 
            &= \x^* - D^{-1} (\b - A\x^{(k)}) - \x^{(k)} \\ 
            &= e^{(k)} - D^{-1} (\b - A\x^{(k)}) && \x^* - \x^{(k)} = e^{(k)} \\
            &= e^{(k)} - D^{-1} (\underbrace{\b - A\x^*}_{\0} + \underbrace{A\x^* - A\x^{(k)}}_{Ae^{(k)}}) \\ 
            &= e^{(k)} - D^{-1}Ae^{(k)} \\ 
            &= (I - D^{-1}A)e^{(k)} 
    \end{aligned}
\end{equation*}
Therefore, 
\begin{equation*}
    \begin{aligned}
        e^{(k + 1)} &= (I - D^{-1}A)e^{(k)}  \\ 
            &= (I - D^{-1} A)^2 e^{(k - 1)} \\ 
            &= (I - D^{-1} A)^3 e^{(k - 2)} \\ 
            &= \hdots \\ 
            &= (I - D^{-1} A)^{k + 1} e^{(0)},
    \end{aligned}
\end{equation*}
where $e^{(0)} = \x^* - \x^{(0)}$ and $\x^{(0)}$ is the initial guess we described above. In any case, $e^{(k + 1)} \mapsto 0$ as $k \mapsto \infty$ only if $I - D^{-1}A$ is ``small.''

\subsection{Result}
If all eigenvalues $\lambda$ of $B = I - D^{-1}A$ satisfy $|\lambda| < 1$, then the Jacobi method converges to $\x^*$ for every choice of $\x^{(0)}$. Put it differently, if the Jacobi method converges to $\x^*$ for all $\x^{(0)}$, then all eigenvalues $\lambda$ of $B$ satisfy $|\lambda| < 1$. 

\bigskip 

Why? Recall from the power method 
\[e^{(0)} = \sum_{i = 1}^{n} c_i v_i,\]
where the $v_i$'s are the eigenvectors of $B$. Then, 
\[B^k e^{(0)} = \sum_{i = 1}^{n} c_i \lambda_i^k v_i = \sum_{i = 1}^{n} \lambda_i^k (c_i v_i) \mapsto 0.\] Note that if all $|\lambda_i| < 1$, then $\lambda_i^k \mapsto 0$ if $k \mapsto \infty$.  

\end{document}