\documentclass[letterpaper]{article}
\usepackage[margin=1in]{geometry}
\usepackage[utf8]{inputenc}
\usepackage{textcomp}
\usepackage{amssymb}
\usepackage{natbib}
\usepackage{graphicx}
\usepackage{gensymb}
\usepackage{amsthm, amsmath, mathtools}
\usepackage[dvipsnames]{xcolor}
\usepackage{enumerate}
\usepackage{mdframed}
\usepackage[most]{tcolorbox}
\usepackage{csquotes}
% https://tex.stackexchange.com/questions/13506/how-to-continue-the-framed-text-box-on-multiple-pages

\tcbuselibrary{theorems}

\newcommand{\R}{\mathbb{R}}
\newcommand{\Z}{\mathbb{Z}}
\newcommand{\N}{\mathbb{N}}
\newcommand{\Q}{\mathbb{Q}}
\newcommand{\C}{\mathbb{C}}
\newcommand{\code}[1]{\texttt{#1}}
\newcommand{\mdiamond}{$\diamondsuit$}
\newcommand{\PowerSet}{\mathcal{P}}
\newcommand{\Mod}[1]{\ (\mathrm{mod}\ #1)}
\DeclareMathOperator{\lcm}{lcm}

%\newtheorem*{theorem}{Theorem}
%\newtheorem*{definition}{Definition}
%\newtheorem*{corollary}{Corollary}
%\newtheorem*{lemma}{Lemma}
\newtheorem*{proposition}{Proposition}


\newtcbtheorem[number within=section]{theorem}{Theorem}
{colback=green!5,colframe=green!35!black,fonttitle=\bfseries}{th}

\newtcbtheorem[number within=section]{definition}{Definition}
{colback=blue!5,colframe=blue!35!black,fonttitle=\bfseries}{def}

\newtcbtheorem[number within=section]{corollary}{Corollary}
{colback=yellow!5,colframe=yellow!35!black,fonttitle=\bfseries}{cor}

\newtcbtheorem[number within=section]{lemma}{Lemma}
{colback=red!5,colframe=red!35!black,fonttitle=\bfseries}{lem}

\newtcbtheorem[number within=section]{example}{Example}
{colback=white!5,colframe=white!35!black,fonttitle=\bfseries}{def}

\newtcbtheorem[number within=section]{note}{Important Note}{
        enhanced,
        sharp corners,
        attach boxed title to top left={
            xshift=-1mm,
            yshift=-5mm,
            yshifttext=-1mm
        },
        top=1.5em,
        colback=white,
        colframe=black,
        fonttitle=\bfseries,
        boxed title style={
            sharp corners,
            size=small,
            colback=red!75!black,
            colframe=red!75!black,
        } 
    }{impnote}
\usepackage[utf8]{inputenc}
\usepackage[english]{babel}
\usepackage{fancyhdr}
\usepackage[hidelinks]{hyperref}
\usepackage{csquotes}

\pagestyle{fancy}
\fancyhf{}
\rhead{Math 170A}
\chead{March 03, 2023 \& March 05, 2023}
\lhead{Lecture 21 \& 22}
\rfoot{\thepage}

\setlength{\parindent}{0pt}
\newcommand{\0}{\mathbf{0}}
\newcommand{\y}{\mathbf{y}}
\renewcommand{\b}{\mathbf{b}}
\newcommand{\x}{\mathbf{x}}
\newcommand{\e}{\mathbf{e}}
\newcommand{\rr}{\mathbf{r}}
\newcommand{\vv}{\mathbf{v}}
\renewcommand{\u}{\mathbf{u}}

\begin{document}

\section{Similar Matrices \& QR Iteration Introduction (5.4)}
Two matrices, $A, B \in \R^{n \times n}$, are \textbf{similar} if there exists an invertible matrix $S \in \R^{n \times n}$ such that $AS = SB$. Equivalently, \[A = SBS^{-1} \qquad B = S^{-1}AS.\]
$A$ and $B$ are called \textbf{orthogonally similar} if $S$ is orthogonal and $A = SBS^{-1}$. In this case, we actually have $A = SBS^T$. 

\begin{theorem}{}{}
    Similar matrices have the same eigenvalues. 
\end{theorem}
That is, if $B = S^{-1}AS$ and $v$ is an eigenvector of $A$ to the eigenvalue $\lambda$, then $S^{-1} v$ is an eigenvector of $B$ with respect to $\lambda$. 

\begin{proof}
    We have 
    \begin{equation}
        \begin{aligned}
            Av = \lambda v &\implies SBS^{-1}v = \lambda v \\ 
                &\implies S^{-1} (SBS^{-1})v = S^{-1}(\lambda v) \\ 
                &\implies (S^{-1} S)BS^{-1}v = \lambda S^{-1}v \\
                &\implies BS^{-1}v = \lambda S^{-1} v.
        \end{aligned}
    \end{equation}
    This means that $S^{-1}v$ is an eigenvalue of $B$ with respect to $\lambda$. 
\end{proof}

\begin{lemma}{Diagonalizing Semisimple Matrices}{}
    Let $A$ be a matrix in $\R^{n \times n}$. $A$ is semisimple if and only if there exists an invertible matrix $V \in \R^{n \times n}$ and a diagonal matrix $D \in \R^{n \times n}$ such that \[A = VDV^{-1}.\]
\end{lemma}
\textbf{Remarks:}
\begin{itemize}
    \item $A$ and $D$ are similar, meaning they have the same eigenvalues\footnote{The eigenvalues of a diagonal matrix is just the entries on the diagonal.}.
    \item $A = VDV^{-1}$ is equivalent to $AV = VD$. This is an eigenvalue/vector equation. 
\end{itemize} 
\emph{Proof Idea.} If $D = \begin{bmatrix}
    \lambda_1 & 0 & \hdots & 0 \\ 
    0 & \lambda_2 & \hdots & 0 \\ 
    \vdots & \vdots & \ddots & \vdots \\ 
    0 & 0 & \hdots & \lambda_n
\end{bmatrix}$ contains eigenvalues and $V = \begin{bmatrix}
    v_1 & v_2 & \hdots & v_n
\end{bmatrix}$ contains eigenvectors, then $V$ is invertible implies that $\begin{bmatrix}
    v_1 & v_2 & \hdots & v_n
\end{bmatrix}$ are linearly independent. Thus, $A$ is semisimple. How do we find the eigenvalues of $A$? 

\subsection{Interlude: Complex Matrices}
Let $\alpha = a + bi \in \C$, where $a, b \in \R$. Then, 
\begin{itemize}
    \item The complex conjugate, $\bar{\alpha} = a - bi \in \C$.
    \item Also, $|\alpha| = \sqrt{a^2 + b^2}$.
\end{itemize}
 % TODO get graph from lecture

Regarding complex matrices, 
\begin{itemize}
    \item \textbf{Generalization of Transpose:} Let $A \in \C^{n \times n}$. Then, $A^* = \bar{A}^T$, known as the generalization of a transposition. $\bar{A}$ is the complex conjugate of every entry. 
    \item \textbf{Generalization of Orthogonality:} Recall that $Q \in \R^{n \times n}$ is orthogonal if $QQ^T = Q^T Q = I$. How do we generalize this to complex matrices? Let $U \in \C^{n \times n}$. $U$ is called \textbf{unitary} if $UU^* = U^*U = I$. 
\end{itemize}

\begin{theorem}{Schur}{}
    Let $A \in \C^{n \times n}$. Then, there exists a unitary matrix $U \in \C^{n \times n}$ and an upper triangular matrix $T \in \C^{n \times n}$ such that $A = UTU^*$.
\end{theorem}
\textbf{Remark:} $A$ is unitarily similar to $T$. So, $A$ and $T$ has the same eignevalues. 

\subsection{Back to Real Matrices}
If $A \in \R^{n \times n}$, we can still apply Schur's theorem\footnote{Recall that $\R \subset \C$.}; that is, \[A = UTU^* \qquad U, T \in \C^{n \times n}.\]
Another version of Schur's Theorem, known as the Real Schur's Theorem, states the following.
\begin{theorem}{Real Schur}{}
    If $A \in \R^{n \times n}$. Then, there exists an orthogonal $Q \in \R^{n \times n}$ and an ``almost'' upper triangular matrix $T \in \R^{n \times n}$ such that $A = QTQ^T$. 
\end{theorem}
We can think of diagonal entries of $T$ as consisting of size $1 \times 1$ or size $2 \times 2$ blocks.

\begin{mdframed}
    (Example.) If $A \in \R^{4 \times 4}$ with eigenvalues $2 + i$, $2 - i$, $5$, and $6$. Then, 
    \begin{itemize}
        \item the complex Schur is 
        \[T = \begin{bmatrix}
            2 + i & * & * & * \\ 
            0 & 2 - i & * & * \\ 
            0 & 0 & 5 & * \\ 
            0 & 0 & 0 & 6
        \end{bmatrix} \in \C^{4 \times 4}.\]
        Note that $\alpha = a + bi$ and $\bar{\alpha} = a - bi$, so \[\alpha \bar{\alpha} = \alpha^2 + b^2.\]

        \item the real Schur is 
        \[T = \begin{bmatrix}
            2 & 1 & * & * \\ 
            -1 & 2 & * & * \\ 
            0 & 0 & 5 & * \\ 
            0 & 0 & 0 & 6
        \end{bmatrix} \in \R^{4 \times 4}.\]
        Notice how we have the $2 \times 2$ ``block'' at the top-left corner, representing the complex eigenvalues $2 + i$ and $2 - i$, respectively. 
    \end{itemize}
\end{mdframed}
\textbf{Remark:} Note that complex eigenvalues always come in \textbf{complex conjugate pairs}. If $a + bi$ is an complex eigenvalue, then $a - bi$ is also a complex eigenvalue. 

\subsection{QR Iteration: A Basic Idea}
The aim is to find the eigenvalues of a matrix $A \in \C^{n \times n}$. The idea behind the iterative procedure is as follows:
\begin{enumerate}
    \item Step-by-step transform $A$ without changing eigenvalues (similarity transformation).
    \item Change into an upper-triangular matrix ($T$ from Schur).  
\end{enumerate}
This method is based on the QR decomposition that we discussed earlier in the quarter.

\bigskip 

\subsubsection{Basic Idea: The Reals}
Consider\footnote{The complex matrix works the same} $A \in \R^{n \times n}$. We know that $A = QR$, where $Q$ is an orthogonal matrix. Then, 
\[A = QR \implies Q^T A = Q^T Q R = R \implies Q^T A Q = RQ.\]
Here, $A$ and $RQ$ have the same eigenvalues. 

\bigskip 

So, the iterative procedure begins by defining $A_0 = A$. The new matrix in iteration is $A_1 = RQ$. $A_0$ and $A_1$ have the same eigenvalues, so we can continue the process. So, the iterative procedure can be described in detailed as follows:
\begin{enumerate}
    \item Iteratively compute QR decomposition. 
    \item Change multiplication order. 
    \item This converges to $T$, the upper-triangular from Schur. 
\end{enumerate}
So, starting with $A_1$, and look for its eigenvalues. Let $A_0 = A$. We can define \[\boxed{A_k = R_k Q_k},\] with $R_k, Q_k$ from the QR decomposition of $A_{k - 1}$; in other words, \[\boxed{A_{k - 1} = Q_k R_k}.\] Then, $A_{k - 1}$ and $A_k$ have the same eigenvalues. 

\bigskip 

More formally, 
\begin{enumerate}
    \item Let $A_0 = A = Q_1 R_1$. Then, $A_1 = R_1 Q_1$. Here, $A_0$ and $A_1$ have the same eigenvalues. 
    \item Let $A_1 = Q_2 R_2$. Then, $A_2 = R_2 Q_2$. Here, $A_1$ and $A_2$ have the same eigenvalues and, in particular, $A_0, A_1, A_2$ all have the same eigenvalues. 
    \item Continue the process\dots
\end{enumerate}
Eventually, $\lim_{k \mapsto \infty} A_k = T$, with $T$ from the Schur decomposition. 

\bigskip 

\textbf{At the end,} the eigenvalues of $A$ are on the diagonal of $T$. If $A$ is real, then $T$ is ``almost'' upper triangular (real Schur decomposition). 

\bigskip 

Because this is an iterative method, we need a stopping criterion\footnote{To be discussed later.}.

\subsubsection{Disadvantages}
There are some significant disadvantages with doing QR iteration. 
\begin{itemize}
    \item \textbf{Flop Count:} QR decomposition needs $\BigO(n^3)$ flops\footnote{$n$ represents the size of the matrix.}, and we need one QR decomposition in every step of the iteration. This is too much work. 
    \item \textbf{Convergence Rate:} Convergence may be slow if the eigenvalues are close together in the absolute value. In case of distinct eigenvalues, $|\lambda_1| > |\lambda_2| > \hdots > |\lambda_n|$, applying QR iteration to $A$ means the elements below the diagonal goes to 0 by the following rate\footnote{$a_{ij}^{(k)}$ means the entry $(i, j)$ in matrix $A_k$.}
    \[(a_{ij}^{(k)}) = \BigO\left(\left|\frac{\lambda_i}{\lambda_j}\right|^k\right)\] for $i > j$ ($i$ below diagonal). 
\end{itemize}

\subsection{Upper Hessenberg Matrix}
\begin{definition}{Upper Hessenberg Matrix}
    An $n \times n$ matrix $H$ is called \textbf{upper Hessenberg} if $h_{ij} = 0$ for $i > j + 1$. 
\end{definition}
\textbf{Note:} If $H$ is also symmetric ($A = A^T$), then we get tridiagonal.

% TODO get matrix representation.





\end{document}