\documentclass[letterpaper]{article}
\usepackage[margin=1in]{geometry}
\usepackage[utf8]{inputenc}
\usepackage{textcomp}
\usepackage{amssymb}
\usepackage{natbib}
\usepackage{graphicx}
\usepackage{gensymb}
\usepackage{amsthm, amsmath, mathtools}
\usepackage[dvipsnames]{xcolor}
\usepackage{enumerate}
\usepackage{mdframed}
\usepackage[most]{tcolorbox}
\usepackage{csquotes}
% https://tex.stackexchange.com/questions/13506/how-to-continue-the-framed-text-box-on-multiple-pages

\tcbuselibrary{theorems}

\newcommand{\R}{\mathbb{R}}
\newcommand{\Z}{\mathbb{Z}}
\newcommand{\N}{\mathbb{N}}
\newcommand{\Q}{\mathbb{Q}}
\newcommand{\C}{\mathbb{C}}
\newcommand{\code}[1]{\texttt{#1}}
\newcommand{\mdiamond}{$\diamondsuit$}
\newcommand{\PowerSet}{\mathcal{P}}
\newcommand{\Mod}[1]{\ (\mathrm{mod}\ #1)}
\DeclareMathOperator{\lcm}{lcm}

%\newtheorem*{theorem}{Theorem}
%\newtheorem*{definition}{Definition}
%\newtheorem*{corollary}{Corollary}
%\newtheorem*{lemma}{Lemma}
\newtheorem*{proposition}{Proposition}


\newtcbtheorem[number within=section]{theorem}{Theorem}
{colback=green!5,colframe=green!35!black,fonttitle=\bfseries}{th}

\newtcbtheorem[number within=section]{definition}{Definition}
{colback=blue!5,colframe=blue!35!black,fonttitle=\bfseries}{def}

\newtcbtheorem[number within=section]{corollary}{Corollary}
{colback=yellow!5,colframe=yellow!35!black,fonttitle=\bfseries}{cor}

\newtcbtheorem[number within=section]{lemma}{Lemma}
{colback=red!5,colframe=red!35!black,fonttitle=\bfseries}{lem}

\newtcbtheorem[number within=section]{example}{Example}
{colback=white!5,colframe=white!35!black,fonttitle=\bfseries}{def}

\newtcbtheorem[number within=section]{note}{Important Note}{
        enhanced,
        sharp corners,
        attach boxed title to top left={
            xshift=-1mm,
            yshift=-5mm,
            yshifttext=-1mm
        },
        top=1.5em,
        colback=white,
        colframe=black,
        fonttitle=\bfseries,
        boxed title style={
            sharp corners,
            size=small,
            colback=red!75!black,
            colframe=red!75!black,
        } 
    }{impnote}
\usepackage[utf8]{inputenc}
\usepackage[english]{babel}
\usepackage{fancyhdr}
\usepackage[hidelinks]{hyperref}
\usepackage{csquotes}

\pagestyle{fancy}
\fancyhf{}
\rhead{Math 170A}
\chead{March 13, 2023}
\lhead{Lecture 25}
\rfoot{\thepage}

\setlength{\parindent}{0pt}
\newcommand{\0}{\mathbf{0}}
\newcommand{\y}{\mathbf{y}}
\renewcommand{\b}{\mathbf{b}}
\newcommand{\x}{\mathbf{x}}
\newcommand{\e}{\mathbf{e}}
\newcommand{\rr}{\mathbf{r}}
\newcommand{\vv}{\mathbf{v}}
\renewcommand{\u}{\mathbf{u}}

\begin{document}

\section{Iterative Methods: Gauss-Seidel Method (8.1-8.3)}
The Gauss-Seidel Method is another iterative method that is based on the Jacobi method discussed earlier. For any matrix $A$, we can write it into three parts: 
\[A = D + L + U.\]
Observing a $4 \times 4$ matrix, this might look like 
\[\underbrace{\begin{bmatrix}
    a_{11} & a_{12} & a_{13} & a_{14} \\ 
    a_{21} & a_{22} & a_{23} & a_{24} \\ 
    a_{31} & a_{32} & a_{33} & a_{34} \\ 
    a_{41} & a_{42} & a_{43} & a_{44} 
\end{bmatrix}}_{A} = \underbrace{\begin{bmatrix}
    a_{11} & 0 & 0 & 0 \\ 
    0 & a_{22} & 0 & 0 \\ 
    0 & 0 & a_{33} & 0 \\ 
    0 & 0 & 0 & a_{44} 
\end{bmatrix}}_{D} + \underbrace{\begin{bmatrix}
    0 & 0 & 0 & 0  \\
    a_{21} & 0 & 0 & 0 \\ 
    a_{31} & a_{32} & 0 & 0 \\ 
    a_{41} & a_{42} & a_{43} & 0 
\end{bmatrix}}_{L} + \underbrace{\begin{bmatrix}
    0 & a_{12} & a_{13} & a_{14} \\ 
    0 & 0 & a_{23} & a_{24}  \\
    0 & 0 & 0 & a_{34}  \\
    0 & 0 & 0 & 0
\end{bmatrix}}_{U}\]

\begin{mdframed}
    (Example.) Suppose \[A = \begin{bmatrix}
        3 & 0 & -8 \\ 
        -2 & -1 & 7 \\ 
        1 & 5 & 2
    \end{bmatrix}.\]
    If we decompose this into $D + L + U$, we have 
    \[D = \begin{bmatrix}
        3 & 0 & 0 \\ 0 & -1 & 0 \\ 0 & 0 & 2
    \end{bmatrix}, \quad L = \begin{bmatrix}
        0 & 0 & 0 \\ -2 & 0 & 0 \\ 1 & 5 & 0
    \end{bmatrix}, \quad U = \begin{bmatrix}
        0 & 0 & -8 \\ 0 & 0 & 7 \\ 0 & 0 & 0
    \end{bmatrix}.\]
\end{mdframed}

\subsection{Deriving from Jacobi Method}
Recall that we defined the iterative process for the Jacobi method 
\[x_i = \frac{1}{a_{ii}}\left(b_{i} - \sum_{\substack{j = 1 \\ j \neq i}}^n a_{ij}x_{j}\right) \qquad i = 1, 2, \hdots, n \text{ and } a_{ii} \neq 0.\]
With this in mind, notice that 
\begin{equation*}
    \begin{aligned}
        x_{i}^{(k + 1)} &= \frac{1}{a_{ii}} \left(b_{i} - \sum_{\substack{j = 1 \\ j \neq i}}^{n} a_{ij} x_{j}^{(k)}\right) \\ 
            &= \frac{1}{a_{ii}} \left(b_{i} - \sum_{j = 1}^{i - 1} a_{ij} x_{j}^{(k)} - \sum_{j = i + 1}^{n} a_{ij}x_{j}^{(k)}\right)
    \end{aligned}
\end{equation*}
Note that $x_{j}^{(k + 1)}$, where $1 \leq j \leq i - 1$, is already known, so we can just use that. Thus, we have  
\[\boxed{x_{i}^{(k + 1)} = \frac{1}{a_{ii}} \left(b_{i} - \sum_{j = 1}^{i - 1} a_{ij} x_{j}^{(k + 1)} - \sum_{j = i + 1}^{n} a_{ij}x_{j}^{(k)}\right)},\]
the Gauss-Seidel Method. Representing and restructuring this in matrix form, we have 
\begin{equation*}
    \begin{aligned}
        x^{(k + 1)} &= D^{-1} (b - Lx^{(k + 1)} - Ux^{(k)}) \\ 
            &\implies D x^{(k + 1)} = b - Lx^{(k + 1)} - Ux^{(k)} \\ 
            &\implies D x^{(k + 1)} + Lx^{(k + 1)} = b - Ux^{(k)} \\ 
            &\implies (D + L)x^{(k + 1)} = b - Ux^{(k)} \\ 
            &\implies \boxed{x^{(k + 1)} = (D + L)^{-1} (b - Ux^{(k)})}
    \end{aligned}
\end{equation*}
\textbf{Remarks:}
\begin{itemize}
    \item $D + L$ is the lower part of $A$. It is invertible if and only if all diagonal entries ($a_{ii}$) are non-zero.
    \item $D + L = A - U$. Thus, you can re-represent the Gauss-Seidel method as 
    \[x^{(k + 1)} = (A - U)^{-1} (b - Ux^{(k)}).\]
\end{itemize} 

\begin{mdframed}
    (Example.) Let $A = \begin{bmatrix}
        3 & 0 & -8 \\ 
        -2 & -1 & 7 \\ 
        1 & 5 & 2
    \end{bmatrix}$, $x^* = \begin{bmatrix}
        1 \\ 1 \\ 1
    \end{bmatrix}$, and $Ax^* = b = \begin{bmatrix}
        -5 \\ 4 \\ 8
    \end{bmatrix}$. Suppose we have the initial guess $x^{(0)} = \begin{bmatrix}
        2 \\ 0 \\ 1
    \end{bmatrix}$. \underline{Using the formula for each entry, we have the following example:}
    \begin{itemize}
        \item Starting at $k = 0$, for our first iteration, we have 
        \begin{equation*}
            \begin{aligned}
                x_{1}^{(1)} &= \frac{1}{a_{11}} \left(b_{1} - \sum_{j = 1}^{1 - 1} a_{1j} x_{j}^{(0 + 1)} - \sum_{j = 1 + 1}^{3} a_{1j}x_{j}^{(0)}\right) \\
                    &= \frac{1}{a_{11}} \left(b_{1} - \sum_{j = 1}^{0} a_{1j} x_{j}^{(1)} - \sum_{j = 2}^{3} a_{1j}x_{j}^{(0)}\right) \\
                    &= \frac{1}{a_{11}} \left(b_{1} - \sum_{j = 2}^{3} a_{1j}x_{j}^{(0)}\right) \\ 
                    &= \frac{1}{a_{11}} \left(b_{1} - (a_{12}x_{2}^{(0)} + a_{13}x_{3}^{(0)})\right).
            \end{aligned}
        \end{equation*}
        \begin{equation*}
            \begin{aligned}
                x_{2}^{(1)} &= \frac{1}{a_{22}} \left(b_{2} - \sum_{j = 1}^{2 - 1} a_{2j} x_{j}^{(0 + 1)} - \sum_{j = 2 + 1}^{3} a_{2j}x_{j}^{(0)}\right) \\ 
                    &= \frac{1}{a_{22}} \left(b_{2} - \sum_{j = 1}^{1} a_{2j} x_{j}^{(1)} - \sum_{j = 3}^{3} a_{2j}x_{j}^{(0)}\right) \\ 
                    &= \frac{1}{a_{22}} \left(b_{2} - a_{21} x_{1}^{(1)} - (a_{23}x_{3}^{(0)})\right).
            \end{aligned}
        \end{equation*}
        \begin{equation*}
            \begin{aligned}
                x_{3}^{(1)} &= \frac{1}{a_{33}} \left(b_{3} - \sum_{j = 1}^{3 - 1} a_{3j} x_{j}^{(0 + 1)} - \sum_{j = 3 + 1}^{3} a_{3j}x_{j}^{(0)}\right) \\ 
                    &= \frac{1}{a_{33}} \left(b_{3} - \sum_{j = 1}^{2} a_{3j} x_{j}^{(1)} - \sum_{j = 4}^{3} a_{3j}x_{j}^{(0)}\right) \\ 
                    &= \frac{1}{a_{33}} \left(b_{3} - \sum_{j = 1}^{2} a_{3j} x_{j}^{(1)}\right) \\
                    &= \frac{1}{a_{33}} \left(b_{3} - (a_{31} x_{1}^{(1)} + a_{32} x_{2}^{(1)})\right).
            \end{aligned}
        \end{equation*}
        This gives us $x^{(1)} = \begin{bmatrix}
            x_{1}^{(1)} \\ 
            x_{2}^{(1)} \\ 
            x_{3}^{(1)} 
        \end{bmatrix}$. The remaining iterations have the same idea.
    \end{itemize}
    \underline{Alternatively, using the matrix-based formula:}
    \begin{itemize}
        \item Starting at $k = 0$, for our first iteration, we have 
        \[x^{(1)} = (D + L)^{-1}(b - Ux^{(0)}).\]
        This requires us to find $(D + L)^{-1}$. Once we find this, we can perform the remaining iterations. 
    \end{itemize}
\end{mdframed}

\subsection{Convergence Result}
The Guass-Seidel method converges for every initial guess $x^{(0)}$ if all eigenvalues $\lambda$ of $G = I - (D + L)^{-1} A = I - (A - U)^{-1} A$ satisfy the following: 
\begin{itemize}
    \item All eigenvalues are distinct. 
    \item $|\lambda| < 1$. 
\end{itemize}
\begin{proof}
    The proof is similar to the proof for Jacobi. We want to show that $e^{(k)} = x^* - x^{(k)} \mapsto 0$. 
    \begin{equation*}
        \begin{aligned}
            e^{(k + 1)} &= x^{*} - x^{(k + 1)} \\ 
                &= x^* - (D + L)^{-1}(b - (A - L - D)x^{(k)}) \\
                &= x^* - (D + L)^{-1}(b - Ax^{(k)} - (L + D)x^{(k)}) \\  
                &= x^* - (D + L)^{-1}(b - Ax^{(k)}) - (D + L)^{-1}((L + D)x^{(k)}) \\ 
                &= x^* - (D + L)^{-1}(b - Ax^{(k)}) - x^{(k)} \\ 
                &= x^* - (D + L)^{-1}(b - A(x^* - e^{(k)})) - x^{(k)} \\ 
                &= x^* - (D + L)^{-1}(b - Ax^* + Ae^{(k)}) - x^{(k)} \\  
                &= x^* - (D + L)^{-1}(b - Ax^* + A(x^* - x^{(k)})) - x^{(k)} \\ 
                &= x^* - (D + L)^{-1}(\underbrace{b - Ax^*}_{0} + Ax^* - Ax^{(k)}) - x^{(k)} \\ 
                &= x^* - (D + L)^{-1}(Ax^* - Ax^{(k)}) - x^{(k)} \\ 
                &= x^* - (D + L)^{-1}(A(x^* - x^{(k)})) - x^{(k)} \\ 
                &= x^* - (D + L)^{-1}(Ae^{(k)}) - x^{(k)} \\ 
                &= x^* - x^{(k)} - (D + L)^{-1}(Ae^{(k)}) \\ 
                &= e^{(k)} - (D + L)^{-1}(Ae^{(k)}) \\ 
                &= (I - (D + L)^{-1}A) e^{(k)} \\ 
        \end{aligned}
    \end{equation*}
    Overall, $e^{(k + 1)} = (I - (D + L)^{-1}A)e^{(k)} = (I - (D + L)^{-1}A)^2 e^{(k - 1)} = \hdots = (I - (D + L)^{-1}A)^{k + 1} e^{(0)}$. So, \[e^{(k)} = (I - (D + L)^{-1} A)^k e^{(0)}.\] As $G = I - (D + L)^{-1} A$, we end up with \[e^{(k)} = G^k e^{(0)}.\] Since all eigenvalues are distinct, this means that $G$ has $n$ linearly independent eigenvectors $v_1, v_2, \hdots, v_n$. Let $\lambda_1, \hdots, \lambda_n$ denote the corresponding eigenvalues of $G$. Then, 
    \[Gv_i = \lambda_i v_i \qquad i = 1, 2, \hdots, n.\]
    As $\{v_i\}$ for $i = 1, 2, \hdots, n$ are linearly independent, we can express $e^{(0)}$ as a linear combination of them, 
    \[e^{(0)} = c_1 v_1 + c_2 v_2 + \hdots + c_n v_n.\]
    Then, the rest of the operation works similarly to the power method. That is, 
    \[Ge^{(0)} = c_1 \lambda_1 v_1 + c_2 \lambda_2 v_2 + \hdots + c_n \lambda_n v_n.\]
    \[G^2 e^{(0)} = c_1 \lambda_1^2 v_1 + c_2 \lambda_2^2 v_2 + \hdots + c_n \lambda_n^2 v_n.\]
    \[G^k e^{(0)} = c_1 \lambda_1^k v_1 + c_2 \lambda_2^k v_2 + \hdots + c_n \lambda_n^k v_n.\]
    Consequently, 
    \[||e^{(k)}|| = ||G^k e^{(0)}|| \leq |c_1| |\lambda_1^k| ||v_1|| + |c_2| |\lambda_2^k| ||v_2|| + \hdots + |c_n| |\lambda_n^k| ||v_n||.\]
    Since $|\lambda_i^k| \mapsto 0$ if and only if $|\lambda_i| < 1$, we have $||e^{(k)}|| \mapsto 0$ for every initial guess $x^{(0)}$ if all eigenvalues $|\lambda| < 1$. 
\end{proof}


\end{document}