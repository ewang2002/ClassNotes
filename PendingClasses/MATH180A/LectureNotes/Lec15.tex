\documentclass[letterpaper]{article}
\usepackage[margin=1in]{geometry}
\usepackage[utf8]{inputenc}
\usepackage{textcomp}
\usepackage{amssymb}
\usepackage{natbib}
\usepackage{graphicx}
\usepackage{gensymb}
\usepackage{amsthm, amsmath, mathtools}
\usepackage[dvipsnames]{xcolor}
\usepackage{enumerate}
\usepackage{mdframed}
\usepackage[most]{tcolorbox}
\usepackage{csquotes}
% https://tex.stackexchange.com/questions/13506/how-to-continue-the-framed-text-box-on-multiple-pages

\tcbuselibrary{theorems}

\newcommand{\R}{\mathbb{R}}
\newcommand{\Z}{\mathbb{Z}}
\newcommand{\N}{\mathbb{N}}
\newcommand{\Q}{\mathbb{Q}}
\newcommand{\C}{\mathbb{C}}
\newcommand{\code}[1]{\texttt{#1}}
\newcommand{\mdiamond}{$\diamondsuit$}
\newcommand{\PowerSet}{\mathcal{P}}
\newcommand{\Mod}[1]{\ (\mathrm{mod}\ #1)}
\DeclareMathOperator{\lcm}{lcm}

%\newtheorem*{theorem}{Theorem}
%\newtheorem*{definition}{Definition}
%\newtheorem*{corollary}{Corollary}
%\newtheorem*{lemma}{Lemma}
\newtheorem*{proposition}{Proposition}


\newtcbtheorem[number within=section]{theorem}{Theorem}
{colback=green!5,colframe=green!35!black,fonttitle=\bfseries}{th}

\newtcbtheorem[number within=section]{definition}{Definition}
{colback=blue!5,colframe=blue!35!black,fonttitle=\bfseries}{def}

\newtcbtheorem[number within=section]{corollary}{Corollary}
{colback=yellow!5,colframe=yellow!35!black,fonttitle=\bfseries}{cor}

\newtcbtheorem[number within=section]{lemma}{Lemma}
{colback=red!5,colframe=red!35!black,fonttitle=\bfseries}{lem}

\newtcbtheorem[number within=section]{example}{Example}
{colback=white!5,colframe=white!35!black,fonttitle=\bfseries}{def}

\newtcbtheorem[number within=section]{note}{Important Note}{
        enhanced,
        sharp corners,
        attach boxed title to top left={
            xshift=-1mm,
            yshift=-5mm,
            yshifttext=-1mm
        },
        top=1.5em,
        colback=white,
        colframe=black,
        fonttitle=\bfseries,
        boxed title style={
            sharp corners,
            size=small,
            colback=red!75!black,
            colframe=red!75!black,
        } 
    }{impnote}
\usepackage[utf8]{inputenc}
\usepackage[english]{babel}
\usepackage{fancyhdr}
\usepackage[hidelinks]{hyperref}

\pagestyle{fancy}
\fancyhf{}
\rhead{MATH 180A}
\chead{Monday, May 2, 2022}
\lhead{Lecture 15}
\rfoot{\thepage}

\setlength{\parindent}{0pt}

\begin{document}

\section{Expected Value and Variance}
\subsection{Examples of Finding Expected Value and Variance}

\begin{mdframed}[]
    (Example.) If $X$ is Bernoulli($p$), then $\mu = p$ and $\sigma^2 = pq$, where $q = 1 - p$. Then, 
    \[\mathbb{E}(X) = 1 \cdot p + 0 \cdot q = p.\]
    Note that $X^2$ has the same distribution as $X$, since $X$ only takes the values $0 = 0^2$ and $1 = 1^2$. Hence, $\mathbb{E}(X^2) = \mathbb{E}(X) = p$. Hence, 
    \[\text{Var}(X) = p - p^2 = p(1 - p) = pq.\]
\end{mdframed}

\begin{mdframed}[]
    (Example.) If $X$ is Binomial($n, p$), then it is the sum of $n$ independent Bernoulli($p$) trials. By the Linearity of Expectation, we know that 
    \[\mathbb{E}(X) = np.\]
    Since the trials are \emph{independent}, we have that 
    \[\text{Var}(X) = npq.\]
\end{mdframed}
Note that this is the special case of the following fact. 
\begin{theorem}{}{}
    Suppose that $X_1, \dots, X_n$ are IID with mean $\mu$ and variance $\sigma^2$. Then, their sum $S_n = \sum_{k = 1}^{n} X_k$ has mean $\mathbb{E}(S_n) = n\mu$ and variance $\text{Var}(S_n) = n\sigma^2$. 
\end{theorem}

\begin{mdframed}[]
    (Example.) If $X$ is Geometric($p$), then $\mathbb{E}(X) = \frac{1}{p}$. So, if $p$ is really small, then we should expect to wait a while before our first success; likewise, if $p$ is large, then we may not need to wait long before our first success. This is intuitive; in particular, the probability of success is $p$, so we should expect about 1 success in every $p$ trials. But, intuition aside, there are several ways to compute this.
    \begin{itemize}
        \item Approach 1.
        \begin{mdframed}[]
            \[\mathbb{E}(X) = \sum_{k = 1}^{\infty} kpq^{k - 1} = p\sum_{k = 1}^{\infty} kq^{k - 1}.\]
            Recall that this is a \emph{geometric} random variable, so we will use the geometric series; in particular, 
            \[\sum_{k = 0}^{\infty} q^k = \frac{1}{1 - q}\]
            and 
            \[\sum_{k = 1}^{\infty} kq^{k - 1} = \frac{d}{dq} \sum_{k = 0}^{\infty} q^k.\]
            Hence,
            \[\mathbb{E}(X) = p \frac{d}{dq} \frac{1}{1 - q} = p\frac{1}{(1 - q)^2} = \frac{p}{p^2} = \frac{1}{p}.\] 
            Similarly, we can show that $\text{Var}(X) = \frac{q}{p^2}$. 
        \end{mdframed}

        \item Approach 2. 
        \begin{mdframed}[]
            Let $X$ be the number of trials until the first success. Then, we have 
            \[\mathbb{E}(X) = 1p + (1 + \mathbb{E}(X))q.\]
            Solving for $\mathbb{E}(X)$ gives us the desired solution.
        \end{mdframed}
    \end{itemize}
\end{mdframed}

\begin{mdframed}[]
    (Example.) A Poisson($\lambda$) has mean and variance $\mu = \sigma^2 = \lambda$. So, 
    \[\mathbb{E}(X) = \sum_{k = 0}^{\infty} ke^{-\lambda} \frac{\lambda^k}{k!} = \sum_{k = 1}^{\infty} e^{-\infty} \frac{\lambda^k}{(k - 1)!} = \lambda \sum_{k = 0}^{\infty} e^{-\lambda} \underbrace{\frac{\lambda^k}{k!}}_{1} = \lambda.\]
    Similarly, you can show that $\mathbb{E}(X^2) = \lambda(1 + \lambda)$ so that $\text{Var}(X) = \lambda(1 + \lambda) - \lambda^2 = \lambda$. 
\end{mdframed}

\begin{mdframed}[]
    (Example.) An Exponential($\lambda$) has $\mu = \frac{1}{\lambda}$ and $\sigma^2 = \frac{1}{\lambda^2}$. For this computation, the theorem following this example will be useful. 

    \bigskip 
    
    If $X$ is Exponential($\lambda$), then it is non-negative and $\PR(X > x) = e^{-\lambda x}.$ Hence, 
    \[\mathbb{E}(X) = \int_{0}^{\infty} e^{-\lambda x} dx = \frac{1}{\lambda}.\]
\end{mdframed}

\begin{theorem}{Expectation Tail Sum for Non-Negative Random Variables}{}
    If $X$ is a non-negative random variable (i.e., $\PR(X \geq 0) = 1$), then 
    \begin{enumerate}
        \item If $X$ is discrete, then 
        \[\mathbb{E}(X) = \sum_{k = 0}^{\infty} \PR(X > k).\]

        \item If $X$ is continuous, then 
        \[\mathbb{E}(X) = \int_{0}^{\infty} \PR(X > x) dx.\]
    \end{enumerate}
\end{theorem}

\begin{mdframed}[]
    \begin{proof}
        (Discrete.) Just note that 
        \begin{equation*}
            \begin{aligned}
                \mathbb{E}(X) &= \sum_{k = 0}^{\infty} kp(k) \\ 
                    &= p(1) + (p(2) + p(2)) + (p(3) + p(3) + p(3)) + \dots \\ 
                    &= (p(1) + p(2) + p(3) + \dots) + (p(2) + p(3) + \dots) + (p(3) + \dots) + \dots \\
                    &= \PR(X > 0) + \PR(X > 1) + \PR(X > 2) + \dots
            \end{aligned}
        \end{equation*}
        Hence, we're done.
    \end{proof}
\end{mdframed}

\begin{mdframed}[]
    (Example.) If $X$ is Normal($\mu, \sigma^2$), then, indeed, $\mu$ is the mean and $\sigma^2$ is its variance. To see why this is the case, see lecture slides.
\end{mdframed}

\begin{mdframed}[]
    (Example.) If $X$ is Cauchy, then $\mathbb{E}(X)$ does not exist. Recall that a (standard) Cauchy has PDF 
    \[f(x) = \frac{1}{\pi(1 + x^2)}.\]
    Note that 
    \[\int_{-\infty}^{\infty} \frac{x}{\pi (1 + x^2)}dx\]
    diverges, since
    \[\int_{0}^{\infty} \frac{x}{1 + x^2}dx = \infty.\]
\end{mdframed}

\end{document}