\documentclass[letterpaper]{article}
\usepackage[margin=1in]{geometry}
\usepackage[utf8]{inputenc}
\usepackage{textcomp}
\usepackage{amssymb}
\usepackage{natbib}
\usepackage{graphicx}
\usepackage{gensymb}
\usepackage{amsthm, amsmath, mathtools}
\usepackage{xcolor}
\usepackage{enumerate}
\usepackage{framed}
\usepackage{tcolorbox}
\tcbuselibrary{theorems}

\newcommand{\R}{\mathbb{R}}
\newcommand{\Z}{\mathbb{Z}}
\newcommand{\N}{\mathbb{N}}
\newcommand{\Q}{\mathbb{Q}}
\newcommand{\code}[1]{\texttt{#1}}
\newcommand{\mdiamond}{$\diamondsuit$}

%\newtheorem*{theorem}{Theorem}
%\newtheorem*{definition}{Definition}
\newtheorem*{proposition}{Proposition}
%\newtheorem*{corollary}{Corollary}
%\newtheorem*{lemma}{Lemma}

\newtcbtheorem[number within=section]{theorem}{Theorem}
{colback=green!5,colframe=green!35!black,fonttitle=\bfseries}{def}

\newtcbtheorem[number within=section]{definition}{Definition}
{colback=blue!5,colframe=blue!35!black,fonttitle=\bfseries}{def}

\newtcbtheorem[number within=section]{corollary}{Corollary}
{colback=yellow!5,colframe=yellow!35!black,fonttitle=\bfseries}{def}

\newtcbtheorem[number within=section]{lemma}{Lemma}
{colback=red!5,colframe=red!35!black,fonttitle=\bfseries}{def}
\usepackage[utf8]{inputenc}
\usepackage[english]{babel}
\usepackage{fancyhdr}
\usepackage[hidelinks]{hyperref}

\pagestyle{fancy}
\fancyhf{}
\rhead{MATH 180A}
\chead{Wednesday, May 4, 2022}
\lhead{Lecture 16}
\rfoot{\thepage}

\setlength{\parindent}{0pt}

\begin{document}

\section{Expected Value and Variance}
\subsection{Conditional Expectation: Discrete Case}
Recall that if $X$ is a discrete random variable with PMF $p$, and $B$ is an event with $\PR(B) > 0$, then 
\[p(x | B) = \frac{p(x)}{\PR(B)}\]
is a probability distribution on $B$. This is the PMF of the random variable $X$, given $B$. 

\begin{definition}{}{}
    Let $X$ be a random variable with PMF $p$. Suppose that $\PR(B) > 0$. Then, the conditional expectation of $X$ given $B$ is 
    \[\mathbb{E}(X | B) = \sum_{x} xp(x | B).\]    
\end{definition}


\subsubsection{Law of Total Expectation}
Just like how there is a Law of Total Probability, there is also a Law of Total Expectation. 
\begin{theorem}{Law of Total Expectation}{}
    Let $X$ be a random variable on sample space $\Omega$. Suppose that $B_1, \dots, B_n$ is a partition $\Omega$. Then, 
    \[\mathbb{E}(X) = \sum_{i = 1}^{n} \mathbb{E}(X | B_i) \PR(B_i).\]
\end{theorem}
This is useful because, often, $\mathbb{E}(X)$ is sometimes difficult to find directly. However, if we condition on a well-chosen $B_i$, then it becomes manageable. 

\begin{mdframed}[]
    (Example.) In the gambling game ``craps,'' a player makes a bet and then rolls a pair of dice. If the sum is 7 or 11, the player wins. If it is 2, 3, or 12, the player loses. If the sum is any other number $s$, the player continues to roll until either another $s$ (they win) or 7 (they lose) occurs (7 is lucky the first time). Now, let $R$ be the number of rolls in a single game of craps.
    \begin{enumerate}
        \item Find $\mathbb{E}(R)$. 
    \end{enumerate}

    \begin{mdframed}[]
        \begin{enumerate}
            \item By the Law of Total Expectation, we have 
            \[\mathbb{E}(R) = \sum_{x = 2}^{12} \mathbb{E}(R | X = x)\PR(X = x),\]
            where $X$ is the initial sum. Note that if 
            \[x \in \{2, 3, 7, 11, 12\},\]
            then \[\mathbb{E}(R | X = x) = 1\]
            since the game is immediately over if we get one of those numbers. In particular,
            \begin{itemize}
                \item There is 1 way to get a 2 (11).
                \item There are 2 ways to get a 3 (12, 21).
                \item There are 6 ways to get a 7 (16, 61, 25, 52, 43, 34).
                \item There are 2 ways to get a 11 (56, 65).
                \item There is 1 way to get a 12 (66).
            \end{itemize}
            Hence, 
            \[\sum_{x \in \{2, 3, 7, 11, 12\}} \mathbb{E}(R | X = x) \PR(X = x) = \frac{12}{36}.\]
            Now, if \[x \in \{4, 5, 6, 8, 9, 10\},\] then by a similar argument to the one above, we have 
            \[\sum_{x \in \{2, 3, 7, 11, 12\}} \mathbb{E}(R | X = x)\PR(X = x) = \frac{24}{36}.\]
        \end{enumerate}
    \end{mdframed}
\end{mdframed}
% 16:00 

\subsection{Conditional Expectation: Continuous Case}
The situation is similar in the continuous case, but we instead have a conditional PDF 
\[f(x | B) = \frac{f(x)}{\PR(B)}\]
and the conditional expectation is given by  
\[\mathbb{E}(X | B) = \int_{-\infty}^{\infty} xp(x | B).\]    

\end{document}