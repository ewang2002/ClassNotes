\documentclass[letterpaper]{article}
\usepackage[margin=1in]{geometry}
\usepackage[utf8]{inputenc}
\usepackage{textcomp}
\usepackage{amssymb}
\usepackage{natbib}
\usepackage{graphicx}
\usepackage{gensymb}
\usepackage{amsthm, amsmath, mathtools}
\usepackage{xcolor}
\usepackage{enumerate}
\usepackage{framed}
\usepackage{tcolorbox}
\tcbuselibrary{theorems}

\newcommand{\R}{\mathbb{R}}
\newcommand{\Z}{\mathbb{Z}}
\newcommand{\N}{\mathbb{N}}
\newcommand{\Q}{\mathbb{Q}}
\newcommand{\code}[1]{\texttt{#1}}
\newcommand{\mdiamond}{$\diamondsuit$}

%\newtheorem*{theorem}{Theorem}
%\newtheorem*{definition}{Definition}
\newtheorem*{proposition}{Proposition}
%\newtheorem*{corollary}{Corollary}
%\newtheorem*{lemma}{Lemma}

\newtcbtheorem[number within=section]{theorem}{Theorem}
{colback=green!5,colframe=green!35!black,fonttitle=\bfseries}{def}

\newtcbtheorem[number within=section]{definition}{Definition}
{colback=blue!5,colframe=blue!35!black,fonttitle=\bfseries}{def}

\newtcbtheorem[number within=section]{corollary}{Corollary}
{colback=yellow!5,colframe=yellow!35!black,fonttitle=\bfseries}{def}

\newtcbtheorem[number within=section]{lemma}{Lemma}
{colback=red!5,colframe=red!35!black,fonttitle=\bfseries}{def}
\usepackage[utf8]{inputenc}
\usepackage[english]{babel}
\usepackage{fancyhdr}
\usepackage[hidelinks]{hyperref}

\pagestyle{fancy}
\fancyhf{}
\rhead{MATH 180A}
\chead{Monday, April 25, 2022}
\lhead{Lecture 12}
\rfoot{\thepage}

\setlength{\parindent}{0pt}

\begin{document}

\section{Distributions and Densities}
\subsection{Types of Continuous Probability Distributions}
The following are some of the most important continuous probability distributions, some of which we've seen before. 
\begin{itemize}
    \item Uniform 
    \item Exponential 
    \item Gamma 
    \item Normal 
    \item Cauchy 
\end{itemize}

\subsubsection{Uniform Distribution}
\begin{definition}{Uniform Distribution}{}
    $U$ has \textbf{Uniform}$[a, b]$ distribution, for $a < b$, if its PDF is 
    \[f(u) = \frac{1}{b - a}\]
    for $a \leq u \leq b$. 
\end{definition}
\textbf{Remark:} Note that $b - a$ is the \emph{length} of $[a, b]$. 

\textbf{Note:} There are also higher-dimensional uniform distributions, but then we replace length with area of volume. 

\subsubsection{Exponential Distribution}
\begin{definition}{Exponential Distribution}{}
    $X$ is Exponential($\lambda$) with rate $\lambda > 0$ if its PDF is 
    \[f(x) = \lambda e^{-\lambda x}\]
    for $x > 0$. 
\end{definition}
Note that there is an important connection between the Exponential and the Poisson, which we will now describe. 
\begin{mdframed}[]
    (Example: Busy Server.) Suppose that a single server queue (e.g. call center, bank, etc.) is very busy, so that there is always someone in the queue. Suppose that service times are independent and Exponential($\lambda$)\footnote{As you wait longer, the time that you wait ``decays,'' or decreases.}. As soon as someone has been served, the next person in the queue starts being served immediately. Let $X_1, X_2, \dots$ be an IID sequence of Exponential($\lambda$) random variables. Then, the time $T_n$ at which the point the $n$th person has been served is distributed as \[\sum_{i = 1}^{n} X_i.\] This sum of $n$ IID Exponential($\lambda$) random variables has a special distribution, called the \textbf{Gamma}($n, \lambda$) distribution. This has PDF\footnote{When $n = 1$, this reduces to $\lambda e^{-\lambda x}$, the PDF of a single Exponential($\lambda$).} 
    \[g(x) = \lambda e^{-\lambda x} \frac{(\lambda x)^{n - 1}}{(n - 1)!}\]
    for $x > 0$.
\end{mdframed}
The Poisson distribution arises when we ask: What is the distribution for the number $N_t$ of people served by time $t > 0$? At any given time $t > 0$, we will (with probability 1, since the service times are continuous) be in the middle of serving someone. This person does not count towards $N_t$. 

\bigskip 

Note that $(N_t : t > 0)$ is called the \textbf{Poisson process}\footnote{This is a fascinating mathematical object with many properties and applications, which won't be covered here.}. This is a \emph{collection}, indexed by time, of random variables\footnote{Such an object is called a \textbf{stochastic process}.} In particular, the random variable $N_t$ has the Poisson($\lambda t$) distribution.

\begin{mdframed}[]
    \begin{proof}
        Note that $N_t = k$ if and only if the $k$th person is served at some time $T_k = s \leq t$, and then the next service $X_{k + 1} > t - s$. In other words, we need to have finished serving $k$ people and be in the middle of serving the ($k + 1$)th person. Since \[T_k = \sum_{i = 1}^{k} X_i\] and $X_{k + 1}$ are independent, it follows that 
        \[\PR(N_t = k) = \int_{0}^t f_{T_k}(s) [1 - F_{X_{k + 1}} (t - s)] ds.\]
        Since $T_k$ is Gamma($k, \lambda$) and $X_{k + 1}$ is Exponential($\lambda$), it follows that 
        \begin{equation*}
            \begin{aligned}
                \PR(N_t = k) &= \int_{0}^t f_{T_k}(s) [1 - F_{X_{k + 1}} (t - s)] ds \\ 
                    &= \int_{0}^{t} \lambda e^{-\lambda s} \frac{(\lambda s)^{k - 1}}{(k - 1)!} \cdot e^{-\lambda (t - s)} ds \\ 
                    &= e^{-\lambda t} \frac{\lambda^k}{(k - 1)!} \int_{0}^{t} s^{k - 1} ds \\ % TODO how? 
                    &= e^{-\lambda t} \frac{(\lambda t)^k}{k!}.
            \end{aligned}
        \end{equation*}
        Hence, this is the PMF of a Poisson, as claimed. 
    \end{proof}
\end{mdframed}

\subsubsection{Normal/Gaussian Distribution}
Recall that a Binomial converges to a Poisson if 
\[p = \lambda / n \mapsto 0\]
as \[n \mapsto \infty.\]
On the other hand, if $p$ is \emph{fixed} (not converging to 0), the Binomial approaches a different distribution as $n \mapsto \infty$ called the \textbf{Normal} or \textbf{Gaussian} distribution. Indeed, as $n$ goes to infinity, we see a bell-shaped curve. 
\begin{definition}{Normal Distribution}{}
    $X$ is \textbf{Normal}($\mu, \sigma^2$) if its PDF is 
    \[\frac{1}{\sigma \sqrt{2\pi}} e^{-\frac{(x - \mu)^2}{2 \sigma^2}}\]
    for $-\infty < x < \infty$. 
\end{definition}
Here, $\mu$ is the ``center'' of the density and $\sigma$ is the measure of the ``spread'' of the density. 

When $\mu = 0$ and $\sigma^2 = 1$, $X$ is called a standard normal, and its PDF is given the special notation 
\[\Phi(x) = \frac{1}{\sqrt{2\pi}} e^{-x^2 / 2}.\]
If $X$ is not standard, then you can ``standardize'' by taking $Z = (X - \mu) / \sigma$. Then, $X$ is Normal($\mu, \sigma^2$) if and only if $Z$ is Normal($0, 1$).

\subsubsection{Cauchy Distribution}
Now, suppose that $X$ and $Y$ are two independent standard Normal random variables. A very interesting distribution arises if we consider the ratio \[Z = X / Y.\] Since $X$ and $Y$ are independent, \[f_{Z}(z) = \int_{S_z} f_{X, Y}(x, y) dx dy,\] where $S_z = \{\{(x, y) \mid x / y = z\}$. We make a change of variables $x = uz$ and $y = u$. Then, as $u$ varies over $\R$, we get the whole set $S_z$. The Jacobian of this transformation is $|u|$, so 
\[f_{Z}(z) = \int_{S_z} f_{X, Y}(x, y) dxdy = \int_{-\infty}^{\infty} |u| f_{X, Y}(uz, u) du.\]
This is the same as 
\[2\left(\frac{1}{\sqrt{2\pi}}\right)^2 \int_{0}^{\infty} ue^{-(uz)^2 / 2 - u^2 / 2} du.\]
It can be shown that 
\[\int_0^{\infty} xe^{-cx^2} dx = \frac{1}{2c}.\]
Hence,
\[f_{Z}(z) = \frac{1}{\pi} \int_0^{\infty} ue^{-\frac{u^2 (1 + z^2)}{2}} du = \frac{1}{\pi (1 + z^2)}.\]
A random variable with this PDF is called a (standard) \textbf{Cauchy} random variable. 

\bigskip 

Note that the Cauchy distribution has some interesting properties. In partivular, it has no expected/average value. So, if you take an IID sequence $X_1, X_2, \dots$ of Cauchy random variables, the limit 
\[\lim_{n \mapsto \infty} \frac{1}{n} \sum_{i = 1}^n X_i\]
does not exist. 

\end{document}