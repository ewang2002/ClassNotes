\documentclass[letterpaper]{article}
\usepackage[margin=1in]{geometry}
\usepackage[utf8]{inputenc}
\usepackage{textcomp}
\usepackage{amssymb}
\usepackage{natbib}
\usepackage{graphicx}
\usepackage{gensymb}
\usepackage{amsthm, amsmath, mathtools}
\usepackage{xcolor}
\usepackage{enumerate}
\usepackage{framed}
\usepackage{tcolorbox}
\tcbuselibrary{theorems}

\newcommand{\R}{\mathbb{R}}
\newcommand{\Z}{\mathbb{Z}}
\newcommand{\N}{\mathbb{N}}
\newcommand{\Q}{\mathbb{Q}}
\newcommand{\code}[1]{\texttt{#1}}
\newcommand{\mdiamond}{$\diamondsuit$}

%\newtheorem*{theorem}{Theorem}
%\newtheorem*{definition}{Definition}
\newtheorem*{proposition}{Proposition}
%\newtheorem*{corollary}{Corollary}
%\newtheorem*{lemma}{Lemma}

\newtcbtheorem[number within=section]{theorem}{Theorem}
{colback=green!5,colframe=green!35!black,fonttitle=\bfseries}{def}

\newtcbtheorem[number within=section]{definition}{Definition}
{colback=blue!5,colframe=blue!35!black,fonttitle=\bfseries}{def}

\newtcbtheorem[number within=section]{corollary}{Corollary}
{colback=yellow!5,colframe=yellow!35!black,fonttitle=\bfseries}{def}

\newtcbtheorem[number within=section]{lemma}{Lemma}
{colback=red!5,colframe=red!35!black,fonttitle=\bfseries}{def}
\usepackage[utf8]{inputenc}
\usepackage[english]{babel}
\usepackage{fancyhdr}
\usepackage[hidelinks]{hyperref}

\pagestyle{fancy}
\fancyhf{}
\rhead{Math 187A}
\chead{Monday, January 09, 2023}
\lhead{Lecture 1}
\rfoot{\thepage}

\setlength{\parindent}{0pt}

\begin{document}
\section{Introduction to Cryptography}
We begin with some common definitions.

\subsection{Terminology}
\begin{definition}{Cipher}{}
    A \textbf{cipher}, or cryptosystem, is a cryptographic method for confidential communication. 
\end{definition}
Generally, a cryptographic method includes algorithms for \emph{encryption} and \emph{decryption}, which are inverse processes that convert between plainly readable information called \emph{plaintext}\footnote{In cryptography, we use \emph{plaintext} and \emph{ciphertext} instead of \emph{plain text} and \emph{cipher text}.} and unintelligible information called \emph{ciphertext}.

\begin{definition}{Sender}{}
    A \textbf{sender}, often named ``Alice'' in abstract cryptographic discussions, \emph{encrypts} her plaintext into ciphertext. 
\end{definition}

\begin{definition}{Receiver}{}
    A \textbf{receiver}, often named ``Bob,'' \emph{decrypts} (or deciphers) the ciphertext back into plaintext. 
\end{definition}
Often times, Bob will use a \emph{key} to decrypt the message. This is sometimes known as a private key or decryption key.

\begin{definition}{Encoding}{}
    The (usually) preliminary step where a message is converted into a format which can then be encrypted is called \textbf{encoding}. 
\end{definition}
Note that encoded text is not secure; it is only secure after encryption. So, we can think of encoding as the pre-processing step. In other words, before we encrypt something, we might \emph{encode} the text so it's easier to encrypt. It should also be noted that if a message had to be encoded before encryption, then it will also need to be decoded after decryption. 

\begin{center}
    \includegraphics[scale=0.8]{../assets/encode_encrypt.png}
\end{center}

\begin{definition}{Adversary}{}
    An \textbf{adversary}, often named ``Eve,'' is one whose aim is to prevent the users of a cryptosystem from achieving their goal. 
\end{definition}
In our case here, an adversary can intercept a ciphertext. Thus, the adversary will not have Bob's decryption key at the beginning. The idea is that, even if the adversary knows what cryptosystem was used to encrypt the message, if the adversary doesn't have this decryption key, she should ideally not be able to decrypt the message. If she does manage to figure out the plaintext, she has \emph{broken} the code.

\begin{definition}{Attack Model}{}
    An \textbf{attack model} specifies what Eve is allowed to do in order to break the code. 
\end{definition}
Some common attack models includes:
\begin{itemize}
    \item Ciphertext-only attack: Eve must recover the plaintext using only the ciphertext.
    \item Known-plaintext attack: Eve may have access to some information about the plaintext (e.g., knowledge of portions of the plaintext), which can be used to recover the plaintext entirely. 
    \item Chosen-plaintext attack: Eve can request or generate ciphertexts corresponding to any plaintext message of her choosing, and she can use this information to recover the plaintext.
\end{itemize}
Classical cryptography was mostly concerned with assuring security against the first two. Modern cryptography tries to assure security against the last. 


\newpage 
\section{Classical Cryptosystems}
We begin with a definition: 
\begin{definition}{$n$-gram}{}
    An $n$-gram is a sequence of $n$ letters.
\end{definition}
For example, a 1-gram is just a single letter; a 2-gram (i.e., \emph{bigram}) is a pair of letters; and so on. Generally, we can group many classical cryptosystems into a few different encryption strategies.

\begin{center}
    \begin{tabular}{|p{1in}|p{5in}|}
        \hline
        \textbf{Strategy} & \textbf{Description} \\ 
        \hline 
        Transposition & Involves rearranging units of plaintext according to some pattern. We'll see just one example of this type of cipher: rectangular transposition. \\ 
        \hline 
        Substitution & Involves replacing units of plaintext with units of ciphertext. We can further group substitution ciphers into some subtypes: 
        \begin{center}
            \begin{tabular}{|p{1in}|p{3.5in}|}
                \hline
                \textbf{Subtype} & \textbf{Description} \\ 
                \hline
                Simple \\ Substitution & In these ciphers, single letters of plaintext are replaced by ciphertext. The substitution scheme stays the same over the course of the entire message. Some examples we'll see include:
                \begin{itemize}
                    \item Masonic cipher 
                    \item Caesar cipher 
                    \item Affine cipher 
                    \item Polybius square 
                \end{itemize}
                In essence, though, there is a 1-1 relationship between the letters of the plaintext and the ciphertext alphabets. \\
                \hline 
                Polygraphic \\ Substitution & In these ciphers, groups of letters in the plaintext are replaced by ciphertext (a group of $n$ letters is called an $n$-gram).The substitution scheme stays the same over the entire message. Some examples we'll see include: 
                \begin{itemize}
                    \item Hill cipher
                    \item Playfair cipher
                \end{itemize}
                So, in essence, polygraphic substitution is just simple substitution but with \emph{groups of letters} instead of individual letters. \\ 
                \hline 
                Polyalphabetic \\ Substitution & In these ciphers, single letters in the plaintext are replaced by ciphertext, and the substitution scheme changes over the course of the message. Some examples include: 
                \begin{itemize}
                    \item Vignere cipher 
                    \item One-time pad 
                \end{itemize} 
                \\ 
                \hline 
            \end{tabular}
        \end{center}
        In practice, however, most cryptosystems employ a combination of these strategies. \\
        \hline 
    \end{tabular}
\end{center}


















\subsection{Rectangular Tranposition}
\textbf{Rectangular tranposition}, known also as \emph{regular columnar transposition}, is a tranposition cipher. The ciphertext is obtained by \emph{permuting} the letters of the plaintext in a particular pattern. The pattern is determined by a secret \emph{keyword}. 

\begin{mdframed}[]
    (Example: Encryption.) Suppose that Alice and Bob share the keyword \code{GUARD}, and that Alice wants to send the following message to Bob: 
    \begin{mdframed}
        \begin{verbatim}
Hide! The baboons are coming for you. \end{verbatim}
    \end{mdframed}

    First, we'll \textbf{encode} the message so that it's easier to encrypt. In our example, we'll remove all spaces and punctuation. 
    \begin{mdframed}
        \begin{verbatim}
HIDETHEBABOONSARECOMINGFORYOU \end{verbatim}
    \end{mdframed}
    Now that encoding is done, we still need to encrypt the message. Notice how the keyword \code{GUARD} has 5 letters; we can break the message up into 5-grams and then stack them into rows:
    \begin{mdframed}
        \begin{verbatim}
HIDET
HEBAB
OONSA
RECOM
INGFO
RYOU\end{verbatim}
    \end{mdframed}
    We then need to insert some random letters at the end of the message so every row has an equal number of letters. Let's use \code{Q}:
    \begin{mdframed}
        \begin{verbatim}
HIDET
HEBAB
OONSA
RECOM
INGFO
RYOUQ\end{verbatim}
    \end{mdframed}
    Now, we begin the \textbf{encryption} process by rearranging the letters in each row based on the alphabetical ranking of the letters of the keyword \code{GUARD}. We note that the alphabetical rankings of the letters of this keyword are 3, 5, 1, 4, 2. 
    \begin{center}
        \includegraphics[scale=0.9]{../assets/rank_crypto.png}
    \end{center}
    Let's suppose each letter in one of the rows of the 5-grams for the plaintext is in the order \textbf{\code{12345}}. The idea is that we want to map the letters so that they're in the order \textbf{\code{35142}}. In order words, 
    \[f(\code{12345}) = \code{35142}.\]
    Consider the first row of the 5-grams for the plaintext (with the numbers representing the order that the letters appear in):
    \begin{mdframed}
        \begin{verbatim}
12345
HIDET\end{verbatim}
    \end{mdframed}
    The corresponding 5-gram for the ciphertext would be: 
    \begin{mdframed}
        \begin{verbatim}
35142
DTHEI\end{verbatim}
    \end{mdframed}
    \begin{center}
        \includegraphics[scale=0.7]{../assets/ex1_hidet.png}
    \end{center}
    Applying this to every row of the 5-gram plaintext, we get 
    \begin{mdframed}
        \begin{verbatim}
DTHEI
BBHAE
NAOSO
CMROE
GOIFN
OQRUY\end{verbatim}
    \end{mdframed}
    Undoing the stacking gives us the ciphertext: 
    \begin{mdframed}
        \begin{verbatim}
DTHEIBBHAENAOSOCMROEGOIFNOQRUY\end{verbatim}
    \end{mdframed}
\end{mdframed}


\begin{mdframed}
    (Example: Decryption.) Consider the above example again. Suppose Alice successfully sends the following ciphertext to Bob:
    \begin{mdframed}
        \begin{verbatim}
DTHEIBBHAENAOSOCMROEGOIFNOQRUY\end{verbatim}
    \end{mdframed}
    Bob knows that the keyword is \code{GUARD}. He can use this keyword to decrypt the message. He can begin by taking the letters of the ciphertext and stacking them into rows of 5, since \code{GUARD} has 5 letters:
    \begin{mdframed}
        \begin{verbatim}
DTHEI
BBHAE
NAOSO
CMROE
GOIFN
OQRUY\end{verbatim}
    \end{mdframed}
    Bob also knows the alphabetical ranking of the letters of \code{GUARD} (which is the same rankings as described above). Now, because each 5-gram is part of the \emph{ciphertext}, the ordering of each letter in a 5-gram is \code{35142}. To decrypt the message, he needs to map the letters in each row (5-gram) so that it's in order \code{12345}. 
    
    \bigskip
    
    Consider the first row of the 5-gram for the ciphertext (numbering shown to indicate ordering):
    \begin{mdframed}
        \begin{verbatim}
35142
DTHEI\end{verbatim}
    \end{mdframed}
    The corresponding 5-gram for the plaintext would be: 
    \begin{mdframed}
        \begin{verbatim}
12345
HIDET\end{verbatim}
    \end{mdframed}

    \begin{center}
        \includegraphics[scale=0.7]{../assets/ex2_hidet.png}
    \end{center}
    Performing this operation for each row gives us the following: 
    \begin{mdframed}
        \begin{verbatim}
HIDET
HEBAB
OONSA
RECOM
INGFO
RYOUQ\end{verbatim}
    \end{mdframed}
    Undoing the stacking gives us:
    \begin{mdframed}
        \begin{verbatim}
HIDETHEBABOONSARECOMINGFORYOUQ\end{verbatim}
    \end{mdframed}
    At this point, Bob needs to make an educated guess as to what the encoded message says (recall that we had to encode the message before encrypting it). By removing the \code{Q} and correctly punctuating the message, we get 
    \begin{mdframed}
        \begin{verbatim}
Hide! The baboons are coming for you.\end{verbatim}
    \end{mdframed}
\end{mdframed}


\begin{mdframed}[nobreak=true]
    (Exercise: Encryption.) \emph{Encrypt the message \code{There is always hope.} using the keyword \code{CRASH}.}

    \begin{mdframed}
        First, we encode the message so that we can easily encrypt it: 
        \begin{mdframed}
            \begin{verbatim}
THEREISALWAYSHOPE\end{verbatim}
        \end{mdframed}

        Noting that \code{CRASH} has length 5, we break the now encoded message into groups of 5 letters (5-grams):
        \begin{mdframed}
            \begin{verbatim}
THERE
ISALW
AYSHO
PE\end{verbatim}
        \end{mdframed}
        
        Let's now add nonsense letters at the end of the last row so every row has 5 letters: 
        \begin{mdframed}
            \begin{verbatim}
THERE
ISALW
AYSHO
PEABC\end{verbatim}
        \end{mdframed}

        Now, we note the alphabetical ranking of each letter in \code{CRASH}:
        \[C \mapsto 2 \quad R \mapsto 4 \quad A \mapsto 1 \quad S \mapsto 5 \quad H \mapsto 3.\]
        So, effectively, if the original positioning of each letter in a row is \code{12345}, we need to remap them so they are in the positioning \code{24153}. This gives us:
        \begin{mdframed}
            \begin{verbatim}
1 2 3 4 5 | 2 4 1 5 3
T H E R E | H R T E E
I S A L W | S L I W A
A Y S H O | Y H A O S
P E A B C | E B P C A\end{verbatim}
        \end{mdframed}
        Unstacking the new rows gives us the ciphertext:
        \begin{mdframed}
            \begin{verbatim}
HRTEESLIWAYHAOSEBPCA\end{verbatim}
        \end{mdframed}
    \end{mdframed}
\end{mdframed}


\begin{mdframed}[nobreak=true]
    (Exercise: Decryption.) \emph{Decrypt the message \code{HGTIEAFREFAESLROEOXS.} using the keyword \code{CRASH}.}

    \begin{mdframed}
        Begin by grouping the letters into 5-grams, since \code{CRASH} has length 5:
        \begin{mdframed}
            \begin{verbatim}
HGTIE
AFREF
AESLR
OEOXS\end{verbatim}
        \end{mdframed}
        Recall that the alphabetical ranking of each letter in \code{CRASH} is \code{24153}. Because the letters in each row are in position \code{24153}, we simply need to map this to \code{12345}. Therefore, we have:
        \begin{mdframed}
            \begin{verbatim}
2 4 1 5 3    1 2 3 4 5
H G T I E -> T H E G I
A F R E F -> R A F F E
A E S L R -> S A R E L
O E O X S -> O O S E X\end{verbatim}
        \end{mdframed}
        Unstacking the new rows gives us the plaintext:
        \begin{mdframed}
            \begin{verbatim}
THEGIRAFFESARELOOSEX\end{verbatim}
        \end{mdframed}
        Decoding the message gives us: 
        \begin{mdframed}
            \begin{verbatim}
The giraffes are loose.\end{verbatim}
        \end{mdframed}
    \end{mdframed}

\end{mdframed}


\subsection{Masonic Cipher}
The masonic cipher (also known as the \emph{pigpen cipher} or \emph{tic-tac-toe cipher}) is a simple substitution cipher that replaces individual letters with certain geometric shapes.

\bigskip 

For example, consider the following diagram, which represents a Masonic cipher for the English letters:
\begin{center}
    \includegraphics[scale=0.75]{../assets/masonic_ex1.png}
\end{center}
The idea is that we can replace a letter (e.g., \code{A}) with a corresponding geometric shape (e.g., the backwards \code{L} represented by the top-left part of the grid.) 

\bigskip 

Some other examples based on the above cipher are shown below:
\begin{center}
    \includegraphics[scale=0.9]{../assets/masonic_ex2.png}
\end{center}
Note that there is \emph{no key} associated with this cipher. There is only a decryption function (which is just mapping the geometric shape back to the letter). Therefore, the adversary, who knows that a message was encrypted using a masonic cipher, can recover the plaintext easily. 


\subsection{Caesar Cipher}
The Caesar cipher, also known as a \emph{shift cipher}, is a simple substitution cipher that \emph{shifts} a letter by some amount $n$. Hence, the key for this cipher is an integer $n$. The idea is that we initially assign each letter an integer, perhaps by their alphabetical ranking (e.g., $A$ is 0, $B$ is 1, and so on.) If we want to shift the letters by some number, we can just ``move'' the letters by that amount. If a letter gets a new integer that's greater than 25, we can ``wrap'' the letter back.  

\bigskip 

Consider the following diagram, which shows the correspondence between the plaintext alphabet and the ciphertext alphabet.
\begin{center}
    \includegraphics[scale=0.75]{../assets/ceasar_1.png}
\end{center}
In this particular diagram, when we apply a shift, we apply the shift to the \emph{plain} row. By doing this, we can translate whatever plaintext we have to ciphertext. 

\begin{mdframed}
    (Example.) If we shift each letter by 3 (i.e., $n = 3$), we have 
    \begin{center}
        \includegraphics[scale=0.7]{../assets/ceasar_2.png}
    \end{center}
    Notice how $A$ now corresponds to 3. Recall that $A$'s original position was 0; if we shift each letter by 3, we essentially add 3 to $A$'s original position to get the new position 
    \[0 + 3 = 3.\]
    The same idea applies to any other letter. One key thing to notice is how $X$, $Y$, and $Z$ were \emph{wrapped back} to the beginning. In any case, let's see how translation would work in this case: 
    \begin{itemize}
        \item To convert a letter from plaintext to ciphertext, look for the letter in the(shifted) plaintext row and then look at the corresponding ciphertext column. For example, $R$ in plaintext would become $U$ in ciphertext. 
        \item To convert a letter from ciphertext to plaintext, look for the letter in the ciphertext row and then look at the corresponding (shifted) plaintext column. For example, $U$ in ciphertext becomes $R$ in plaintext. 
    \end{itemize}
\end{mdframed}

\begin{mdframed}
    (Example.) If we shift each letter by -2 (i.e., $n = -2$), we have 
    \begin{center}
        \includegraphics[scale=0.7]{../assets/ceasar_3.png}
    \end{center}
\end{mdframed}

As with rectangular tranposition, we should encode the message by removing any non-alphabetic characters and capitalizing everything. 

\begin{mdframed}
    (Exercise.) 
    \begin{itemize}
        \item \emph{Using a shift of 3, encrypt the message \code{Meet at La Jolla Shores.}}
        \begin{mdframed}
            Encoding the message gives us \code{MEETATLAJOLLASHORES}. Then, we can use the example above (with the shift of 3) to give us the proper correspondence.
            \begin{verbatim}
    plain       M E E T A T L A J O L L A S H O R E S
    cipher      P H H W D W O D M R O O D V K R U H V\end{verbatim}
            This gives us \code{PHHWDWODMROODVKRUHV}.
        \end{mdframed}
        \item \emph{Using a shift of 3, decrypt the message \code{PHHWDWVXQJRGODZQ}}
        \begin{mdframed}
            Using the example above (with the shift of 3), we have
            \begin{verbatim}
    cipher      P H H W D W V X Q J R G O D Z Q
    plain       M E E T A T S U N G O D L A W N\end{verbatim}
            Decoding this gives us \code{Meet at Sun God Lawn.}
        \end{mdframed}
    \end{itemize}
\end{mdframed}

\begin{mdframed}
    (Exercise.) \emph{You are Eve. You have just intercepted the following message that Alice was trying to send to Bob: \code{Q TQDM IB QPWCAM}. You know that Alice used a Caesar cipher, but she didn't remove spaces before encrypting: she left the spaces in her original message as-is. What is the original message?}
    
    \begin{mdframed}
        \code{Q} itself could be a word; specifically, it could either be \code{A} or \code{I}. We can try to figure out what the message by guessing which word the first word could be. 
        \begin{itemize}
            \item If \code{Q} maps to \code{A}, then we have 
            \begin{center}
                \includegraphics[scale=0.6]{../assets/ceasar_4.png}
            \end{center}
            Partially decrypting the ciphertext gives us \code{A DANW}, but \code{DANW} is meaningless. Therefore, it cannot be \code{A}. 

            \item If \code{Q} maps to \code{I}, then we have
            \begin{center}
                \includegraphics[scale=0.6]{../assets/ceasar_5.png}
            \end{center}
            Decrypting this gives us: 
            \begin{verbatim}
                I LIVE AT IHOUSE\end{verbatim}
        \end{itemize}
        Therefore, the message is \code{I LIVE AT IHOUSE}. The shift was 8.
    \end{mdframed}
\end{mdframed}

\end{document}