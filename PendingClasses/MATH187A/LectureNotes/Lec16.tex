\documentclass[letterpaper]{article}
\usepackage[margin=1in]{geometry}
\usepackage[utf8]{inputenc}
\usepackage{textcomp}
\usepackage{amssymb}
\usepackage{natbib}
\usepackage{graphicx}
\usepackage{gensymb}
\usepackage{amsthm, amsmath, mathtools}
\usepackage[dvipsnames]{xcolor}
\usepackage{enumerate}
\usepackage{mdframed}
\usepackage[most]{tcolorbox}
\usepackage{csquotes}
% https://tex.stackexchange.com/questions/13506/how-to-continue-the-framed-text-box-on-multiple-pages

\tcbuselibrary{theorems}

\newcommand{\R}{\mathbb{R}}
\newcommand{\Z}{\mathbb{Z}}
\newcommand{\N}{\mathbb{N}}
\newcommand{\Q}{\mathbb{Q}}
\newcommand{\C}{\mathbb{C}}
\newcommand{\code}[1]{\texttt{#1}}
\newcommand{\mdiamond}{$\diamondsuit$}
\newcommand{\PowerSet}{\mathcal{P}}
\newcommand{\Mod}[1]{\ (\mathrm{mod}\ #1)}
\DeclareMathOperator{\lcm}{lcm}

%\newtheorem*{theorem}{Theorem}
%\newtheorem*{definition}{Definition}
%\newtheorem*{corollary}{Corollary}
%\newtheorem*{lemma}{Lemma}
\newtheorem*{proposition}{Proposition}


\newtcbtheorem[number within=section]{theorem}{Theorem}
{colback=green!5,colframe=green!35!black,fonttitle=\bfseries}{th}

\newtcbtheorem[number within=section]{definition}{Definition}
{colback=blue!5,colframe=blue!35!black,fonttitle=\bfseries}{def}

\newtcbtheorem[number within=section]{corollary}{Corollary}
{colback=yellow!5,colframe=yellow!35!black,fonttitle=\bfseries}{cor}

\newtcbtheorem[number within=section]{lemma}{Lemma}
{colback=red!5,colframe=red!35!black,fonttitle=\bfseries}{lem}

\newtcbtheorem[number within=section]{example}{Example}
{colback=white!5,colframe=white!35!black,fonttitle=\bfseries}{def}

\newtcbtheorem[number within=section]{note}{Important Note}{
        enhanced,
        sharp corners,
        attach boxed title to top left={
            xshift=-1mm,
            yshift=-5mm,
            yshifttext=-1mm
        },
        top=1.5em,
        colback=white,
        colframe=black,
        fonttitle=\bfseries,
        boxed title style={
            sharp corners,
            size=small,
            colback=red!75!black,
            colframe=red!75!black,
        } 
    }{impnote}
\usepackage[utf8]{inputenc}
\usepackage[english]{babel}
\usepackage{fancyhdr}
\usepackage{hyperref}
\usepackage{csquotes}

\DeclareMathOperator{\ord}{ord}

\pagestyle{fancy}
\fancyhf{}
\rhead{Math 187A}
\chead{Friday, March 03, 2023}
\lhead{Lecture 16}
\rfoot{\thepage}

\setlength{\parindent}{0pt}

\begin{document}

\section{Modern Cryptography}
(Continued from previous notes.)

\subsection{Interlude: Order}
Consider the following definition of order: 
\begin{definition}{Order}{3-3:1}
    Fix a positive integer $n$. If $a$ is an integer with $\gcd(a, n) = 1$, the order of $a$ mod $n$, denoted $\ord_{n}(a)$, is the smallest positive integer $k$ such that $a^k \equiv 1 \Mod{n}$.
\end{definition}
For example, suppose $n = 7$ and $a = 2$. We then compute 
\[2^2 = 4 \Mod{7}.\]
\[2^3 = 8 \equiv 1 \Mod{7}.\]
Here, 3 is the smallest positive exponent such that raising 2 to the power gives us something congruent 1 mod 7, which means $\ord_{7}(2) = 3$.

\subsubsection{Order Lemmas}
Note that $\phi(7) = 6$ and $\ord_{7}(2) = 3$ happens to be a divisor of 6. This is no coincidence. 

\begin{lemma}{First Order Lemma}{3-3:2}
    Fix a positive integer $n$ and an integer $a$ with $\gcd(a, n) = 1$. If $m$ is an integer with $a^m \equiv 1 \Mod{n}$, then $\ord_{n}(a)$ divides $m$. In particular, $\ord_{n}(a)$ divides $\phi(n)$. 
\end{lemma}
The First Order Lemma makes it easier to compute the order of an element. Suppose, for example, we are interested in $n = 7$ and $a = 3$. The lemma guarantees that $\ord_{7}(3)$ must be a divisor of $\phi(7) = 6$, so it can only be 1, 2, 3, or 6. We check
\begin{equation*}
    \begin{aligned}
        &3^1 \not\equiv 1 \Mod{7} \\ 
        &3^2 = 9 \equiv 2 \not\equiv 1 \Mod{7} \\ 
        &3^3 = 27 \equiv 6 \not\equiv 1 \Mod{7} \\ 
        &3^6 = 729 \equiv 1 \Mod{7}.
    \end{aligned}
\end{equation*}
So, $\ord_{7}(3)$ cannot be 1, 2, or 3 and thus must be 6.

\begin{mdframed}
    (Exercise.) Calculate the following orders. 
    \begin{enumerate}[(a)]
        \item $\ord_{5}(2)$
        \begin{mdframed}
            We need to find the smallest integer $k$ such that $2^k \equiv 1 \Mod{5}$. We find 
            \[2^1 = 2 \not\equiv 1 \Mod{5}\]
            \[2^2 = 4 \not\equiv 1 \Mod{5}\]
            \[2^3 = 8 \equiv 3 \not\equiv 1 \Mod{5}\]
            \[2^4 = 16 \equiv 1 \Mod{5},\]
            so $\ord_{5}(2) = 4$. 
        \end{mdframed}
        \item $\ord_{9}(4)$
        \begin{mdframed}
            We need to find the smallest integer $k$ such that $4^k \equiv 1 \Mod{9}$. We find 
            \[4^1 = 4 \not\equiv 1 \Mod{9}\]
            \[4^2 = 16 \not\equiv 1 \Mod{9}\]
            \[4^3 = 64 \equiv 1 \Mod{9},\]
            so $\ord_{9}(4) = 3$.
        \end{mdframed}
        \item $\ord_{10}(3)$
        \begin{mdframed}
            We need to find the smallest integer $k$ such that $3^k \equiv 1 \Mod{10}$. We find 
            \[3^1 = 3 \not\equiv 1 \Mod{10}\]
            \[3^2 = 9 \not\equiv 1 \Mod{10}\]
            \[3^3 = 27 \not\equiv 1 \Mod{10}\]
            \[3^4 = 81 \not\equiv 1 \Mod{10},\]
            so $\ord_{10}(3) = 4$.     
        \end{mdframed}

        \item $\ord_{11}(7)$
        \begin{mdframed}
            We note that \[\phi(11) = 11\prod_{\substack{p | 11 \\ p \text{ prime}}}\left(1 - \frac{1}{p}\right) = 11\left(1 - \frac{1}{11}\right) = 11\left(\frac{10}{11}\right) = 10.\] By the First Order Lemma, we know that $\ord_{11}(7)$ divides $\phi(11)$. So, $\ord_{11}(7)$ can only be 1, 2, 5, or 10. Let's try the different values:
            \[7^1 = 7 \not\equiv 1 \Mod{11}\]
            \[7^2 = 49 \not\equiv 1 \Mod{11}\]
            \[7^5 = 7^4 7 = (7^2)^2 7 = 49^2 7 \equiv 5^2 7 = 25 \cdot 7 \equiv 3 \cdot 7 = 21 \not\equiv 1 \Mod{11}\]
            \[7^{10} = (7^2)^5 = 49^5 \equiv 5^5 = 5^4 5 = (5^2)^2 5 = 25^2 5 \equiv 3^2 5 = 45 \equiv 1 \Mod{11},\]
            so $\ord_{11}(7) = 10$. 
        \end{mdframed}
        \item $\ord_{13}(1)$
        \begin{mdframed}
            As usual, we find the smallest integer $k$ such that $1^k \equiv 1 \Mod{13}$. Conveniently, we find that $k = 1$ and so $\ord_{13}(1) = 1$. 
        \end{mdframed}
    \end{enumerate}
\end{mdframed}

\begin{lemma}{Second Order Lemma}{3-3:3}
    Fix a positive integer $n$ and an integer $a$ with $\gcd(a, n) = 1$ and let $k = \ord_{n}(a)$. Then, $a^i \equiv a^k \Mod{n}$ if and only if $i \equiv j \Mod{k}$. In particular, the numbers $a^0, a^1, a^2, a^3, \hdots, a^{k - 1}$ are all incongruent mod $n$. 
\end{lemma}

\subsubsection{Primitive Roots and Discrete Logarithms}
The First Order Lemma tells us that $\phi(n)$ is the largest possible order mod $n$ that any integer could have, since the order must always be a divisor of $\phi(n)$. The situation when this maximum is achieved gets a special name.

\begin{definition}{Primitive Root}{3-3:4}
    Fix an integer $n \geq 2$. An integer $g$ with $\gcd(g, n) = 1$ and $\ord_{n}(g) = \phi(n)$ is called a primitive root mod $n$. 
\end{definition}
For example, we saw above that $\ord_{7}(3) = 6 = \phi(7)$, so 3 is a primitive root mod 7. The Second Order Lemma tells us that $3^0, 3^1, 3^2, 3^3, 3^4, 3^5$ are all incongruent mod 7, but there are only 6 nonzero congruence classes mod 7, so the fact that all the nonzero congruence classes mod 7 must be represented among the integers $3^0, 3^1, 3^2, 3^3, 3^4, 3^5$. Let's check this explicitly. 
\[3^0 \equiv 1 \Mod{7}\]
\[3^1 \equiv 3 \Mod{7}\]
\[3^2 \equiv 2 \Mod{7}\]
\[3^3 \equiv 6 \Mod{7}\]
\[3^4 \equiv 4 \Mod{7}\]
\[3^5 \equiv 5 \Mod{7}\]
All of the nonzero remainders mod 7 appear in this list. This generalizes. 
\begin{lemma}{Existence of Discrete Logarithms}{3-3:5}
    Fix an integer $n \geq 2$ and suppose $g$ is a primitive root mod $n$. If $\gcd(a, n) = 1$, then there exists a unique $k$ such that $0 \leq k \leq \phi(n)$ and $g^k \equiv a \Mod{n}$. This integer $k$ is called the \emph{discrete log base} $g$ of $a$ mod $n$, and is denoted $\log_{g}(a \Mod{n})$.
\end{lemma}
So, our calculations above show that the discrete log base 3 of 6 mod 7 is 3, since $3^3 \equiv 6 \Mod{7}$. 

\begin{mdframed}
    (Exercise.) For each of the following, determine whether or not the proposed value of $g$ is actually a primitive root mod $n$. 
    \begin{enumerate}[(a)]
        \item $n = 11, g = 2$
        \begin{mdframed}
            Recall that $\phi(11) = 10$. By the First Order Lemma, $\ord_{11}(2)$ must either be 1, 2, 5, or 10. So, 
            \[2^1 = 2 \not\equiv 1 \Mod{11},\]
            \[2^2 = 4 \not\equiv 1 \Mod{11},\]
            \[2^5 = 32 \equiv 10 \not\equiv 1 \Mod{11},\]
            \[2^{10} = (2^5)^2 = 32^2 \equiv 10^2 = 100 \equiv 1 \Mod{11}.\]
            So, in particular, we find that $\ord_{11}(2) = 10$. By the definition of the primitive root, since $\ord_{11}(2) = 10 = \phi(11)$, $g = 2$ is a primitive root. 
        \end{mdframed}
        \item $n = 11, g = 3$
        \begin{mdframed}
            Recall that $\phi(11) = 10$. By the First Order Lemma, $\ord_{11}(3)$ must either be 1, 2, 5, or 10. So, 
            \[3^1 = 3 \not\equiv 1 \Mod{11},\]
            \[3^2 = 9 \not\equiv 1 \Mod{11},\]
            \[3^5 = 3^3 \cdot 3^2 = 27 \cdot 3^2 \equiv 5 \cdot 9 = 45 \equiv 1 \Mod{11}.\]
            So, $\ord_{11}(3) = 5$, but because $\ord_{11}(3) \neq \phi(11)$, $g = 3$ is not a primitive root.
        \end{mdframed}
        \item $n = 11, g = 4$
        \begin{mdframed}
            Recall that $\phi(11) = 10$. By the First Order Lemma, $\ord_{11}(4)$ must either be 1, 2, 5, or 10. So, 
            \[4^1 = 4 \not\equiv 1 \Mod{11},\]
            \[4^2 = 16 \equiv 5 \not\equiv 1 \Mod{11},\]
            \[4^5 = (4^2)^2 4 = 16^2 4 \equiv 5^2 4 = 25 \cdot 4 = 100 \equiv 1 \Mod{11}.\]
            So, $\ord_{11}(4) = 5$, but because $\ord_{11}(4) \neq \phi(11)$, $g = 4$ is not primitive root.
        \end{mdframed}
    \end{enumerate}
\end{mdframed}

\begin{mdframed}
    (Exercise.) For each of the following values of $n$, find \emph{all} of the primitive roots mod $n$. 
    \begin{itemize}
        \item $n = 5$
        \begin{mdframed}
            We find that \[\phi(5) = 5\prod_{\substack{p | 5 \\ p \text{ prime}}} \left(1 - \frac{1}{p}\right) = 5\left(1 - \frac{1}{5}\right) = 5\frac{4}{5} = 4.\]
            By the definition of the Primitive Root (\ref{def:3-3:4}), we know that an integer $g$ with $\gcd(g, 5) = 1$ and $\ord_{5}(g) = \phi(5) = 4$ is called a primitive root.
            
            \bigskip 
            
            Let's consider all $1 \leq g \leq 4$ (since, for $g > 5$, we can mod $g$ such that it's between $0 \leq g \leq 4$; also, for $g = 0$, $g^n = 0$ and $\gcd(0, 5) = 5$.)

            \begin{center}
                \begin{tabular}{|c|c c c c|}
                    \hline 
                    $g$ & $g^1 \Mod{5}$ & $g^2 \Mod{5}$ & $g^3 \Mod{5}$ & $g^4 \Mod{5}$ \\ 
                    \hline 
                    1   & 1             &               &               &               \\ 
                    2   & 2             & 4             & 3             & 1             \\ 
                    3   & 3             & 4             & 2             & 1             \\ 
                    4   & 4             & 1             &               &               \\ 
                    \hline 
                \end{tabular}
            \end{center}
            So, in particular, the order of  
            \begin{itemize}
                \item $g = 1$ is 1, 
                \item $g = 2$ is 4, 
                \item $g = 3$ is 4,
                \item $g = 4$ is 1.
            \end{itemize}
            Because $\phi(5) = 4$ and $\ord_{5}(2) = \ord_{5}(3) = 4$, it follows that 2 and 3 are the primitive roots. 
        \end{mdframed}
        \item $n = 7$
        \begin{mdframed}
            We know that $\phi(7) = 6$. By the definition of the Primitive Root (\ref{def:3-3:4}), we know that an integer $g$ with $\gcd(g, 7) = 1$ and $\ord_{7}(g) = \phi(7) = 6$ is called a primitive root. 
            
            \bigskip 

            Let's consider all $1 \leq g \leq 6$.
            \begin{center}
                \begin{tabular}{|c|c c c c c c|}
                    \hline 
                    $g$ & $g^1 \Mod{7}$ & $g^2 \Mod{7}$ & $g^3 \Mod{7}$ & $g^4 \Mod{7}$ & $g^5 \Mod{7}$ & $g^6 \Mod{7}$ \\ 
                    \hline 
                    1   & 1           &                 &               &                &              &               \\ 
                    2   & 2           & 4               & 1             &                &              &               \\ 
                    3   & 3           & 2               & 6             & 4              & 5            & 1             \\ 
                    4   & 4           & 2               & 1             &                &              &               \\ 
                    5   & 5           & 4               & 6             & 2              & 3            & 1             \\ 
                    6   & 6           & 1               &               &                &              &               \\ 
                    \hline 
                \end{tabular}
            \end{center}

            So, in particular, the order of  
            \begin{itemize}
                \item $g = 1$ is 1, 
                \item $g = 2$ is 3, 
                \item $g = 3$ is 6,
                \item $g = 4$ is 3, 
                \item $g = 5$ is 6,
                \item $g = 6$ is 2.
            \end{itemize}
            Because $\phi(7) = 6$ and $\ord_{7}(3) = \ord_{7}(5) = 6$, it follows that 3 and 5 are the primitive roots. 
        \end{mdframed}
        \item $n = 11$
        \begin{mdframed}
            We know that $\phi(11) = 10$. By the definition of the Primitive Root (\ref{def:3-3:4}), we know that an integer $g$ with $\gcd(g, 11) = 1$ and $\ord_{11}(g) = \phi(11) = 10$ is called a primitive root.
            
            \bigskip 

            Let's consider all $1 \leq g \leq 10$ (note that the columns $g^x$ for $x = 1, 2, \hdots$ are mod 11.) 
            \begin{center}
                \begin{tabular}{|c|c c c c c c c c c c|}
                    \hline 
                    $g$ & $g^1$ & $g^2$ & $g^3$ & $g^4$ & $g^5$ & $g^6$ & $g^7$ & $g^8$ & $g^9$ & $g^{10}$ \\ 
                    \hline 
                    1   & 1     &       &       &       &       &       &       &       &       &          \\ 
                    2   & 2     & 4     & 8     & 5     & 10    & 9     & 7     & 3     & 6     & 1        \\ 
                    3   & 3     & 9     & 5     & 4     & 1     &       &       &       &       &          \\ 
                    4   & 4     & 5     & 9     & 3     & 1     &       &       &       &       &          \\ 
                    5   & 5     & 3     & 4     & 9     & 1     &       &       &       &       &          \\ 
                    6   & 6     & 3     & 7     & 9     & 10    & 5     & 8     & 4     & 2     & 1        \\ 
                    7   & 7     & 5     & 2     & 3     & 10    & 4     & 6     & 9     & 8     & 1        \\ 
                    8   & 8     & 9     & 6     & 4     & 10    & 3     & 2     & 5     & 7     & 1        \\ 
                    9   & 9     & 4     & 3     & 5     & 1     &       &       &       &       &          \\ 
                    10  & 10    & 1     &       &       &       &       &       &       &       &          \\ 
                    \hline 
                \end{tabular}
            \end{center}
            So, in particular, because $\phi(11) = 10$ and $\ord_{11}(2) = \ord_{11}(6) = \ord_{11}(7) = \ord_{11}(8) = 10$, it follows that 2, 6, 7, 8 are the primitive roots.
        \end{mdframed}
    \end{itemize}
\end{mdframed}

\begin{mdframed}
    (Exercise.) For each of the following, find the discrete log base $g$ of $a$ mod $n$. 
    \begin{enumerate}[(a)]
        \item $n = 7, g = 3, a = 5$
        \begin{mdframed}
            We know that $\phi(7) = 6$, so by lemma (\ref{lem:3-3:5}) there exists a unique integer $k$ such that $0 \leq k \leq 6$ and $3^k \equiv 5 \Mod{7}$. So, 
            \[3^0 = 1 \not\equiv 5 \Mod{7},\]
            \[3^1 = 3 \not\equiv 5 \Mod{7},\]
            \[3^2 = 9 \equiv 2 \not\equiv 5 \Mod{7},\]
            \[3^3 = 27 \equiv 6 \not\equiv 5 \Mod{7},\]
            \[3^4 = 81 \equiv 4 \not\equiv 5 \Mod{7},\]
            \[3^5 = 3^4 3 = 9^2 3 = 81(3) \equiv 4(3) = 12 \equiv 5 \Mod{7}.\]
            So, in particular, $k = 5$. 
        \end{mdframed}
        \item $n = 5, g = 2, a = 4$
        \begin{mdframed}
            We know that $\phi(5) = 4$, so by lemma (\ref{lem:3-3:5}) there exists a unique integer $k$ such that $0 \leq k \leq 4$ and $2^k \equiv 4 \Mod{5}$. So, 
            \[2^0 = 1 \not\equiv 4 \Mod{5},\]
            \[2^1 = 2 \not\equiv 4 \Mod{5},\]
            \[2^2 = 4 \Mod{5}.\]
            By said lemma, we have $k = 2$. 
        \end{mdframed}
        \item $n = 11, g = 2, a = 3$
        \begin{mdframed}
            We know that $\phi(11) = 10$, so by lemma (\ref{lem:3-3:5}) there exists a unique integer $k$ such that $0 \leq k \leq 10$ and $2^k \equiv 3 \Mod{11}$. Additionally, by lemma (\ref{lem:3-3:3}) we know that $2^0, 2^1, \hdots, 2^8, 2^9$ are all incongruent mod 11, so we only care about $0 \leq k \leq 9$. So, 
            \[2^0 = 1 \not\equiv 3 \Mod{11},\]
            \[2^1 = 2 \not\equiv 3 \Mod{11},\]
            \[2^2 = 4 \not\equiv 3 \Mod{11},\]
            \[2^3 = 8 \not\equiv 3 \Mod{11},\]
            \[2^4 = 16 \equiv 5 \not\equiv 3 \Mod{11},\]
            \[2^5 = 2^4 2 \equiv 5 \cdot 2 = 10 \not\equiv 3 \Mod{11},\]
            \[2^6 = 2^5 2 \equiv 10 \cdot 2 = 20 \equiv 9 \not\equiv 3 \Mod{11},\]
            \[2^7 = 2^6 2 \equiv 9 \cdot 2 = 18 \equiv 7 \not\equiv 3 \Mod{11},\]
            \[2^8 = 2^7 2 \equiv 7 \cdot 2 = 14 \equiv 3 \Mod{11}.\]
            So, by said former lemma, $k = 8$. 
        \end{mdframed}
    \end{enumerate}
\end{mdframed}

\subsubsection{Existence of Primitive Roots}
We haven't yet shown that primitive roots always exist, and in fact, it is not true that primitive roots always exist. Here is the statement:
\begin{theorem}{Primitive Root Theorem}{3-3:6}
    Fix an integer $n \geq 2$. Then, there exists a primitive root mod $n$ if and only if $n = 2, 4, p^k, 2p^k$ for an odd prime $p$ and a positive integer $k$. In particular, there always exists a primitive root mod $p$ (a prime). 
\end{theorem}

\begin{mdframed}
    (Exercise.) Use the Primitive Root Theorem to find the 5 smallest integers $n \geq 2$ such that there does \emph{not} exist a primitive root mod $n$.

    \begin{mdframed}
        Referring to theorem (\ref{th:3-3:6}), we know that every prime has a primitive root. In other words, we know that 
        \begin{itemize}
            \item 2, 4 are special cases. 
            \item 3, 5, 7, 11, 13, 17, 19, etc. are all primes.
            \item 6, 10, 14, 22, 26, etc. all have primitive roots (these are just primes multiplied by 2, i.e., $2p^1$, but we omitted 2 since we only care about odd primes). 
            \item 9, 25, 49, 121, etc. all have primitive roots (these are just the primes multiplied by themselves, i.e., $p^2$).
            \item 18, 50, 98, etc. all have primitive roots (these are just $2p^2$, but notice how we omitted 8 because powers only apply to odd primes). 
        \end{itemize} 
        So, in particular, 8, 12, 15, 16, 20.
    \end{mdframed}
\end{mdframed}





\end{document}