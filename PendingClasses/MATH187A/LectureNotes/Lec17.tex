\documentclass[letterpaper]{article}
\usepackage[margin=1in]{geometry}
\usepackage[utf8]{inputenc}
\usepackage{textcomp}
\usepackage{amssymb}
\usepackage{natbib}
\usepackage{graphicx}
\usepackage{gensymb}
\usepackage{amsthm, amsmath, mathtools}
\usepackage[dvipsnames]{xcolor}
\usepackage{enumerate}
\usepackage{mdframed}
\usepackage[most]{tcolorbox}
\usepackage{csquotes}
% https://tex.stackexchange.com/questions/13506/how-to-continue-the-framed-text-box-on-multiple-pages

\tcbuselibrary{theorems}

\newcommand{\R}{\mathbb{R}}
\newcommand{\Z}{\mathbb{Z}}
\newcommand{\N}{\mathbb{N}}
\newcommand{\Q}{\mathbb{Q}}
\newcommand{\C}{\mathbb{C}}
\newcommand{\code}[1]{\texttt{#1}}
\newcommand{\mdiamond}{$\diamondsuit$}
\newcommand{\PowerSet}{\mathcal{P}}
\newcommand{\Mod}[1]{\ (\mathrm{mod}\ #1)}
\DeclareMathOperator{\lcm}{lcm}

%\newtheorem*{theorem}{Theorem}
%\newtheorem*{definition}{Definition}
%\newtheorem*{corollary}{Corollary}
%\newtheorem*{lemma}{Lemma}
\newtheorem*{proposition}{Proposition}


\newtcbtheorem[number within=section]{theorem}{Theorem}
{colback=green!5,colframe=green!35!black,fonttitle=\bfseries}{th}

\newtcbtheorem[number within=section]{definition}{Definition}
{colback=blue!5,colframe=blue!35!black,fonttitle=\bfseries}{def}

\newtcbtheorem[number within=section]{corollary}{Corollary}
{colback=yellow!5,colframe=yellow!35!black,fonttitle=\bfseries}{cor}

\newtcbtheorem[number within=section]{lemma}{Lemma}
{colback=red!5,colframe=red!35!black,fonttitle=\bfseries}{lem}

\newtcbtheorem[number within=section]{example}{Example}
{colback=white!5,colframe=white!35!black,fonttitle=\bfseries}{def}

\newtcbtheorem[number within=section]{note}{Important Note}{
        enhanced,
        sharp corners,
        attach boxed title to top left={
            xshift=-1mm,
            yshift=-5mm,
            yshifttext=-1mm
        },
        top=1.5em,
        colback=white,
        colframe=black,
        fonttitle=\bfseries,
        boxed title style={
            sharp corners,
            size=small,
            colback=red!75!black,
            colframe=red!75!black,
        } 
    }{impnote}
\usepackage[utf8]{inputenc}
\usepackage[english]{babel}
\usepackage{fancyhdr}
\usepackage{hyperref}
\usepackage{csquotes}

\DeclareMathOperator{\ord}{ord}

\pagestyle{fancy}
\fancyhf{}
\rhead{Math 187A}
\chead{Wednesday, March 08, 2023}
\lhead{Lecture 17}
\rfoot{\thepage}

\setlength{\parindent}{0pt}

\begin{document}

\section{Modern Cryptography}
(Continued from previous notes.)

\subsection{Elgamal Cryptosystem}
The Elgamal cryptosystem is a public-key cryptosystem like RSA, named after the Egyptian cryptographer Taher Elgamal.

\subsubsection{How Elgamal Works}
The process begins with Bob choosing a public key. He picks a prime number $p$ and a primitive root $g$ of $p$. He chooses a random integer $x$ with $0 \leq x < p - 1$. This is his \underline{private} key. He then computes $h = g^x \Mod{p}$ and his \underline{public} key is the triple $(p, g, h)$. 

\bigskip 

Suppose Alice wants to send Bob a message. She first encodes her message as an integer $m$ between $0$ and $p - 1$ (e.g., by using the same ``base 26'' strategy that we employed for RSA.) Then, she chooses a random integer $y$ between $0$ and $p - 1$ called the \textbf{ephemeral key}. Alice will have to choose a different ephemeral key for every message she sends, but Bob does not have to know the value of this key beforehand. Alice computes $s = h^y \Mod{p}$, $c_1 = g^y \Mod{p}$, and $c_2 = ms \Mod{p}$. Note that she can compute $s$ and $c_1$ quickly using binary exponentation. The pair $(c_1, c_2)$ is the ciphertext that she sends to Bob. 

\bigskip 

To decrypt the ciphertext $(c_1, c_2)$, Bob first computes $c_1^x \Mod{p}$. Bob can do this quickly with binary exponentation. Notice that 
\[c_1^x \equiv (g^y)^x = g^{xy} = (g^x)^y \equiv h^y \equiv s \Mod{p}.\]
In other words, Bob found the same value of $s$ that Alice had, even though he does not know the value of the ephemeral key $y$. He then computes an inverse mod $p$ of $c_1^x$ using the extended Euclidian algorithm. From there, he computes 
\[c_2 (c_1^x)^{-1} \equiv c_2 s^{-1} \equiv (ms)s^{-1} \equiv m \cdot 1 = m \Mod{p},\]
thus allowing him to recover Alice's message $m$.

\begin{mdframed}
    (Example.) Suppose Bob picks the prime $p = 4115549$ and $g = 2$ is his primitive root. He then picks a random integer $x = 2634326$. From there, he can compute \[h = g^x \Mod{p},\] getting $h = 1149114$. Thus, the triple $(4115549, 2, 1149114)$ is his public key. $x = 2634326$ must be kept secret. 

    \bigskip 

    Suppose Alice wants to send Bob the message \code{Hi Bob}. She begins by converting this message to the integer $m = 3340481$. Then, she chooses an ephemeral key $y = 2775147$. She keeps this value of $y$ secret, and then computes \[s = h^y \Mod{p} = 962840\] using binary exponentation. Alice also keeps $s$ a secret. She also computes \[c_1 = g^y \Mod{p} = 621674\] using binary exponentation. Finally, she computes \[c_2 = ms \Mod{p} = 1911501.\] From there, $(c_1, c_2) = (621674, 1911501)$ is the ciphertext she sends to Bob.

    \bigskip 

    Bob receives the pair $(c_1, c_2) = (621674, 1911501)$. He computes 
    \[c_1^x \Mod{p} = 962840\]
    using binary exponentation. This is the same value that Alice found for $s$. Then, he computes an inverse mod $p$ and finds $s^{-1} \equiv 2329074 \Mod{p} = 4115549$. From there, he computes \[c_2 s^{-1} \Mod{p} = 3340481,\] and then converts this message back to the text \code{HIBOB}. 
\end{mdframed}

\begin{mdframed}
    (Exercise.) Suppose Bob picks the prime $p = 29$ and the primitive root $g = 2$. 
    \begin{enumerate}[(a)]
        \item Suppose Bob picks $x = 3$. What is his public key? 
        \begin{mdframed}
            We compute \[h = g^x \Mod{p} = 2^3 \Mod{29} = 8 \Mod{29}.\] Therefore, Bob's public key is the triple $(p, g, h) = (29, 2, 8)$. 
        \end{mdframed}
        \item Suppose Alice wants to send Bob the plaintext integer $m = 7$. What is the corresponding ciphertext pair? 
        \begin{mdframed}
            Suppose Alice selects ephemeral key $y = 3$. Then, Alice can compute \[s = h^y \Mod{p} = 8^3 \Mod{29} = 19 \Mod{29},\] \[c_1 = g^y \Mod{p} = 2^{3} \Mod{29} = 8 \Mod{29},\] \[c_2 = ms \Mod{p} = 7(19) \Mod{29} = 17 \Mod{29}.\] The pair, $(8, 17)$, is the ciphertext pair. 
        \end{mdframed}
        \item Suppose Bob receives the ciphertext pair $(3, 9)$ from Alice. What is the plaintext integer $m$? 
        \begin{mdframed}
            Bob computes \[c_1^x \Mod{p} = 3^3 \Mod{29} = 27 \Mod{29}.\] This value is $s$; that is, $s = 27 \Mod{29}$. From there, we want to find the inverse of $c_1^x = 27$ mod 29. To do this, let's find Bezout's coefficient; 
            \[29 = 27q + r \implies 29 = 27(1) + 2 \implies 2 = 29 + 27(-1)\]
            \[27 = 2q + r \implies 27 = 2(13) + 1 \implies 1 = 27 + 2(-13)\]
            \[2 = 1q + r \implies 2 = 1(2) + 0.\]
            From this, $\gcd(27, 29) = 1$ so we can find the Bezout coefficient.
            \begin{equation*}
                \begin{aligned}
                    1 &= 27 + 2(-13) \\ 
                        &= 27 + (29 + 27(-1))(-13) \\
                        &= 27 + 29(-13) + 27(-1)(-13) \\ 
                        &= 27 + 29(-13) + 27(13) \\ 
                        &= 27(14) + 29(-13).
                \end{aligned}
            \end{equation*}
            From this, it follows that the Bezout coefficients are $x = 14$ and $y = -13$; more importantly, we find that the inverse of $c_1^x = 27$ mod 29 is $x = 14$. So, 
            \[c_2 s^{-1} \Mod{p} = 9(14) \Mod{29} = 10 \Mod{29},\]
            so $m = 10$. 
        \end{mdframed}
    \end{enumerate}
\end{mdframed}

\begin{mdframed}
    (Exercise.) If Bob wants to be able to receive messages with $r = 10$ characters, how large must he choose $p$ to be? What if $r = 100$? $r = 1000$? 

    \begin{mdframed}
        Assuming we choose to use the ``base 26'' strategy for encoding the message, the largest possible 10 character message would be \code{ZZZZZZZZZZ}. Here, \code{Z} corresponds to the number 25, so we can encode this message as follows: \[\sum_{i = 0}^{9} 25 \cdot 26^i = 26^{10} - 1.\] Recall that the integer encoding of the message $m$ must be between 0 and $p - 1$, i.e., $0 \leq m \leq p - 1$. So, $p > 26^{10} - 1 \implies p -1 > 26^{10} - 2$. The same reasoning applies for $r = 100$ and $r = 1000$. 
    \end{mdframed}
\end{mdframed}

\begin{mdframed}
    (Exercise.) Bob's Eigamal public key has $p = 29$, $g = 3$, and $h = 27$. Alice wants to send Bob the message \code{C}. She generates an ephemeral key $y = 10$. What is the ciphertext that she sends Bob? 

    \begin{mdframed}
        Encoding \code{C} gives us $m = 2$, the base 26 representation. Now, note that 
        \[s = h^y \Mod{p} = 27^{10} \Mod{29} = (-2)^{10} \Mod{29} = 9 \Mod{29},\]
        \[c_1 = g^y \Mod{p} = 3^{10} \Mod{29} = 5 \Mod{29},\]
        \[c_2 = ms \Mod{p} = 2(9) \Mod{29} = 18 \Mod{29}.\]
        Therefore, Alice sends Bob $(c_1, c_2) = (5, 18)$.
    \end{mdframed}
\end{mdframed}

\subsubsection{Why Elgamal is Probably Secure (For Now...)}
There are at least two strategies Eve might employ to recover the plaintext $m$ from the ciphertext $(c_1, c_2)$. 
\begin{itemize}
    \item Eve can try to find Bob's decryption key $x$ so she can follow Bob's decryption strategy but, in order to do this, she needs to find the discrete log base $g$ of $h$ mod $p$. 
    \item Even can try to find Alice's ephemeral key $y$, but then she needs to find the discrete log base $h$ of $c_1$ mod $p$. 
\end{itemize}
In any case, Eve needs to find a discrete log base $g$ mod $p$. So, the security of the Elgamal cryptosystem relies on the presumed difficulty of the following: 
\begin{mdframed}[nobreak=true]
    (Discrete Logarithm Problem.) Suppose you are given a prime $p$, a primitive root $g$ mod $p$, and an integer $a$ not divisible by $p$. Find the discrete log base $g$ of $a$ mod $n$. In other words, find the unique integer $k$ such that $0 \leq k \leq p - 1$ such that $g^k \equiv a \Mod{p}$. 
\end{mdframed}
As $p$ gets larger, the problem becomes difficult for classical computers. The naive method to solving this problem would be to try all possible values of $k$ from 1 to $p - 1$, but this is linear in $p$ and exponential in the number of digits of $p$. Although there are faster algorithms out there, they are not faster by much\footnote{There are no known algorithm that accomplishes this task that is polynomial in the number of digits of $p$.}.


\subsection{Diffie-Hellman Key Exchange}
The Diffie-Hellman key exchange is \emph{not} quite a cryptosystem for exchanging messages, but rather it is a protocol that allows Alice and Bob to share a secret, but neither has full control over the content of the shared secret. The shared secret can be used as the key for a symmetric key cipher like a one-time pad. 

\bigskip 

The procedure is as follows:
\begin{itemize}
    \item Alice and Bob publicly agree to fix a prime $p$ and a primitive root $g$ mod $p$. 
    \item Alice then chooses a secret integer $0 \leq a < p - 1$ and sends Bob $x = g^a \Mod{p}$. She can compute this value quickly using binary exponentation. 
    \item Bob similarly chooses a secret integer $0 \leq b < p - 1$ and sends Alice $y = g^b \Mod{p}$. 
    \item Alice computes $s = y^a \Mod{p}$ and Bob computes $s = x^b \Mod{p}$.  
\end{itemize}
The two values of $s$ that Alice and Bob computes are the same, because
\[y^a \equiv (g^b)^a = g^{ab} = (g^a)^b \equiv x^b \Mod{p}.\]
Thus, Alice and Bob now share a secret, $s$. Neither of them have full control over the shared secret, so this cannot be regarded as Alice or Bob sending a message to the other.
\begin{mdframed}
    (Exercise.) Suppose Alice and Bob agree to use $p = 11$ and $g = 2$. Alice chooses the integer $a = 3$. She receives the integer $y = 5$ from Bob. What is her shared secret $s$ with Bob? 

    \begin{mdframed}
        The shared secret is \[s = y^a \Mod{p} = 5^3 \Mod{11} = 4.\]
    \end{mdframed}
\end{mdframed}

\begin{mdframed}
    (Exercise.) Alice and Bob agree to perform a Diffie-Hellman key exchange using $p = 31$ and $g = 3$. 
    
    \begin{enumerate}[(a)]
        \item Alice chooses the secret integer $a = 11$. What is the integer $x$ that she sends to Bob?
        \begin{mdframed}
            We know that 
            \[x = g^a \Mod{p},\]
            so 
            \[x = 3^{11} \Mod{31} = 13 \Mod{31}.\] 
        \end{mdframed}

        \item Using $a = 11$, Alice receives the integer $y = 2$ from Bob. What is her shared secret with Bob? 
        \begin{mdframed}
            We know that 
            \[s = y^a \Mod{p},\]
            so 
            \[s = 2^{11} \Mod{31} = 2 \Mod{31}.\]
        \end{mdframed}

        \item Eve sees Alice send Bob the integer $x = 9$ and Bob send Alice the integer $y = 27$. What is Alice and Bob's shared secret? 
        \begin{mdframed}
            We know that 
            \[x = g^a \Mod{p} = 3^a \Mod{31}.\]
            Here, $a = 2$. Note that, in general, Eve needs to try values of $a = 0, 1, \hdots, 30$ until she finds 9. With $a = 2$, we know that 
            \[s = y^a \Mod{p} = 27^{2} \Mod{31} = 16 \Mod{31}.\]
        \end{mdframed}
    \end{enumerate}
\end{mdframed}


\end{document}