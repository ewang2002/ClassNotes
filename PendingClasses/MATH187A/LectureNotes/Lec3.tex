\documentclass[letterpaper]{article}
\usepackage[margin=1in]{geometry}
\usepackage[utf8]{inputenc}
\usepackage{textcomp}
\usepackage{amssymb}
\usepackage{natbib}
\usepackage{graphicx}
\usepackage{gensymb}
\usepackage{amsthm, amsmath, mathtools}
\usepackage[dvipsnames]{xcolor}
\usepackage{enumerate}
\usepackage{mdframed}
\usepackage[most]{tcolorbox}
\usepackage{csquotes}
% https://tex.stackexchange.com/questions/13506/how-to-continue-the-framed-text-box-on-multiple-pages

\tcbuselibrary{theorems}

\newcommand{\R}{\mathbb{R}}
\newcommand{\Z}{\mathbb{Z}}
\newcommand{\N}{\mathbb{N}}
\newcommand{\Q}{\mathbb{Q}}
\newcommand{\C}{\mathbb{C}}
\newcommand{\code}[1]{\texttt{#1}}
\newcommand{\mdiamond}{$\diamondsuit$}
\newcommand{\PowerSet}{\mathcal{P}}
\newcommand{\Mod}[1]{\ (\mathrm{mod}\ #1)}
\DeclareMathOperator{\lcm}{lcm}

%\newtheorem*{theorem}{Theorem}
%\newtheorem*{definition}{Definition}
%\newtheorem*{corollary}{Corollary}
%\newtheorem*{lemma}{Lemma}
\newtheorem*{proposition}{Proposition}


\newtcbtheorem[number within=section]{theorem}{Theorem}
{colback=green!5,colframe=green!35!black,fonttitle=\bfseries}{th}

\newtcbtheorem[number within=section]{definition}{Definition}
{colback=blue!5,colframe=blue!35!black,fonttitle=\bfseries}{def}

\newtcbtheorem[number within=section]{corollary}{Corollary}
{colback=yellow!5,colframe=yellow!35!black,fonttitle=\bfseries}{cor}

\newtcbtheorem[number within=section]{lemma}{Lemma}
{colback=red!5,colframe=red!35!black,fonttitle=\bfseries}{lem}

\newtcbtheorem[number within=section]{example}{Example}
{colback=white!5,colframe=white!35!black,fonttitle=\bfseries}{def}

\newtcbtheorem[number within=section]{note}{Important Note}{
        enhanced,
        sharp corners,
        attach boxed title to top left={
            xshift=-1mm,
            yshift=-5mm,
            yshifttext=-1mm
        },
        top=1.5em,
        colback=white,
        colframe=black,
        fonttitle=\bfseries,
        boxed title style={
            sharp corners,
            size=small,
            colback=red!75!black,
            colframe=red!75!black,
        } 
    }{impnote}
\usepackage[utf8]{inputenc}
\usepackage[english]{babel}
\usepackage{fancyhdr}
\usepackage[hidelinks]{hyperref}

\pagestyle{fancy}
\fancyhf{}
\rhead{Math 187A}
\chead{Friday, January 13, 2023}
\lhead{Lecture 3}
\rfoot{\thepage}

\setlength{\parindent}{0pt}

\begin{document}

\section{Classical Cryptosystems}
(Continued from Lecture 2.)
% 11:19am
\subsection{Interlude: GCDs}
\begin{definition}{Greatest Common Divisor}{}
    The \textbf{greatest common divisor} (or \emph{GCD}) of two integers $a$ and $b$ that are not both zero is denoted $\gcd(a, b)$ and is defined to be the largest integer which is both a divisor of $a$ and a divisor of $b$.
\end{definition}

\begin{mdframed}
    (Example.) Suppose we wanted to compute $\gcd(14, 21)$. 
    \begin{itemize}
        \item The factors of 14 are 1, 2, 7, and 14.
        \item The factors of 21 are 1, 3, 7, and 21. 
    \end{itemize}
    Therefore, as 7 is the \emph{largest integer} which is both a divisor of 14 and 21, it follows that $\gcd(14, 21) = 7$.
\end{mdframed}
Note that, while intuitive, this is actually not the best way of finding GCDs. Finding the factors of a number, especially a large one, is difficult. However, there exists algorithms that we can use to quickly calculate GCDs.

\begin{mdframed}
    (Example.) Suppose $a$ is a nonzero integer. What is $\gcd(a, 0)$?

    \begin{mdframed}
        The answer is $\gcd(a, 0) = |a|$. To see why this is the case, consider the following points. 

        \begin{enumerate}
            \item If $a \neq 0$, the largest value that divides $a$ is $|a|$. 
            \begin{mdframed}
                For example, the largest value that divides $100$ is $|100| = 100$. Likewise, the largest value that divides $-100$ is still $|-100| = 100$. 
            \end{mdframed}

            \item If you think about it, all integers divide 0. 
            \begin{mdframed}
                Recall that, if $a$ and $b$ are integers, $a$ divides $b$ if there is an integer $c$ such that 
                \[ac = b.\]
                Here, we write that $a | b$ to mean that $a$ divides $b$. 

                \bigskip 

                With this in mind, we note that 
                \[a \cdot 0 = 0\]
                and therefore 
                \[a | 0.\]
            \end{mdframed}

            \item Therefore, it follows that $\gcd(a, 0) = |a|$. 
            \begin{mdframed}
                To see this, note that the factors of $10$ and $-10$ are 
                \[\{-10, -5, -2, -1, 1, 2, 5, 10\},\]
                and we know that all factors of $0$ are effectively all integers. Therefore, it follows that 10 would be the answer here. 
            \end{mdframed}
        \end{enumerate}
    \end{mdframed}
\end{mdframed}

\subsubsection{Euclidean Algorithm}
The Euclidean Algorithm for computing GCDs relies on the following observation, defined as a lemma. 
\begin{lemma}{}{}
    Let $n$ be a positive integer and $a \equiv b \Mod{n}$. Then, $\gcd(a, n) = \gcd(b, n)$. 
\end{lemma}

\begin{proof}
    Let $c = \gcd(a, n)$ and $d = \gcd(b, n)$. Let $k$ be an integer such that \[a - b = nk.\] Since $c$ is a factor of both $a$ and $n$, it is also a factor of $a - nk = b$. Thus, $c$ is a common factor of both $b$ and $n$ as well, so $c \leq d$ bby definition of $d$. On the other hand, the same logic shows that $d$ is a common factor of both $a$ and $n$, so $d \leq c$ and thus $d = c$. 
\end{proof}

\begin{corollary}{}{}
    Let $n$ be a positive integer and let $r$ be the remainder when an integer $a$ is divided by $n$. Then, $\gcd(a, n) = \gcd(r, n)$. 
\end{corollary}

This brings us to the Euclidean Algorithm: 
\begin{mdframed}
    Suppose $a$ and $b$ are two positive integers, and assume without loss of generality (WLOG) that $b \geq a$. To find $\gcd(a, b)$, we can do the following: 
    \begin{itemize}
        \item Divide $b$ by $a$ and let $r$ be the remainder. Then, 
        \begin{itemize}
            \item If $r = 0$, output $a$. 
            \item Otherwise, replace $b$ with $a$ and $a$ with $r$. Then, repeat. 
        \end{itemize}
    \end{itemize}
\end{mdframed}

\begin{mdframed}
    (Example.) Suppose we wanted to compute $\gcd(115, 35)$. We divide the bigger number by the smaller one and get 
    \[115 = 3 \cdot 35 + 10.\]
    The remainder, $r = 10$, is nonzero, so we'll divide again, but this time, we'll divide the dividend (35) by the remainder (10) to get 
    \[35 = 3 \cdot 10 + 5.\]
    The remainder is nonzero again, so we repeat to get 
    \[10 = 2 \cdot 5 + 0.\]
    Since the remainder is 0, we output the dividend: \boxed{5}. Therefore, 
    \[\gcd(115, 35) = 5.\]
\end{mdframed}

\begin{mdframed}
    (Exercise.) Compute the following GCDs using the Euclidean Algorithm.

    \begin{itemize}
        \item $\gcd(180, 120).$
        \begin{mdframed}
            \begin{center}
                \begin{tabular}{|c|c|c|c|c|}
                    \hline 
                    $\mathbf{a}$ & $\mathbf{b}$ & $\mathbf{b = aq + r}$ & $\mathbf{q}$ & $\mathbf{r}$ \\ 
                    \hline 
                    120 & 180 & $180 = 120q + r$ & 1   & 60 \\ 
                    60  & 120 & $120 = 60q + r$  & 2   & 0  \\ 
                    \hline 
                \end{tabular}
            \end{center}
            Therefore, the answer must be \boxed{60}.
        \end{mdframed}
        \item $\gcd(180, 81).$
        \begin{mdframed}
            \begin{center}
                \begin{tabular}{|c|c|c|c|c|}
                    \hline 
                    $\mathbf{a}$ & $\mathbf{b}$ & $\mathbf{b = aq + r}$ & $\mathbf{q}$ & $\mathbf{r}$ \\ 
                    \hline 
                    81 & 180 & $180 = 81q + r$ & 2 & 18 \\ 
                    18 & 81 & $81 = 18q + r$ & 4 & 9 \\ 
                    9 & 18 & $18 = 9q + r$ & 2 & 0 \\ 
                    \hline 
                \end{tabular}
            \end{center}
            Therefore, the answer must be \boxed{9}.
        \end{mdframed}
        \item $\gcd(121, 77).$
        \begin{mdframed}
            \begin{center}
                \begin{tabular}{|c|c|c|c|c|}
                    \hline 
                    $\mathbf{a}$ & $\mathbf{b}$ & $\mathbf{b = aq + r}$ & $\mathbf{q}$ & $\mathbf{r}$ \\ 
                    \hline 
                    77 & 121 & $121 = 77q + r$ & 1 & 44 \\ 
                    44 & 77 & $77 = 44q + r$ & 1 & 33 \\ 
                    33 & 44 & $44 = 33q + r$ & 1 & 11 \\ 
                    11 & 33 & $33 = 11q + r$ & 3 & 0 \\ 
                    \hline 
                \end{tabular}
            \end{center}
            Therefore, the answer must be \boxed{11}.
        \end{mdframed}
    \end{itemize}
\end{mdframed}

\subsubsection{Bezout's Theorem}
\begin{theorem}{Bezout's Theorem}{}
    Suppose $a$ and $b$ are integers not both 0. Then, $\gcd(a, b)$ can be written as an \emph{integer linear combination} of $a$ and $b$, i.e., it can be written as $ax + by$ for some integers $x$ and $y$. Integers $x$ and $y$ such that \[\gcd(a, b) = ax + by\] are called \textbf{Bezout's coefficients}.
\end{theorem}

We can use the Euclidean Algorithm to find the Bezout coefficients, as seen in the example below. 
\begin{mdframed}
    (Example.) Suppose we want to find the Bezout coefficients for $\gcd(115, 35)$. Recall the sequence of operations we had to do:
    \[115 = 3 \cdot 35 + 10.\]
    \[35 = 3 \cdot 10 + 5.\]
    \[10 = 2 \cdot 5 + 0.\]
    Suppose we rearrange the first and second equations, like so: 
    \[10 = 115 - 3 \cdot 35.\] 
    \[5 = 35 - 3 \cdot 10.\]
    Plugging in the first equation into the second equation gives us 
    \[5 = 35 - 3 \cdot (115 - 3 \cdot 35).\]
    Simplifying this gives us 
    \begin{equation*}
        \begin{aligned}
            5 &= 35 - 3 \cdot (115 - 3 \cdot 35) \\ 
                &= 35 - 3(115) + 9(35) \\ 
                &= 10(35) - 3(115).
        \end{aligned}
    \end{equation*}
    Notice how we wrote $\gcd(115, 35)$ as an integer linear combination of those two numbers.
\end{mdframed}

Essentially, the steps are as follows: 
\begin{enumerate}
    \item Find the GCD using the Euclidean Algorithm.
    \item Rewrite the division for the \emph{last nonzero remainder}. 
    \item Alternate between substitution for the remainder directly above, and then simplify. Alternatively, start from the last equation with a nonzero remainder and then keep using the equations before that equation (e.g., from equation $n$, the last equation with a nonzero remainder, substitute equation $n - 1$ in the next step. Then, in the next step, substitute equation $n - 2$. Keep doing this until you reach equation 1.)
\end{enumerate}

\begin{mdframed}
    (Example.) Suppose we want to find the Bezout coefficients for $\gcd(240, 46)$. 
    \begin{enumerate}
        \item First, let's compute the GCD, keeping note of the sequence of operations we made. 
        \begin{center}
            \begin{tabular}{|c|c|c|c|c|}
                \hline 
                $\mathbf{a}$ & $\mathbf{b}$ & $\mathbf{b = aq + r}$ & $\mathbf{q}$ & $\mathbf{r}$ \\ 
                \hline 
                46 & 240 & $240 = 46q + r$ & 5 & 10 \\ 
                10 & 46 & $46 = 10q + r$ & 4 & 6 \\ 
                6 & 10 & $10 = 6q + r$ & 1 & 4 \\ 
                4 & 6 & $6 = 4q + r$ & 1 & 2 \\ 
                2 & 4 & $4 = 2q + r$ & 2 & 0 \\
                \hline 
            \end{tabular}
        \end{center}
        This tells us that $\gcd(240, 46) = 2$. The operations we did were 
        \begin{itemize}
            \item (Eq. 1) $240 = 46(5) + 10 \implies 10 = 240 - 46 \cdot 5$
            \item (Eq. 2) $46 = 10(4) + 6 \implies 6 = 46 - 10 \cdot 4$
            \item (Eq. 3) $10 = 6(1) + 4 \implies 4 = 10 - 6 \cdot 1$
            \item (Eq. 4) $6 = 4(1) + 2 \implies 2 = 6 - 4 \cdot 1$
            \item (Eq. 5) $4 = 2(2) + 0$
        \end{itemize}

        \item Rewriting the division for the last equation with the nonzero remainder (Eq. 4) gives us $2 = 6 - 4 \cdot 1.$
        \item Starting from the division for the last nonzero remainder, let's rewrite it: 
        \begin{equation*}
            \begin{aligned}
                2 &= 6 - 4 \cdot 1 && \text{From Eq. 4} \\ 
                    &= 6 - (\underbrace{10 - 6 \cdot 1}_{\text{Eq. 3}}) \cdot 1 && \text{Substitute Eq. 3} \\ 
                    &= 6 - 10 + 6 && \text{Expand} \\ 
                    &= 2 \cdot 6 - 1 \cdot 10 && \text{Rewrite to group like terms} \\ 
                    &= 2 \cdot (\underbrace{46 - 10 \cdot 4}_{\text{Eq. 2}}) - 1 \cdot 10 && \text{Substitute Eq. 2} \\ 
                    &= 2 \cdot 46 - 2 \cdot 10 \cdot 4 - 1 \cdot 10 && \text{Expand} \\ 
                    &= 2 \cdot 46 - 8 \cdot 10 - 1 \cdot 10 && \text{Simplify} \\ 
                    &= 2 \cdot 46 - 9 \cdot 10 && \text{Rewrite to group like terms} \\ 
                    &= 2 \cdot 46 - 9 \cdot (\underbrace{240 - 46 \cdot 5}_{\text{Eq. 1}}) && \text{Substitute Eq. 1} \\ 
                    &= 2 \cdot 46 - 9 \cdot 240 + 46 \cdot 5 \cdot 9 && \text{Expand} \\ 
                    &= 2 \cdot 46 - 9 \cdot 240 + 46 \cdot 45 && \text{Simplify} \\ 
                    &= 47 \cdot 46 - 9 \cdot 240 && \text{Rewrite to group like terms}
            \end{aligned}
        \end{equation*}
        Notice how the Bezout coefficients are $47$ and $-9$. 
    \end{enumerate}
\end{mdframed}

\begin{mdframed}
    (Exercise.) Calculate Bezout's coefficients for the following GCDs using the extended Euclidean Algorithm.
    \begin{itemize}
        \item $\gcd(180, 120).$
        \begin{mdframed}
            \begin{enumerate}
                \item First, compute the GCD. We already did this in a previous exercise, but just to reiterate: 
                \begin{center}
                    \begin{tabular}{|c|c|c|c|c|}
                        \hline 
                        $\mathbf{a}$ & $\mathbf{b}$ & $\mathbf{b = aq + r}$ & $\mathbf{q}$ & $\mathbf{r}$ \\ 
                        \hline 
                        120 & 180 & $180 = 120q + r$ & 1   & 60 \\ 
                        60  & 120 & $120 = 60q + r$  & 2   & 0  \\ 
                        \hline 
                    \end{tabular}
                \end{center}
                Therefore, the GCD is 60. The operations that we did were 
                \begin{itemize}
                    \item (Eq. 1) $180 = 120(1) + 60 \implies 60 = 180 - 120(1)$
                    \item (Eq. 2) $120 = 60(2) + 0$
                \end{itemize}


                \item Next, we just need to rewrite the last equation with a nonzero remainder. 
                \[180 = 120(1) + 60 \implies 60 = 180 - 120(1)\]

                \item Finally, we need to work backwards, substituting the previous equations. Because we only have one operation which resulted in a non-zero remainder, it follows that we only need to do: 
                \[60 = 180 - 120(1).\]
                Therefore, the Bezout coefficients are \boxed{1} and \boxed{-1}.
            \end{enumerate}
        \end{mdframed}
        \item $\gcd(180, 81).$
        \begin{mdframed}
            \begin{enumerate}
                \item First, we need to compute the GCD. We already did this in a previous exercise, but to reiterate: 
                
                \begin{center}
                    \begin{tabular}{|c|c|c|c|c|}
                        \hline 
                        $\mathbf{a}$ & $\mathbf{b}$ & $\mathbf{b = aq + r}$ & $\mathbf{q}$ & $\mathbf{r}$ \\ 
                        \hline 
                        81 & 180 & $180 = 81q + r$ & 2 & 18 \\ 
                        18 & 81 & $81 = 18q + r$ & 4 & 9 \\ 
                        9 & 18 & $18 = 9q + r$ & 2 & 0 \\ 
                        \hline 
                    \end{tabular}
                \end{center}
                Therefore, the GCD is 9. The operations we did were 
                \begin{itemize}
                    \item (Eq. 1) $180 = 81(2) + 18 \implies 18 = 180 - 81(2)$
                    \item (Eq. 2) $81 = 18(4) + 9 \implies 9 = 81 - 18(4)$ 
                    \item (Eq. 3) $18 = 9(2) + 0$
                \end{itemize}
        


                \item Next, we need to rewrite the last equation with a nonzero remainder. 
                \[81 = 18(4) + 9 \implies 9 = 81 - 18(4).\]

                \item Finally, we need to work backwards, substituting the previous equations as needed. 
                \begin{equation*}
                    \begin{aligned}
                        9 &= 81 - 18(4) \\ 
                            &= 81 - (\underbrace{180 - 81(2)}_{\text{Eq. 1}}) \cdot 4 \\ 
                            &= 81 - 180(4) + 81(8) \\ 
                            &= 81(9) - 180(4) 
                    \end{aligned}
                \end{equation*}

                Therefore, the Bezout coefficients are \boxed{9} and \boxed{-4}.
            \end{enumerate}



        \end{mdframed}
        \item $\gcd(121, 77).$
        \begin{mdframed}
            \begin{enumerate}
                \item First, compute the GCD. To reiterate:
                \begin{center}
                    \begin{tabular}{|c|c|c|c|c|}
                        \hline 
                        $\mathbf{a}$ & $\mathbf{b}$ & $\mathbf{b = aq + r}$ & $\mathbf{q}$ & $\mathbf{r}$ \\ 
                        \hline 
                        77 & 121 & $121 = 77q + r$ & 1 & 44 \\ 
                        44 & 77 & $77 = 44q + r$ & 1 & 33 \\ 
                        33 & 44 & $44 = 33q + r$ & 1 & 11 \\ 
                        11 & 33 & $33 = 11q + r$ & 3 & 0 \\ 
                        \hline 
                    \end{tabular}
                \end{center}
                Therefore, the GCD is 11. The operations that we did were 
                \begin{itemize}
                    \item (Eq. 1) $121 = 77(1) + 44 \implies 44 = 121 - 77(1)$ 
                    \item (Eq. 2) $77 = 44(1) + 33 \implies 33 = 77 - 44(1)$
                    \item (Eq. 3) $44 = 33(1) + 11 \implies 11 = 44 - 33(1)$
                    \item (Eq. 4) $33 = 11(3) + 0$
                \end{itemize}

                \item Next, rewrite the last equation with a nonzero remainder.
                \[44 = 33(1) + 11 \implies 11 = 44 - 33(1).\]

                \item Finally, work backwards. 
                \begin{equation*}
                    \begin{aligned}
                        11 &= 44 - 33(1) \\ 
                            &= 44 - (\underbrace{77 - 44(1)}_{\text{Eq. 2}}) \cdot 1 \\
                            &= 44 - 77 + 44(1) \\ 
                            &= 44(2) - 77 \\ 
                            &= (\underbrace{121 - 77(1)}_{\text{Eq. 1}}) \cdot 2 - 77 \\ 
                            &= 121(2) - 77(2) - 77 \\ 
                            &= 121(2) - 77(3).
                    \end{aligned}
                \end{equation*}
                Thereforem the Bezout coefficients are \boxed{2} and \boxed{-3}.
            \end{enumerate}
        \end{mdframed}
    \end{itemize}
\end{mdframed}

\begin{mdframed}
    (Exercise.) Observe that $\gcd(42, 12) = 6$. Show taht the pairs $(-1, 4)$ and $(1, -3)$ are both Bezout coefficients for 42 and 12. 

    \begin{mdframed}
        \begin{itemize}
            \item For the pair $(-1, 4)$, we have 
            \[42(-1) + 12(4) = -42 + 48 = 6.\]

            \item For the pair $(1, -3)$, we have 
            \[42(1) + 12(-3) = 42 - 36 = 6.\]
        \end{itemize}
    \end{mdframed}
\end{mdframed}


\subsubsection{Modular Inversion}
Suppose you are asked to solve the equation 
\[5z = 7.\]
Intuitively, we can just divide both sides by 5. Stated differently, we can multiply both sides by $\frac{1}{5}$:
\[\left(\frac{1}{5}\right) \cdot 5z = \left(\frac{1}{5}\right) 7 \implies z = \frac{5}{7}.\]
In other words, we're able to ``cancel out'' the 5 that appears on the left-hand side, thus isolating $z$. 

\bigskip 

With modular inversion, we can recreate this process with \emph{congruences}. For example, suppose we want to solve 
\[5z \equiv 7 \Mod{11}.\]
We cannot ``divide both sides by 5'' because congruences only make sense when both sides of the congruence are \emph{integers}. But, if we find an integer $x$ with the property that 
\[5x \equiv 1 \Mod{11},\] 
then we can multiply both sides of our congruence by $x$ to effectively eliminate the 5 on the left-hand side. In this example, there \emph{is} an integer: $x = 9$. Using this integer, we have 
\[5x = 9 \cdot 5 = 45 \equiv 1 \Mod{11}.\]
Therefore, multiplying both sides of our congruence by 9 gives us 
\[z = 1 \cdot z \equiv (5 \cdot 9)z = 9 \cdot (5z) \equiv 9 \cdot 7 \Mod{11}.\]
Thus, 
\[z \equiv 9 \cdot 7 = 63 \equiv 8 \Mod{11},\]
and we've solved our congruence: $z \equiv 8 \Mod{11}$. While we solved this congruence, note that we basically guessed what the solution is. However, there's a way to get such $x$.

\begin{definition}{}{}
    Fix a positive integer $n$. An integer $a$ is \emph{invertible} mod $n$ (or a \emph{unit} mod $n$) if there exists another integer $x$ such that $ax \equiv 1 \Mod{n}$. The number $x$ is then called an \emph{inverse of} $a$ mod $n$ and, in symbols, one writes $x \equiv a^{-1} \Mod{n}$. 
\end{definition}
So, in the above example, we found that $9 \equiv 5^{-1} \Mod{11}$ because $5 \cdot 9 \equiv 1 \Mod{11}$. 

\begin{mdframed}
    (Exercise.) Explain why 2 is not invertible mod 4.
    \begin{mdframed}
        Essentially, we need to show why there does not exist an integer $x$ such that 
        \[2x \equiv 1 \Mod{4}.\]
        However, notice that both 2 and 4 are even. Therefore, multiplying 2 by any integer gives us an even number. Because 4 is even as well, it follows that we'll never be able to find an $x$ such that $2x \equiv 1 \Mod{4}$. 
    \end{mdframed}
\end{mdframed}

\begin{theorem}{Modular Inversion Theorem}{l3:1}
    Fix a positive integer $n$ and another integer $a$. Then, $a$ is invertible mod $n$ if and only if $\gcd(a, n) = 1$. Moreover, if $\gcd(a, n) = 1$ and $x$ and $y$ are Bezout coefficients for $a$ and $n$, then $x$ is an inverse of $a$ mod $n$. 
\end{theorem}

\begin{mdframed}
    (Example.) Suppose we want to find the inverse of $7 \Mod{23}$. Using the Euclidean Algorithm to compute $\gcd(23, 7)$, we get 
    \[23 = 3 \cdot 7 + 2\]
    \[7 = 3 \cdot 2 + 1\]
    \[2 = 2 \cdot 1 + 0.\]
    So, $\gcd(23, 7) = 1$ and thus 7 is in fact invertible mod 23. Working backwards, we find that 
    \begin{equation*}
        \begin{aligned}
            1 &= 7 - 3 \cdot 2 \\ 
                &= 7 - 3 \cdot (23 - 3 \cdot 7) \\ 
                &= 10 \cdot 7 - 3 \cdot 23.
        \end{aligned}
    \end{equation*}
    Therefore, the Modular Inversion Theorem tells us that 10 is the inverse of 7 mod 23. 
\end{mdframed}

\begin{mdframed}
    (Exercise.) For each of the following, determine whether $a$ is invertible mod $n$. If it is, find an inverse of $a$ mod $n$. 
    \begin{itemize}
        \item $a = 14, n = 21$.
        \begin{mdframed}
            First, let's calculate $\gcd(14, 21)$. 
            \begin{center}
                \begin{tabular}{|c|c|c|c|c|}
                    \hline 
                    $\mathbf{a}$ & $\mathbf{b}$ & $\mathbf{b = aq + r}$ & $\mathbf{q}$ & $\mathbf{r}$ \\ 
                    \hline 
                    14 & 21 & $21 = 14q + r$ & 1 & 7 \\ 
                    7 & 14 & $14 = 7q + r$ & 2 & 0 \\ 
                    \hline 
                \end{tabular}
            \end{center}
            Therefore, $\gcd(14, 21) = 7$. By Theorem (\ref{th:l3:1}), it follows that $14$ is not invertible mod $21$. 
        \end{mdframed}

        \item $a = 3, n = 7$.
        \begin{mdframed}
            First, we calculate $\gcd(3, 7)$. 
            \begin{center}
                \begin{tabular}{|c|c|c|c|c|}
                    \hline 
                    $\mathbf{a}$ & $\mathbf{b}$ & $\mathbf{b = aq + r}$ & $\mathbf{q}$ & $\mathbf{r}$ \\ 
                    \hline 
                    3 & 7 & $7 = 3q + r$ & 2 & 1 \\ 
                    1 & 3 & $3 = 1q + r$ & 3 & 0 \\ 
                    \hline 
                \end{tabular}
            \end{center}
            Therefore, $\gcd(3, 7) = 1$. By Theorem (\ref{th:l3:1}), it follows that $3$ is invertible mod $7$.

            \bigskip 

            With this in mind, let's find the Bezout coefficients. We note that the equations we used to find the GCD were
            \begin{itemize}
                \item (Eq. 1) $7 = 3(2) + 1 \implies 1 = 7 - 3(2)$
                \item (Eq. 2) $3 = 1(3) + 0$
            \end{itemize}
            Starting with the last equation with a nonzero remainder, which is Eq. 1, we have 
            \[7 = 3(2) + 1 \implies 1 = 7 - 3(2).\]
            Because we are able to write an equation in terms of 3 and 7, we find that 
            \[\gcd(3, 7) = 1 = 3(-2) + 7(1).\]
            From this, it follows that $x = -2$ and $y = 1$. So, by Theorem (\ref{th:l3:1}), it follows that $-2$ is an inverse of $3 \Mod{7}$. 

            \bigskip 

            We should note that Bezout coefficients are not unique. If we wanted a positive answer, we note that 
            \[-2 \equiv 5 \Mod{7}\]
            so that another possible answer is \boxed{5}.
        \end{mdframed}

        \item $a = 41, n = 50$.
        \begin{mdframed}
            First, we calculate $\gcd(41, 50)$. 
            \begin{center}
                \begin{tabular}{|c|c|c|c|c|}
                    \hline 
                    $\mathbf{a}$ & $\mathbf{b}$ & $\mathbf{b = aq + r}$ & $\mathbf{q}$ & $\mathbf{r}$ \\ 
                    \hline 
                    41 & 50 & $50 = 41q + r$ & 1 & 9 \\ 
                    9 & 41 & $41 = 9q + r$ & 4 & 5 \\ 
                    5 & 9 & $9 = 5q + r$ & 1 & 4 \\ 
                    4 & 5 & $5 = 4q + r$ & 1 & 1 \\ 
                    1 & 4 & $4 = 1q + r$ & 4 & 0 \\ 
                    \hline 
                \end{tabular}
            \end{center}
            Therefore, $\gcd(41, 50) = 1$. By Theorem (\ref{th:l3:1}), it follows that $41$ is invertible mod $50$.

            \bigskip 

            Next, we need to find the Bezout coefficients. We note that the equations we used to find the GCD were
            \begin{itemize}
                \item (Eq. 1) $50 = 41(1) + 9 \implies 9 = 50 - 41(1)$
                \item (Eq. 2) $41 = 9(4) + 5 \implies 5 = 41 - 9(4)$
                \item (Eq. 3) $9 = 5(1) + 4 \implies 4 = 9 - 5(1)$
                \item (Eq. 4) $5 = 4(1) + 1 \implies 1 = 5 - 4(1)$
                \item (Eq. 5) $4 = 1(4) + 0$
            \end{itemize}
            
            Now, working backwards from the last equation with a nonzero remainder (i.e., Eq. 4):
            \begin{equation*}
                \begin{aligned}
                    1 &= 5 - 4(1) \\ 
                        &= 5 - (\underbrace{9 - 5(1)}_{\text{Eq. 3}})(1) \\ 
                        &= 5 - 9 + 5 \\ 
                        &= 5(2) - 9 \\ 
                        &= (\underbrace{41 - 9(4)}_{\text{Eq. 2}})(2) - 9 \\ 
                        &= 41(2) - 9(4)(2) - 9 \\ 
                        &= 41(2) - 9(8) - 9 \\ 
                        &= 41(2) - 9(9) \\ 
                        &= 41(2) - (\underbrace{50 - 41(1)}_{\text{Eq. 1}})(9) \\ 
                        &= 41(2) - 50(9) + 41(9) \\ 
                        &= 41(11) - 50(9)
                \end{aligned}
            \end{equation*}
            Therefore, we have 
            \[\gcd(41, 50) = 1 = 41(11) + 50(-9)\]
            and so $x = 11$ and $y = -9$. From this, by Theorem (\ref{th:l3:1}) it follows that $\boxed{11}$ is an inverse of $41 \Mod{50}$. 
        \end{mdframed}
    \end{itemize}
\end{mdframed}

\begin{mdframed}
    (Exercise.) Solve the following congruences for $z$. 
    \begin{itemize}
        \item $2z \equiv 3 \Mod{11}$
        \begin{mdframed}
            Trivially, $\gcd(2, 11) = 1$. However, let's find the GCD using the Euclidean Algorithm regardless.
            \begin{center}
                \begin{tabular}{|c|c|c|c|c|}
                    \hline 
                    $\mathbf{a}$ & $\mathbf{b}$ & $\mathbf{b = aq + r}$ & $\mathbf{q}$ & $\mathbf{r}$ \\ 
                    \hline 
                    2 & 11 & $11 = 2q + r$ & 5 & 1 \\ 
                    1 & 2 & $2 = 1q + r$ & 2 & 0 \\ 
                    \hline 
                \end{tabular}
            \end{center}
            Therefore, the GCD is 1. We can now find the Bezout coefficients. Note that the equations used to find the GCD were 
            \begin{itemize}
                \item (Eq. 1) $11 = 2(5) + 1$
                \item (Eq. 2) $2 = 1(2) + 0$
            \end{itemize} 
            Starting with the last equation with a nonzero remainder, which is Eq. 1, we have 
            \[1 = 11 - 2(5).\]
            Immediately, it follows that 
            \[\gcd(2, 11) = 1 = 11(1) + 2(-5).\]
            Hence, by Theorem (\ref{th:l3:1}), $x = -5 \equiv 6 \Mod{11}$ is the inverse of $2 \Mod{11}$.

            \bigskip 

            With this in mind, we now know that 
            \begin{equation*}
                \begin{aligned}
                    2z &\equiv 3 \Mod{11} \\ 
                        &\implies 6(2z) \equiv 6(3) \Mod{11} \\ 
                        &\implies 12z \equiv 18 \Mod{11} \\ 
                        &\implies z \equiv 7 \Mod{11}.
                \end{aligned}
            \end{equation*}
            Therefore, the answer is $z \equiv \boxed{7} \Mod{11}$.
        \end{mdframed}
        
        \item $3z \equiv 2 \Mod{7}$
        \begin{mdframed}
            Using the strategy of trial-and-error, we find that $z \equiv 3 \Mod{7}$. 
        \end{mdframed}
        
        \item $5z \equiv 3 \Mod{15}$
        \begin{mdframed}
            We note that $\gcd(5, 15) = 5$. Therefore, by Theorem (\ref{th:l3:1}), there is no solution that satisfies this congruence.
        \end{mdframed}

        \item $5z \equiv 17 \Mod{101}$
        \begin{mdframed}
            First, we want to find $\gcd(5, 101)$. Using the Euclidean Algorithm gives us:
            \begin{center}
                \begin{tabular}{|c|c|c|c|c|}
                    \hline 
                    $\mathbf{a}$ & $\mathbf{b}$ & $\mathbf{b = aq + r}$ & $\mathbf{q}$ & $\mathbf{r}$ \\ 
                    \hline 
                    5 & 101 & $101 = 5q + r$ & 20 & 1 \\ 
                    1 & 5 & $5 = 1q + r$ & 5 & 0 \\ 
                    \hline 
                \end{tabular}
            \end{center}
            Therefore, the GCD is 1. We can now find the Bezout coefficients. Note that the equations used to find the GCD were 
            \begin{itemize}
                \item (Eq. 1) $101 = 5(20) + 1 \implies 1 = 101 - 5(20)$
                \item (Eq. 2) $5 = 1(5) + 0$
            \end{itemize}
            
            Starting with the last equation with a nonzero remainder, which is Eq. 1, we have 
            \[1 = 101 - 5(20).\]
            Immediately, it follows that 
            \[\gcd(5, 101) = 1 = 101(1) + 5(-20).\]
            Hence, by Theorem (\ref{th:l3:1}), $x = -20 \equiv 81 \Mod{11}$ is the inverse of $5 \Mod{101}$.

            \bigskip 

            With this in mind, we now know that 
            \begin{equation*}
                \begin{aligned}
                    5z &\equiv 17 \Mod{101} \\ 
                        &\implies 81(5z) \equiv 81(17) \Mod{101} \\ 
                        &\implies 405z \equiv 1377 \Mod{101} \\ 
                        &\implies z \equiv 64 \Mod{101}.
                \end{aligned}
            \end{equation*}
            Therefore, the answer is $z \equiv \boxed{64} \Mod{101}$.

            \begin{mdframed}
                If we use $x = -20$ instead, we have 
                \begin{equation*}
                    \begin{aligned}
                        5z &\equiv 17 \Mod{101} \\ 
                            &\implies -20(5z) \equiv -20(17) \Mod{101} \\ 
                            &\implies -100z \equiv -340 \Mod{101} \\ 
                            &\implies z \equiv -340 \Mod{101} \\ 
                            &\implies z \equiv 64 \Mod{101}.
                    \end{aligned}
                \end{equation*}
            \end{mdframed}
        \end{mdframed}
    \end{itemize}
\end{mdframed}
So, in summary, given the congruence $az \equiv b \Mod{n}$, the steps for solving for $z$ are as follows: 
\begin{enumerate}
    \item Find $\gcd(a, n)$. If $\gcd(a, n) \neq 1$, then there are no possible solutions.
    \item Find the Bezout coefficients for $\gcd(a, n)$. Specifically, for \[\gcd(a, n) = ax + ny,\] find $x$ (the Bezout coefficients for $a$). This represents your inverse of $a \Mod{n}$.
    \item Multiply both sides of the congruence by $x$; that is, 
    \[x(az) \equiv x(b) \Mod{n},\]
    and then simplify.
\end{enumerate}

As you can tell, Bezout coefficients are not unique, and inverses aren't strictly unique either. Notice, for example, that $3(2) \equiv 1 \Mod{5}$ and $8(2) \equiv 1 \Mod{5}$ so that $8$ and $3$ are both inverses of $2 \Mod{5}$. However, notice that $8 \equiv 3 \Mod{5}$. In other words, inverses are \emph{kind of} unique when they exist: they are unique mod $n$. 

\begin{lemma}{}{}
    Fix a positive integer $n$ and suppose $a$ is invertible mod $n$. If $x$ and $x'$ are both inverses of $a$ mod $n$, then 
    \[x \equiv x' \Mod{n}.\]
\end{lemma}

\end{document}