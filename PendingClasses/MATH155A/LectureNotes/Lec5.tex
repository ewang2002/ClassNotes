\documentclass[letterpaper]{article}
\usepackage[margin=1in]{geometry}
\usepackage[utf8]{inputenc}
\usepackage{textcomp}
\usepackage{amssymb}
\usepackage{natbib}
\usepackage{graphicx}
\usepackage{gensymb}
\usepackage{amsthm, amsmath, mathtools}
\usepackage[dvipsnames]{xcolor}
\usepackage{enumerate}
\usepackage{mdframed}
\usepackage[most]{tcolorbox}
\usepackage{csquotes}
% https://tex.stackexchange.com/questions/13506/how-to-continue-the-framed-text-box-on-multiple-pages

\tcbuselibrary{theorems}

\newcommand{\R}{\mathbb{R}}
\newcommand{\Z}{\mathbb{Z}}
\newcommand{\N}{\mathbb{N}}
\newcommand{\Q}{\mathbb{Q}}
\newcommand{\C}{\mathbb{C}}
\newcommand{\code}[1]{\texttt{#1}}
\newcommand{\mdiamond}{$\diamondsuit$}
\newcommand{\PowerSet}{\mathcal{P}}
\newcommand{\Mod}[1]{\ (\mathrm{mod}\ #1)}
\DeclareMathOperator{\lcm}{lcm}

%\newtheorem*{theorem}{Theorem}
%\newtheorem*{definition}{Definition}
%\newtheorem*{corollary}{Corollary}
%\newtheorem*{lemma}{Lemma}
\newtheorem*{proposition}{Proposition}


\newtcbtheorem[number within=section]{theorem}{Theorem}
{colback=green!5,colframe=green!35!black,fonttitle=\bfseries}{th}

\newtcbtheorem[number within=section]{definition}{Definition}
{colback=blue!5,colframe=blue!35!black,fonttitle=\bfseries}{def}

\newtcbtheorem[number within=section]{corollary}{Corollary}
{colback=yellow!5,colframe=yellow!35!black,fonttitle=\bfseries}{cor}

\newtcbtheorem[number within=section]{lemma}{Lemma}
{colback=red!5,colframe=red!35!black,fonttitle=\bfseries}{lem}

\newtcbtheorem[number within=section]{example}{Example}
{colback=white!5,colframe=white!35!black,fonttitle=\bfseries}{def}

\newtcbtheorem[number within=section]{note}{Important Note}{
        enhanced,
        sharp corners,
        attach boxed title to top left={
            xshift=-1mm,
            yshift=-5mm,
            yshifttext=-1mm
        },
        top=1.5em,
        colback=white,
        colframe=black,
        fonttitle=\bfseries,
        boxed title style={
            sharp corners,
            size=small,
            colback=red!75!black,
            colframe=red!75!black,
        } 
    }{impnote}
\usepackage[utf8]{inputenc}
\usepackage[english]{babel}
\usepackage{fancyhdr}
\usepackage[hidelinks]{hyperref}

\pagestyle{fancy}
\fancyhf{}
\rhead{MATH 155A}
\chead{Tuesday, April 12, 2022}
\lhead{Lecture 5}
\rfoot{\thepage}

\setlength{\parindent}{0pt}

\begin{document}

\section{Moving to \texorpdfstring{$\R^3$}{Three-Dimensions}}
In $\R^3$, we often have that $y$ is facing towards us. However, in this course, $y$ will be facing upwards while $z$ is facing towards us. 

\bigskip 

Let $f(\mathbf{x}) = M\mathbf{x}$, where $M$ is a $3 \times 3$ matrix. Here, $M$ is denoted by 
\[\begin{bmatrix}
    f(\mathbf{i}) & f(\mathbf{j}) & f(\mathbf{k})
\end{bmatrix}.\]
Additionally, note that the unit vectors in $\R^3$ are given by 
\[\mathbf{i} = \begin{bmatrix}
    1 \\ 0 \\ 0
\end{bmatrix} \qquad \mathbf{j} = \begin{bmatrix}
    0 \\ 1 \\ 0
\end{bmatrix} \qquad \mathbf{k} = \begin{bmatrix}
    0 \\ 0 \\ 1
\end{bmatrix}.\]

\subsection{Scaling}
For $\alpha \in \R$, we say that $S_{\alpha}(\mathbf{x}) = \alpha \mathbf{x}$ is known as \emph{uniform scaling}.

\bigskip 

Now, for $a, b, c \in \R$, we say that $S_{\cyclic{a, b, c}}(\cyclic{x_1, x_2, x_3}) =\cyclic{ax_1, bx_2, bx_3}$ is known as \emph{non-uniform scaling}.


\begin{mdframed}[]
    (Example.) Suppose we have $S = S_{\cyclic{1/2, 1, 1}}$ and $R = R_{90^{\circ}, \mathbf{j}}$. Find the following: 
    \begin{itemize}
        \item $S \circ R$.
        \item $R \circ S$. 
    \end{itemize}
\end{mdframed}


\end{document}