\documentclass[letterpaper]{article}
\usepackage[margin=1in]{geometry}
\usepackage[utf8]{inputenc}
\usepackage{textcomp}
\usepackage{amssymb}
\usepackage{natbib}
\usepackage{graphicx}
\usepackage{gensymb}
\usepackage{amsthm, amsmath, mathtools}
\usepackage{xcolor}
\usepackage{enumerate}
\usepackage{framed}
\usepackage{tcolorbox}
\tcbuselibrary{theorems}

\newcommand{\R}{\mathbb{R}}
\newcommand{\Z}{\mathbb{Z}}
\newcommand{\N}{\mathbb{N}}
\newcommand{\Q}{\mathbb{Q}}
\newcommand{\code}[1]{\texttt{#1}}
\newcommand{\mdiamond}{$\diamondsuit$}

%\newtheorem*{theorem}{Theorem}
%\newtheorem*{definition}{Definition}
\newtheorem*{proposition}{Proposition}
%\newtheorem*{corollary}{Corollary}
%\newtheorem*{lemma}{Lemma}

\newtcbtheorem[number within=section]{theorem}{Theorem}
{colback=green!5,colframe=green!35!black,fonttitle=\bfseries}{def}

\newtcbtheorem[number within=section]{definition}{Definition}
{colback=blue!5,colframe=blue!35!black,fonttitle=\bfseries}{def}

\newtcbtheorem[number within=section]{corollary}{Corollary}
{colback=yellow!5,colframe=yellow!35!black,fonttitle=\bfseries}{def}

\newtcbtheorem[number within=section]{lemma}{Lemma}
{colback=red!5,colframe=red!35!black,fonttitle=\bfseries}{def}
\usepackage[utf8]{inputenc}
\usepackage[english]{babel}
\usepackage{fancyhdr}
\usepackage[hidelinks]{hyperref}

\pagestyle{fancy}
\fancyhf{}
\rhead{MATH 155A}
\chead{Tuesday, April 05, 2022}
\lhead{Lecture 3}
\rfoot{\thepage}

\setlength{\parindent}{0pt}

\begin{document}

% 2.5, 3.4, 3.5, 3.6, 3.7

\section{Graphics Pipeline, Linear \& Affine Transformations in \texorpdfstring{$\R^2$}{R2}}

\subsection{Inverses of Linear and Affine Transformations}
We begin with a definition. 
\begin{definition}{}{}
    Let $A$ and $B$ be transformations of $\R^2$. We say that $B$ is the \textbf{inverse} of $A$, written $B = A^{-1}$, if: 
    \begin{enumerate}
        \item $A \circ B = I$. 
        \item $B \circ A = I$. 
    \end{enumerate}
\end{definition}

\subsubsection{Finding Inverses of Linear Transformations}
How do we compute inverses? One way to do so is to do it by matrices. Let $A: \R^2 \mapsto \R^2$ be represented by the matrix $M = \begin{bmatrix}
    a & b \\ c & d
\end{bmatrix}$. Then, $A^{-1}$ is represented by $M^{-1}$, where 
\[M^{-1} = \begin{bmatrix}
    d & -b \\ 
    -c & a
\end{bmatrix} \frac{1}{ad - bc}.\]

\subsubsection{Finding Inverses of Affine Transformations}
Let $A(\mathbf{x})$ be defined by 
\[A(\mathbf{x}) = B(\mathbf{x}) + \mathbf{u},\]
where $B$ is linear and $\mathbf{u} \in \R$ so that $A$ is affine. How do we compute its inverse? Now, to find $A^{-1}$, we need to solve 
\[\mathbf{y} = B(\mathbf{x}) + \mathbf{u}\]
to get $\mathbf{x}$ in terms of $\mathbf{y}$. Now, we have 
\begin{equation*}
    \begin{aligned}
        \mathbf{y} &= B(\mathbf{x}) + \mathbf{u} \\ 
            &\implies B(\mathbf{x}) = \mathbf{y} - \mathbf{u} \\ 
            &\implies \mathbf{x} = B^{-1}(\mathbf{y} - \mathbf{u}) \\ 
            &\implies \mathbf{x} = B^{-1}(\mathbf{y}) - B^{-1}(\mathbf{u}) \\ 
            &\implies \mathbf{x} = \underbrace{B^{-1}(\mathbf{y})}_{\text{Linear Part}} + \overbrace{(-B^{-1}(\mathbf{u}))}^{\text{Translation Part}}.
    \end{aligned}
\end{equation*}

\subsection{Homogeneous Coordinates}
We can represent points in $\R^2$ with triples $\cyclic{x, y, u}$; this represents $\cyclic{\frac{x}{u}, \frac{y}{u}}$ for $u \neq 0$. 

\begin{mdframed}[]
    (Example.) Some examples of homogeneous coordinates for $\cyclic{2, 1} \in \R^2$ are: 
    \begin{itemize}
        \item $\cyclic{2, 1, 1}$. 
        \item $\cyclic{4, 2, 2}$. 
        \item $\cyclic{8, 4, 4}$. 
        \item $\cyclic{-2, -1, -1}$. 
    \end{itemize}
\end{mdframed}

If $f: \R^2 \mapsto \R^2$ and if
\[f(\mathbf{x}) = \begin{bmatrix}
    a & b \\ c & d
\end{bmatrix} \mathbf{x} + \begin{bmatrix}
    e \\ f 
\end{bmatrix},\]
then $f$ -- acting in homogeneous coordinates -- is represented by the $3 \times 3$ matrix 
\[\begin{bmatrix}
    a & b & e \\ c & d & f \\ 0 & 0 & 1
\end{bmatrix}.\]



\end{document}