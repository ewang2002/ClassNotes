\documentclass[letterpaper]{article}
\usepackage[margin=1in]{geometry}
\usepackage[utf8]{inputenc}
\usepackage{textcomp}
\usepackage{amssymb}
\usepackage{natbib}
\usepackage{graphicx}
\usepackage{gensymb}
\usepackage{amsthm, amsmath, mathtools}
\usepackage[dvipsnames]{xcolor}
\usepackage{enumerate}
\usepackage{mdframed}
\usepackage[most]{tcolorbox}
\usepackage{csquotes}
% https://tex.stackexchange.com/questions/13506/how-to-continue-the-framed-text-box-on-multiple-pages

\tcbuselibrary{theorems}

\newcommand{\R}{\mathbb{R}}
\newcommand{\Z}{\mathbb{Z}}
\newcommand{\N}{\mathbb{N}}
\newcommand{\Q}{\mathbb{Q}}
\newcommand{\C}{\mathbb{C}}
\newcommand{\code}[1]{\texttt{#1}}
\newcommand{\mdiamond}{$\diamondsuit$}
\newcommand{\PowerSet}{\mathcal{P}}
\newcommand{\Mod}[1]{\ (\mathrm{mod}\ #1)}
\DeclareMathOperator{\lcm}{lcm}

%\newtheorem*{theorem}{Theorem}
%\newtheorem*{definition}{Definition}
%\newtheorem*{corollary}{Corollary}
%\newtheorem*{lemma}{Lemma}
\newtheorem*{proposition}{Proposition}


\newtcbtheorem[number within=section]{theorem}{Theorem}
{colback=green!5,colframe=green!35!black,fonttitle=\bfseries}{th}

\newtcbtheorem[number within=section]{definition}{Definition}
{colback=blue!5,colframe=blue!35!black,fonttitle=\bfseries}{def}

\newtcbtheorem[number within=section]{corollary}{Corollary}
{colback=yellow!5,colframe=yellow!35!black,fonttitle=\bfseries}{cor}

\newtcbtheorem[number within=section]{lemma}{Lemma}
{colback=red!5,colframe=red!35!black,fonttitle=\bfseries}{lem}

\newtcbtheorem[number within=section]{example}{Example}
{colback=white!5,colframe=white!35!black,fonttitle=\bfseries}{def}

\newtcbtheorem[number within=section]{note}{Important Note}{
        enhanced,
        sharp corners,
        attach boxed title to top left={
            xshift=-1mm,
            yshift=-5mm,
            yshifttext=-1mm
        },
        top=1.5em,
        colback=white,
        colframe=black,
        fonttitle=\bfseries,
        boxed title style={
            sharp corners,
            size=small,
            colback=red!75!black,
            colframe=red!75!black,
        } 
    }{impnote}
\usepackage[utf8]{inputenc}
\usepackage[english]{babel}
\usepackage{fancyhdr}
\usepackage[hidelinks]{hyperref}

\pagestyle{fancy}
\fancyhf{}
\rhead{CSE 100}
\chead{September 14th, 2021}
\lhead{Course Notes}
\rfoot{\thepage}

\setlength{\parindent}{0pt}

\begin{document}

\begin{titlepage}
    \begin{center}
        \vspace*{1cm}
            
        \Huge
        \textbf{CSE 100 Notes}
            
        \vspace{0.5cm}
        \LARGE
        Advanced Data Structures
            
        \vspace{1.5cm}
            
        \vfill
            
        Fall 2021\\
        Taught by Professor Niema Moshiri
    \end{center}
\end{titlepage}

\pagenumbering{gobble}

\newpage 

\pagenumbering{gobble}
\begingroup
    \renewcommand\contentsname{Table of Contents}
    \tableofcontents
\endgroup

\newpage
\pagenumbering{arabic}

\section{A Brief Introduction}
In this course, we will primarily be building off of our prior knowledge of data structures (CSE 12). In particular, we will: 
\begin{itemize}
    \item Analyze data structures for both time and space complexity. 
    \item Describe the strengths and weaknesses of a data structure. 
    \item Implement complex data structures correctly and efficiently. 
\end{itemize}


% https://www.youtube.com/watch?v=_vpy1Flh__4&list=PLM_KIlU0WoXmkV4QB1Dg8PtJaHTdWHwRS&index=2
\subsection{Data Structures vs. Abstract Data Types}
When talking about data, we often hear about data structures and abstract data types. 

\begin{center}
    \begin{tabular}{|p{7cm}|p{7cm}|}
        \hline 
        \textbf{Data Structures} (DS) & \textbf{Abstract Data Type} (ADT) \\
        \hline 
        Data structures are collections that contain: 
        \begin{itemize}
            \item Data values. 
            \item Relationships among the data. 
            \item Operations applied to the data. 
        \end{itemize}
        It also describes how the data are organized and how tasks are performed. So, a data structure defines every single detail about anything relating to the data. 
        &
        Abstract data types are defined primarily by its \underline{behavior} from the view of the \underline{user}. So, not necessarily how the operations are done, but rather what operations it must have from a completely abstract point of view.  
    
        \bigskip 
    
        Specifically, it describes only what needs to be done, not how it's done. \\
        \hline 
    \end{tabular}
\end{center}

Consider the \code{ArrayList} (DS) vs. the \code{List} (ADT).
\begin{itemize}
    \item A \code{List} will most likely have the following operations: 
    \begin{itemize}
        \item \code{add}: Adds an element to the list.
        \item \code{find}: Does an element exist in the list? 
        \item \code{remove}: Remove an element from the list. 
        \item \code{size}: How many elements are in this list? 
        \item \code{ordered}: Each element should be ordered in the way we added it. For example, if we added \code{5}, and \emph{then} added \code{3}, and \emph{then} added \code{10}, our list should look like: \code{[5, 3, 10]}.   
    \end{itemize}

    Of course, as an abstract data type, \code{List} isn't going to define how these operations work. It just lists all operations that any implementing data structure must have. In other words, we can think of \code{List}, or any abstract data type, as a \emph{blueprint} for future data structures. 

    \item An \code{ArrayList} is simply an array that is expandable. It is internally backed by an \underline{array}. So, we can perform the following operations: 
    \begin{itemize}
        \item We can \code{add} an element to the \code{ArrayList}. In this case, we add the element to the next available slot in the array, expanding the array if necessary. 
        \item We can \code{find} an element in the \code{ArrayList}. In this case, we can search through each slot of the array until we find the array or we reach the end of the array.
        \item We can \code{remove} an element from the \code{ArrayList}. In this case, we can simply move every element after the specified element back one slot. 
        \item We can get the \code{size} of the \code{ArrayList}. In this case, this is as simple as seeing how many elements are in this \code{ArrayList}.
        \item And, we know that the \code{ArrayList} is \code{ordered}. In this case, this is already done via the \code{add} and \code{remove} methods.  
    \end{itemize}
    Notice how \code{ArrayList} specifies how each operation defined by \code{List} works. In this sense, we say that \code{ArrayList} essentially implements \code{List} because we need to define \emph{how} the tasks defined by \code{List} are performed. 
\end{itemize}
So, the key takeaways are: 
\begin{itemize}
    \item An abstract data type (in our case, \code{List}) specifies what needs to be done without specifying how it's done. 
    \item A data structure (in our case, \code{ArrayList}) actually defines \textbf{how} the data is organized, how the different operations are performed, and how exactly everything is represented.
\end{itemize}



\newpage 
\section{Introduction to C++}
Here, we will talk about C++, the programming language that we will use in this course. 

% https://www.youtube.com/watch?v=8FGvlugzS5A&list=PLM_KIlU0WoXmkV4QB1Dg8PtJaHTdWHwRS&index=4
\subsection{Data Types}
First, we'll compare the data types in Java and C++. 
\begin{center}
    \begin{tabular}{|c|c|c|}
        \hline 
        \textbf{Data Type} & \textbf{Java} & \textbf{C++} \\ 
        \hline 
        \code{byte} & 1 byte & 1 byte \\ 
        \code{short} & 2 bytes & 2 bytes \\ 
        \code{int} & 4 bytes & 4 bytes \\ 
        \code{long} & 8 bytes & 8 bytes \\ 
        \code{long long} & & 16 bytes \\ 
        \hline 
        \code{float} & 4 bytes & \code{4 bytes} \\ 
        \code{double} & 8 bytes & \code{8 bytes} \\ 
        \hline 
        \code{boolean} & Usually 1 byte & \\ 
        \code{bool} &  & Usually 1 byte \\ 
        \code{char} & 2 bytes & 1 byte \\ 
        \hline 
    \end{tabular}
\end{center}
It should be mentioned that: 
\begin{itemize}
    \item In Java, you can only have signed data types. 
    \item In C++, you can have both signed and unsigned data types. 
    \item \code{boolean} (Java) and \code{bool} (C++) are effectively the same thing: they represent either \code{true} or \code{false}. 
\end{itemize}


\end{document}